\chapter{Conclusion} \label{conclusion}
As mentioned in Chapter \ref{introduction}, previous investigations of voice syncretism have been sporadic in the literature and implicit in nature, and a general cross-linguistic picture of the phenomenon has so far been lacking \citep[3f.]{malchukov:2017}. The main goal of this book has been to fill this gap by providing the first systematic typological investigation of syncretism between passive\is{passive voice}, antipassive\is{antipassive voice}, reflexive\is{reflexive voice}, reciprocal\is{reciprocal voice}, anticausative\is{anticausative voice}, causative\is{causative voice}, and applicative\is{applicative voice} verbal voiec marking based on a survey of 222 languages (see Appendix A). This final chapter provides a summary and overview of the main findings of the previous chapters (\sectref{conclusion:summary}) before addressing prospects for further research (\sectref{conclusion:future}).
 
\section{Summary and main findings} \label{conclusion:summary}
Chapter \ref{defining-voices} was dedicated to the definitions of the seven voices of focus in the book. Existing voice definitions commonly rely on certain notions like an \isi{argument-adjunct distinction}, \isi{transitivity}, \isi{grammatical roles} and an \isi{active voice} that are intuitively clear but notoriously difficult to establish as comparative concepts\is{comparative concept}. Rather than attempting to redefine such notions once again (as has often been done in the past), the notions have been avoided altogether in this book. Instead, Chapter \ref{defining-voices} proposed a new approach to voice definition based on a comparison between two clausal constructiones (i.e. diatheses\is{diathesis}) and their formal verbal marking in addition to their numbers of semantic participants\is{semantic participant} and the semantic roles of these. It was demonstrated in the chapter that these criteria alone suffice to define passives\is{passive voice}, antipassives\is{antipassive voice}, reflexives\is{reflexive voice}, reciprocals\is{reciprocal voice}, anticausatives\is{anticausative voice}, causatives\is{causative voice}, and applicatives\is{applicative voice} for use in a typological investigation of voice syncretism. Given their few criteria and wide scopes, the definitions can potentially be employed in future cross-linguistic research pertaining not only to voice syncretism but also to other typological aspects of voice.

Next, Chapter \ref{voice-syncretism} gave an overview of previous research on voice syncretism, recognising two main approaches in the literature: one with a semantic core meaning as its point of reference in the investigation of voices and their syncretism, and another with formal marking as its point of reference. The former approach has been common in studies of the infamous \isi{middle voice} where only voices believed to involve some kind of \isi{subject} \isi{affectedness} (e.g. \citealt{klaiman:1991}) have been in focus, notably the reflexive\is{reflexive voice}, reciprocal\is{reciprocal voice}, anticausative\is{anticausative voice} and passive\is{passive voice} voices. By contrast, the latter approach is essentially unrestricted in its semantic scope due to its focus on formal marking and therefore makes it suitable for the exploration of voice syncretism, for which reason it has been adopted in this book. In practice, this approach allows individual markers to be inspected with regard to their full range of semantic functions, and only markers that feature one or more functions qualifying as one of the seven voices of focus in this book have been further examined in terms of voice syncretism. Chapter \ref{voice-syncretism} also established three types of voice syncretism based on resemblance in voice marking. Type 1 syncretism denotes full resemblance in voice marking\is{voice syncretism, full resemblance -- type 1} (e.g. \ili{Gurr-Goni} reflexive-reciprocal \example{-yi}: \example{bu-yi-} ‘to hit self’ or ‘to hit e.o.’, \citealt[214]{green:1995}), type 2 syncretism denotes partial resemblance in voice marking\is{voice syncretism, partial resemblance -- type 2} (e.g. \ili{Assiniboine} applicative\is{applicative voice} \example{ki-} and reciprocal\is{reciprocal voice} \example{kicʰí-}: \example{ki-yúkʰą́} ‘to make room for sb.’, \example{kicʰí-pažipa} ‘to poke e.o.’, \citealt[263, 271]{cumberland:2005}), and type 3 syncretism denotes reverse resemblance in voice marking\is{voice syncretism, reverse resemblance -- type 3} (e.g. \ili{Alamblak} causative\is{causative voice} \example{hay-} and applicative\is{applicative voice} \example{-hay}: \example{hay-ni} ‘to make sb. go’, \example{suh-hay} ‘to fall for the benefit of sb.’, \citealt[209, 250, 255]{bruce:1979}). Type 1 syncretism can in turn be divided into two subtypes based on whether the full resemblance\is{voice syncretism, full resemblance -- type 1} in question is unconditioned (like in \ili{Gurr-Goni} above) or conditioned (for instance, in \ili{Sandawe} causative\is{causative voice} and applicative\is{applicative voice} voice marking is only identical, \example{-kw}, before a vowel: \example{mântshà-kw-ꜜé} ‘to make him eat sth.’ or ‘to eat sth. for his benefit’, \citealt[189]{steeman:2012}). These types are not restricted to voice syncretism, but can be applied to the investigation of other kinds of syncretism as well.

Having defined voice and voice syncretism, Chapter \ref{simpl-syncr} provided a systematic cross-linguistic synchronic investigation of simplex voice syncretism\is{voice syncretism, simplex}, denoting two voices sharing the same voice marking (e.g. reflexive-reciprocal syncretism). Given the seven voices of focus in this book, 21 patterns of such syncretism can logically be posited, and each of these patterns was covered by the chapter (see \tabref{tab:ch4:simplex-patterns} on page \pageref{tab:ch4:simplex-patterns}). The patterns were approached and examined from the perspective of minimal syncretism\is{voice syncretism, minimal}, which means that voice marking was discussed in relation to two voices at a time, even if the marking in question happens to have additional voice functions. Nevertheless, for the sake of transparency, maximal syncretism\is{voice syncretism, maximal} -- or the full range of functions -- of any given voice marking was duly described as well. Prior research on simplex voice syncretism\is{voice syncretism, simplex} has tended to focus only on certain patterns of simplex voice syncretism\is{voice syncretism, simplex}, notably \isi{middle syncretism} (involving the passive\is{passive voice}, reflexive\is{reflexive voice}, reciprocal\is{reciprocal voice}, and/or anticausative\is{anticausative voice} voices), yet Chapter \ref{simpl-syncr} demonstrated that most of the 21 patterns mentioned above are attested in one or more language. In fact, only one pattern remains unattested altogether, applicative-anticausative syncretism, which is not particularly surprising considering the seemingly disparate functions of the applicative\is{applicative voice} and anticausative\is{anticausative voice} functions: the former voice is generally associated with a reduction in semantic participants\is{semantic participant}, while the latter voice is associated with an increase. However, other seemingly incongruous patterns were actually attested in the survey, for example causative-anticausative syncretism and passive-antipassive syncretism in four languages each. The attestations of such unexpected patterns suggest that disparity and incongruity defined in theory is not necessarily always reflected in practice. 

Whereas Chapter \ref{simpl-syncr} focused on simplex voice syncretism\is{voice syncretism, simplex}, Chapter \ref{sec:complex-syncretism} provided a cross-linguistic synchronic investigation of complex voice syncretism\is{voice syncretism, complex}, which refers to more than two voices sharing the same voice marking. Unlike the previous chapter, this chapter approached voice syncretism from the perspective of maximal syncretism\is{voice syncretism, maximal}, looking at the full range of voice functions of any given voice marking. Given the seven voices of focus in this book, 99 patterns of such syncretism can logically be posited, but only seventeen of these were actually attested in the survey (see \tabref{tab:ch5:complex-patterns} on page \pageref{tab:ch5:complex-patterns}), leaving 82 patterns unattested altogether. Twelve of the seventeen patterns involve three voices (e.g. reflexive-reciprocal-anticausative syncretism), four patterns involve four voices (e.g. anti\-pas\-sive-re\-flex\-ive-reciprocal-anticausative syncretism), while a single pattern involves five voices (i.e. passive-antipassive-reflexive-reciprocal-anticausative syncretism). The latter pattern has so far only been attested in the Permic languages \ili{Udmurt} and \ili{Komi} as well as in the Slavic language \ili{Russian} (all three \lang{ea}). It is hardly surprising that no complex patterns\is{voice syncretism, complex} involving six or seven voices have been attested in the survey, as such patterns would entail a high degree of functional ambiguity. This is even true for the complex voice syncretism\is{voice syncretism, complex} in \ili{Udmurt}, \ili{Komi} and \ili{Russian}, yet in this unique case the context and the semantics of verbs apparently suffice to tell the voice meanings apart. Thus, it seems that the voice syncretism in these three languages currently represents the upper limit of how many voices might share the same voice marking. 

Chapter \ref{sect:distribution} presented a statistical distributional overview of simplex\is{voice syncretism, simplex} and complex syncretism\is{voice syncretism, complex} attested in the language sample preceded by brief discussions of the distribution of voices in general and voice marking dedicated to a single voice. 104 of the 222 languages in the sample (46.8 percent) were found to feature some kind of voice syncretism (see \tabref{tab:ch6:voice-syncretism} on page \pageref{tab:ch6:voice-syncretism}), in the vast majority of cases type 1a syncretism. However, it is worth noting that 25 of these 104 languages (24.0 percent) feature type 2 voice syncretism\is{voice syncretism, partial resemblance -- type 2}, and this type of syncretism thus seems to be more prevalent cross-linguistically than generally acknowledged. By contrast, type 1b and type 3 voice syncretism\is{voice syncretism, reverse resemblance -- type 3} are rare, yet the attestations of these types in six languages each show that resemblance in voice marking can be an intricate phenomenon in its own right (see \tabref{tab:ch6:voice-syncretism-type-1} and \tabref{tab:ch6:voice-syncretism-type-2} on page \pageref{tab:ch6:voice-syncretism-type-1}). The reflexive\is{reflexive voice} voice was found to be more prone to be syncretic than the other six voices of focus in the book (see \tabref{tab:ch6:voice-syncretism-macroarea-2} on page \pageref{tab:ch6:voice-syncretism-macroarea-2} and \figref{ch6:fig:scale-syncretism} on page \pageref{fig:ch07:syncretism-diachrony}) which indicates that the traditional attention given to the reflexive\is{reflexive voice} voice in discussions of syncretism is not unfounded (e.g. \citealt{geniusiene:1987}). Individual patterns of voice syncretism were approached both from the perspective of minimal syncretism\is{voice syncretism, minimal} and from the perspective of maximal syncretism\is{voice syncretism, maximal}. The former discussion showed that \isi{middle syncretism} is undoubtedly more prevalent cross-linguistically than other patterns, yet patterns of causative-applicative as well as causative-passive syncretism and not least patterns of antipassive\is{antipassive voice} syncretism are comparatively common as well (see \tabref{tab:ch6:voice-syncretism-simplex} on page \pageref{tab:ch6:voice-syncretism-simplex} and \figref{ch6:fig:attestations-minimal} on page \pageref{ch6:fig:attestations-minimal}). Thus, it seems that patterns other than those associated with \isi{middle syncretism} might deserve more attention (as also suggested by, e.g., \citealt{malchukov:2017} in relation to causative-applicative syncretism and by \citealt{janic:2010} in relation to antipassive syncretism), and it is not unlikely that they have been overlooked in many languages and genera\is{genus} outside the language sample employed in this book. In terms of maximal syncretism\is{voice syncretism, maximal}, only seven patterns were attested in more than five languages, six of which are simplex\is{voice syncretism, simplex} and only one pattern complex\is{voice syncretism, complex}. All other patterns of maximal\is{voice syncretism, maximal} simplex\is{voice syncretism, simplex} and complex patterns\is{voice syncretism, complex} have only been attested in less than a handful of languages (see \tabref{tab:ch6:voice-syncretism-simplex} on page \pageref{tab:ch6:voice-syncretism-simplex} and \tabref{tab:ch6:voice-syncretism-maximal-simplex-macroarea} on page \pageref{tab:ch6:voice-syncretism-maximal-simplex-macroarea}). In terms of geography, it has proved difficult to make any broad generalisations about the macroareal\is{macroarea} distribution of voice syncretism due to the sporadic and limited attestations of most patterns. However, it can be noted that voice syncretism seems to be most prevalent in Australia and most diverse in North America, while it is least prevalent and least diverse in Papunesia (see \tabref{tab:ch6:voice-syncretism} on page \pageref{tab:ch6:voice-syncretism} and \tabref{tab:ch6:voice-syncretism-macroarea-minimal} on page \pageref{tab:ch6:voice-syncretism-macroarea-minimal}).

Finally, Chapter \ref{sec:diachrony} provided a diachronic investigation\is{diachronic development} of voice syncretism, or more specifically an investigation of cases of syncretic voice marking for which it can be plausibly demonstrated that one voice function evolved prior to other voice functions. Given the seven voices of focus in this book, 42 directional paths of voice development can logically be posited. Plausible evidence was found and discussed for twenty of these paths (see \tabref{tab:ch7:developmental-paths} on page \pageref{tab:ch7:developmental-paths}), several of which have received little or no prior treatment in the literature. Twelve of the twenty paths represent six bidirectional developments, some of which have traditionally been considered unidirectional\is{diachronic development, unidirectional} in the literature (e.g. reflexive-reciprocal syncretism of \isi{reflexive origin}). Thus, the findings presented in the chapter indicate that the diachrony of many patterns of voice syncretism may be more complicated and unpredictable than previously believed (see \figref{fig:ch07:syncretism-diachrony} on \pageref{fig:ch07:syncretism-diachrony}). 

\section{Prospects for further research} \label{conclusion:future}
Having mapped the cross-linguistic and typological variation in the syncretism between passives\is{passive voice}, antipassives\is{antipassive voice}, reflexives\is{reflexive voice}, reciprocals\is{reciprocal voice}, anticausatives\is{anticausative voice}, causatives\is{causative voice}, and applicatives\is{applicative voice}, this book naturally invites for further research. Voice syncretism is a broad topic and this book has only touched upon certain aspects of the phenomenon, while other aspects have not been covered in detail. Most notably, syntactic aspects have only been mentioned sporadically, and potential correlations between voice syncretism and syntactic language-internal characteristics (e.g. morphosyntactic \isi{alignment}, head and dependent marking, etc.) have not been discussed at all. Neither have relationships between voice syncretism and semantic verb classes\is{verb class} (see, e.g., \citealt{malchukov:2015} and \citealt{wichmann:2015}). Furthermore, as mentioned repeatedly in the previous chapters, individual voices are commonly associated with various additional semantic functions (e.g. reciprocals\is{reciprocal voice} with sociativity and antipassives\is{antipassive voice} with \isi{aspect}) which have not been covered systematically in this book for practical reasons due to their sheer numbers. As demonstrated in Chapter \ref{sec:diachrony} on diachrony, some of these functions are clearly relevant to the evolution of voice syncretism in some languages, but the extent of their cross-linguistic relevance is yet to be determined more exactly. In other words, it remains unclear how widely applicable many of the proposed diachronic explanations\is{diachronic development} are cross-linguistically due to the limited evidence available for many developmental paths and genera\is{genus}. 

As hinted throughout the previous chapters, there are indications of certain diachronic developments\is{diachronic development} in some languages (for instance in the form of synchronic distribution of functions) but without additional comparative and/or historical data it is difficult to confirm that such indications are valid. Moreover, some diachronic developments\is{diachronic development} appear to be bidirectional, but it remains unclear what conditions the directionality. Say, why does reflexive\is{reflexive voice} voice marking develop a reciprocal\is{reciprocal voice} function in some languages but reciprocal\is{reciprocal voice} voice marking develop a reflexive\is{reflexive voice} function in others. Consider, for instance, the reflexive-reciprocal prefix \example{ze-} in the Tupi-Guaraní language \ili{Emerillon} (\lang{sa}) and the reflexive-reciprocal suffix \example{-nʸji} in the Gunwinyguan language \ili{Nunggubuyu} (\lang{au}). The former affix reflects \ili{Proto-Tupi-Guaraní} reflexive\is{reflexive voice} \example{*je-} and has entirely replaced reciprocal\is{reciprocal voice} \example{*jo-} \citep{jensen:1998}, while the latter affix reflects \ili{Proto-Gunwinyguan} reciprocal\is{reciprocal voice} \example{*-nci} and has almost entirely replaced reflexive\is{reflexive voice} \example{*-yi} \citep{alpher:al:2003}. Furthermore, it remains unknown to what extent rare patterns of voice syncretism are the result of coincidental convergence or the result of more systematic (albeit infrequent) processes of development. Although various functional explanations can be -- and have been -- proposed for the rise of such patterns, evidence remains scarce and restricted to a few languages. One obvious place to look for more evidence would be among related languages through genus-\is{genus} or family-specific case studies. Case studies could also show if patterns of voice syncretism in individual languages reflect genus-\is{genus} or family-wide tendencies. Many of the languages with the most complex voice syncretism\is{voice syncretism, complex} attested in this book belong to big and rather well-documented language families (e.g. Uto-Aztecan, Oto-Manguean, Iroquoian, Turkic, and of course Indo-European) and such studies should therefore be feasible. In turn, macroarea-specific\is{macroarea} case studies might turn up more evidence for tendencies in the geographic distribution of voice syncretism.

Voice syncretism is evidently a diverse and multifaceted phenomenon, and it is hoped that the findings and approach of this book can serve as inspiration and as a starting point for future typological exploration of the matter as well as for the investigation of other linguistic phenomena pertaining to voice and syncretism.