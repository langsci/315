\chapter{Introduction} \label{introduction}
This book is a typological study of resemblance in formal verbal marking between two or more of the following seven clausal constructions: passives\is{passive voice}, antipassives\is{antipassive voice}, reflexives\is{reflexive voice}, reciprocals\is{reciprocal voice}, anticausatives\is{anticausative voice}, causatives\is{causative voice}, and applicatives\is{applicative voice}. Following \citet{malchukov:2015, malchukov:2016, malchukov:2017}, \citet{creissels:2016}, and \citet{zuniga:kittila:2019}, these constructions are called \textsc{voices}. In turn, their formal marking is called \textsc{voice marking}, and any resemblance in voice marking is called \textsc{voice syncretism}. The latter term here denotes resemblance in formal marking regardless of whether the marking in two or more voices is related semantically and/or diachronically (\citealt[233f.]{zuniga:kittila:2019}). Thus, the term refers strictly to the polyfunctionality or \isi{coexpression} of voice marking (\citealt[21]{haspelmath:2019}). As discussed in \chapref{defining-voices}, voice itself has been a topic of much debate and innumerable definitions of the seven voices mentioned above have been proposed in the literature. Many definitions rely on notions like an \isi{argument-adjunct distinction}, \isi{transitivity}, \isi{grammatical roles} and/or an \isi{active voice} that are intuitively clear yet difficult to apply to different languages in a uniform manner. The sheer amount of literature dedicated to capturing the essence of the individual notions testifies to their elusive nature, and there does not seem to be any consensus as to how they are best defined for use in cross-linguistic investigations (\sectref{voices-revisited}). Rather than attempting to (re)define the notions once again, this book employs alternative voice definitions that avoid the notions altogether. The definitions instead rely solely on i) a comparison of two clausal constructions, ii) the number of semantic participants\is{semantic participant} in the constructions, iii) the semantic roles of certain semantic participants\is{semantic participant} in the constructions, and iv) the formal verbal marking of the constructions (\sectref{voices-redefined}). Observe that this book covers only voices that are formally marked on the verb, while periphrastic constructions of various kinds are largely excluded from the discussion.

It is well-known that two or more of the seven voices of focus in this book share the same voice marking in some languages. For instance, languages in which the reflexive\is{reflexive voice} and reciprocal\is{reciprocal voice} voices share the same marking can be found throughout the world. This \textsc{pattern} of voice syncretism (i.e. reflexive-reciprocal syncretism) is illustrated in \tabref{tab:ch1:examples} by examples from the Bantu language Namibian Fwe\il{Fwe, Namibian} of Africa \citep[269f.]{gunnink:2018}, the South-Central Dravidian language \ili{Telugu} of Eurasia \citep[226, 233]{subbarao:murthy:1999}, the Mangrida language \ili{Nakkara} of Australia \citep[251]{eather:2011}, the West Bougainville language \ili{Rotokas} of Papunesia \citep[101, 222]{robinson:s:2011}, the Mixe-Zoque language Ayutla Mixe\il{Mixe, Ayutla} of North America \citep[371f.]{romero-mendez:2009}, and the language isolate \ili{Kamsá} of South America \citep[129]{obrien:2018}. While some patterns of voice syncretism have been the focus of much scrutiny (like reflexive-reciprocal syncretism), discussions of most patterns of voice syncretism are generally sporadic and implicit in the literature, and a comprehensive typological survey of the phenomenon has hitherto not been undertaken \citep[3f.]{malchukov:2017}. This book strives to fill this gap with a systematic investigation of voice syncretism from both synchronic and diachronic perspectives through a survey of a language sample encompassing 222 languages (\sectref{sample}). 

\begin{table}
	\setlength{\tabcolsep}{2.7pt}
	\begin{tabularx}{\textwidth}{lllll}
		\lsptoprule
		& \textsc{refl} & & \textsc{recp} & \\
		\midrule
		Namib. Fwe\il{Fwe, Namibian} & \example{-\textbf{rì}-kùnkùmún-} & ‘to brush self’ & \example{-\textbf{rì}-shák-} & ‘to love e.o.’ \\
		\ili{Telugu} & \example{gillu-\textbf{konn}-} & ‘to pinch self’ & \example{tiṭṭu-\textbf{konn}-} & ‘to scold e.o.’ \\
		\ili{Nakkara} & \example{bburda-\textbf{ndjiya}-} & ‘to hit self’ & \example{kkulakki-\textbf{ndjiya}-} & ‘to wake e.o.’ \\
		\ili{Rotokas} & \example{\textbf{ora}-karekare-} & ‘to scratch self’ & \example{\textbf{ora}-uugaa-} & ‘to kiss e.o.’ \\
		Ayutla Mixe\il{Mixe, Ayutla} & \example{\textbf{nay}-tsuk-} & ‘to cut self’ & \example{\textbf{nay}-akook-} & ‘to kill e.o.’ \\
		\ili{Kamsá} & \example{\textbf{en}-onÿ-} & ‘to see self’ & \example{\textbf{en}-chwaye-} & ‘to greet e.o.’ \\
		\lspbottomrule
	\end{tabularx}
	\caption{Reflexive-reciprocal syncretism across the world}
	\label{tab:ch1:examples}
\end{table} 

The reflexive-reciprocal syncretism illustrated in \tabref{tab:ch1:examples} represents one of 21 logically possible patterns of voice syncretism when one considers two of the seven voices sharing the same voice marking. As discussed in \chapref{voice-syncretism}, previous research on voice syncretism has often focused on a subset of these patterns (notably patterns involving the reflexive\is{reflexive voice}, reciprocal\is{reciprocal voice}, anticausative\is{anticausative voice} and/or passive\is{passive voice} voices) and on full resemblance in voice marking\is{voice syncretism, full resemblance -- type 1} between voices (\sectref{previous-research}). However, only one of the 21 patterns actually remains unattested in the language sample and voice syncretism therefore seems to deserve more attention than it has received in the literature so far. For instance, growing evidence shows that syncretism involving the \isi{antipassive voice} is genealogically and geographically widespread (see, e.g., \citealt{janic:2010}) and the same is true for causative-applicative syncretism (see, e.g., \citealt{malchukov:2017}). In terms of marking, it is worth observing that in some languages the voice marking in one voice does not necessarily bear full resemblance to the voice marking in another voice under all conditions, only under certain conditions -- or the resemblance might be only partial\is{voice syncretism, partial resemblance -- type 2} in the first place (\sectref{resemblance}). Such variation in voice marking has received little attention with the notable exception of \citet[243f.]{nedjalkov:2007d} who distinguishes between “combined markers” and “complex morphological markers” in his investigation of reciprocal syncretism. According to \citeauthor{nedjalkov:2007d}, the former kind of markers indicate that “both meanings are expressed by the same marker”. In turn, the latter kind of markers “share a common component”, as illustrated in \tabref{tab:ch1:nedjalkov-examples} by examples from the North Halmaheran language \ili{Tidore} (\lang{pn}), the Northern Pama-Nyungan language \ili{Uradhi} (\lang{au}), Bolivian Quechua\il{Quechua, Bolivian} (\lang{sa}), and the Algic language \ili{Yurok} (\lang{na}). The main focus of this book is full resemblance in voice marking\is{voice syncretism, full resemblance -- type 1} (like in \tabref{tab:ch1:examples}) but partial resemblance (like in \tabref{tab:ch1:nedjalkov-examples}) is covered as well for the sake of linguistic diversity.

\begin{table} 
	\setlength{\tabcolsep}{2pt}
	\begin{tabularx}{\textwidth}{lccl}
		\lsptoprule
		& \textsc{refl} & \textsc{recp} & \\
		\midrule
		\ili{Tidore} & \example{ma-} & \example{ma-ku-} & \\
		\ili{Yurok} & \example{-ep} & \example{-ep-ew} & \\
		\ili{Uradhi} & \example{-ːni} & \example{-ːni-βa} & (e.g. \example{uta-ni} ‘to cut self’, \example{uta-ni-βa} ‘to cut e.o.’)  \\
		Quechua\il{Quechua, Bolivian} & \example{-ku} & \example{-na-ku} & (e.g. \example{riku-ku} ‘to look at self’, \example{riku-na-ku} ‘to look at e.o.’)\\
		\lspbottomrule
	\end{tabularx}
	\caption{Partial resemblance in voice marking \citep[244]{nedjalkov:2007d}}
	\label{tab:ch1:nedjalkov-examples}
\end{table}

For the sake of convenience, in this book the term \textsc{simplex voice syncretism}\is{voice syncretism, simplex} serves as a shorthand for the 21 patterns of voice syncretism that are logically possible when one considers \textit{two} of the seven voices sharing the same voice marking. These 21 patterns of simplex voice syncretism\is{voice syncretism, simplex} are discussed in \chapref{simpl-syncr} which provides a general overview of voice syncretism and offers easy access to examples and information about each pattern. In the chapter the 21 patterns are divided into four groupings: \isi{middle syncretism} (\sectref{sec:simple-syncretism:middle}), antipassive syncretism (\sectref{sec:simple-syncretism:antipassive}), causative syncretism (\sectref{sec:simple-syncretism:causative}), and applicative syncretism (\sectref{sec:simple-syncretism:applicative}). While these groupings are essentially arbitrary and primarily serve to facilitate the discussion of the many patterns of voice syncretism in a clear and structured manner, the groupings do reflect the frequencies of the various patterns to some extent. For instance, \isi{middle syncretism} (involving the reflexive\is{reflexive voice}, reciprocal\is{reciprocal voice}, anticausative\is{anticausative voice} and/or \isi{passive voice}) is considerably more common than applicative syncretism. Furthermore, in some languages one or more of the 21 patterns of simplex voice syncretism\is{voice syncretism, simplex} can form part of extended voice syncretism, in which more than two voices share the same voice marking (e.g. reflexive-reciprocal-anticausative syncretism). The term \textsc{complex voice syncretism}\is{voice syncretism, complex} serves as a shorthand for the 99 patterns of voice syncretism that are logically possible when one considers \textit{more than two} voices sharing the same marking. However, only seventeen of the 99 patterns have actually been attested in the language sample and these patterns are discussed groupwise in \chapref{sec:complex-syncretism} in terms of \isi{middle syncretism} (\sectref{sec:complex-syncretism:middle}), antipassive syncretism (\sectref{sec:complex-syncretism:antipassive}) and causative syncretism (\sectref{sec:complex-syncretism:causative}). Each of these groupings covers voice marking shared by three or four voices. In turn, a fourth grouping is reserved for voice marking exceptionally shared by five voices. While several patterns of voice syncretism involving three or four voices have been attested cross-linguistically, syncretism involving five voices has so far only been found in Permic languages and in the Slavic language \ili{Russian} (\sectref{sec:complex-syncretism:multiplex}).

This book maintains an important distinction between \textsc{maximal syncretism}\is{voice syncretism, maximal} and \textsc{minimal syncretism}\is{voice syncretism, minimal} which essentially represent two different manners of approaching syncretism. The former term refers to syncretic marking and its maximal scope -- or full range -- of functions, whereas the latter term refers minimally to precisely two functions of syncretic marking, even if the marking in question also has additional functions. For the sake of illustration, consider the reflexive-reciprocal syncretism of the languages \ili{Telugu} and \ili{Rotokas} shown in \tabref{tab:ch1:examples} on page \pageref{tab:ch1:examples}. In \ili{Telugu} the suffix \example{-kon(n)} not only has a reflexive\is{reflexive voice} and a reciprocal\is{reciprocal voice} function but also an anticausative\is{anticausative voice} function. Thus, the maximal functional scope of the suffix \example{-kon(n)} is complex reflexive-reciprocal-anticausative syncretism\is{voice syncretism, complex}, but if the syncretism is viewed minimally in terms of pairwise patterns, it entails reflexive-reciprocal, reflexive-anticausative, and reciprocal-anticausative simplex syncretism\is{voice syncretism, simplex}. By contrast, in Rotokas the prefix \example{ora-} has only a reflexive\is{reflexive voice} and a reciprocal\is{reciprocal voice} function, and the maximal syncretism\is{voice syncretism, maximal} of the prefix therefore equals its minimal syncretism\is{voice syncretism, minimal}. These two cases are summarised in \tabref{tab:ch1:maximal-minimal}. Evidently, both \ili{Rotokas} and \ili{Telugu} clearly feature voice marking shared by both the reflexive\is{reflexive voice} and reciprocal\is{reciprocal voice} voices (minimal syncretism)\is{voice syncretism, minimal}. Nevertheless, it is important to keep in mind that the voice marking in \ili{Telugu} also has an anticausative\is{anticausative voice} function (full syncretism), for which reason the suffix \example{-kon(n)} in this language does not fully correspond to the prefix \example{ora-} in \ili{Rotokas} despite the shared reflexive-reciprocal syncretism. This distinction is generally not maintained in the literature in which affixes like \ili{Telugu} \example{-kon(n)} and Rotokas prefix \example{ora-} tend to be treated on par with each other despite their apparent functional differences. To account for such differences, the distinction is maintained throughout this book, although it is only explicitly mentioned when relevant. \chapref{simpl-syncr} treats simplex voice syncretism\is{voice syncretism, simplex} primarily in terms of minimal syncretism\is{voice syncretism, minimal}, whereas \chapref{sec:complex-syncretism} on complex voice syncretism\is{voice syncretism, complex} focuses exclusively on maximal syncretism\is{voice syncretism, maximal}.

\begin{table} 
 	\begin{tabularx}{.72\textwidth}{lll}
 		\lsptoprule
 		\multicolumn{3}{l}{Focus: Maximal voice syncretism} \\
 		\midrule
 		\ili{Rotokas} & \example{ora-} & \textsc{refl-recp} \\
 		\ili{Telugu} & \example{-kon(n)} & \textsc{refl-recp-antc} \\
 		\midrule\midrule
 		\multicolumn{3}{l}{Focus: Minimal voice syncretism} \\
 		\midrule
 		\ili{Rotokas} & \example{ora-} & \textsc{refl-recp} \\
 		\ili{Telugu} & \example{-kon(n)} & \textsc{refl-recp}, \textsc{refl-antc}, \textsc{recp-antc} \\
 		\lspbottomrule
 	\end{tabularx}
 	\caption{Maximal and minimal voice syncretism compared}
 	\label{tab:ch1:maximal-minimal}
\end{table}

The cross-linguistic distribution of voice syncretism is explored in \chapref{sect:distribution}. As the presence of two or more voices in a language are a natural prerequisite for voice syncretism to be attested, the chapter in question also covers the distribution of voices more generally (\sectref{dist:voices}). The chapter even provides a brief treatment of \isi{dedicated voice marking}, in other words voice marking restricted to a single voice in a language (\sectref{dist:dedicated}). For the sake of transparency, the distribution of voice syncretism itself is discussed first in terms of minimal\is{voice syncretism, minimal} simplex syncretism\is{voice syncretism, simplex} (\sectref{dist:minimal}) and then in terms of maximal\is{voice syncretism, maximal} simplex and complex syncretism\is{voice syncretism, complex} (\sectref{dist:maximal}). As demonstrated in the discussion of minimal\is{voice syncretism, minimal} simplex voice syncretism\is{voice syncretism, simplex}, patterns of \isi{middle syncretism} (involving the reflexive\is{reflexive voice}, reciprocal\is{reciprocal voice}, anticausative\is{anticausative voice} and/or passive\is{passive voice} voices) are among the most common patterns of voice syncretism attested in the language sample. However, causative-applicative and causative-passive syncretism are also rather frequent cross-linguistically, and the same is true for patterns of antipassive syncretism. By contrast, most other patterns are only marginally attested in the language sample. For example, eight patterns are attested in less than five languages each. When it comes to maximal\is{voice syncretism, maximal} complex voice syncretism\is{voice syncretism, complex}, only one pattern -- reflexive-reciprocal-anticausative syncretism -- is attested in more than five languages, one of which is \ili{Telugu} already discussed above. \chapref{sect:distribution} also deals with the geographic distribution of voice syncretism and resemblance in voice marking.

The diachrony of certain patterns of voice syncretism has received considerable attention in the literature, most notably patterns of \isi{middle syncretism}, particularly in relation to Indo-European languages. It is, for instance, well-known that the \ili{Proto-Indo-European} \isi{reflexive pronoun} \example{*s(u)e} has grammaticalised\is{grammaticalisation} and developed reciprocal\is{reciprocal voice}, anticausative\is{anticausative voice}, passive\is{passive voice} and/or antipassive\is{antipassive voice} uses in several descendant languages (\sectref{diachrony:reflexive}). Likewise, it has been known for more than one and a half century that causative\is{causative voice} marking in some languages has developed a passive\is{passive voice} function (\sectref{diachrony:caus2pass}). By contrast, the diachrony of most other patterns of voice syncretism has only received sporadic and often scarce treatment in the literature, or been ignored altogether. Prior research on the diachrony of voice syncretism is reviewed in \chapref{sec:diachrony} which also presents new evidence for a variety of diachronic voice developments. Assuming that each of the seven voices of focus in this book can theoretically be the origin for each of the other voices, there are logically 42 potential paths of development. However, it has only been possible to find plausible evidence for twenty of these paths, and those twenty paths constitute the main focus of the discussion on the diachrony of voice syncretism. The paths are discussed according to origin: \isi{reflexive origin} (\sectref{diachrony:reflexive}), \isi{reciprocal origin} (\sectref{diachrony:reciprocal}), \isi{anticausative origin} (\sectref{diachrony:anticausative}), \isi{passive origin} (\sectref{diachrony:passive}), \isi{causative origin} (\sectref{diachrony:causative}), and \isi{applicative origin} (\sectref{diachrony:applicative}). Antipassive voice\is{antipassive voice} marking has so far not been observed to have developed any of the other six voice functions in any language. The chapter demonstrates that several voice developments can potentially be bidirectional,\is{diachronic development, bidirectional} including ones that have traditionally been considered unidirectional.\is{diachronic development, unidirectional} For instance, it has often been stated that \isi{reflexive voice} marking might develop a reciprocal\is{reciprocal voice} function but not vice versa (e.g. \citealt{heine:2000, heine:miyashita:2008}), yet it appears that \isi{reciprocal voice} marking actually has developed a reflexive\is{reflexive voice} function in several languages across the world (\sectref{diachrony:recp2refl}). The findings thus suggest that the diachrony of voice syncretism is less predictable and more diverse than generally assumed.

Finally, \chapref{conclusion} concludes the book with a summary of the main findings presented in previous chapters (\sectref{conclusion:summary}) in addition to a look at prospects for further research (\sectref{conclusion:future}).

\section{Language sample} \label{sample}
As mentioned in the introduction, this book provides a typological survey of voice syncretism in 222 languages. These languages represent a \isi{variety sample} which has been designed according to the Genus-Macroarea\is{macroarea}\is{genus} sampling method conceived by \citet{miestamo:2003, miestamo:2005} and further elaborated by \citet[247ff.]{miestamo:al:2016}. In the spirit of \citet{bell:1978} and \citet{dryer:1989, dryer:1992, dryer:2000}, this method incorporates stratification for genealogical and geographical affiliation, and there-by ensures a high degree of interlingual independence which, in turn, promotes linguistic diversity. \citet[238ff.]{miestamo:al:2016} define a \textsc{genus}\is{genus} as “a level of genealogical classification intended to be comparable across the world in terms of time depth” which “is not more than approximately 3,500 to 4,000 years”, and \textsc{macroareas}\is{macroarea} as “continent-size linguistic areas which are independent of each other, but within which languages are to some extent typologically similar due to either (ancient) contact or (very deep) genealogical affinity” (see also \citealt[84]{dryer:1992}). The number of genera\is{genus} in the world has variously been estimated to be 478 \citep{bell:1978}, 322 \citep{dryer:1989}, 458 \citep{dryer:2005}, 413 \citep{miestamo:2005} and 521 \citep{dryer:2013, miestamo:al:2016}. In this book 542 genera\is{genus} are acknowledged in accordance with the World Atlas of Language Structures (WALS) as of August 2019. Likewise, following WALS, six macroareas\is{macroarea} are recognised: Africa, Eurasia, Papunesia, Australia, North America, and South America (for a more detailed discussion of these macroareas\is{macroarea}, see \citealt{hammarstrom:donohue:2014}). \tabref{tab:ch1:wals} provides an overview of these genera\is{genus} and macroareas\is{macroarea}. 

\begin{table}
	\begin{tabularx}{.4\textwidth}{lcc}
		\lsptoprule
		& \# & \% \\
		\midrule
		Africa & 77 & 14.2 \\
		Eurasia & 82 & 15.1 \\
		Australia & 42 & 7.8 \\
		Papunesia & 136 & 25.1 \\
		North America & 101 & 18.6 \\
		South America & 104 & 19.2 \\
		\lspbottomrule
	\end{tabularx}
	\caption{Genera and macroareas according to WALS}
	\label{tab:ch1:wals}
\end{table} 

The variety sample employed in this book represents a so-called core sample in which all languages belong to different genera\is{genus} \citep[250ff.]{miestamo:al:2016}. The 222 languages of the sample have been chosen one by one from different macroareas\is{macroarea} in alternating turns. The genera\is{genus} have been chosen largely on a random basis, yet availability of data has naturally had an effect on the choice of genera\is{genus} as well. For instance, genera\is{genus} encompassing languages for which little data are currently available have been ignored, and recent comprehensive descriptive grammars have generally been preferred over older less detailed grammatical descriptions. Furthermore, an attempt has been made to include genera\is{genus} from as many distinct language families as possible. However, as noted by \citet[257f.{miestamo:al:2016}, “[u]nless the size of the sample is very small, the number of distinct language families is soon exhausted for some macroareas”,\is{macroarea} in which case genera\is{genus} from the same language families must be chosen. Thus, certain language families are represented by more than one genus\is{genus} in the language sample. \tabref{tab:ch1:sample} shows the geographical distribution of the genera\is{genus} represented in the sample. 

\begin{table}
	\begin{tabularx}{.4\textwidth}{lcc}
		\lsptoprule
		& \# & \% \\
		\midrule
		Africa & 39 & 50.7 \\
		Eurasia & 41 & 50.0 \\
		Australia & 21 & 50.0 \\
		Papunesia & 48 & 35.3 \\
		North America & 36 & 35.6 \\
		South America & 37 & 35.6 \\
		\lspbottomrule
	\end{tabularx}
	\caption{Language sample according to genera and macroareas}
	\label{tab:ch1:sample}
\end{table} 

\newpage

The percentages in \tabref{tab:ch1:sample} are based on the total numbers of genera\is{genus} in the individual macroareas\is{macroarea} and not on the total number of genera\is{genus} in the world. Collectively, the 222 genera\is{genus} shown in the table represent roughly 41 percent of the world’s 542 genera\is{genus}. There are currently considerably more satisfactory data readily available for genera\is{genus} of the African, Eurasian, and Australian macroareas\is{macroarea} than for genera\is{genus} of the Papuan, North American, and South American macroareas\is{macroarea}, and this \isi{bibliographical bias} \citep[106f.]{bakker:2010} explains the percentual differences in coverages of the six macroareas\is{macroarea} in \tabref{tab:ch1:sample}. The language sample is evidently proportionally biased slightly towards the Old World and Australia, though it is worth observing that the New World and Papunesia are better represented in absolute numbers. A restricted sample \citep[250f.]{miestamo:al:2016} could alternatively be extracted from the core sample by lowering the percentages of the African, Eurasian, and Australian macroareas\is{macroarea} from 50 to 35 percent, but this would inevitably lead to loss of diversity and has therefore not been done here. In any case, a chi-squared goodness-of-fit test of the sample in \tabref{tab:ch1:sample} based on the expected proportions for each macroarea\is{macroarea} listed in \tabref{tab:ch1:wals} shows that the differences in the distribution of genera\is{genus} across the macroareas\is{macroarea} are not statistically significant (p-value = 0.241). Thus, the geographical distribution of genera\is{genus} in the sample is considered reasonably balanced. 

The 222 languages in the sample representing 222 genera\is{genus} are plotted onto the map in \figref{fig:ch1:map} to give an idea of their geographic distribution. The individual languages are all listed in Appendix A alongside information about macroarea\is{macroarea} and genus\is{genus}. 

% https://simplemaps.com/resources/svg-world

\begin{sidewaysfigure}
	\centering
	\def\svgwidth{\textwidth}
	\input{figures/map.pdf_tex}
	\caption{Map of languages in language sample}
	\label{fig:ch1:map}
\end{sidewaysfigure}

\section{Sources and data} \label{sources}
Descriptive grammars have served as the primary data sources for the typological survey of this book. The data for most of the 222 languages discussed in the previous section come from a single source each, but for a few languages data have been obtained from multiple sources, including articles and dictionaries. In cases where more than one source has been consulted for the same language, care has been taken to ensure that all sources represent the same variety or dialect. Additionally, some data for some languages have been obtained through fieldwork as well as personal correspondence, and some data are based on personal knowledge. Data of these kinds are duly noted in the book where relevant, while all primary sources are listed in Appendix A. The actual data are given in Appendices B and C which cover voice attestations and syncretic voice marking, respectively. More details about the data are provided in the respective appendices.
