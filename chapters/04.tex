\chapter{Simplex voice syncretism} \label{simpl-syncr}
Given the seven voices of focus in this book (i.e. passive\is{passive voice}, antipassive\is{antipassive voice}, reflexive\is{reflexive voice}, reciprocal\is{reciprocal voice}, anticausative\is{anticausative voice}, causative\is{causative voice}, applicative\is{applicative voice}), 21 patterns of voice syncretism can logically be posited when one considers \textit{two} voices sharing the same voice marking (\tabref{tab:ch4:simplex-patterns}). In this chapter these patterns of simplex voice syncretism\is{voice syncretism, simplex} are discussed in terms of minimal syncretism\is{voice syncretism, minimal}. In other words, syncretic voice marking is here discussed in relation to two voices at a time, even if the marking in question happens to have additional voice functions. Nevertheless, it is duly noted throughout the following sections if a pattern of simplex voice syncretism\is{voice syncretism, simplex} features voice marking that is shared by other patterns and voices as well (maximal\is{voice syncretism, maximal} syncretism). If the voice marking in any given pattern of simplex voice syncretism\is{voice syncretism, simplex} is not mentioned to have additional voice uses, the minimal syncretism\is{voice syncretism, minimal} of the marking equals its maximal syncretism\is{voice syncretism, maximal}. This distinction between minimal\is{voice syncretism, minimal} and maximal voice syncretism\is{voice syncretism, maximal} has been described in Chapter \ref{introduction}. Complex voice syncretism\is{voice syncretism, complex} is discussed in terms of maximal syncretism\is{voice syncretism, maximal} in the next chapter.  

\begin{table}
	\begin{tabularx}{.70\textwidth}{cccc}
		\lsptoprule
		Middle & Antipassive & Causative & Applicative \\
		\midrule
		\textsc{refl-recp} & \textsc{antp-refl} & \textsc{caus-appl} & \textsc{appl-pass} \\
		\textsc{refl-antc} & \textsc{antp-recp} & \textsc{caus-pass} & \textsc{appl-antp} \\
		\textsc{recp-antc} & \textsc{antp-antc} & \textsc{caus-antp} & \textsc{appl-refl} \\
		\textsc{pass-refl} & \textsc{pass-antp} & \textsc{caus-refl} & \textsc{appl-recp} \\
		\textsc{pass-recp} & & \textsc{caus-recp} & \textsc{appl-antc} \\
		\textsc{pass-antc} & & \textsc{caus-antc} & \\
		\lspbottomrule
	\end{tabularx}
	\caption{Patterns of minimal simplex voice syncretism}
	\label{tab:ch4:simplex-patterns}
\end{table} 

The 21 patterns of simplex voice syncretism\is{voice syncretism, simplex} covered by this chapter are divided into the four groupings shown in \tabref{tab:ch4:simplex-patterns} to facilitate their discussion in a convenient manner. Middle syncretism\is{middle syncretism} here refers to voice syncretism involving two of the following four voices: passive\is{passive voice}, reflexive\is{reflexive voice}, reciprocal\is{reciprocal voice}, and anticausative\is{anticausative voice} (\sectref{sec:simple-syncretism:middle}). In turn, antipassive\is{antipassive voice} syncretism here refers to voice syncretism involving the antipassive\is{antipassive voice} voice and one of the voices associated with \isi{middle syncretism} (\sectref{sec:simple-syncretism:antipassive}). By contrast, causative\is{causative voice} syncretism refers to any pattern of voice syncretism involving the causative\is{causative voice} voice (\sectref{sec:simple-syncretism:causative}), while applicative\is{applicative voice} syncretism refers to voice syncretism involving the applicative\is{applicative voice} voice and another voice except the causative\is{causative voice} voice (\sectref{sec:simple-syncretism:applicative}). The four groupings are essentially arbitrary, though it can be noted that the groupings reflect the frequencies of the various patterns in the language sample to some extent: patterns of \isi{middle syncretism} are generally more frequent than other patterns of syncretism cross-linguistically, while patterns of applicative\is{applicative voice} syncretism tend to be less frequent than other patterns. There are, however, a handful of exceptions to this generalisation. For instance, causative-applicative syncretism is more frequent than many patterns of middle\is{middle syncretism} and antipassive\is{antipassive voice} syncretism. In any case, the exact frequencies (and distribution) of the various patterns will not be discussed here but in Chapter \ref{sect:distribution}. Furthermore, observe that the order in which two voices are listed in any given pattern of syncretism is completely arbitrary and does not denote any particular \isi{diachronic development}. For instance, the pattern called “causative-passive” in \sectref{sec:simple-syncretism:caus-pass} could just as well have been called “passive-causative”, and the term itself does not necessarily indicate that the use of a causative\is{causative voice} marker has extended its functional scope to cover passivity\is{passive voice} (nor vice versa). Diachrony of voice syncretism is only briefly addressed in this chapter, but treated more extensively in Chapter \ref{sec:diachrony}. 

\section{Middle syncretism} \label{sec:simple-syncretism:middle}
Patterns of \isi{middle syncretism} are among the most common patterns of voice syncretism among the world’s languages (see \tabref{tab:ch6:voice-syncretism-simplex} on page \pageref{tab:ch6:voice-syncretism-simplex}), and the linguistic literature on \isi{middle syncretism} is accordingly vast, although the syncretism has generally been discussed rather implicitly (\sectref{middle-voice}). For practical reasons it is not feasible to describe and illustrate \isi{middle syncretism} in all the languages in which it is attested in the language sample nor is it possible to address and discuss all previous research dealing with the syncretism. Instead, as \isi{middle syncretism} is already a well-known phenomenon, the various patterns of the syncretism are only briefly described in the following sections. 

\subsection{Reflexive-reciprocal} \label{sec:simple-syncretism:refl-recp}
Reflexive-reciprocal syncretism is well-attested (\citealt{geniusiene:1987, knjazev:1998, nedjalkov:2007a}), although the extent of its prevalence has occasionally been questioned. For instance, \cite[66]{creissels:2016} argues that “[t]he re\-flex\-ive-reciprocal syncretism does not seem to be particularly widespread in the languages of the world, but it is found in several branches of Indo-European”. Nevertheless, re\-flex\-ive-reciprocal syncretism is by far the most frequently attested pattern of voice syncretism attested in the language sample. Indeed, the syncretism in question is attested in close to one fifth of all the languages in the sample, that is 49 languages (see \tabref{tab:ch6:voice-syncretism-simplex} on page \pageref{tab:ch6:voice-syncretism-simplex}) -- only one of which belongs to the Indo-European language family, Eastern Armenian\il{Armenian, Eastern} (\lang{ea}). The reflexive-reciprocal syncretism attested in the language sample is primarily of type 1\is{voice syncretism, full resemblance -- type 1}, though a handful of languages feature reflexive-reciprocal type 2 syncretism\is{voice syncretism, partial resemblance -- type 2}. For the sake of illustration, reflexive-reciprocal type 1 syncretism\is{voice syncretism, full resemblance -- type 1} is here described for one language. In the Algic language \ili{Arapaho} (\lang{na}) reflexive-reciprocal syncretism is characterised by the suffix \example{-etí}, as seen in the following \isi{voice relation} (\ref{ex:Arapaho:speak:a}↔\ref{ex:Arapaho:speak:b}). The voice in (\ref{ex:Arapaho:speak:b}) qualifies as either reflexive\is{reflexive voice} or reciprocal\is{reciprocal voice} depending on context. \cite[140]{cowell:moss:2008} argue that “[w]hen the person inflection is plural, either meaning can be possible and only context makes clear which is intended”. For comparison, the verb \example{henéétetí3-etí-noo} marked by the first person singular suffix \example{-noo} can only have a reflexive\is{reflexive voice} reading, ‘I am speaking to myself’ \citep[139]{cowell:moss:2008}. Note that the underlying verbal stem in both (\ref{ex:Arapaho:speak:a}) and (\ref{ex:Arapaho:speak:b}) is the same, \example{eeneti3}. The prefixal part \example{hen-} results from an “initial change” (glossed \textsc{ic}), “a morphophonological\is{morphophonology} process that serves grammatically to indicate either present \isi{tense} and ongoing \isi{aspect} or present perfect \isi{tense} and \isi{aspect} in affirmative order verbs and conjunct \isi{iterative} verbs”, and the differences in pitch are also morphophonologically\is{morphophonology} conditioned \citep[22ff., 73]{cowell:moss:2008}.

\ea \ili{Arapaho} \citep[110, 140]{cowell:moss:2008}
\ea\label{ex:Arapaho:speak:a}
	\gll	heneenéti3-é3en	\\
			\textsc{ic}.speak-\textsc{1sg/2sg} \\
	\glt	‘I am speaking to you’.
\ex\label{ex:Arapaho:speak:b}
	\gll	henéénetí3-\textbf{etí}-no’ \\
			\textsc{ic}.speak-\textsc{refl/recp-du} \\
	\glt	‘We are speaking to ourselves’.
	\glt	‘We are speaking to each other’.
	\z 
\z

Similar reflexive-reciprocal type 1 syncretism\is{voice syncretism, full resemblance -- type 1} has already been discussed and exemplified for the West Bougainville language \ili{Rotokas} (\lang{pn}) in \sectref{resemblance-type1a} (see examples \ref{ex:GurrGoni:hit:a}--\ref{ex:Rotokas:kill:c} on page \pageref{ex:GurrGoni:hit:a}). While the \ili{Arapaho} suffix \example{-etí} and the Rotokas prefix \example{ora-} serve as voice marking only in the reflexive\is{reflexive voice} and reciprocal\is{reciprocal voice} voices, in other languages voice marking found in the reflexive\is{reflexive voice} and reciprocal\is{reciprocal voice} voices might have additional voice functions as well (see \sectref{sec:complex-syncretism:middle} for multiple examples). In turn, reflexive-reciprocal type 2 syncretism\is{voice syncretism, partial resemblance -- type 2} has already been illustrated with examples from the North Halmaheran language Tidore (\lang{pn}), the Northern Pama-Nyungan language Uradhi (\lang{au}), Bolivian Quechua (\lang{sa}), and the Hokan language Yurok (\lang{na}) in Chapter \ref{introduction} (see \tabref{tab:ch1:nedjalkov-examples} on page \pageref{tab:ch1:nedjalkov-examples}). In terms of diachrony, reflexive-reciprocal syncretism is often assumed to have a \isi{reflexive origin}, meaning that the syncretic voice marking in question originally had a reflexive\is{reflexive voice} function before eventually developing a reciprocal\is{reciprocal voice} function (\sectref{diachrony:refl2recp}). However, evidence from some languages indicate that the opposite development can occur as well (\sectref{diachrony:recp2refl}).

\subsection{Reflexive-anticausative} \label{sec:simple-syncretism:refl-antc}
Reflexive-anticausative syncretism is also cross-linguistically prevalent and is attested exclusively as type 1 syncretism\is{voice syncretism, full resemblance -- type 1} in the language sample. This kind of syncretism is here illustrated by examples from the Torricelli language \ili{Yeri} (\lang{pn}). In this language the prefix \example{d-} serves as voice marking not only in the reflexive\is{reflexive voice} (\ref{ex:Yeri:cover:a}↔\ref{ex:Yeri:cover:b}) and anticausative\is{anticausative voice} voices (\ref{ex:Yeri:hang:a}↔\ref{ex:Yeri:hang:b}), but also in the reciprocal\is{reciprocal voice} voices (see \tabref{tab:ch5:middle} on page \pageref{tab:ch5:middle}). The lack of a broader context in (\ref{ex:Yeri:hang:a}) makes the example somewhat opaque. However, the author explicitly states that the verb in the example “involves the act of hanging an item” \citep[370]{wilson:2017}, in this case an implicit item (marked by the infix \example{<he>} language-specifically\is{language-specific}). Observe that the difference in the glossing of the prefix \example{w-} is not a mistake: the prefix is syncretic in the language and can indicate both a third personal female person and third person plural persons.

\ea \ili{Yeri} \citep[369f., 451]{wilson:2017}
\ea\label{ex:Yeri:cover:a}
	\gll	n-altou yewal w-ei=de-n n-aruba-i-bai \\
			\textsc{3sg.m}-cover.\textsc{real} eye \textsc{rel-pl=3-sg.m} \textsc{3sg.m}-do.well.\textsc{real-pl-rdpl} \\
	\glt	‘He covered his eyes very very carefully’.
\ex\label{ex:Yeri:cover:b}
	\gll	te-Ø w-\textbf{d}-altou \\
			\textsc{3-sg.f} \textsc{3sg.f-refl}-cover.\textsc{real} \\
	\glt	‘She covered herself’.
\ex\label{ex:Yeri:hang:a}
	\gll	peigɨlia-i w-goba w-a<he>-wɨl \\
			some-\textsc{pl} \textsc{3pl}-bend.in.half.\textsc{real} \textsc{3pl}-hang.\textsc{real<sg.f>} \\
	\glt	‘Some just break and hang it’.
\ex\label{ex:Yeri:hang:b}
	\gll	hɨwol wanagawɨl yot-ua-Ø, w-\textbf{d}-awɨl \\
			breadfruit breadfruit \textsc{dem-dist-sg.f} \textsc{3sg.f-antc}-hang.\textsc{real} \\
	\glt	‘The breadfruit's fruit there, it hangs’.
	\z 
\z

Reflexive-reciprocal voice marking is very often syncretic with voice marking in other voices (as in \ili{Yeri}), though some languages do feature voice marking that is exclusive to the reflexive\is{reflexive voice} and reciprocal\is{reciprocal voice} voices. This is, for instance, true for the South American language isolate \ili{Urarina} (cf. \example{ne-} in Appendix C) and the Eurasian language isolate Nivkh (see \tabref{tab:ch7:refl-antc-world} on page \pageref{tab:ch7:refl-antc-world}). Other examples of reflexive-reciprocal syncretism are provided in \sectref{sec:complex-syncretism:middle}. The syncretism in question generally has a \isi{reflexive origin} (\sectref{diachrony:refl2antc}), though an \isi{anticausative origin} has been proposed for reflexive-anticausative syncretism in at least one language, Indo-European \ili{Hittite} (\sectref{diachrony:antc2refl}).

\subsection{Reciprocal-anticausative} \label{sec:simple-syncretism:recp-antc}
Like reflexive-reciprocal syncretism, reciprocal-anticausative syncretism is rather well-attested as both type 1\is{voice syncretism, full resemblance -- type 1} and type 2 syncretism\is{voice syncretism, partial resemblance -- type 2}. The former type of reciprocal-anticausative syncretism is here illustrated for the Northern Chu\-kot\-ko-Kam\-chat\-kan language \ili{Chukchi} (\lang{ea}) by a reciprocal\is{reciprocal voice} \isi{voice relation} (\ref{ex:Chukchi:hug:a}↔\ref{ex:Chukchi:hug:b}) as well as an anticausative\is{anticausative voice} \isi{voice relation} (\ref{ex:Chukchi:close:a}↔\ref{ex:Chukchi:close:b}). Evidently, the suffix \example{-tku/-tko} conditioned by vowel harmony \citep[48]{dunn:1999} serves as voice marking in both the reciprocal\is{reciprocal voice} voice (\ref{ex:Chukchi:hug:b}) and the anticausative\is{anticausative voice} voice (\ref{ex:Chukchi:close:b}). \cite[221f.]{nedjalkov:2006} calls the suffix “the most syncretic suffix in \ili{Chukchi}”, noting that it can also be found in the antipassive\is{antipassive voice} and reflexive\is{reflexive voice} voices (see \tabref{tab:ch5:antp-refl-recp-antc} on page \pageref{tab:ch5:antp-refl-recp-antc}).

\ea \ili{Chukchi} \citep[222]{nedjalkov:2006}
\ea\label{ex:Chukchi:hug:a}
	\gll	ommačajpə-nen \\
			hug-\textsc{aor.3sg:3sg} \\
	\glt	‘He hugged him’.
\ex\label{ex:Chukchi:hug:b}
	\gll	ommačajpə-\textbf{tko}-ɣʔat \\
			hug-\textsc{recp-aor.3pl} \\
	\glt	‘They hugged each other’.
\ex\label{ex:Chukchi:close:a}
	\gll	ejpə-nin \\
			close-\textsc{aor.3sg:3sg} \\
	\glt	‘He closed it’.
\ex\label{ex:Chukchi:close:b}
	\gll	ejpə-\textbf{tku}-ɣʔi \\
			close-\textsc{antc-aor.3sg} \\
	\glt	‘It closed’.
	\z 
\z

Reciprocal-anticausative voice marking is -- like reflexive-reciprocal voice marking -- very often syncretic with voice marking in other voices (as in the case of \ili{Chukchi}). In fact, it has only been possible to find one language in which a given voice marker is restricted exclusively to the reciprocal\is{reciprocal voice} and anticausative\is{anticausative voice} voices. In the South Omotic language \ili{Hamar} (\lang{af}) the suffix \example{-Vm} is marginally productive\is{productivity} as both a reciprocal\is{reciprocal voice} marker (e.g. \example{sunq-} ‘to kiss sb.’ ↔ \example{sunq-um-} ‘to kiss e.o.’) and an anticausative\is{anticausative voice} marker (e.g. \example{bax-} ‘to cook sth.’ ↔ \example{bax-em-} ‘to cook’, \citealt[148ff.]{petrollino:2016}). Other cases of reciprocal-anticausative type 1\is{voice syncretism, full resemblance -- type 1} syncretism are provided in \sectref{sec:complex-syncretism:middle}, \sectref{sec:complex-syncretism:antp-refl}, and \sectref{sec:complex-syncretism:multiplex}. By contrast, reciprocal-anticausative type 2 syncretism\is{voice syncretism, partial resemblance -- type 2} is not exemplified elsewhere in this book, for which reason it is here illustrated for the Central Arawakan language \ili{Paresi-Haliti} (\lang{sa}) in \tabref{tab:ch4:recp-antc}. In this language the suffix \example{-kakoa} serves as voice marking in the reciprocal\is{reciprocal voice} voice while the suffix \example{-oa} serves as voice marking in the anticausative\is{anticausative voice} voice. The stem-final vowel /a/ in the anticausative\is{anticausative voice} examples is deleted “when suffixes are attached” \citep[68f.]{brandao:2014}. \cite[259]{brandao:2014} argues that the suffix \example{-kakoa} “may be further analyzed as formed by reciprocal\is{reciprocal voice} \example{-kak} and the \isi{middle voice} form \example{-oa}”, which reflect the reciprocal\is{reciprocal voice} suffix \example{*-kʰakʰ} and the reflexive\is{reflexive voice} suffix \example{*-wa} reconstructed\is{reconstruction} by \cite{wise:1990} for \ili{Proto-Arawakan}, respectively. The suffix \example{-oa} retains the reflexive\is{reflexive voice} function in Paresi-Haliti (e.g. \example{fehanatya} ‘to bless sb.’ ↔ \example{fehanaty-oa} ‘to bless self’, \citealt[251]{brandao:2014}).

\begin{table}
	\begin{tabularx}{\textwidth}{llllll}
		\lsptoprule
		\multicolumn{6}{l}{\ili{Paresi-Haliti} \citep[248ff., 256, 367, 372]{brandao:2014}} \\
		\midrule
		\textsc{recp} & \example{zakolo} & ‘to hug sb.’ & ↔ & \example{zakolo-\textbf{kakoa}} & ‘to hug e.o.’ \\
		\textsc{recp} & \example{xaka} & ‘to shoot sb.’ & ↔ & \example{xaka-\textbf{kakoa}} & ‘to shoot e.o.’ \\
		\textsc{antc} & \example{txiholatya} & ‘to open sth.’ & ↔ & \example{txiholaty-\textbf{oa}} & ‘to open’ \\
		\textsc{antc} & \example{etolitsa} & ‘to lay sth. down’ & ↔ & \example{etolits-\textbf{oa}} & ‘to lie down’ \\
		\lspbottomrule
	\end{tabularx}
	\caption{Reciprocal-anticausative syncretism in Paresi-Haliti}
	\label{tab:ch4:recp-antc}
\end{table}

As briefly mentioned in the previous two sections, it is well-known that reflexive\is{reflexive voice} voice marking can develop both reciprocal\is{reciprocal voice} and anticausative\is{anticausative voice} functions. By contrast, evidence for voice development from reciprocal\is{reciprocal voice} to anticausative\is{anticausative voice} (\sectref{diachrony:recp2antc}) and vice versa (\sectref{diachrony:antc2recp}) is rare.

\subsection{Passive-reflexive} \label{sec:simple-syncretism:pass-refl}
Passive-reflexive type 1 syncretism\is{voice syncretism, full resemblance -- type 1} is here illustrated for the Tangkic language \ili{Kayardild} (\lang{au}), in which the suffixal lengthening of the last vowel of a verbal stem characterises both the passive\is{passive voice} and reflexive\is{reflexive voice} voices (\ref{ex:Kayardild:hit:a}↔\ref{ex:Kayardild:hit:b}) as well as the anticausative\is{anticausative voice} voice (see \tabref{tab:ch5:middle} on page \pageref{tab:ch5:middle}). Other languages feature voice marking restricted to the passive\is{passive voice} and reflexive\is{reflexive voice} voices, including the African language isolate \ili{Chabu} (cf. \example{-we} in Appendix C) and the South American languages Filomeno Mata Totonac\il{Totonac, Filomeno Mata} and \ili{Bora} (see \tabref{tab:ch7:pass-refl-totonacan-bora} on page \pageref{tab:ch7:pass-refl-totonacan-bora}). Additional examples of passive-reflexive type 1 syncretism\is{voice syncretism, full resemblance -- type 1} are given in \sectref{sec:complex-syncretism:middle}, \sectref{sec:complex-syncretism:pass-antp}, \sectref{sec:complex-syncretism:caus-pass}, and \sectref{sec:complex-syncretism:multiplex}.

\ea \ili{Kayardild} \citep[307, 352]{evans:1995}
	\ea\label{ex:Kayardild:hit:a}
	\gll	ngada kurulutha bala-tha niwan-ji wangalk-ur \\
			\textsc{1sg.nom} hard/intensely hit-\textsc{active} him-\textsc{loc} boomerang-\textsc{prop} \\
	\glt	‘I hit him hard with the boomerang’.
\ex\label{ex:Kayardild:hit:b}
	\gll	ngada bala-\textbf{a}-ja karwa-wuru \\
			\textsc{1sg.nom} hit-\textsc{pass/refl-active} club-\textsc{prop} \\
	\glt	‘I was hit with a club’.
	\glt	‘I hit myself with a club’.
	\z 
\z

Passive-reflexive type 2 syncretism\is{voice syncretism, partial resemblance -- type 2} was briefly mentioned in \sectref{resemblance-type3} in relation to the language isolate \ili{Mosetén} (\lang{sa}) but is properly exemplified here. As \tabref{tab:ch4:pass-refl} shows, the suffix \example{-ti} in this language can serve as voice marking not only in the reciprocal\is{reciprocal voice} voice, but also in the reflexive\is{reflexive voice} and passive\is{passive voice} voices in combination with the affix \example{-ja/ja-}. Note that the verbs \example{ji-ti-}, \example{tyáph-yi-}, and \example{jo-yi-} appear as \example{ji-te-}, \example{tyáph-ye-}, and \example{jo-ye-} in the original source, respectively, because the stem-final /i/ becomes /e/ before “\isi{transitive} cross-reference forms which do not trigger vowel harmony” \citep[45]{sakel:2004}. The affix \example{-ja/ja-} generally appears as a prefix but can appear as a suffix on verbs featuring the verbal stem marker \example{-yi} \citep[229]{sakel:2004}. The affix may “have developed from a causative\is{causative voice} with the form \example{ja-}, though synchronically such a form does not exist” \citep[303]{sakel:2004}.

\begin{table}
	\setlength{\tabcolsep}{3pt}
	\begin{tabularx}{\textwidth}{llllll}
		\lsptoprule
		\multicolumn{6}{l}{\ili{Mosetén} \citep[42, 137, 155, 194, 251ff., 304]{sakel:2004}} \\
		\midrule
		\textsc{pass} & \example{ji-ti-} & ‘to send sth.’ & ↔ & \example{\textbf{ja}-ji-ti-\textbf{ti}-} & ‘to be sent [by sb.]’ \\
		\textsc{pass} & \example{tyáph-yi-} & ‘to grab sth.’ & ↔ & \example{tyáph-\textbf{já}-yi-\textbf{ti}-} & ‘to be grabbed [by sb.]’ \\
		\textsc{refl} & \example{jo-yi-} & ‘to serve sth.’ & ↔ & \example{jo-yi-\textbf{ti}-} & ‘to serve self’ \\
		\textsc{refl} & \example{kaw-i-} & ‘to see sth.’ & ↔ & \example{kaw-i-\textbf{ti}-} & ‘to see self’ \\
		\lspbottomrule
	\end{tabularx}
	\caption{Passive-reflexive syncretism in Mosetén}
	\label{tab:ch4:pass-refl}
\end{table}

In terms of diachrony, passive-reflexive syncretism is generally believed to evolve from reflexivity\is{reflexive voice} through an intermediary stage of anticausativity\is{anticausative voice} (\sectref{diachrony:reflexive}). However, this scenario is not particularly convincing for languages in which passive-reflexive voice marking does not have an anticausative\is{anticausative voice} function nor traces thereof, including Chabu, Filomeno Mata Totonac and Bora mentioned further above (\sectref{diachrony:refl2pass}). Moreover, there is some sparse evidence pointing towards a \isi{passive origin} for passive-reflexive syncretism in a few languages (\sectref{diachrony:pass2refl}).

\subsection{Passive-reciprocal} \label{sec:simple-syncretism:pass-recp}
In terms of type 1 syncretism\is{voice syncretism, full resemblance -- type 1}, passive-reciprocal syncretism is the least frequent pattern of \isi{middle syncretism} in the sample, though quite a few languages feature passive-reciprocal type 2 syncretism\is{voice syncretism, partial resemblance -- type 2} instead. The former type of passive-reciprocal syncretism is here illustrated by examples from the East Chadic language \ili{Baraïn} (\lang{af}). As seen in the following \isi{voice relation} (\ref{ex:barain:look:a}↔\ref{ex:barain:look:b}), in this language the suffix \example{-ɟó} evidently serves as voice marking in both the passive\is{passive voice} and reciprocal\is{reciprocal voice} voices. \cite[148ff.]{lovestrand:2012} also considers a reflexive\is{reflexive voice} function of this suffix but concludes that it is “less natural” (and the only potential example he provides is preceded by a question mark), noting instead that reflexivity\is{reflexive voice} in the language is expressed periphrastically. Consequently, a reflexive\is{reflexive voice} function for the suffix \example{-ɟó} is not recognised here. However, in many other languages voice marking found in the passive\is{passive voice} and reciprocal\is{reciprocal voice} voices does indeed also have additional voice functions (for examples, see \sectref{sec:complex-syncretism:middle}, \sectref{sec:complex-syncretism:caus-pass}, and \sectref{sec:complex-syncretism:multiplex}).

\ea \ili{Baraïn} \citep[137, 150]{lovestrand:2012}
\ea\label{ex:barain:look:a}
	\gll	Músà ɲár-gà Mámːàt \\
			\textsc{name} search-\textsc{obj.3.m} \textsc{name} \\
	\glt	‘Moussa is looking for Mammat’.
\ex\label{ex:barain:look:b}
	\gll	nándáŋgá ɲár-ō-\textbf{ɟó} \\
			children search-\textsc{prf-recp/pass} \\
	\glt	‘The children were looked for’.
	\glt	‘The children looked for each other’.
	\z 
\z

Passive-reciprocal type 2 syncretism\is{voice syncretism, partial resemblance -- type 2} is illustrated in \tabref{tab:ch4:pass-recp} by examples from the Central Cushitic language \ili{Khimt’anga} (\lang{af}), in which the suffix \example{-ʃit/-ʃɨt} serves as voice marking in both the passive\is{passive voice} and reciprocal\is{reciprocal voice} voices, in the latter voice accompanied by \isi{reduplication}. Observe that the schwa in the reduplicated forms\is{reduplication} is simply a “linking vowel” \citep[xxi]{belay:2015}. Interestingly, as described in \sectref{sec:simple-syncretism:caus-appl}, in \ili{Khimt’anga} \isi{reduplication} is even used to differentiate the causative\is{causative voice} and applicative\is{applicative voice} voices, which are otherwise both marked by the suffix \example{-s} (see \tabref{tab:ch4:caus-appl-khimtanga} on page \pageref{tab:ch4:caus-appl-khimtanga}).

\begin{table}
	\begin{tabularx}{.97\textwidth}{llllll}
		\lsptoprule
		\multicolumn{6}{l}{\ili{Khimt’anga} \citep[239]{belay:2015}} \\
		\midrule
		\textsc{pass} & \example{kʼɨw-} & ‘to kill sb.’ & ↔ & \example{kʼɨw-\textbf{ɨʃit}-} & ‘to be killed [by sb.]’ \\
		\textsc{pass} & \example{kəβ-} & ‘to help sb.’ & ↔ & \example{kəβ-\textbf{ɨʃit}-} & ‘to be helped [by sb.]’ \\
		\textsc{recp} & \example{kʼɨw-} & ‘to kill sb.’ & ↔ & \example{\textbf{kʼɨw-ə}-kʼɨw-\textbf{ɨʃit}-} & ‘to kill e.o.’ \\
		\textsc{recp} & \example{kəβ-} & ‘to help sb.’ & ↔ & \example{\textbf{kəβ-ə}-kəβ-\textbf{ɨʃit}-} & ‘to help e.o.’ \\
		\lspbottomrule
	\end{tabularx}
	\caption{Passive-reciprocal syncretism in Khimt’anga}
	\label{tab:ch4:pass-recp}
\end{table}

The diachrony of passive-reciprocal syncretism is not as well-known as the diachrony of the patterns of syncretism discussed in the previous sections. Currently there does not appear to be any concrete evidence for reciprocal\is{reciprocal voice} voice marking developing a passive\is{passive voice} function (\sectref{diachrony:recp2pass}), and there is only scarce evidence for passive\is{passive voice} voice marking developing a reciprocal\is{reciprocal voice} function in a single language, the Highland East Cushitic
language Sidaama (\sectref{diachrony:pass2recp}). However, it is well-known and well-attested that passive-reciprocal syncretism in many Indo-European languages ultimately has a \isi{reflexive origin} (\sectref{diachrony:reflexive}).

\subsection{Passive-anticausative} \label{sec:simple-syncretism:pass-antc}
Passive-anticausative syncretism is -- like reflexive-anticausative syncretism -- attested exclusively as type 1 syncretism\is{voice syncretism, full resemblance -- type 1} in the sample. This type of passive-anticausative syncretism is here illustrated for the Tibeto-Burman language \ili{Dhimal} (\lang{ea}) in which the “middle morpheme” \example{-nha} serves as voice marking not only in the passive\is{passive voice} (\ref{ex:Dhimal:take:a}↔\ref{ex:Dhimal:take:b}) and anticausative\is{anticausative voice} voices (\ref{ex:Dhimal:cook:a}↔\ref{ex:Dhimal:cook:b}), but also in the reflexive\is{reflexive voice} voice (e.g. \example{ce:-} ‘to cut sth.’ ↔ \example{ce:-nha-} ‘to cut self’, \citealt[527]{king:j:2009}). According to \cite[239]{khatiwada:2016}, the passive\is{passive voice} use of the suffix has likely evolved under the influence of the Indo-Aryan language \ili{Nepali} (\lang{ea}) and can thus be regarded as a recent innovation. 

\ea \ili{Dhimal} \citep[189, 459, 604]{king:j:2009}
\ea\label{ex:Dhimal:take:a}
	\gll	kalau insa cuma-hi la \\
			so like.that take-\textsc{pst} \textsc{mir} \\
	\glt	‘And so he took him’.
\ex\label{ex:Dhimal:take:b}
	\gll	hiso cuma-\textbf{nha}-hi ede jamal \\
			whither take-\textsc{pass-pst} this child \\
	\glt	‘Where was this child taken?’
\ex\label{ex:Dhimal:cook:a}
	\gll	me-ta pundhui oŋ-gha \\
			fire-\textsc{loc} brain cook-\textsc{pst.1sg} \\
	\glt	‘I cooked the brain in the fire’.
\ex\label{ex:Dhimal:cook:b}
	\gll	tui rem-pha oŋ-\textbf{nha}-hi \\
			egg be.good-do cook-\textsc{antc-pst} \\
	\glt	‘The egg cooked well’.
	\z 
\z

Passive-anticausative voice marking is -- like reflexive-reciprocal and re\-ci\-pro\-cal-anticausative voice marking -- often syncretic with voice marking in other voices (like in \ili{Dhimal}), though a few languages seem to feature voice marking exclusively used for the passive\is{passive voice} and anticausative\is{anticausative voice} voices. This is, for example, the case for the Northern Atlantic language Ganja Balanta\il{Balanta, Ganja} (\lang{af}; e.g. \example{tɛɛ} ‘to spread sth.’ ↔ \example{tɛɛ-l.ɛ} ‘to spread’ or ‘to be spread [by sb.]’, \citealt[211]{creissels:biaye:2016}) and the language isolate \ili{Korean} (\lang{ea}; see \tabref{tab:ch7:antc2pass-korean} on page \pageref{tab:ch7:antc2pass-korean}). Various other examples of passive-anticausative syncretism are provided in \sectref{sec:complex-syncretism:middle}, \sectref{sec:complex-syncretism:pass-antp}, and \sectref{sec:complex-syncretism:multiplex}. In terms of diachrony, it is well-known that reflexive\is{reflexive voice} voice marking can develop an anticausative\is{anticausative voice} function (\sectref{diachrony:refl2antc}) and subsequently a passive\is{passive voice} function (\sectref{diachrony:antc2pass}). However, it is worth noting that a \isi{passive origin} has been proposed for passive-anticausative syncretism in a few languages (\sectref{diachrony:pass2antc}).

\section{Antipassive syncretism} \label{sec:simple-syncretism:antipassive}
Antipassive\is{antipassive voice} syncretism has received less attention in the typological literature than \isi{middle syncretism}, although the phenomenon has been acknowledged at least since the late 1960s. For instance, \cite[40ff.]{nedjalkov:silnickij:1969} provide early cross-linguistic examples of syncretism between antipassive\is{antipassive voice} (\textit{абсолютивно-потенциальное} “absolutive-potential”) and anticausative\is{anticausative voice} (\textit{декаузативное} “\isi{decausative}”) voice marking. Furthermore, \cite[314]{polinsky:2017} has strongly argued that “[i]n the majority of languages that mark the antipassive\is{antipassive voice} verbally, the affix indexes other categories as well” and often the “antipassive\is{antipassive voice} is syncretic with detransitivizing\is{detransitivisation} affixes such as anticausative\is{anticausative voice}, reflexive\is{reflexive voice}/reciprocal\is{reciprocal voice}, middle, or passive\is{passive voice} markers” (see also \citealt[139]{heaton:2020}). However, observations on antipassive\is{antipassive voice} syncretism remain largely implicit and sporadic in the literature, though typological interest in the phenomenon has been on the rise since the turn of the millennium. Interestingly, patterns of antipassive\is{antipassive voice} voice syncretism very often form part of complex voice syncretism\is{voice syncretism, complex}, and many of the languages attested with antipassive\is{antipassive voice} syncretism in the language sample are therefore only mentioned briefly in this chapter before being discussed in more detail in the next chapter. Furthermore, note that all attestations of antipassive\is{antipassive voice} syncretism in the language sample represent type 1 syncretism\is{voice syncretism, full resemblance -- type 1} without exception.

\subsection{Antipassive-reflexive} \label{sec:simple-syncretism:antp-refl}
In a rare explicit typological study of antipassive\is{antipassive voice} syncretism, \cite[158]{janic:2010} provides a brief cross-linguistic overview of antipassive-reflexive syncretism in which she argues that “[i]n addition to Australian [i.e. Northern Pama-Nyungan] and Slavic languages, Romance, Ca\-ri\-ban, Tacanan, Manding [i.e. Western Mande], South Caucasian [i.e. Kartvelian], and [Northern] Chukotko-Kamchatkan languages can be mentioned among language families in which the re\-flex\-ive-anti-passive \isi{polysemy} is attested”. However, the antipassive-reflexive syncretism discussed by \citeauthor{janic:2010} for the Kartvelian language \ili{Laz}, the Slavic languages \ili{Bulgarian} and \ili{Polish} (all three \lang{ea}), and the Western Mande language \ili{Bambara} (\lang{af}) is not recognised by this book. Firstly, the purported antipassivity\is{antipassive voice} of the prefix \example{i-} in \ili{Laz} is uncertain. \citeauthor{janic:2010} (and also, e.g., \citealt[193]{sanso:2017}) argues that the prefix has an antipassive\is{antipassive voice} function based on \citeauthor{lacroix:2009}’s (\citeyear[467]{lacroix:2009}; \citeyear[181f.]{lacroix:2012}) discussion of the prefix in relation to the two verbs \example{(o-)gur} and \example{i-gur}. \citeauthor{lacroix:2009} translates these verbal forms ‘to teach sth. to sb.’ and ‘to learn sth.’, respectively, indicating a reflexive\is{reflexive voice} rather than antipassive\is{antipassive voice} function of the prefix, at least with the verb in question (‘to teach self sth.’). \cite[181]{lacroix:2012} is cautious in his description of \example{i-} as antipassive\is{antipassive voice} himself, saying that it cannot be “analysed as a prototypical\is{prototype} antipassive\is{antipassive voice}”. Secondly, the antipassives\is{antipassive voice} and reflexives\is{reflexive voice} in \ili{Bulgarian}, \ili{Polish}, and \ili{Bambara} do not feature verbal voice marking and thus lie beyond the scope of this book. However, antipassive-reflexive syncretism is attested in the Slavic language \ili{Russian} (\sectref{sec:complex-syncretism:multiplex}) and the Western Mande language \ili{Soninke} (e.g. \example{còró} ‘to cook sth.’ ↔ \example{còr-é} ‘to cook [sth.]’, \example{bóorà} ‘to undress sb.’ ↔ \example{bóor-è} ‘to undress self’, \citealt[10]{creissels:2012}). The difference in the tone of the suffix \example{-e} in these examples is not inherent to the voice marking itself. 

The remaining cases of antipassive-reflexive syncretism mentioned by \cite{janic:2010} are readily acknowledged here, characterised by the prefix \example{öt-} in the Cariban language \ili{Ye’kwana} (\lang{sa}; see \citealt{gildea:al:2016}),  by the suffixes \example{-gali} and \example{-:dji} in the Northern Pama-Nyungan languages \ili{Warrungu} and Yidiny (\lang{au}; see \citealt{terrill:1997}), by the circumfixes \example{k(a)-…-ti} and \example{xa-…-ki} in the Tacanan languages \ili{Cavineña} and \ili{Ese Ejja} (\lang{sa}), and by the suffix \example{-tku} in the Northern Chukotko-Kamchatkan language \ili{Chukchi} (\lang{ea}). Additionally, antipassive-reflexive syncretism has been noted by \cite[75ff.]{vigus:2016} for the language isolate \ili{Oksapmin} (\lang{pn}) characterised by the prefix \example{t-}; and by \cite[193ff.]{sanso:2017} for the Na-Dene language \ili{Tlingit} (\lang{na}) and the Turkic language \ili{Tuvan} (\lang{ea}) characterised by the affixes \example{dzi-/da-} and \example{-n}, respectively. Several of the voice markers mentioned here also have additional voice functions, as further discussed in the following sections.

\ili{Ese Ejja}, \ili{Chukchi}, and \ili{Oksapmin} are also included in the language sample of this book. In addition to these languages, antipassive-reflexive syncretism has been attested in seven other languages of the sample: the Turkic language \ili{Tatar}, the Permic language \ili{Udmurt} (both \lang{ea}), the Gunwinyguan language \ili{Nunggubuyu}, the Mangarrayi-Maran language \ili{Mangarrayi} (both \lang{au}), the Oto-Man\-guean language Acazulco Otomí\il{Otomí, Acazulco}, the Southern Iroquoian language \ili{Cherokee} (both \lang{na}), and the Katukinan language \ili{Katukina-Kanamari} (\lang{sa}). \cite[169]{heaton:2017} hints at antipassive-reflexive syncretism in both Queretaro Otomí\il{Otomí, Queretaro} and \ili{Cherokee} but does not pursue the matter further, only remarking that the languages have “antipassive\is{antipassive voice} uses for \isi{middle voice} morphemes”. The antipassive-reflexive syncretism in each of the ten languages forms part of complex voice syncretism\is{voice syncretism, complex} and are discussed in more detail in the next chapter. Nevertheless, for illustrative purposes, glossed examples demonstrating antipassive-reflexive syncretism in \ili{Ese Ejja} are provided below in the form of an antipassive\is{antipassive voice} \isi{voice relation} (\ref{ex:EseEjja:wait:a}↔\ref{ex:EseEjja:wait:b}) and a reflexive\is{reflexive voice} \isi{voice relation} (\ref{ex:EseEjja:comb:a}↔\ref{ex:EseEjja:comb:b}). As seen in these examples, the circumfix \example{xa-…-ki} can serve as voice marking in both the antipassive\is{antipassive voice} (\ref{ex:EseEjja:wait:b}) and reflexive\is{reflexive voice} voices (\ref{ex:EseEjja:comb:b}). \cite[162]{janic:2010} notes that the circumfix \example{k(a)-…-ti} in the closely related language \ili{Cavineña} is similar to the Ese Ejja circumfix \example{xa-…-ki} in this respect (e.g. \example{peta} ‘to look at sth.’ ↔ \example{ka-peta-ti} ‘to look at [sth.]’ or ‘to look at self’, \citealt[268]{guillaume:2008}). Note that the circumfix in \ili{Ese Ejja} also can serve as voice marking in the reciprocal\is{reciprocal voice} and anticausative\is{anticausative voice} voices (see \tabref{tab:ch5:antp-refl-recp-antc} on page \pageref{tab:ch5:antp-refl-recp-antc}). 

\ea \ili{Ese Ejja} \citep[520ff.]{vuillermet:2012}
\ea\label{ex:EseEjja:wait:a}
	\gll	ekwaa motor ishwa-’axa-naje \\
			\textsc{1excl.erg} motorboat wait-\textsc{frust-pst} \\
	\glt	‘We vainly waited for the motorboat’.
\ex\label{ex:EseEjja:wait:b}
	\gll	jama=ya esea ani-ani, \textbf{xa}-ishwa-\textbf{ki}-ani-ani \\
			so=\textsc{foc} \textsc{1incl.abs} sit-\textsc{ipfv} \textsc{antp}-wait-\textsc{antp-ipfv-prs} \\
	\glt	‘So we usually sit and wait’.
\ex\label{ex:EseEjja:comb:a}
	\gll	eyaya ekwe=bakwa jabe-je \\
			\textsc{1sg.erg} \textsc{1sg.poss}=child comb-\textsc{fut} \\
	\glt	‘I will comb my child’.
\ex\label{ex:EseEjja:comb:b}
	\gll	epona \textbf{xa}-jabe-\textbf{ki}-ani \\
			woman.\textsc{abs} \textsc{refl}-comb-\textsc{refl-prs} \\
	\glt	‘The woman is combing herself’.
	\z 
\z

In terms of diachrony, it seems that antipassive-reflexive syncretism generally has a \isi{reflexive origin} (see \sectref{diachrony:reflexive} and \sectref{diachrony:refl2antp}), while there is currently no evidence for antipassive\is{antipassive voice} voice marking developing a reflexive\is{reflexive voice} function in any language.

\subsection{Antipassive-reciprocal} \label{sec:simple-syncretism:antp-recp}
Antipassive-reciprocal syncretism has received less attention in the literature than the antipassive-reflexive syncretism discussed in the previous section, yet antipassive-reciprocal syncretism actually appears to be slightly more prevalent cross-linguistically (see \tabref{tab:ch6:voice-syncretism-simplex} on page \pageref{tab:ch6:voice-syncretism-simplex}). Nevertheless, previous observations on the phenomenon can be found sporadically in the literature. For instance, \cite{janic:2010} briefly notes the existence of antipassive-reciprocal syncretism in several of the languages mentioned in the previous section, including the Cariban language \ili{Ye’kwana}, the Tacanan languages \ili{Cavineña} and \ili{Ese Ejja} (all three \lang{sa}), and the Northern Chukotko-Kamchatkan language \ili{Chukchi} (\lang{ea}). Likewise, \cite{sanso:2017} attests the syncretism in the Na-Dene language \ili{Tlingit} (\lang{na}) also mentioned in the previous section, as well as in the Surmic language \ili{Tirmaga} (\lang{af}) characterised by the suffix \example{-inɛ(n)}. Antipassive-reciprocal syncretism can additionally be observed in a number of Oceanic and Bantu languages (\sectref{diachrony:recp2antp}). In addition to \ili{Ese Ejja} and \ili{Chukchi}, antipassive-reciprocal syncretism is attested in nine other languages in the language sample, eight of which have already been mentioned in relation to antipassive-reflexive syncretism in the previous section: the Turkic language \ili{Tatar}, the Permic language \ili{Udmurt} (both \lang{ea}), the Gunwinyguan language \ili{Nunggubuyu}, the Mangarrayi-Maran language \ili{Mangarrayi} (both \lang{au}), the Oto-Manguean language Acazulco Otomí\il{Otomí, Acazulco} (\lang{na}) and the Katukinan language \ili{Katukina-Kanamari} (\lang{sa}). The remaining two languages are the Eskimo language Central Alaskan Yupik\il{Yupik, Central Alaskan} (\lang{na}) and the Kordofanian language \ili{Lumun} (\lang{af}). In these languages antipassive-reciprocal voice marking is mostly syncretic with voice marking in other voices, except in Lumun and Tatar further described here. The antipassive-reciprocal syncretism in the remaining languages is described in more detail in the next chapter.

In \ili{Lumun} there are two affixes which can serve as voice marking in both the antipassive\is{antipassive voice} and reciprocal\is{reciprocal voice} voices but not in other voices: “\example{(a)rɔ} replaces a final or last vowel \example{ɔ} or comes after a final or last vowel \example{a}”, while “\example{ttɔ} is typically attached to stems with a final or last \example{ɛ}” \citep[550f.]{smits:2017}. The former affix has the allomorphs\is{allomorphy} \example{<ar>}, \example{<rɔ>} and \example{-rɔ}, while the latter affix has the allomorphs\is{allomorphy} \example{-ttɔ} and \example{<ttɔ>}. The antipassive\is{antipassive voice} and reciprocal\is{reciprocal voice} functions of the former affix are here illustrated by an antipassive\is{antipassive voice} \isi{voice relation} (\ref{ex:Lumun:take:a}↔\ref{ex:Lumun:take:b}) and a reciprocal\is{reciprocal voice} \isi{voice relation} (\ref{ex:Lumun:know:a}↔\ref{ex:Lumun:know:b}). In turn, the same functions of the latter affix are exemplified by the following verbs: \example{a.kkwɛ} ‘to beat sb.’ ↔ \example{á.kkwɛ́-ttɔ} ‘to beat [sb.]’, \example{accɛ} ‘to lick sb.’ ↔ \example{accɛ-ttɔ} ‘to lick e.o.’ \citep[551, 559, 734]{smits:2017}. The tonal differences in the various examples are related to the overall tone system of Lumun and do not form part of the voice marking itself. \cite[558]{smits:2017} explicitly remarks that the \ili{Lumun} affixes “do not only give an ‘each other’-reading, but also a non-reciprocal reading with a human object that is not (nominally or pronominally) referred to, i.e. an antipassive\is{antipassive voice}”. \citeauthor{smits:2017} also makes it clear that the verb in (\ref{ex:Lumun:take:b}) does not have the meaning *‘to take each other’ but denotes a river taking a human \isi{semantic participant} which cannot be expressed syntactically. The word \example{tɪ́at̪\~{}t̪ɪak} functions as an adverb indicating intensity or repetition, hence \citeauthor{smits:2017}’ idiomatic translation ‘to take many people’s lives’ and the gloss ‘very’. 

\ea \ili{Lumun} \citep[505, 573, 558, 742]{smits:2017}
\ea\label{ex:Lumun:take:a}
	\gll	akka.ɪ̂n a-t̪t̪ɔ́má p-á.ɪ́k p-á.nɛ́kɔ́-n \\
			why \textsc{conj}-friend \textsc{agr}-be.\textsc{prs} \textsc{agr}-take.\textsc{ipfv-1obj} \\
	\glt	‘Why, my friend is carrying me’.
\ex\label{ex:Lumun:take:b}
	\gll	tʊɛ t-ɔká.t t-ɔ́nʊ́ ŋəɽɪ ŋ-ɔppɔt ɪ-a.nɛ́k<\textbf{ar}>ɔ tɪ́at̪\~{}t̪ɪak \\
			river \textsc{agr}-be.\textsc{pfv} \textsc{agr}-have water \textsc{agr}-many \textsc{restr}-take.\textsc{ipfv<antp>} very\~{}very \\
	\glt	‘There was a river that had a lot of water and that took many people’s lives’. (lit. ‘[…] that took very’)
\ex\label{ex:Lumun:know:a}
	\gll	kəllán k-ɪna lɔ́n l-ɔppɔ́t \\
			old.woman \textsc{agr}-know.\textsc{ipfv} words/things \textsc{agr}-many \\
	\glt	‘The old woman knows many things’.
\ex\label{ex:Lumun:know:b}
	\gll	ɔ-kɪ́n t̪-ɪ́na-\textbf{rɔ} acɪ́n-t̪a \\
			\textsc{agr-3pl} \textsc{agr}-know.\textsc{ipfv-recp} when-\textsc{q} \\
	\glt	‘When will they get to know each other?’
	\z 
\z

Next, as illustrated in \tabref{tab:ch4:antp-recp}, in \ili{Tatar} antipassive-reciprocal syncretism is characterised by the suffix \example{-š}. The suffix is generally associated with reciprocity\is{reciprocal voice}, but it is widely described as also having a function that qualifies as antipassive\is{antipassive voice} (\citealt[192f.]{zinnatullina:1969};; \citeyear[179]{zinnatullina:1993}; \citealt[297f.]{nedjalkov:2007d};; \citealt[490]{burbiel:2018}). Most diachronic evidence suggests that antipassive-reciprocal syncretism generally has a \isi{reciprocal origin}, as in the case of the \ili{Tatar} suffix \example{-š} and possibly also the \ili{Lumun} affix \example{<ar>/<rɔ>/-rɔ} and various other languages (\sectref{diachrony:recp2antp}). By contrast, there is currently no convincing evidence for antipassive\is{antipassive voice} voice marking developing a reciprocal\is{reciprocal voice} function in any language.\is{antipassive origin}

\begin{table}
	\begin{tabularx}{.90\textwidth}{llllll}
		\lsptoprule
		\multicolumn{6}{l}{\ili{Tatar} \citep[295, 298, 318]{nedjalkov:2007d}} \\
		\midrule
		\textsc{antp} & \example{alda-} & ‘to deceive sb.’ & ↔ & \example{alda-\textbf{š}-} & ‘to deceive [sb.]’ \\
		\textsc{antp} & \example{jaz-} & ‘to write sth.’ & ↔ & \example{jaz-\textbf{əš}-} & ‘to write [sth.]’ \\
		\textsc{recp} & \example{üb-} & ‘to kiss sb.’ & ↔ & \example{üb-\textbf{eš}-} & ‘to kiss e.o.’ \\
		\textsc{recp} & \example{sug-} & ‘to hit sb.’ & ↔ & \example{sug-\textbf{əš}-} & ‘to hit e.o.’ \\
		\lspbottomrule
	\end{tabularx}
	\caption{Antipassive-reciprocal syncretism in Tatar}
	\label{tab:ch4:antp-recp}
\end{table}

\subsection{Antipassive-anticausative} \label{sec:simple-syncretism:antp-antc}
\cite[314]{polinsky:2017} argues that voice marking in the antipassive\is{antipassive voice} voice in many languages “is syncretic with detransitivizing\is{detransitivisation} affixes such as anticausative\is{anticausative voice}”, yet typological literature on antipassive-anticausative syncretism remains scarce and mostly consists of sporadic observations. For example, \cite[40ff.]{nedjalkov:silnickij:1969} briefly address antipassive-anticausative syncretism in the Slavic language \ili{Russian} (\sectref{sec:complex-syncretism:multiplex}), and the syncretism has been observed by \cite[167]{janic:2010} and \cite[76]{vigus:2016} in the Northern Chukotko-Kamchatkan language \ili{Chukchi} and the language isolate \ili{Oksapmin}, respectively. \cite[165f.]{janic:2010} mentions the antipassive-reflexive syncretism of the suffix \example{-gali} in the Northern Pama-Nyungan language \ili{Warrungu} (\lang{au}), but does not mention its anticausative\is{anticausative voice} function (see instead \citealt[523]{tsunoda:2011}). Additionally, antipassive-anticausative syncretism has been described for the Bantu language \ili{Citumbuka} \citep{chavula:2016} and for the Northwest Caucasian language \ili{Adyghe} \citep{letuchiy:2007}. 

In addition to \ili{Chukchi} and \ili{Oksapmin}, antipassive-anticausative syncretism has been attested in eight other languages in the sample, six of which have been mentioned in one or both of the previous sections as well: the Turkic language \ili{Tatar}, the Permic language \ili{Udmurt} (both \lang{ea}), the Gunwinyguan language \ili{Nunggubuyu} (\lang{au}), the Oto-Manguean language Acazulco Otomí\il{Otomí, Acazulco}, the Southern Iroquoian language \ili{Cherokee} (both \lang{na}), and the Tacanan language \ili{Ese Ejja} (\lang{sa}). The two remaining languages in which antipassive-anticausative syncretism has been attested are the language isolate \ili{Mosetén} (\lang{sa}) and the Surmic language \ili{Majang} (\lang{af}). The latter language is the only language attested with voice marking restricted to the antipassive\is{antipassive voice} and anticausative\is{anticausative voice} voices. In the remaining languages antipassive-anticausative voice marking is syncretic with voice marking in other voices as well. The syncretism in \ili{Majang} is described in this section, while the syncretism in the other languages is discussed in the next chapter. It is here worth remarking that the Solomons East Papuan language \ili{Savosavo} (\lang{pn}) in the language sample has a “detransitivizing”\is{detransitivisation} suffix \example{-za}, one function of which is anticausative\is{anticausative voice} (e.g. \example{pili} ‘to turn sth. around’ ↔ \example{pili-za} ‘to turn around’, \citealt[275, 376]{wegener:2012}). \cite[171]{wegener:2012} also describes another function of the suffix which is reminiscent of an antipassive\is{antipassive voice}: “[t]he subject is unchanged, only the object\is{object, direct} is removed” (e.g. \example{ghogho} ‘to swear at sb.’ ↔ \example{ghogho-za} ‘to swear’). However, according to \citeauthor{wegener:2012} this particular function of \example{-za} is rare and has hitherto not been attested with any other verb but \example{ghogho} (p.c., December 4th 2019). Consequently, antipassive-anticausative syncretism is not recognised for Savosavo.

In \ili{Majang} four suffixes can serve as voice marking in both the antipassive\is{antipassive voice} and anticausative\is{anticausative voice} voices: “conjoint” \example{-ìː} and “disjoint” \example{-iː\textsuperscript{L}} (with “most a-class verbs”), and “conjoint” \example{-ɗìː} and “disjoint” \example{-ɗiː\textsuperscript{L}} (with verbs of other \isi{language-specific} classes, \citealt[227]{joswig:2019}). The conjoint-disjoint distinction is maintained throughout the verbal system of Majang and is not unique to antipassives\is{antipassive voice} and anticausatives\is{anticausative voice}. According to \cite[132]{joswig:2019}, the distinction is “conditioned by the case and the \isi{topicality} status of the following \textsc{np}”. The antipassive\is{antipassive voice} use of both conjoint \example{-ɗìː} (glossed \textsc{cj}) and disjoint \example{-ɗiː\textsuperscript{L}} (glossed \textsc{dj}) are illustrated in (\ref{ex:Majang:bite:a}↔\ref{ex:Majang:bite:b} and \ref{ex:Majang:bite:a}↔\ref{ex:Majang:bite:c}), and so is the anticausative\is{anticausative voice} use of the latter suffix (\ref{ex:Majang:break:a}↔\ref{ex:Majang:break:b}). Verbs marked by one of the suffixes under discussion “often change their stem tone” (\citealt[229]{joswig:2019}; cf. \ref{ex:Majang:bite:a}↔\ref{ex:Majang:bite:b}) though not always (cf. \ref{ex:Majang:break:a}↔\ref{ex:Majang:break:b}). However, the effects are the same in both the antipassive\is{antipassive voice} and anticausative\is{anticausative voice} voices (cf., e.g., the antipassive\is{antipassive voice} \isi{voice relation} \example{ɓòkòt} ‘to kill sb.’ ↔ \example{ɓòkò-ɗìː} ‘to kill [sb.]’). 

\ea \ili{Majang} \citep[228, 361]{joswig:2019}
\ea\label{ex:Majang:bite:a}
	\gll	kàw-ɛ́ wâr èːɟɛ́ \\
			bite-\textsc{3sg.dj} dog.\textsc{sg.erg} cat.\textsc{sg.abs} \\
	\glt	‘A dog bites a cat’.
\ex\label{ex:Majang:bite:b}
	\gll	káw-\textbf{ɗíː\textsuperscript{L}} wár kɛ́kàr \\
			bite-\textsc{antp.dj} dog.\textsc{sg.nom} again \\
	\glt	‘The dog bites again’.
\ex\label{ex:Majang:bite:c}
	\gll	káw-\textbf{ɗìː} wár\textsuperscript{L} kɛ́kàr \\
			bite-\textsc{antp.cj} dog.\textsc{sg.nom} again \\
	\glt	‘A dog bites again’.
\ex\label{ex:Majang:break:a}
	\gll	ŋùːl-è béá\textsuperscript{L} \\
			break-\textsc{3sg.cj} spear.\textsc{sg.abs} \\
	\glt	‘He broke a spear’.
\ex\label{ex:Majang:break:b}
	\gll	ŋùːl-\textbf{ɗìː} béá\textsuperscript{L} nɛ̀ːk-ɛ̂ː=ŋ \\
			break-\textsc{antc.dj} spear.\textsc{sg.nom} \textsc{poss.3sg-nom=top} \\
	\glt	‘And his spear broke’.
	\z 
\z

In terms of diachrony, there does not appear to be any evidence for antipassive\is{antipassive voice} voice marking developing an anticausative\is{anticausative voice} function or for anticausative\is{anticausative voice} voice marking developing an antipassive\is{antipassive voice} function in any language. However, it appears that antipassive-anticausative syncretism can ultimately have a \isi{reflexive origin} (\sectref{diachrony:refl2antc} and \sectref{diachrony:refl2antp}).

\subsection{Passive-antipassive} \label{sec:simple-syncretism:pass-antp}
Passive-antipassive syncretism is discussed rather seldom in the literature, yet it is worth noting that the syncretism has a long tradition of study in the Slavic language \ili{Russian} (e.g. \citealt{nedjalkov:silnickij:1969}). A rare explicit description of passive-antipassive syncretism in another language but Russian is provided by \cite[241]{zuniga:kittila:2019} who observe the syncretism in the Arauan language \ili{Paumarí} (\lang{sa}; e.g. \example{soko-} ‘to wash sth.’ ↔ \example{soko-a-} ‘to wash [sth.]’ or ‘to be washed [by sb.]’, \citealt[298]{chapman:derbyshire:1991}). \cite[241]{zuniga:kittila:2019} further argue that similar syncretism is “rather difficult to find”. Interestingly, \cite[10]{creissels:2012} argues – quite to the contrary – that passive-antipassive (and other patterns of middle\is{middle syncretism} and antipassive\is{antipassive voice} syncretism) are “extremely common cross-linguistically”, noting that such syncretism is “found in particular in languages belonging to various branches of the Indo-European family (Romance, Slavic, Germanic, etc.), as the result of the evolution of the \ili{Proto-Indo-European} reflexive pronoun \example{*se}” (\sectref{diachrony:reflexive}). However, the antipassives\is{antipassive voice} in the languages mentioned by \citeauthor{creissels:2012} rarely feature verbal voice marking (with a few exceptions, notably \ili{Russian} mentioned above) and therefore lie outside the scope of this book. However, \citeauthor{creissels:2012} (\citeyear[10]{creissels:2012}; \citeyear[54]{creissels:2016}) does provide interesting examples of passive-antipassive syncretism in the Western Mande language \ili{Soninke} (\sectref{sec:simple-syncretism:antp-refl}), but does not otherwise mention other languages featuring passive-antipassive syncretism (e.g. \ili{Soninke} \example{ñígá} ’to eat sth.’ ↔ \example{ñíg-é} ‘to eat [sth.]’ or ‘to be eaten [by sb.]’).

\cite[151f.]{dixon:1994} and \citeauthor{dixon:aikhenvald:2000} (\citeyear[11]{dixon:aikhenvald:2000}; \citeyear[51]{dixon:aikhenvald:2011}) argue for antipas\-sive-passive syncretism in certain Australian languages, for instance in the Northern Pama-Nyungan language \ili{Kuku-Yalanji} characterised by the suffix \example{-ji} and in the Central Pama-Nyungan language \ili{Diyari} characterised by the suffix \example{-tharri} (alternatively \example{-t̪adi} or \example{-thadi}). However, it is not entirely clear from the limited available data on these languages (see \citealt{patz:2002} on \ili{Kuku-Yalanji} and \citealt{austin:2013} on \ili{Diyari}) whether or not the suffixes can have an antipassive\is{antipassive voice} function according to the definitions employed in this book (\sectref{def:passives-antipassives}). On the one hand, both \cite[148]{patz:2002} and \cite[162]{austin:2013} explicitly explain that the \isi{agent} can be left unexpressed in the passive\is{passive voice} voice in the languages, and the passive\is{passive voice} function of the suffixes \example{-ji} and \example{-tharri} is readily accepted. On the other hand, it is not clear if the same holds true for the \isi{semantic participant} which is not an \isi{agent} in the antipassive\is{antipassive voice} voice. In the various examples provided by  \cite[151]{patz:2002} and \cite[160]{austin:2013} all semantic participants\is{semantic participant} seem equally likely to be expressed syntactically but are marked differently in terms of \isi{language-specific} case marking\is{case}. Differences of this kind alone do not qualify as antipassive\is{antipassive voice} in this book. Likewise, \cite{janic:2016} argues for passive-antipassive syncretism in the Oceanic languages \ili{Mokilese} and \ili{Kara} (both \lang{pn}) characterised by the suffixes \example{-ek} and \example{-ai}, respectively, but as in the case of \ili{Kuku-Yalanji} and \ili{Diyari} above, data on these languages are too scarce to determine if the suffixes have an antipassive\is{antipassive voice} function according to the definitions employed in this book (see \citealt{harrison:1976} on \ili{Mokilese} and \citealt{schlie:1983} on \ili{Kara}). 

Passive-antipassive syncretism is attested in only four languages in the language sample, three of which have already been mentioned in one or more of the previous sections: the Permic language \ili{Udmurt}, the Turkic language \ili{Tatar} (both \lang{ea}), and the language isolate \ili{Mosetén} (\lang{sa}). The fourth language is the Algonquian language \ili{Arapaho}. Passive-antipassive syncretism has already been described for Arapaho in \sectref{resemblance-type1a}, while it is discussed for the other three languages in the next chapter. Nevertheless, due to the rare nature of the syncretism and the little attention it has received in the literature, it is discussed in turn for each of the languages in this section as well. Moreover, note that the verbal marking in the antipassive-like voice described for the Salishan languages \ili{Nxa’amxcin} and \ili{Musqueam} (both \lang{na}) in \sectref{def:passives-antipassives} bears resemblance to the voice marking in the passive\is{passive voice} voice in these languages. However, as this syncretism does not qualify as proper passive-antipassive syncretism in this book, the syncretism in these two languages is ignored in this section (but see \sectref{sec:complex-syncretism:pass-antp} for a few examples from Musqueam). \citeauthor{vuillermet:2012}’s (\citeyear[519]{vuillermet:2012}) discussion of the circumfix \example{xa-…-ki} in the Tacanan language \ili{Ese Ejja} (\lang{sa}) superficially suggests the existence of passive-antipassive syncretism in this language, as she specifically states that the circumfix can have a “reflexive\is{reflexive voice}, reciprocal\is{reciprocal voice}, antipassive\is{antipassive voice}, anticausative\is{anticausative voice}, and passive-like” function. On the one hand, in its antipassive\is{antipassive voice} function a \isi{semantic participant} which is not an \isi{agent} is “typically omitted but may be encoded by an oblique” \citep[520]{vuillermet:2012}, complying with the definitions of antipassives\is{antipassive voice} employed in this book (see examples \ref{ex:EseEjja:wait:a} and \ref{ex:EseEjja:wait:b} on page \pageref{ex:EseEjja:wait:a}). On the other hand, no \isi{semantic participant} seems to be more or less likely to be omitted in her purported passive\is{passive voice} examples. On the contrary, \citeauthor{vuillermet:2012} suggests that perhaps a passive\is{passive voice} reading is simply not possible if a \isi{semantic participant} is omitted, and also remarks that the purported passive\is{passive voice} function of the circumfix is fairly rare in the first place (p.c., November 13th 2019). For these reasons, passive-antipassive syncretism is not recognised for Ese Ejja here.

In \ili{Mosetén} passive-antipassive syncretism is characterised by the suffix \example{-ki}, as seen in the following passive\is{passive voice} \isi{voice relation} (\ref{ex:Moseten:eat:a}↔\ref{ex:Moseten:eat:b}) and antipassive\is{antipassive voice} \isi{voice relation} (\ref{ex:Moseten:work:a}↔\ref{ex:Moseten:work:b}). This suffix is discussed in more detail later in \sectref{sec:complex-syncretism:pass-antp}. Note that the thematic “verbal stem marker” \example{-(ty)i} in a stem becomes \example{-(ty)e} when followed by “\isi{transitive} cross-reference forms which do not trigger \isi{vowel harmony}” \citep[45]{sakel:2004}, including the third person female object marker \example{-’} in (\ref{ex:Moseten:work:a}). The same stem marker changes to \example{-(ty)a} before certain suffixes, including \example{-ki} \citep[47, 308]{sakel:2004}. The underlying stem of the verbs in the passive\is{passive voice} \isi{voice relation} (\ref{ex:Moseten:eat:a}↔\ref{ex:Moseten:eat:b}) is \example{jeb-i-} (the third person plural inclusive object marker \example{-ksi} is another suffix that prompts the preceding verbal stem marker \example{-i} to change into \example{-a}). \cite[308]{sakel:2004} explicitly discusses passive-antipassive syncretism in \ili{Mo\-se\-tén}: “Many verbs can be marked by both the antipassive\is{antipassive voice} and the middle. When the forms are similar, only context and common knowledge clarifies the intended meaning of the speaker. Hence, a vermin bites more than getting bitten itself […], whereas a woman most probably gets bitten more than biting someone herself”. Compare examples (\ref{ex:Moseten:bite:a}) and (\ref{ex:Moseten:bite:b}). \cite[306ff.]{sakel:2004} makes a distinction between the antipassive\is{antipassive voice} voice on the one hand, and a “middle\is{middle voice} (voice)” covering the passive\is{passive voice} and anticausative\is{anticausative voice} voices on the other hand, but explicitly maintains that the voices share the exact same marking.

\ea \ili{Mosetén} \citep[231, 306, 311]{sakel:2004}
\ea\label{ex:Moseten:eat:a}
	\gll	me’-tya-ksi-’ katyi’ mö’-yä’ jike iji jeb-a-ksi-’ \\
			so-\textsc{th-3pl.obj-f} \textsc{evid} \textsc{3f-ade} \textsc{pst} ucumari eat-\textsc{th-3pl.obj-f} \\
	\glt	‘So it did this to them, the ucumari-monster, it ate them’.
\ex\label{ex:Moseten:eat:b}
	\gll	khin’-cchata’ aj jeb-a-\textbf{ki}-’ phen-yäe \\
			now-\textsc{mod} yet eat-\textsc{th-pass-f} woman-\textsc{1sg.poss} \\
	\glt	‘Now truly my wife has been eaten’.
\ex\label{ex:Moseten:work:a}
	\gll	tsin khin’ i-ya’ jäe’mä karij-tye-’ öi texto en Mosetén \\
			\textsc{1pl} now \textsc{m-ade} uh hard-\textsc{th-f} \textsc{dem.f} text in Mosetén \\
	\glt	‘Here, we now work on this text in Mosetén’.
\ex\label{ex:Moseten:work:b}
	\gll	mi’-ya’ karij-tya-\textbf{ki} jiri-s yomodye’ \\
			\textsc{3m-ade} hard-\textsc{th-antp} one-\textsc{f} year \\
	\glt	‘There I worked for one year’.
	\z 
\z

\ea \ili{Mosetén} \citep[306ff.]{sakel:2004}
\ea\label{ex:Moseten:bite:a}
	\gll	mö’ raem’-ya-\textbf{ki}-’ ïnöj yomo’ \\
			\textsc{3f.sg} bite-\textsc{th-pass-f} moment night \\
	\glt	‘She was bitten [by sb.] last night’.
\ex\label{ex:Moseten:bite:b}
	\gll	mö’ roro’ raem’-ya-\textbf{ki}-’ \\
			\textsc{3f.sg} vermin bite-\textsc{th-antp-f} \\
	\glt	‘This vermin has bitten [sb.]’.
	\z 
\z

Next, as illustrated in \tabref{tab:ch4:pass-antp}, passive-antipassive syncretism is characterised by the suffix \example{-ee} in \ili{Arapaho}, by the suffix \example{-n} in \ili{Tatar}, and by the suffix \example{-śk} in \ili{Udmurt}. Other examples of passive-antipassive syncretism can be found for \ili{Arapaho} in \tabref{tab:ch3:type1a-examples-2} on page \pageref{tab:ch3:type1a-examples-2}, for \ili{Tatar} in \tabref{tab:ch5:pass-antp-refl-antc} on page \pageref{tab:ch5:pass-antp-refl-antc}, and for \ili{Udmurt} in \tabref{tab:ch5:multiplex} on page \pageref{tab:ch5:multiplex}. Evidently and unsurprisingly, passive-antipassive voice marking always appears to be syncretic with marking in other voices, and it has hitherto not been possible to find any language featuring voice marking exclusively used in the passive\is{passive voice} and antipassive\is{antipassive voice} voices.

\begin{table}
	\setlength{\tabcolsep}{3pt}
	\begin{tabularx}{\textwidth}{llllll}
		\lsptoprule
		\multicolumn{6}{l}{\ili{Arapaho} \citep[133, 135f., 155f., 229, 276, 280, 307]{cowell:moss:2008}} \\
		\midrule
		\textsc{pass} & \example{neh’-} & ‘to kill sb.’ & ↔ & \example{neh’-\textbf{ee}-} & ‘to be killed [by sb.]’ \\
		\textsc{pass} & \example{nestoow-} & ‘to warn sb.’ & ↔ & \example{nestoow-\textbf{ee}-} & ‘to get warned [by sb.]’ \\
		\textsc{antp} & \example{otoon-oo-} & ‘to buy sth.’ & ↔ & \example{otoon-\textbf{ee}-} & ‘to buy [sth.]’ \\
		\textsc{antp} & \example{ceit-oo-} & ‘to visit sb.’ & ↔ & \example{ceit-\textbf{ee}-} & ‘to visit [sb.]’ \\
		\midrule\midrule
		\multicolumn{6}{l}{\ili{Tatar} (\citealt[198, 201]{ganiev:1997}; \citealt[473, 485]{burbiel:2018})} \\
		\midrule
		\textsc{pass} & \example{taşla-} & ‘to throw sth.’ & ↔ & \example{taşla-\textbf{n}-} & ‘to be thrown [by sb.]’ \\
		\textsc{pass} & \example{ülçä-} & ‘to measure sth.’ & ↔ & \example{ülçä-\textbf{n}-} & ‘to be measured [by sb.]’ \\
		\textsc{antp} & \example{tikşer-} & ‘to investigate sth.’ & ↔ & \example{tikşer-\textbf{en}-} & ‘to investigate [sth.]’ \\
		\textsc{antp} & \example{ezlä-} & ‘to search for sth.’ & ↔ & \example{ezlä-\textbf{n}-} & ‘to search for [sth.]’ \\
		\midrule\midrule
		\multicolumn{6}{l}{\ili{Udmurt} (\citealt[227f.]{perevoscikov:1962};; \citealt{kirillova:2008})} \\
		\midrule
		\textsc{pass} & \example{leśt-} & ‘to build sth.’ & ↔ & \example{leśt-\textbf{ïśk}-} & ‘to be built [by sb.]’ \\
		\textsc{pass} & \example{birj-} & ‘to elect sb.’ & ↔ & \example{birj-\textbf{iśk}-} & ‘to be elected [by sb.]’ \\
		\textsc{antp} & \example{pyž-} & ‘to bake sth.’ & ↔ & \example{pyž-\textbf{iśk}-} & ‘to bake [sth.]’ \\
		\textsc{antp} & \example{gožja-} & ‘to write sth.’ & ↔ & \example{gožja-\textbf{śk}-} & ‘to write [sth.]’ \\
		\lspbottomrule
	\end{tabularx}
	\caption{Examples of passive-antipassive syncretism}
	\label{tab:ch4:pass-antp}
\end{table}

\newpage

It is well-known that passive-antipassive syncretism in Indo-European languages like \ili{Russian} has a \isi{reflexive origin} (\sectref{diachrony:reflexive}). This is partly true for \ili{Tatar} as well (\sectref{sec:complex-syncretism:pass-antp}). By contrast, relatively little is known about the origin of passive-antipassive syncretism in other languages, and it has not been possible to establish the exact diachrony for the syncretism in \ili{Mosetén}, \ili{Arapaho} or \ili{Udmurt} (\sectref{sec:complex-syncretism:multiplex}). Nevertheless, as argued by \cite[180]{janic:2016}, it is very likely that passive-antipassive syncretism can arise from a generalised function that syntactically suppresses any \isi{semantic participant}. A similar view is shared by \cite[24]{malchukov:2017}. The suffix \example{-ki} in \ili{Mosetén} illustrated in examples (\ref{ex:Moseten:eat:b}) and (\ref{ex:Moseten:work:b}) would be a particularly good example of such a generalised function.

\section{Causative syncretism} \label{sec:simple-syncretism:causative}
Causative-applicative syncretism and causative-passive syncretism are rather well-known and widely attested cross-linguistically. By contrast, other patterns of causative\is{causative voice} syncretism have received little attention in the past, yet all kinds of causative\is{causative voice} syncretism are attested in the language sample, as shown in the following sections.

\subsection{Causative-applicative} \label{sec:simple-syncretism:caus-appl}
Early observations on the cross-linguistic similarities between voice marking in the causative\is{causative voice} and applicative\is{applicative voice} voices are provided by \cite[36f.]{nedjalkov:silnickij:1969}, for example in relation to the affix \example{r-/n-(…-et/-at)} in the Northern Chukotko-Kamchatkan language \ili{Chukchi}, the suffix \example{-se} in Yukaghir (both \lang{ea}), the suffix \example{-isa} in the Bantu language \ili{Zulu} (\lang{af}), and the suffix \example{-kan} in the Malayo-Sumbawan language \ili{Indonesian} (\lang{pn}). \ili{Chukchi} and Yukaghir (more specifically Tundra Yukaghir\il{Yukaghir, Tundra}) both form part of the language sample in this book, while the Bantu and Malayo-Sumbawan genera\is{genus} are represented in the sample by Namibian Fwe\il{Fwe, Namibian} and \ili{Madurese}, respectively. Causative-applicative syncretism is acknowledged for each of these languages. \citeauthor{nedjalkov:silnickij:1969} even argue for causative-applicative syncretism in the Penutian language \ili{Miwok} and the Oregon Coast language \ili{Siuslaw} (both \lang{na}), but it has not been possible to confirm these claims due to lack of data. \cite[116ff.]{shibatani:pardeshi:2002} observe cau\-sa\-tive-applicative syncretism in various additional languages, for instance in the Northern Pama-Nyungan language \ili{Yidiny} (\lang{au}) and the Yuman language \ili{Hualapai} (\lang{na}) as well as in the Panoan language \ili{Matsés} (\lang{sa}). The Northern Pama-Nyungan \isi{genus} is not included in the language sample, but the related Western Pama-Nyungan \isi{genus} is represented by the language Mparntwe Arrernte\il{Arrernte, Mparntwe}, which also features causative-applicative syncretism (for an overview of causative-applicative syncretism in Australian languages in general, see \citealt{austin:2005}). The Yuman and Panoan genera\is{genus} are also part of the sample, represented by the languages \ili{Jamul Tiipay} and \ili{Chácobo}, respectively. Causative-applicative syncretism has been attested in the latter language \citep{tallman:2018}, but not in the former \citep{miller:a:2001}. Additional typological discussions of causative-applicative syncretism are provided by \cite[183]{comrie:1989}, \citeauthor{kulikov:2001} (\citeyear[984]{kulikov:2001}; \citeyear[394]{kulikov:2010}), \cite[1139]{haspelmath:muller-bardey:2004}, \citeauthor{malchukov:2015} (\citeyear[115f.]{malchukov:2015};; \citeyear[403ff.]{malchukov:2016};; \citeyear[6ff., 9ff.]{malchukov:2017}), and recently by \cite[234ff.]{zuniga:kittila:2019} and \cite{franco:2019}.	

Causative-applicative syncretism in \ili{Chukchi}, Tundra Yukaghir\il{Yukaghir, Tundra}, Namibian Fwe\il{Fwe, Namibian}, \ili{Madurese}, Mparntwe Arrernte\il{Arrernte, Mparntwe}, and \ili{Chácobo} qualifies more precisely as type 1 syncretism\is{voice syncretism, full resemblance -- type 1}. Causative-applicative voice marking in Chácobo bears partial resemblance to voice marking\is{voice syncretism, partial resemblance -- type 2} in the passive\is{passive voice} voice (see \tabref{tab:ch5:caus-appl-pass} on page \pageref{tab:ch5:caus-appl-pass}), while causative-applicative voice marking in the other languages is restricted to the causative\is{causative voice} and applicative\is{applicative voice} voices. Causative-applicative type 2 syncretism\is{voice syncretism, partial resemblance -- type 2} has been attested in the language isolate \ili{Kwaza} and the Central Cushitic language \ili{Khimt’anga} (both \lang{af}), and causative-applicative type 3 syncretism\is{voice syncretism, reverse resemblance -- type 3} has been attested in the language isolate \ili{Ainu} (\lang{ea}) and the Sepik Hill language \ili{Alamblak} (\lang{pn}). The syncretism has already been exemplified for \ili{Kwaza} in \tabref{tab:ch3:type2-examples} on page \pageref{tab:ch3:type2-examples} and for \ili{Ainu} and \ili{Alamblak} in \tabref{tab:ch3:type3-examples} on page \pageref{tab:ch3:type3-examples}. For practical reasons, in this section it is not possible to illustrate causative-applicative syncretism for all these languages (nor for all the other languages in the language sample featuring causative-applicative syncretism), but for illustrative purposes the syncretism is here described for five geographically diverse languages: the North Halmaheran language \ili{Ternate} (\lang{pn}), the language isolate \ili{Chabu} (\lang{af}) and the Uto-Aztecan language \ili{Pima Bajo} (\lang{na}) in addition to Mparntwe Arrernte\il{Arrernte, Mparntwe} (\lang{au}) and Tundra Yukaghir\il{Yukaghir, Tundra} (\lang{ea}) already mentioned above.

Causative-applicative syncretism in \ili{Ternate} is here illustrated by a causative \isi{voice relation} (\ref{ex:Ternate:sleep:a}↔\ref{ex:Ternate:sleep:b}) and an applicative\is{applicative voice} \isi{voice relation} (\ref{ex:Ternate:open:a}↔\ref{ex:Ternate:open:b}). As seen in these voice relations\is{voice relation}, the prefix \example{si-} serves as voice marking in both the causative\is{causative voice} voice (\ref{ex:Ternate:sleep:b}) and the applicative\is{applicative voice} voice (\ref{ex:Ternate:open:b}). The causative-applicative syncretism in the language is explicitly noted by \cite[132]{hayami-allen:2001}, who remarks that the non-causative use of the prefix adds “an implication that the action is done purposefully, for someone else’s benefit, by someone else’s order, or by an instrument”. In the case of (\ref{ex:Ternate:open:b}), the action is done for someone else’s benefit, and in the given context the verb \example{si-hoi} does not have the meaning *‘to make sb. open sth.’

\ea \ili{Ternate} \citep[130ff.]{hayami-allen:2001}
\ea\label{ex:Ternate:sleep:a}
	\gll	ma-ngofa gee hotu \\
			\textsc{poss}-child \textsc{dem} sleep \\
	\glt	‘The child is sleeping’.
\ex\label{ex:Ternate:sleep:b}
	\gll	ma-yaya \textbf{si}-hotu ma-ngofa gee \\
			\textsc{poss}-mother \textsc{caus}-sleep \textsc{poss}-child \textsc{dem} \\
	\glt	‘The mother put the child to sleep’.
\ex\label{ex:Ternate:open:a}
	\gll	mina hoi ngara \\
			\textsc{3sg.f} open door \\
	\glt	‘She opened the door’.
\ex\label{ex:Ternate:open:b}
	\gll	kanang mina \textbf{si}-hoi ngara, ngori to=wosa \\
			a.while.ago \textsc{3sg.f} \textsc{appl}-open door \textsc{1sg} \textsc{1sg}=enter \\
	\glt	‘A while ago she opened the door [for me], and I entered’.
	\z 
\z

As illustrated in \tabref{tab:ch4:caus-appl}, causative-applicative syncretism is characterised by the suffix \example{-(u)mba} in \ili{Chabu}, by the suffix \example{-id/-di} in \ili{Pima Bajo}, by the suffix \example{-lhile} in Mparntwe Arrernte\il{Arrernte, Mparntwe}, and by the suffix \example{-re} in Tundra Yukaghir\il{Yukaghir, Tundra}. The applicative\is{applicative voice} use of the suffix \example{-lhile} in Mparntwe Arrertne\il{Arrernte, Mparntwe} is only attested with the two verbs \example{therre-} and \example{artne-} presented in the table. Furthermore, note that the Tundra Yukaghir\il{Yukaghir, Tundra} causative-applicative suffix \example{-re} can be found in the closely related language Kolyma Yukaghir\il{Yukaghir, Kolyma} as well \citep[224]{maslova:2003}. In Tundra Yukaghir\il{Yukaghir, Tundra} “[t]he suffix is confined to the \isi{semelfactive} [\isi{aspect}]” \citep[160]{schmalz:2013} and is therefore generally followed by the \isi{semelfactive} suffix \example{-j} (cf. \example{mojaγa-re-j-} ‘to make sth. soft’, \example{porčaγa-re-j-} ‘to sprinkle sth.’). However, the \isi{semelfactive} suffix is not exclusive to the causative\is{causative voice} and applicative\is{applicative voice} voices (cf., e.g., \example{tiwaγa-} ‘to wink’ ↔ \example{tiwaγa-j-} ‘to wink once’) and it is therefore not included in \tabref{tab:ch4:caus-appl} -- except in the case of the verb \example{köčegej-} in which the \isi{semelfactive} suffix appears to have become lexicalised\is{lexicalisation} \citep[28, 153]{schmalz:2013}. Observe that the voice relations \example{sal’γa-} ↔ \example{sal’γa-re-} and \example{köčegej-} ↔ \example{köčegej-re-} appear as \example{sal’γač} ↔ \example{sal’γarejm} and \example{köčegeč} ↔ \example{köčegejrem} in the original source \citep[160]{schmalz:2013}. The final element \example{-č} results from the affrication \citep[54]{schmalz:2013} of the semelfactive suffix \example{-j} and the third person intransitive marker \example{-j} (i.e. \example{sal’γač} < *\example{sal’γa-j-j} and \example{köčegeč} < *\example{köčege-j-j}), while the final \example{-m} is simply a language-specific third person transitive marker.

\begin{table}
	\setlength{\tabcolsep}{3pt}
	\begin{tabularx}{\textwidth}{llllll}
		\lsptoprule
		\multicolumn{6}{l}{\ili{Chabu} \citep[276, 279]{kibebe:2015}} \\
		\midrule
		\textsc{caus} & \example{ate-} & ‘to open’ & ↔ & \example{ate-\textbf{mba}-} & ‘to open sth.’ \\
		\textsc{caus} & \example{gɛt-} & ‘to move/turn’ & ↔ & \example{gɛt-\textbf{umba}-} & ‘to move/turn sth.’ \\
		\textsc{appl} & \example{tʼakʼo-} & ‘to pestle sth.’ & ↔ & \example{tʼakʼo-\textbf{mba}-} & ‘to pestle sth. for sb.’ \\
		\textsc{appl} & \example{aɗit-} & ‘to winnow sth.’ & ↔ & \example{aɗit-\textbf{umba}-} & ‘to winnow sth. for sb.’ \\
		\midrule\midrule
		\multicolumn{6}{l}{\ili{Pima Bajo} \citep[84, 122, 166, 169, 174, 214]{fernandez:2014}} \\
		\midrule
		\textsc{caus} & \example{hoin} & ‘to rock’ & ↔ & \example{hoin-\textbf{id}} & ‘to rock sth.’ \\
		\textsc{caus} & \example{tood} & ‘to be frightened’ & ↔ & \example{tood-\textbf{id}} & ‘to frighten sb.’ \\
		\textsc{appl} & \example{hink} & ‘to shout’ & ↔ & \example{hink-\textbf{id}} & ‘to shout at sb.’ \\
		\textsc{appl} & \example{som} & ‘to sew sth.’ & ↔ & \example{som-\textbf{di}} & ‘to sew sth. for sb.’ \\
		\midrule\midrule
		\multicolumn{6}{l}{Mparntwe Arrernte\il{Arrernte, Mparntwe} \citep[258]{wilkins:1989}} \\
		\midrule
		\textsc{caus} & \example{tnye-} & ‘to fall’ & ↔ & \example{tnye-\textbf{lhile}-} & ‘to make sth. fall’ \\
		\textsc{caus} & \example{pwernke-} & ‘to split open’ & ↔ & \example{pwernke-\textbf{lhile}-} & ‘to split sth. open’ \\
		\textsc{appl} & \example{therre-} & ‘to laugh’ & ↔ & \example{therre-\textbf{lhile}-} & ‘to laugh at sb.’ \\
		\textsc{appl} & \example{artne} & ‘to cry’ & ↔ & \example{artne-\textbf{lhile}-} & ‘to cry at sb.’ \\
		\midrule\midrule
		\multicolumn{6}{l}{Tundra Yukaghir\il{Yukaghir, Tundra} \citep[28, 111, 153f., 160]{schmalz:2013}} \\
		\midrule
		\textsc{caus} & \example{mojaγa-} & ‘to get soft’ & ↔ & \example{mojaγa-\textbf{re}-} & ‘to make sth. soft’ \\
		\textsc{caus} & \example{sal’γa-} & ‘to break’ & ↔ & \example{sal’γa-\textbf{re}-} & ‘to break sth.’ \\
		\textsc{appl} & \example{köčegej-} & ‘to gallop’ & ↔ & \example{köčegej-\textbf{re}-} & ‘to rush/jump at sb.’ \\
		\textsc{appl} & \example{porčaγa-} & ‘to splash’ & ↔ & \example{porčaγa-\textbf{re}-} & ‘to splash at sth.’ \\
		\lspbottomrule
	\end{tabularx}
	\caption{Examples of causative-applicative syncretism}
	\label{tab:ch4:caus-appl}
\end{table}

\newpage

Next, as shown in \tabref{tab:ch4:caus-appl-khimtanga}, in \ili{Khimt’anga} the suffix \example{-s} serves as voice marking in both the causative\is{causative voice} and applicative\is{applicative voice} voices, in the latter voice in combination with full \isi{reduplication}. The schwa in the reduplicated\is{reduplication} verbal forms is a “linking vowel” \citep[xxi]{belay:2015}. The applicative\is{applicative voice} voices are translated ‘[Guleshe] supported [them] break [the wood]’ and ‘[Aderu] supported [Guleshe] buy [the cow]’ in the original source, respectively \citep[231f.]{belay:2015}. It is they (i.e. ‘them’) who break the wood in the former example, and Guleshe only supports them in doing so. Likewise, Guleshe buys the cow in the latter example, and Aderu only supports him in doing so. Thus, there is no \isi{causer} present in neither voice, and the voice relations\is{voice relation} qualify as applicative\is{applicative voice} (\sectref{def:causatives-anticausatives}). The applicative\is{applicative voice} use of the suffix \example{-s} is tellingly called “\isi{adjutative}” by \citeauthor{belay:2015}. As already mentioned earlier in this section, the causative-applicative syncretism in \ili{Khimt’anga} evidently represents type 2 syncretism\is{voice syncretism, partial resemblance -- type 2}, while the causative-ap\-pli\-cative syncretism illustrated for \ili{Chabu}, \ili{Pima Bajo}, Mparntwe Arrernte\il{Arrernte, Mparntwe}, and Tundra Yukaghir\il{Yukaghir, Tundra} in \tabref{tab:ch4:caus-appl} represent type 1 syncretism\is{voice syncretism, full resemblance -- type 1}.

\begin{table}
	\begin{tabularx}{\textwidth}{llllll}
		\lsptoprule
		\multicolumn{6}{l}{\ili{Khimt’anga} \citep[127, 161, 229--237]{belay:2015}} \\
		\midrule
		\textsc{caus} & \example{χʷ-} & ‘to eat sth.’ & ↔ & \example{χʷ-\textbf{ɨs}-} & ‘to make sb. eat sth.’ \\
		\textsc{caus} & \example{qal-} & ‘to see sth.’ & ↔ & \example{qal-\textbf{s}-} & ‘to make sb. see sth.’ \\
		\textsc{appl} & \example{kil-} & ‘to break sth.’ & ↔ & \example{\textbf{kil-ə}-kil-\textbf{s}-} & ‘to break sth. with \\
		& & & & & \multicolumn{1}{r}{support from sb.’} \\
		\textsc{appl} & \example{dʒɨβ-} & ‘to buy sth.’ & ↔ & \example{\textbf{dʒɨβ-ə}-dʒɨβ-\textbf{ɨs}-} & ‘to buy sth. with \\
		& & & & & \multicolumn{1}{r}{support from sb.’} \\
		\lspbottomrule
	\end{tabularx}
	\caption{Causative-applicative syncretism in Khimt’anga}
	\label{tab:ch4:caus-appl-khimtanga}
\end{table} 

In many languages there is a close relationship between the causative and applicative voices, and cross-linguistic evidence suggests that causative-applicative syncretism can have either a causative\is{causative voice}\is{causative origin} (\sectref{diachrony:caus2appl}) or an \isi{applicative origin} (\sectref{diachrony:appl2caus}).

\subsection{Causative-passive} \label{sec:simple-syncretism:caus-pass}
Causative-passive syncretism has been the subject of much scrutiny in the literature, and observations on the phenomenon date back more than one and a half centuries. As noted by \cite[4f.]{nedyalkov:1991}, “[i]t was H. C. von der Gabelentz who in 1861 drew attention to the existence of such causative\is{causative voice} forms which may fulfil passive\is{passive voice} function” (see \citealt[516--529]{von-der-gabelentz:1861}). Renewed interest in the syncretism in question is in turn generally credited to the aforementioned \citeauthor{nedyalkov:1991}’s father \citeauthor{nedjalkov:1964}’s (\citeyear{nedjalkov:1964}) study “on the link between causativity\is{causative voice} and passivity\is{passive voice}” (\textit{О связи каузативности и пассивности}), as well as \cite[38ff.]{nedjalkov:silnickij:1969}. As observed in these and later studies, causative-passive syncretism appears to be particularly widespread among Altaic or Trans-Eurasian languages, including \ili{Korean} as well as Mongolic, Tungusic, and Turkic languages (\citealt{robbeets:2007}). Korean is included in the language sample of this book and so are representatives of the other three genera\is{genus}: the Mongolic language \ili{Mongolian}, the Tungusic language \ili{Kilen}, and the Turkic language \ili{Tatar}. Causative-passive syncretism is attested in the first three languages, but not in \ili{Tatar} \citep{zinnatullina:1993, burbiel:2018}. However, causative-passive syncretism can be found in other Turkic languages (e.g. Old Turkic\il{Turkic, Old} \example{bak-} ‘to look at sth.’ ↔ \example{bak-ït-} ‘to make sb. look at sth.’ and \example{kov-} ‘to follow/chase sb.’ ↔ \example{kov-ït-} ‘to be chased [by sb.]’, \citealt[291f.]{robbeets:2015}). In fact, \citeauthor{robbeets:2007} (\citeyear[178ff.]{robbeets:2007};; \citeyear[290ff.]{robbeets:2015}) reconstructs\is{reconstruction} a causative-passive suffix for \ili{Proto-Turkic}, \example{*-ti}. The reconstructed\is{reconstruction} suffix is reflected by the suffix \example{-t} in \ili{Tatar}, which has a causative\is{causative voice} function but not a passive\is{passive voice} function. \citeauthor{robbeets:2007} (\citeyear[165f.]{robbeets:2007};; \citeyear[276f.]{robbeets:2015}) also argues for causative-passive syncretism in \ili{Proto-Japonic} characterised by the suffix \example{*-ta} reflected by Old Japanese\il{Japanese, Old} \example{-t}, but provides no convincing examples of its purported passive\is{passive voice} use, and causative-passive syncretism is therefore not recognised for this language here. The Japonic \isi{genus} is represented in the language sample by the language \ili{Irabu} (\lang{ea}), which does not feature causative-passive syncretism either \citep{shimoji:2008}.

\citeauthor{kulikov:2001} (\citeyear[894]{kulikov:2001}; \citeyear[394]{kulikov:2010}) remarks that causative-passive syncretism has additionally been attested in “some West African languages (Songhai, Dogon), \ili{Bella Coola} (Amerindian), and some other languages of the world”, but provides no examples. In the language sample of this book the Songhay and Dogon genera\is{genus} are represented by the languages \ili{Humburi Senni} and \ili{Yanda Dom}, which do indeed feature causative-passive syncretism. Neither the language \ili{Bella Coola} nor the \isi{genus} of the same name is included in the sample, but two related languages are, the Central Salish language \ili{Musqueam} and the Interior Salish language \ili{Nxa’amxcin} (both \lang{na}). However, these languages do not feature causative-passive syncretism \citep{suttles:2004, willett:2003}. Finally, causative-passive syncretism has also been discussed to various extents by \cite[840]{shibatani:1985}, \cite[46ff.]{haspelmath:1990}, \cite{knott:1995}, \cite[31]{dixon:2000}, and \cite[400ff.]{malchukov:2016}.

Causative-passive syncretism in \ili{Mongolian}, \ili{Kilen}, \ili{Korean}, \ili{Humburi Senni}, and \ili{Yanda Dom} qualifies as type 1 syncretism\is{voice syncretism, full resemblance -- type 1}. Causative-passive voice marking in \ili{Korean} is syncretic with voice marking in the anticausative\is{anticausative voice} voice (\sectref{diachrony:caus2antc}), whereas causative-passive voice marking in the other languages is restricted to the causative\is{causative voice} and passive\is{passive voice} voices. The syncretism in each of these languages is illustrated in this section alongside examples of causative-passive type 1 syncretism\is{voice syncretism, full resemblance -- type 1} in the Mixe-Zoque language Ayutla Mixe\il{Mixe, Ayutla} (\lang{na}) and examples of causative-passive type 2 syncretism\is{voice syncretism, partial resemblance -- type 2} in the Finnic language \ili{Finnish} (\lang{ea}) and the Lowland East Cushitic language \ili{Konso} (\lang{af}). Causative-passive type 1 syncretism\is{voice syncretism, full resemblance -- type 1} has already been illustrated for the North Omotic language \ili{Wolaytta} (\lang{af}) in \tabref{tab:ch3:type1b-examples-1} on page \pageref{tab:ch3:type1b-examples-1} and for San Francisco del Mar Huave\il{Huave, San Francisco del Mar} (\lang{na}) in \tabref{tab:ch3:type1b-examples-2} on page \pageref{tab:ch3:type1b-examples-2}, while causative-passive type 2 syncretism\is{voice syncretism, partial resemblance -- type 2} has been exemplified for the Kxa language \ili{ǂHȍã} (\lang{af}) in \tabref{tab:ch3:type2-examples} on page \pageref{tab:ch3:type2-examples}. Additional examples are given in the next chapter. 

\newpage

For glossed examples of causative-passive syncretism, consider the following causative\is{causative voice} \isi{voice relation} (\ref{ex:AyutlaMixe:die:a}↔\ref{ex:AyutlaMixe:die:b}) and a passive\is{passive voice} \isi{voice relation} (\ref{ex:AyutlaMixe:take:a}↔\ref{ex:AyutlaMixe:take:b}) in Ayutla Mixe\il{Mixe, Ayutla}. As shown in these voice relations\is{voice relation}, the prefix \example{ak-} serves as voice marking in both the causative\is{causative voice} (\ref{ex:AyutlaMixe:die:b}) and passive\is{passive voice} voices (\ref{ex:AyutlaMixe:take:b}). \cite[370]{romero-mendez:2009} notes that “[t]he same phenomenon is observed in other Mixe languages”, including \ili{Olutec} \citep{zavala:2000}. 

\ea Ayutla Mixe\il{Mixe, Ayutla} \citep[482, 495]{romero-mendez:2009}
\ea\label{ex:AyutlaMixe:die:a}
	\gll	ta atäm n-jëntsën y-ook-yë’n \\
			\textsc{dem.med} \textsc{1pl.incl} \textsc{1poss}-chief \textsc{3sbj}-die-\textsc{1.incl} \\
	\glt	‘[…] then our leader died’.
\ex\label{ex:AyutlaMixe:die:b}
	\gll	pës n-\textbf{ak}-ook-ë’m yë’ë tsä’äny \\
			\textsc{disc} \textsc{1a-caus}-die-\textsc{1pl.excl} \textsc{dem.m} snake \\
	\glt	‘We have to kill the snake’.
\ex\label{ex:AyutlaMixe:take:a}
	\gll	ja’a pää’äy ojts w<y>ä’äke’ek-y \\
			\textsc{dem.dist} savage \textsc{pst} take<\textsc{3.obj.inv}>-\textsc{asp} \\
	\glt	‘The savage people took her there’.
\ex\label{ex:AyutlaMixe:take:b}
	\gll	ps jam ojts y-\textbf{ak}-wä’äke’ek-y \\
			\textsc{disc} \textsc{dem.dist} \textsc{pst} \textsc{3sg-pass}-take-\textsc{asp} \\
	\glt	‘She was taken there’.
	\z 
\z

Causative-passive syncretism in the four Trans-Eurasian languages \ili{Mongolian}, \ili{Kilen}, \ili{Korean}, and \ili{Finnish} is illustrated in \tabref{tab:ch4:caus-pass}. The syncretism is characterised by the suffix \example{-uul} in \ili{Mongolian}, by the suffix \example{-wu} in \ili{Kilen}, and by the suffix \example{-(C)i} in \ili{Korean}. In \ili{Finnish} the syncretism is characterised by the suffix \example{-ta/-tä} which is always accompanied by the suffix \example{-an/-än} in the passive\is{passive voice} voice but not in the causative\is{causative voice} voice. The allomorphs\is{allomorphy} of this Finnish suffix are conditioned by \isi{vowel harmony}. Next, causative-passive syncretism in the three African languages \ili{Yanda Dom}, \ili{Humburi Senni}, and \ili{Konso} is illustrated in \tabref{tab:ch4:caus-pass-2}. The syncretism is only marginally productive\is{productivity} in \ili{Yanda Dom} and \ili{Humburi Senni}. In \ili{Yanda Dom} the suffix \example{-mɛ́} serves as voice marking in both the causative\is{causative voice} and passive\is{passive voice} voices, though only three verbs of perception are attested in the latter voice. \cite[237]{heath:2017b} notes that the passive\is{passive voice} sense of the suffix “can be \isi{semelfactive}, e.g. ‘was seen (once)’, as well as \isi{habitual}”, and also remarks that the suffix can have a sense of potentiality depending on context (e.g. ‘to be findable’). \ili{Humburi Senni} is rather similar to \ili{Yanda Dom} in this respect and features the suffix \example{-(y)éyndí} that serves as voice marking in both the causative\is{causative voice} and passive\is{passive voice} voices. \cite[382]{heath:2014} calls the latter voice “\isi{potential passive}” and observes that “[t]he most common sense of the \isi{potential passive} is ‘be \textsc{verb}-able’ or ‘be habitually\is{habitual} \textsc{verb}-ed’”. However, he additionally remarks that “a more general passive\is{passive voice} function is also possible” \citep[382]{heath:2014}. In turn, in Konso the suffix \example{-aɗ} serves as voice marking in the passive\is{passive voice} voice, and also forms part of suffix \example{-acciis} (< \example{-aɗ} + \example{-ciis/-siis}) serving as voice marking in the causative\is{causative voice} voice \citep[139]{orkaydo:2013}.

\begin{table}
	\setlength{\tabcolsep}{4.7pt}
	\begin{tabularx}{\textwidth}{llllll}
		\lsptoprule
		\multicolumn{6}{l}{\ili{Mongolian} (\citealt[123]{tserenpil:kullmann:2008}; \citealt[249f.]{janhunen:2012})} \\
		\midrule
		\textsc{caus} & \example{asg-} & ‘to empty sth.’ & ↔ & \example{asg-\textbf{uul}-} & ‘to make/let sb. empty sth.’ \\
		\textsc{caus} & \example{id-} & ‘to eat sth.’ & ↔ & \example{id-\textbf{uul}-} & ‘to make/let sb. eat sth.’ \\
		\textsc{pass} & \example{id-} & ‘to eat sth.’ & ↔ & \example{id-\textbf{uul}-} & ‘to be eaten [by sb.]’ \\
		\textsc{pass} & \example{xaz-} & ‘to bite sth.’ & ↔ & \example{xaz-\textbf{uul}-} & ‘to be bitten [by sb.]’ \\
		\midrule\midrule
		\multicolumn{6}{l}{\ili{Kilen} \citep[59, 116f., 173, 188f.]{paiyu:2013}} \\
		\midrule
		\textsc{caus} & \example{ənə} & ‘to go’ & ↔ & \example{ənə-\textbf{wu}} & ‘to make sb. go’ \\
		\textsc{caus} & \example{tanta} & ‘to hit sth.’ & ↔ & \example{tanta-\textbf{wu}} & ‘to make sb. hit sb.’ \\
		\textsc{pass} & \example{tanta} & ‘to hit sth.’ & ↔ & \example{tanta-\textbf{wu}} & ‘to be hit [by sb.]’ \\
		\textsc{pass} & \example{dʑəfə} & ‘to eat’ & ↔ & \example{dʑəfə-\textbf{wu}} & ‘to be eaten [by sb.]’ \\
		\midrule\midrule
		\multicolumn{6}{l}{\ili{Korean} \citep[369, 375]{sohn:h-m:1999}} \\
		\midrule
		\textsc{caus} & \example{wus} & ‘to laugh’ & ↔ & \example{wus-\textbf{ki}} & ‘to make sb. laugh’ \\
		\textsc{caus} & \example{kwul} & ‘to roll’ & ↔ & \example{kwul-\textbf{li}} & ‘to make sth. roll’ \\
		\textsc{pass} & \example{ccoch} & to chase sb.‘’ & ↔ & \example{ccoch-\textbf{ki}} & ‘to be chased [by sb.]’ \\
		\textsc{pass} & \example{kkul} & ‘to pull sth.’ & ↔ & \example{kkul-\textbf{li}} & ‘to be pulled [by sb.]’ \\
		\midrule\midrule
		\multicolumn{6}{l}{\ili{Finnish} (personal knowledge)} \\
		\midrule
		\textsc{caus} & \example{alene-} & ‘to descend’ & ↔ & \example{alen-\textbf{ta}-} & ‘to lower sth.’ \\
		\textsc{caus} & \example{heikene-} & ‘to weaken’ & ↔ & \example{heiken-\textbf{tä}-} & ‘to weaken sth.’ \\
		\textsc{pass} & \example{lue-} & ‘to read sth.’ & ↔ & \example{lue-\textbf{ta-an}} & ‘to be read [by sb.]’ \\
		\textsc{pass} & \example{iske-} & ‘to hit sth.’ & ↔ & \example{iske-\textbf{tä-än}} & ‘to be hit [by sb.]’ \\	
		\lspbottomrule
	\end{tabularx}
	\caption{Examples of causative-passive syncretism (I)}
	\label{tab:ch4:caus-pass}
\end{table}

\begin{table}
	\setlength{\tabcolsep}{4.9pt}
	\begin{tabularx}{\textwidth}{llllll}
		\lsptoprule
		\multicolumn{6}{l}{\ili{Yanda Dom} \citep[227, 237]{heath:2017b}} \\
		\midrule
		\textsc{caus} & \example{jé} & ‘to dance’ & ↔ & \example{jé-\textbf{mɛ́}} & ‘to make sb. dance’ \\
		\textsc{caus} & \example{yɛ́} & ‘to weep’ & ↔ & \example{yɛ́-\textbf{mɛ́}} & ‘to make sb. weep’ \\
		\textsc{caus} & \example{nɔ́} & ‘to hear sth.’ & ↔ & \example{nɔ́-\textbf{mɛ́}} & ‘to make sb. hear sth.’ \\
		\textsc{pass} & \example{nɔ́} & ‘to hear sth.’ & ↔ & \example{nɔ́-\textbf{mɛ́}} & ‘to be heard [by sb.]’ \\
		\textsc{pass} & \example{tɛ́mbɛ́} & ‘to find sth.’ & ↔ & \example{tɛ́mbɛ́-\textbf{mɛ́}} & ‘to be found [by sb.]’ \\
		\textsc{pass} & \example{wɔ́} & ‘to see sth.’ & ↔ & \example{wɔ́-\textbf{mɛ́}} & ‘to be seen [by sb.]’ \\
		\midrule\midrule
		\multicolumn{6}{l}{\ili{Humburi Senni} \citep[280, 283]{heath:2014}} \\
		\midrule
		\textsc{caus} & \example{tóː} & ‘to become full’ & ↔ & \example{tóː-\textbf{yéyndí}} & ‘to fill sth.’ \\
		\textsc{caus} & \example{zéː} & ‘to swear’ & ↔ & \example{zéː-\textbf{yéyndí}} & ‘to make sb. swear’ \\
		\textsc{pass} & \example{nóː} & ‘to give sb. sth.’ & ↔ & \example{nóː-\textbf{yéyndí}} & ‘to be given [by sb. to sb.]’ \\
		\textsc{pass} & \example{díː} & ‘to see sth.’ & ↔ & \example{dí-\textbf{yéyndí}} & ‘to be seen [by sb.]’ \\
		\midrule\midrule
		\multicolumn{6}{l}{\ili{Konso} \citep[143ff., 222]{orkaydo:2013}} \\
		\midrule
		\textsc{caus} & \example{ʛot-} & ‘to dig sth.’ & ↔ & \example{ʛot-\textbf{acciis}} & ‘to make sb. dig sth.’ \\
		\textsc{caus} & \example{mur-} & ‘to cut sth.’ & ↔ & \example{mur-\textbf{acciis}} & ‘to make sb. cut sth.’ \\
		\textsc{pass} & \example{kup-} & ‘to burn sth.’ & ↔ & \example{kup-\textbf{aɗ}-} & ‘to be burned [by sb.]’ \\
		\textsc{pass} & \example{χor-} & ‘to fine sb.’ & ↔ & \example{χor-\textbf{aɗ}-} & ‘to be fined [by sb.]’ \\	
		\lspbottomrule
	\end{tabularx}
	\caption{Examples of causative-passive syncretism (II)}
	\label{tab:ch4:caus-pass-2}
\end{table}

It is well-known that causative-passive syncretism commonly has a \isi{causative origin}, as in the case of the four Trans-Eurasian languages included in \tabref{tab:ch4:caus-pass} (\sectref{diachrony:caus2pass}). By contrast, it has hitherto not been possible to find evidence for passive\is{passive voice} voice marking developing a causative\is{causative voice} function in any language.

\subsection{Causative-antipassive} \label{sec:simple-syncretism:caus-antp}
Discussions of causative-antipassive syncretism are considerably more scarce in the typological literature than those of causative-passive syncretism and mostly consist of sporadic observations on a few languages. For instance, \cite[295ff.]{creissels:nouguier-voisin:2008} describe causative-antipassive syncretism for the Atlantic language \ili{Wolof} (e.g. \example{génn} ‘to go out’ ↔ \example{génn-e} ‘to take sth. out’, i.e. ‘to make sth. go out’ and \example{màtt} ‘to bite sth.’ ↔ \example{màtt-e} ‘to bite [sth.]’), and \cite[20]{creissels:diagne:2013} describe similar syncretism for the Western Mande language \ili{Soninke} (e.g. \example{bònò} ‘to become spoilt’ ↔ \example{bònò-ndì} ‘to damage sth.’, i.e. ‘to make sth. become spoilt’ ↔ \example{bònò-ndì-ndì} ‘to damage [sth.]’). Both languages are spoken in Africa. Outside of Africa, \cite[240, 244f.]{juarez:gonzalez:2017} have described causative-antipassive syncretism for the South Guaicuruan language \ili{Mocoví} spoken in South America (e.g. \example{[ɾ-]eda} ‘to move’ ↔ \example{[y-]ida-ɢan} ‘to move sth.’ and \example{-ta-} ‘to sniff sth.’ ↔ \example{-ta-ɢan} ‘to sniff [sth.]’). Observe that the third person \isi{agreement} markers \example{ɾ-} and \example{y-} are here included in square brackets only to show how they affect the following vowel phonologically. Neither the vowel variation nor the \isi{agreement} markers themselves are part of the voice marking in the language.

The Atlantic, Western Mande, and South Guaicuruan genera\is{genus} are represented in the language sample by the languages Ganja Balanta\il{Balanta, Ganja}, \ili{Jalkunan}, and \ili{Pilagá}, respectively. However, causative-antipassive syncretism is only attested in Ganja Balanta. In addition to this language, only one other language in the sample features causative-antipassive syncretism, the Timor-Alor-Pantar language \ili{Makalero} (\lang{pn}). The syncretism in \ili{Makalero} has already been exemplified in \tabref{tab:ch3:type1a-examples-1} on page \pageref{tab:ch3:type1a-examples-1}, while the syncretism in Ganja Balanta\il{Balanta, Ganja} has been illustrated in \tabref{tab:ch3:type1b-examples-2} on page \pageref{tab:ch3:type1b-examples-2}. Nevertheless, due to the low prevalence of causative-antipassive syncretism cross-linguistically, it is discussed in this section as well. Causative-antipassive syncretism in \ili{Makalero} is more specifically of type 1a\is{voice syncretism, full resemblance -- type 1} and characterised by the suffix \example{-ini} as seen in the following causative\is{causative voice} \isi{voice relation} (\ref{ex:Makalero:break:a}↔\ref{ex:Makalero:break:b}) and the antipassive\is{antipassive voice} \isi{voice relation} (\ref{ex:Makalero:weave:a}↔\ref{ex:Makalero:weave:b}). \cite[340]{huber:2011} explicitly notes that the suffix in question “can function to either add or remove a participant to or from the sentence”. In the causative\is{causative voice} voice the suffix is obligatorily accompanied by the auxiliary verb \example{mei} (\ref{ex:Makalero:break:b}) which has the meaning ‘to take’ “if used as a lexical verb” \citep[203]{huber:2011}. However, as this verb does not constitute verbal marking, it is not considered to form part of the voice marking in the causative\is{causative voice} voice (\sectref{resemblance-type1a}). Furthermore, note that from a \isi{language-specific} perspective \cite[340]{huber:2011} argues that the \isi{causer} and \isi{causee} in the causative\is{causative voice} voice stand in separate clauses as the result of the inclusion of the auxiliary verb. Nevertheless, the use of the verb appears to be fully grammaticalised\is{grammaticalisation} (with no indication of its original lexical meaning), and the causative\is{causative voice} example (\ref{ex:Makalero:break:b}) is therefore treated as a single clause from a cross-linguistic perspective. 

\ea \ili{Makalero} \citep[299, 340f.]{huber:2011}
\ea\label{ex:Makalero:break:a}
	\gll	kopu ere hai da’al, ira hai mu’a-isa \\
			glass \textsc{1dem} \textsc{nsit} break water \textsc{nsit} ground-go.down \\
	\glt	‘This glass broke and the water spilled’.
\ex\label{ex:Makalero:break:b}
	\gll	mata ka’u=ni kopu ere	mei=ni da’al-\textbf{ini} \\
			child small=\textsc{ctr} glass \textsc{1dem} take=\textsc{link} break-\textsc{caus} \\
	\glt	‘The child broke the glass’.
\ex\label{ex:Makalero:weave:a}
	\gll	ani sedang heru=ua ei=ua so’ot ere heru \\
			\textsc{1sg} \textsc{prog} cloth=\textsc{rel} \textsc{2s=rel} want \textsc{1dem} weave \\
	\glt	‘I’m weaving the cloth that you asked for’.
\ex\label{ex:Makalero:weave:b}
	\gll	tufuraa k-asu=ni uere=ni omar-ik’a lopu-ika’ isa-\textbf{ini} tina-\textbf{ini} heru-\textbf{ini} \\
			woman \textsc{2.und}-for=\textsc{ctr} \textsc{2dem=ctr} stilt.house-up.in house-up.in bake-\textsc{antp} cook-\textsc{antp} weave-\textsc{antp}\\
	\glt	‘(Work) for the women is to stay at home, bake, cook, weave, [...]’
	\z 
\z

Causative-antipassive syncretism in Ganja Balanta\il{Balanta, Ganja} qualifies as type 1b syncretism\is{voice syncretism, full resemblance -- type 1}, and is characterised by the suffix \example{-t} as seen in \tabref{tab:ch4:caus-antp-ganja}. The verbs in the causative\is{causative voice} and antipassive\is{antipassive voice} voices differ in terms of \isi{verb class} as indicated by the different infinitive vowels: verbs in the causative\is{causative voice} voice belong to a so-called class A or C, while verbs in the antipassive\is{antipassive voice} voice belong to class B \citep[142ff., 208, 211]{creissels:biaye:2016}. Furthermore, it can be noted that the suffix \example{-t} seemingly has the allomorph\is{allomorphy} \example{-Vt} in the causative\is{causative voice} voice (e.g. \example{yisim} ‘to sneeze’ ↔ \example{yisim-it} ‘to make sb. sneeze’, \citealt[209]{creissels:biaye:2016}) but not in the antipassive\is{antipassive voice} voice. Finally, observe that \citeauthor{creissels:biaye:2016} only have attested the antipassive\is{antipassive voice} use of the suffix \example{-t} with four verb stems in Ganja Balanta (the other two verbs are given in \tabref{tab:ch3:type1b-examples-2} on page \pageref{tab:ch3:type1b-examples-2}).

\begin{table}
	\begin{tabularx}{\textwidth}{llllll}
		\lsptoprule
		\multicolumn{6}{l}{Ganja Balanta\il{Balanta, Ganja} \citep[209ff.]{creissels:biaye:2016}} \\
		\midrule
		\textsc{caus} & \example{sιιg} & ‘to drink sth.’ & ↔ & \example{sιιg-\textbf{t}.ι} & ‘to make sb. drink sth.’ \\
		\textsc{caus} & \example{θɔɔb} & ‘to be(come) slim’ & ↔ & \example{θɔɔb-\textbf{t}.ι} & ‘to make sb. be(come) slim’ \\
		\textsc{antp} & \example{lɔt} & ‘to cook sth.’ & ↔ & \example{lɔt-\textbf{t}.ɛ} & ‘to cook [sth.]’ \\
		\textsc{antp} & \example{sʊg} & ‘to sew/sow sth.’ & ↔ & \example{sʊg-\textbf{t}.ɛ} & ‘to sew/sow [sth.]’ \\
		\lspbottomrule
	\end{tabularx}
	\caption{Causative-antipassive syncretism in Ganja Balanta}
	\label{tab:ch4:caus-antp-ganja}
\end{table} 

In terms of diachrony, little is known about the emergence of causative-antipas\-sive syncretism, and there is currently no evidence for a development from cau\-sa\-tive\is{causative voice} to antipassive\is{antipassive voice} nor from antipassive\is{antipassive voice} to causative\is{causative voice} in any language. However, it can here be mentioned that \cite[18]{creissels:2015} has proposed that the causative\is{causative voice} and antipassive\is{antipassive voice} functions of the suffix \example{-ndi} in \ili{Soninke} mentioned at the beginning of this section “result from the grammaticalization\is{grammaticalisation} of the same \ili{Proto-West-Mande} verb (\example{*tin} ‘do’) in two different constructions: a causative\is{causative voice} periphrasis and an antipassive\is{antipassive voice} periphrasis”. The same appears to be true for the \ili{Makalero} causative-antipassive\is{antipassive voice} suffix \example{-ini} which relates to the verb \example{kini} ‘to do/make’ \citep[128]{huber:2011}. Nevertheless, the more specific \isi{bridging context} as well as the order in which the causative\is{causative voice} and antipassive\is{antipassive voice} functions of the two suffixes evolved remain obscure.

\subsection{Causative-reflexive} \label{sec:simple-syncretism:caus-refl}
Previous research on causative-reflexive appears to be very scant or altogether non-existent, as it has not been possible to find a single discussion of the phenomenon in the literature. \cite{pederson:1991} investigates “universals in the syncretism of reflexive\is{reflexive voice} and causative\is{causative voice} constructions”, but treats reflexive\is{reflexive voice} and causative\is{causative voice} syncretism separately and does not address causative-reflexive syncretism. The syncretism in question has been attested in only two languages in the sample, and the lack of literature concerning the phenomenon is therefore not surprising. These two languages are the North Omotic language \ili{Wolaytta} (\lang{af}) and the Northern Chukotko-Kamchatkan language \ili{Chukchi} (\lang{ea}), and in both languages causative-reflexive voice marking is syncretic with marking in other voices as well. It has hitherto not been possible to find any language featuring voice marking restricted exclusively to the causative\is{causative voice} and reflexive\is{reflexive voice} voices. Both \ili{Wolaytta} and \ili{Chukchi} are treated in more detail in the next chapter (see \sectref{sec:complex-syncretism:caus-pass} and \sectref{sec:complex-syncretism:caus-refl-antc}, respectively), but due to the rare nature of causative-reflexive syncretism and for the sake of illustration in this section, the syncretism in the languages is briefly exemplified here. Accordingly, causative-reflexive syncretism in \ili{Wolaytta} is here illustrated by a causative \isi{voice relation} (\ref{ex:Wolaytta:tasty:a}↔\ref{ex:Wolaytta:tasty:b}) and a passive\is{passive voice} \isi{voice relation} (\ref{ex:Wolaytta:respect:a}↔\ref{ex:Wolaytta:respect:b}). As seen in these examples, the suffix \example{-ett} serves as voice marking in the causative\is{causative voice} voice (\ref{ex:Wolaytta:tasty:b}) and the suffix \example{-ett/-étt} serves as voice marking in the passive\is{passive voice} voice (\ref{ex:Wolaytta:respect:b}).

\ea \ili{Wolaytta} \citep[706, 797, 1029, 1072]{wakasa:2008}
\ea\label{ex:Wolaytta:tasty:a}
	\gll	hageeti 7ubb-ái-kka maLL-óosona \\
			these all-\textsc{nom.m.sg}-too be.tasty-\textsc{ipfv.3pl} \\
	\glt	‘These [bulbs of garlic, cabbages, onions] are all tasty’.
\ex\label{ex:Wolaytta:tasty:b}
	\gll	zaar-ídi 7á wáát-i maLL-\textbf{ett}-úuteetii? \\
			return-\textsc{cvb.3pl} it.\textsc{abs} do.what-\textsc{cvb.2pl} be.tasty-\textsc{caus-q} \\
	\glt	‘[…] how will you make it tasty again?’
\ex\label{ex:Wolaytta:respect:a}
	\gll	7alb-é-nné tiit-ú banta-7aaw-áa bonc-óosona \\
			\textsc{name-nom}-and \textsc{name-nom} own-father-\textsc{abs.m.sg} respect-\textsc{ipfv.3pl} \\
	\glt	‘Albe and Tito respect their father’.
\ex\label{ex:Wolaytta:respect:b}
	\gll	bonc-\textbf{étt}-a \\
			respect-\textsc{refl-opt.2sg} \\
	\glt	‘Respect yourself!’
	\z 
\z

\newpage

By comparison, as \tabref{tab:ch4:caus-refl} shows, in \ili{Chukchi} the suffix \example{-et} serves as voice marking in both the reflexive\is{reflexive voice} and causative\is{causative voice} voices, in the latter voice in combination with the prefix \example{r-/n-}. Evidently, \ili{Wolaytta} features causative-reflexive type 1b syncretism\is{voice syncretism, full resemblance -- type 1}, and \ili{Chukchi} features causative-reflexive type 2 syncretism\is{voice syncretism, partial resemblance -- type 2}. 

\begin{table}
	\begin{tabularx}{.90\textwidth}{llllll}
		\lsptoprule
		\multicolumn{6}{l}{\ili{Chukchi} (\citealt[72, 206, 256]{dunn:1999}; \citealt[186]{kurebito:2012}} \\
		\midrule
		\textsc{caus} & \example{qit} & ‘to freeze’ & ↔ & \example{\textbf{r-/n-ə}-qit-\textbf{et}} & ‘to freeze sth.’ \\
		\textsc{caus} & \example{lw} & ‘to burn’ & ↔ & \example{\textbf{r-/n-ə}-lw-\textbf{et}} & ‘to burn sth.’ \\
		\textsc{refl} & \example{ejup} & ‘to prick sb.’ & ↔ & \example{ejup-\textbf{et}} & ‘to prick self’ \\
		\textsc{refl} & \example{qetw} & ‘to stab sb.’ & ↔ & \example{qetw-\textbf{et}} & ‘to stab self’ \\
		\lspbottomrule
	\end{tabularx}
	\caption{Causative-reflexive syncretism in Chukchi}
	\label{tab:ch4:caus-refl}
\end{table} 

It is difficult to draw any conclusions about the diachrony of causative-reflexive syncretism based on data from \ili{Wolaytta} and \ili{Chukchi} alone. The diachrony of the \ili{Wolaytta} suffix \example{-ett} (and \example{-étt}) is  unknown, and the functions of the \ili{Chukchi} suffix \example{-et} is described as having “unpredictable semantic or syntactic features” by \cite[243]{dunn:1999} as further discussed in \sectref{sec:simple-syncretism:appl-antc}. Thus, the emergence and development of causative-reflexive syncretism remains obscure for the time being.

\subsection{Causative-reciprocal} \label{sec:simple-syncretism:caus-recp}
Causative-reciprocal syncretism has been described for a few languages in the literature, most notably for the Arawakan language \ili{Yine} (or Piro; \lang{sa}) in which the syncretism is characterised by the suffix \example{-kaka} (\citealt[38]{nedjalkov:silnickij:1969}; \citealt{kulikov:nedjalkov:1992}; \citealt[894]{kulikov:2001}; \citealt[292]{nedjalkov:2007d}). \cite[286]{nedjalkov:2007d} also observes causative-reciprocal syncretism in the related language \ili{Wayuu} (or Guajiro) characterised by the suffix \example{-hira}. Unfortunately, however, it has not been possible to obtain concrete examples of the latter suffix, and the purported syncretism in Wayuu can therefore not be confirmed here. \cite[894]{kulikov:2001} argues that “[t]his rare type of syncretism” also occurs in some Austronesian languages, including the Oceanic languages \ili{Nakanai} and \ili{Tangga} (both \lang{pn}), but provides no examples (see instead \citealt[181f.]{johnston:1978} and \citealt[286]{nedjalkov:2007d}). Causative-reciprocal syncretism is also found in, for example, the Nilotic language \ili{Bari} characterised by the prefix \example{tɔ-} \citep[285ff.]{nedjalkov:2007d} and in the Northern Atlantic language \ili{Wolof} (both \lang{af}) characterised by the suffix \example{-e}, though the reciprocal\is{reciprocal voice} function of the suffix \example{-e} in the Wolof is “not very productive”\is{productivity} \citep[298]{creissels:nouguier-voisin:2008}.

The language sample of this book includes \ili{Yine} and also covers the Oceanic, Nilotic, and Northern Atlantic genera\is{genus}, represented by the languages \ili{Cheke Holo}, \ili{Luwo}, and Ganja Balanta\il{Balanta, Ganja}, respectively. While causative-reciprocal syncretism is indeed attested in Yine, none of the three other languages features the syncretism in question. However, four other languages in the sample do feature causative-reciprocal syncretism: the North Omotic language \ili{Wolaytta} (\lang{af}), the Northwest Sumatra-Barrier Islands language \ili{Gayo} (\lang{pn}), the Dizoid language \ili{Sheko} (\lang{af}), and the language isolate \ili{Nivkh} (\lang{ea}). The causative-reciprocal syncretism in \ili{Yine} and \ili{Wolaytta} qualifies as type 1 syncretism\is{voice syncretism, full resemblance -- type 1}, in \ili{Gayo} and \ili{Sheko} as type 2 syncretism\is{voice syncretism, partial resemblance -- type 2}, and in \ili{Nivkh} as type 3 syncretism\is{voice syncretism, reverse resemblance -- type 3}. The syncretism in the latter language has already been illustrated in Table \ref{tab:ch3:type3-examples} on page \pageref{tab:ch3:type3-examples}, while the syncretism in \ili{Wolaytta} is discussed in \sectref{sec:complex-syncretism:caus-pass}. Causative-reciprocal syncretism in the remaining languages is illustrated in this section, starting with the following causative\is{causative voice} \isi{voice relation} (\ref{ex:Yine:run:a}↔\ref{ex:Yine:run:b}) and reciprocal\is{reciprocal voice} \isi{voice relation} (\ref{ex:Yine:hit:a}↔\ref{ex:Yine:hit:b}) in \ili{Yine}. In this language the causative-reciprocal marker \example{-kaka} bears some resemblance to the passive\is{passive voice} marker \example{-ka} (see \tabref{tab:ch5:caus-recp-pass} on page \pageref{tab:ch5:caus-recp-pass}).

\ea \ili{Yine} \citep[33, 191, 269f.]{hanson:2010}
\ea\label{ex:Yine:run:a}
	\gll	r-hasɨka-na pimri-ne \\
			\textsc{3}-run-\textsc{3pl} other-\textsc{pl} \\
	\glt	‘The others ran off’.
\ex\label{ex:Yine:run:b}
	\gll	wale hasɨka-\textbf{kaka}-na-na \\
			\textsc{3sg.m} run-\textsc{caus-cmpv-3pl} \\
	\glt	‘He made them run’.
\ex\label{ex:Yine:hit:a}
	\gll	t-hiylaka-hima-ta-lɨ \\
			\textsc{3sg.f}-hit-\textsc{quot-th-3sg.m} \\
	\glt	‘She hit it, reportedly’. 
\ex\label{ex:Yine:hit:b}
	\gll	r-hiylaka-\textbf{kaka}-na-na sɨwa-yma hawa mhenoklɨ-ne-yma \\
			\textsc{3}-hit-\textsc{recp-cmpv-3pl} anteater-\textsc{com} and jaguar-\textsc{pl-com} \\
	\glt	‘They fought each other, the anteater and the jaguars’.
	\z 
\z

Next, as illustrated in \tabref{tab:ch4:caus-recp}, in \ili{Sheko} the reciprocal\is{reciprocal voice} voice is characterised by the suffix \example{-s-ǹ} which is composed of the causative\is{causative voice} suffix \example{-s} and the suffix \example{-ǹ} which can, for instance, have an anticausative\is{anticausative voice} function on its own (e.g. \example{gàz-} ‘to snap sth.’ ↔ \example{gàz-ǹ} ‘to snap’, \citealt[284]{hellenthal:2010}). The causative\is{causative voice} suffix is generally “coupled with L tone on the verb stem and \isi{vowel shortening} (if the root has a long vowel)” \citep[373]{hellenthal:2010}. \cite[395]{hellenthal:2010} explicitly addresses the causative-passive syncretism in \ili{Sheko}, and comments that syncretism of this kind is uncommon in other Omotic languages. Next, as also illustrated in \tabref{tab:ch4:caus-recp}, in \ili{Gayo} the suffix \example{-nen} serves as voice marking in both the causative\is{causative voice} and reciprocal\is{reciprocal voice} voices, in the latter voice accompanied by the prefix \example{bersi-} or, alternatively, by the prefix \example{be(r)-} plus \isi{reduplication} (e.g. \example{tulak} ‘to push sb.’ ↔ \example{be-te~tulak-an} ‘to push e.o.’, \citealt[154]{eades:2005}). The suffix in question has four allomorphs:\is{allomorphy} \example{-nan} found on verb stems ending in the vowel /a/, \example{-nen} on verb stems ending in any other vowel, \example{-an} on consonant-final verb stems with the vowel /a/ in the last syllable, and \example{-en} on consonant-final verb stems with any other vowel in the last syllable. \cite[39f.]{eades:2005} adds that “[t]he forms \example{-nen} and \example{-nan} are often reduced to \example{-n}, which is in free variation with the longer forms”. Being a typical western Austronesian language, \ili{Gayo} features three so-called “voice” or “orientation” affixes (e.g. \isi{undergoer} orientation \example{i-}, \isi{actor} orientation \example{mun-}, and unintentional \isi{undergoer} \example{ter-}). However, \cite[167]{eades:2005} explicitly argues that the phenomenon of voice in Gayo “contrasts with primarily syntactically motivated explanations for voice” and that “voice affixation signals the semantic \isi{macrorole} of the \isi{subject} argument in a clause that involves two semantic participants\is{semantic participant}”. In other words, the function of “voice” in \ili{Gayo} is dependent primarily on discourse continuity, and is not considered voice marking in relation to the causative\is{causative voice} and reciprocal\is{reciprocal voice} voices by \cite[162f., 186ff.]{eades:2005} and this book (\sectref{def:principles}). Thus, the various Gayo verbs in \tabref{tab:ch4:caus-recp} are given without any orientation affixes. 

\begin{table}
	\setlength{\tabcolsep}{4.3pt}
	\begin{tabularx}{\textwidth}{llllll}
		\lsptoprule
		\multicolumn{6}{l}{\ili{Sheko} \citep[195, 374, 394, 433]{hellenthal:2010}} \\
		\midrule
		\textsc{caus} & \example{sár-} & ‘to be hot’ & ↔ & \example{sar-\textbf{s}} & ‘to heat sth.’ \\
		\textsc{caus} & \example{door-} & ‘to run’ & ↔ & \example{dor-\textbf{s}} & ‘to make sb. run’ \\
		\textsc{recp} & \example{tùfkù-} & ‘to bump into sb.’ & ↔ & \example{tùfkù-\textbf{s-ǹ}} & ‘to bump into e.o.’ \\
		\textsc{recp} & \example{tʼùùs-} & ‘to know sb.’ & ↔ & \example{tʼùs-\textbf{ùs-ǹ}} & ‘to know e.o.’ \\
		\midrule\midrule
		\multicolumn{6}{l}{\ili{Gayo} \citep[14, 39, 124, 162, 171, 187f.]{eades:2005}} \\
		\midrule
		\textsc{caus} & \example{tangkuh} & ‘to go out’ & ↔ & \example{tangkuh-\textbf{n}} & ‘to make sb. go out’ \\
		\textsc{caus} & \example{ayo} & ‘to enter’ & ↔ & \example{ayo-\textbf{n(en)}} & ‘to make sb. enter’ \\
		\textsc{recp} & \example{dere} & ‘to hit sb.’ & ↔ & \example{\textbf{bersi}-dere-\textbf{n(en)}} & ‘to hit e.o.’ \\
		\textsc{recp} & \example{tipak} & ‘to kick sb.’ & ↔ & \example{\textbf{bersi}-tipak-\textbf{an}} & ‘to kick e.o.’ \\
		\lspbottomrule
	\end{tabularx}
	\caption{Causative-reciprocal syncretism}
	\label{tab:ch4:caus-recp}
\end{table} 

\newpage

The diachrony of causative-reciprocal syncretism is not well-known, but there is some cross-linguistic evidence for a \isi{reciprocal origin} in some languages, in part facilitated by \isi{comitativity} (\sectref{diachrony:recp2caus}). By contrast, currently there does not appear to be any evidence for causative\is{causative voice} voice marking developing a reciprocal\is{reciprocal voice} function in any language. It can be mentioned here that the causative-reciprocal syncretism in the Oceanic language \ili{Nakanai} mentioned at the beginning of this section seems to be the result of coincidental phonological \isi{convergence} of \ili{Proto-Oceanic} reciprocal\is{reciprocal voice} \example{*pa\textsc{r}i-} and causative\is{causative voice} \example{*paka-} following the loss of the phonemes \example{*\textsc{r}} and \example{*k} in the language (e.g. \example{va-ubi} ‘to shoot e.o.’ and \example{va-lolo} ‘to make sb. hear sth.’, \citealt[286]{nedjalkov:2007d}). The diachrony of \example{*pa\textsc{r}i-} is discussed in more detail in \sectref{diachrony:recp2refl}, \sectref{diachrony:recp2antc}, and \sectref{diachrony:recp2antp}.

\subsection{Causative-anticausative} \label{sec:simple-syncretism:caus-antc}
Discussions of causative-anticausative syncretism in the typological literature are difficult to come by. \cite[244]{zuniga:kittila:2019} state that they “have found only one clear case of it in the literature”, in the language isolate \ili{Ainu} (\lang{ea}), and further argue that “[t]he causative-anticausative syncretism is especially striking, given the semantic and syntactic disparity of the two effects”. Nevertheless, it is worth noting that causative-anticausative syncretism has in fact been observed in at least one other language in the literature, \ili{Japanese} (\lang{ea}). For instance, \cite[310]{comrie:2006} remarks that “[o]ne of the striking characteristics of \isi{inchoative}-causative pairs in \ili{Japanese} is that the suffix \example{-e} is used with some verbs to mark the \isi{inchoative}, with other verbs to mark the causative\is{causative voice}”, and goes on to provide two “[c]omprehensive lists of 36 pairs where \example{-e} marks the anticausative\is{anticausative voice} and 57 where it marks the causative\is{causative voice}” (see also \citealt[197ff.]{jacobsen:1982}). The “\isi{inchoative}” mentioned by \citeauthor{comrie:2006} is compatible with the anticausative\is{anticausative voice} voice in this book. Some of the 93 verbal pairs are also listed by \cite[116]{haspelmath:1993}. Consider for example the following voice relations\is{voice relation} in \ili{Japanese} (in the original source these verbs are followed by the non-past suffix \example{-(r)u} which has here been omitted for clarity): \example{sizum-} ‘to sink’ ↔ \example{sizum-e-} ‘to sink sth.’ and \example{or-} ‘to break sth.’ ↔ \example{or-e-} ‘to break’ \citep[311f.]{comrie:2006}. \ili{Ainu} is included in the language sample of this book while the Japonic \isi{genus} is represented by the language \ili{Irabu} which -- unlike Japanese -- does not feature causative-anticausative syncretism. In addition to \ili{Ainu}, causative-anticausative syncretism has been attested in four other languages in the sample: the language isolate \ili{Korean}, the Ugric language Northern Mansi\il{Mansi, Northern} (both \lang{ea}), the language isolate \ili{Kutenai} (\lang{na}), and the Northern Chukotko-Kamchatkan language \ili{Chukchi} (\lang{ea}). The syncretism qualifies as type 2 syncretism\is{voice syncretism, partial resemblance -- type 2} in Chukchi, and as type 1 syncretism\is{voice syncretism, full resemblance -- type 1} in the other languages.

Causative-anticausative syncretism has already been illustrated for Northern Mansi\il{Mansi, Northern} in \tabref{tab:ch3:type1a-examples-2} on page \pageref{tab:ch3:type1a-examples-2} and for \ili{Kutenai} in \sectref{resemblance-type2}, while it is discussed for \ili{Korean} in \sectref{sec:complex-syncretism:caus-pass} and for \ili{Chukchi} in \sectref{sec:complex-syncretism:caus-refl-antc}. However, due to the little attention causative-anticausative syncretism has received in the literature, it is briefly exemplified for each of the languages here in this section. Glossed examples of causative-anticausative type 2 syncretism\is{voice syncretism, partial resemblance -- type 2} in \ili{Chukchi} are provided below in the form of a causative\is{causative voice} \isi{voice relation} (\ref{ex:Chukchi:open:a}↔\ref{ex:Chukchi:open:b}) and an anticausative\is{anticausative voice} \isi{voice relation} (\ref{ex:Chukchi:close:a-2}↔\ref{ex:Chukchi:close:b-2}). The examples show that the suffix \example{-et} serves as voice marking in both the anticausative\is{anticausative voice} (\ref{ex:Chukchi:close:b-2}) and causative\is{causative voice} voices, in the latter in combination with the prefix \example{r-/n-} (\ref{ex:Chukchi:open:b}). It is worth noting, however, that the anticausative\is{anticausative voice} use of the suffix is only marginally productive\is{productivity} in the language, attested with just three verbs \citep[187]{kurebito:2012}.

\ea \ili{Chukchi} (\citealt[6]{stenin:2017}; \citealt[187]{kurebito:2012})
\ea\label{ex:Chukchi:open:a}
	\gll	qerɣəsʔ-ə-n sinit went-ə-ɣʔ-i \\
			window-\textsc{ep-abs.sg} self open-\textsc{ep-th-3sg.sbj} \\
	\glt	‘A window opened itself’.
\ex\label{ex:Chukchi:open:b}
	\gll	ɣəm-nan qerɣəsʔ-ə-n t-ə-\textbf{n}-went-\textbf{et}-ɣʔe-n \\
			\textsc{1sg-erg} window-\textsc{ep-abs.sg} \textsc{1sg-ep-caus}-open-\textsc{caus-th-3sg.obj} \\
	\glt	‘I opened the window’.
\ex\label{ex:Chukchi:close:a-2}
	\gll	t-ejp-ɣʔe-n qerɣəsʔ-ə-n \\
			\textsc{1sg}-close-\textsc{th-3sg.obj} window-\textsc{ep-abs.sg} \\
	\glt	‘I closed the window’.
\ex\label{ex:Chukchi:close:b-2}
	\gll	qerɣəsʔ-ə-n ejp-\textbf{et}-ɣʔ-i \\
			window-\textsc{ep-abs.sg} close-\textsc{antc-th-3sg.sbj} \\
	\glt	‘The window closed’.
	\z 
\z

As illustrated in \tabref{tab:ch4:caus-antc}, causative-anticausative type 1a syncretism\is{voice syncretism, full resemblance -- type 1} is characterised by the suffix \example{-(C)i} in \ili{Korean}, by the suffix \example{-ke} in \ili{Ainu}, and by the suffix \example{-l} in Northern Mansi\il{Mansi, Northern}. In the two latter languages the illustrated anticausative\is{anticausative voice} voices are defined according to an \isi{equipollent} causative-anticausative \isi{voice relation}, and the marking in the voices is thus in variation with verbal marking in the contrasting causative\is{causative voice} voices (\sectref{def:causatives-anticausatives}). As noted by \cite[244]{zuniga:kittila:2019}, “[t]his may aid the speakers in keeping the two functions of this syncretic marker apart”. Finally, as also illustrated in \tabref{tab:ch4:caus-antc}, causative-anticausative type 1b\is{voice syncretism, full resemblance -- type 1} in \ili{Kutenai} is characterised by a glottal stop \example{-ʔ} which has the allomorph\is{allomorphy} \example{-p} in the anticausative\is{anticausative voice} voice but not in the causative\is{causative voice} voice. As already noted in \sectref{resemblance-type1b}, the anticausative\is{anticausative voice} allomorph\is{allomorphy} \example{-ʔ} appears before “the invariantly encliticized Indicative Marker [\example{-ni}], and the invariantly encliticized Locative Marker [\example{-ki}]” and the allomorph\is{allomorphy} \example{-p} appears elsewhere \citep[336]{morgan:1991}. Compare, for example, the verbs \example{¢̓aqa-ʔ-ni} ‘it (proximate) is greasy’ and \example{¢̓aqa-p-si} ‘it (obviate) is greasy’. The suffixal \example{-a} in the verb \example{ʔiʔtwum-a-ʔ} is simply epenthetic.

\begin{table}
	\setlength{\tabcolsep}{4.6pt}
	\begin{tabularx}{\textwidth}{llllll}
		\lsptoprule
		\multicolumn{6}{l}{\ili{Korean} (\citealt[82f.]{baek:1997};; \citealt[375]{sohn:h-m:1999})} \\
		\midrule
		\textsc{caus} & \example{cwul-} & ‘to decrease’ & ↔ & \example{cwul-\textbf{li}-} & ‘to reduce sth.’ \\
		\textsc{caus} & \example{nwup-} & ‘to lie down’ & ↔ & \example{nwup-\textbf{hi}-} & ‘to lay sth. down’ \\
		\textsc{antc} & \example{yel-} & ‘to open sth.’ & ↔ & \example{yel-\textbf{li}-} & ‘to open’ \\
		\textsc{antc} & \example{mak-} & ‘to block sth.’ & ↔ & \example{mak-\textbf{hi}-} & ‘to block’ \\
		\midrule\midrule
		\multicolumn{6}{l}{\ili{Ainu} \citep[44]{shibatani:1990}} \\
		\midrule
		\textsc{caus} & \example{ray} & ‘to die’ & ↔ & \example{ray-\textbf{ke}} & ‘to kill sb.’ \\
		\textsc{caus} & \example{ahun} & ‘to enter’ & ↔ & \example{ahun-\textbf{ke}} & ‘to make sb. enter’ \\
		\textsc{caus} & \example{sat} & ‘to dry’ & ↔ & \example{sat-\textbf{ke}} & ‘to dry sth.’ \\
		\textsc{antc} & \example{mak-a} & ‘to open sth.’ & ↔ & \example{mak-\textbf{ke}} & ‘to open’ \\
		\textsc{antc} & \example{kom-o} & ‘to bend sth.’ & ↔ & \example{kom-\textbf{ke}} & ‘to bend’ \\
		\textsc{antc} & \example{mes-u} & ‘to tear sth. off’ & ↔ & \example{mes-\textbf{ke}} & ‘to come off’ \\
		\midrule\midrule
		\multicolumn{6}{l}{Northern Mansi\il{Mansi, Northern} \citep[154, 160]{rombandeeva:1973}} \\
		\midrule
		\textsc{caus} & \example{pons-} & ‘to cure’ & ↔ & \example{pons-\textbf{l}-} & ‘to cure sth.’ \\
		\textsc{caus} & \example{tōs-} & ‘to dry’ & ↔ & \example{tōs-\textbf{l}-} & ‘to dry sth.’ \\
		\textsc{caus} & \example{āst-} & ‘to end’ & ↔ & \example{āst-\textbf{l}-} & ‘to end sth.’ \\
		\textsc{antc} & \example{sawa-t-} & ‘to torment sb.’ & ↔ & \example{sawa-\textbf{l}-} & ‘to stuffer’ \\
		\textsc{antc} & \example{xari̮ɣ-t-} & ‘to extinguish sth.’ & ↔ & \example{xari̮ɣ-\textbf{l}-} & ‘to extinguish’ \\
		\textsc{antc} & \example{xali-t-} & ‘to split sth.’ & ↔ & \example{xali-\textbf{l}-} & ‘to split’ \\
		\midrule\midrule
		\multicolumn{6}{l}{\ili{Kutenai} \citep[25, 297, 336, 337]{morgan:1991}} \\
		\midrule
		\textsc{caus} & \example{yik̓ta} & ‘to spill’ & ↔ & \example{yik̓ta-\textbf{ʔ}} & ‘to spill sth.’ \\
		\textsc{caus} & \example{ʔiʔtwum} & ‘to become pregnant’ & ↔ & \example{ʔiʔtwum-a-\textbf{ʔ}} & ‘to impregnate sb.’ \\
		\textsc{antc} & \example{¢̓aqa} & ‘to grease sth.’ & ↔ & \example{¢̓aqa-\textbf{ʔ}} & ‘to be greasy’ \\
		\textsc{antc} & \example{¢uku} & ‘to light sth.’ & ↔ & \example{¢uku-\textbf{ʔ}} & ‘to become lit’ \\
		\lspbottomrule
	\end{tabularx}
	\caption{Examples of causative-anticausative syncretism}
	\label{tab:ch4:caus-antc}
\end{table}

\newpage

The causative-anticausative syncretism discussed for \ili{Korean} is very likely of \isi{causative origin}, and the same might be true for \ili{Ainu} (\sectref{diachrony:caus2antc}). By contrast, there is currently no evidence for anticausative\is{anticausative voice} voice marking developing a causative\is{causative voice} function in any language.\is{anticausative origin}

\section{Applicative syncretism} \label{sec:simple-syncretism:applicative}
Patterns of applicative\is{applicative voice} syncretism are among the least common patterns of voice syncretism attested in the language sample. In fact, applicative-anticausative syncretism remains unattested altogether, while applicative-reflexive syncretism is attested only as type 2\is{voice syncretism, partial resemblance -- type 2} and type 3 syncretism\is{voice syncretism, reverse resemblance -- type 3}. In any case, all kinds of applicative\is{applicative voice} syncretism are explicitly discussed in the following sections for the sake of linguistic diversity.

\subsection{Applicative-passive} \label{sec:simple-syncretism:appl-pass}
Applicative-passive syncretism has received little prior explicit treatment in the literature, though it has been extensively discussed implicitly in relation to the syncretism between the passive\is{passive voice} voice and a so-called “\isi{adversative} passive” in some languages which qualifies as applicative-passive syncretism in this book (\sectref{def:applicatives}). For instance, consider the following \ili{Japanese} (\lang{ea}) voice relations\is{voice relation}: \example{koros-} ‘to kill sb.’ ↔ \example{koros-are-} ‘to be killed [by sb.]’ and \example{sin-} ‘to die’ ↔ \example{sin-are-} ‘to die to the detriment of sb.’ (\citealt[244]{zuniga:kittila:2019}). See also \cite[608f.]{malchukov:nedjalkov:2015} for similar examples from the Tungusic language \ili{Evenki}. \cite[81]{zuniga:kittila:2019} discuss the \ili{Japanese} \isi{voice relation} in terms of “subjective undergoer nucleatives”\is{subjective undergoer nucleative} because “unlike applicatives\is{applicative voice}, these operations install these [non-agentive] arguments as subjects\is{subject}”. This distinction is not maintained in this book, and the \ili{Japanese} \isi{voice relation} in question qualifies as applicative\is{applicative voice}. Applicative-passive type 1 syncretism\is{voice syncretism, full resemblance -- type 1} similar to that described for \ili{Japanese} can be found in the related Japonic language \ili{Irabu} which is included in the language sample of this book. The syncretism can additionally be found in the language isolate \ili{Kutenai} (\lang{na}), in the Panoan language \ili{Chácobo}, and the language isolate \ili{Mosetén} (both \lang{sa}). Applicative-passive syncretism in \ili{Irabu} and \ili{Kutenai} qualifies as type 1 syncretism\is{voice syncretism, full resemblance -- type 1}, in \ili{Chácobo} as type 2 syncretism\is{voice syncretism, partial resemblance -- type 2}, and in \ili{Mosetén} as type 3 syncretism\is{voice syncretism, reverse resemblance -- type 3}. The syncretism in the latter language has already been discussed in \sectref{resemblance-type3}, while it is described for \ili{Kutenai} and \ili{Chácobo} in \sectref{sec:complex-syncretism:caus-appl-pass}. 

\newpage

In turn, glossed examples of applicative-passive syncretism in \ili{Irabu} are provided here in the form of an applicative\is{applicative voice} \isi{voice relation} (\ref{ex:Irabu:take:a}↔\ref{ex:Irabu:take:b}) and a passive\is{passive voice} \isi{voice relation} (\ref{ex:Irabu:scold:a}↔\ref{ex:Irabu:scold:b}). A similar applicative\is{applicative voice} \isi{voice relation} has already been discussed in \sectref{def:applicatives} (see examples \ref{ex:Irabu:fall:a}↔\ref{ex:Irabu:fall:b} on page \pageref{ex:Irabu:fall:a}). As seen in these examples, in \ili{Irabu} the suffix \example{-ai} serves as voice marking in both the applicative\is{applicative voice} (\ref{ex:Irabu:take:b}) and passive\is{passive voice} voices (\ref{ex:Irabu:scold:b}). The suffix \example{-a} in (\ref{ex:Irabu:scold:a}) is simply a “thematic vowel” which is found on some verbs when followed by “certain inflectional suffixes”, including the “finite irrealis intentional suffix \example{-di}” \citep[260f.]{shimoji:2008}. Furthermore, the underlying stem in (\ref{ex:Irabu:scold:a}↔\ref{ex:Irabu:scold:b}) is actually \example{ž}. The geminate form \example{žž} is the result of a “geminate copy insertion rule” described by \cite[69]{shimoji:2008} in the following manner: “if underlyingly moraic //C// and //(G)V// are adjacent in a word-plus, then a geminate copy of //C// is inserted to produce a surface /CiCi(G)V/”. This rule applies to both the thematic vowel \example{-a} and the applicative-passive suffix \example{-ai} \citep[70, 297]{shimoji:2008}. 

\ea \ili{Irabu} \citep[193, 297, 496]{shimoji:2008}
\ea\label{ex:Irabu:take:a}
	\gll	taugagara=nu jaa=ju=du tur-tar \\
			someone=\textsc{nom} house=\textsc{acc=foc} take-\textsc{pst} \\
	\glt	‘Someone took a house (by force)’.
\ex\label{ex:Irabu:take:b}
	\gll	kari=a taugagara=n jaa=ju=d tur-\textbf{ai}-tar \\
			\textsc{3sg=top} someone=\textsc{dat} house=\textsc{acc=foc} take-\textsc{appl-pst} \\
	\glt	‘He was troubled by the fact that someone took his house (by force)’.
\ex\label{ex:Irabu:scold:a}
	\gll	ba=ga ffa-gama=u=du žž-a-di \\
			\textsc{1sg=nom} child-\textsc{dim=acc=foc} scold-\textsc{th-int} \\
	\glt	‘I will scold (my) little child’.
\ex\label{ex:Irabu:scold:b}
	\gll	ba=a sinsii=n=du žž-\textbf{ai}-tar \\
			\textsc{1sg=top} teacher=\textsc{dat=foc} scold-\textsc{pass-pst} \\
	\glt	‘I was scolded by the teacher’.
	\z 
\z

\cite[244]{zuniga:kittila:2019} argue that the similarity between applicatives\is{applicative voice} (in their terminology, “subjective undergoer nucleatives”) and passives\is{passive voice} “is unsurprising given the grammatical relations involved in both kinds of constructions”. Here they refer to the similarities in how the \isi{applicative participant} in the applicative\is{applicative voice} voice and the \isi{semantic participant} which is not the \isi{agent} in the passive\is{passive voice} voice are treated (cf. \example{kari=a} ‘he’ in example \ref{ex:Irabu:take:b} and \example{ba=a} ‘I’ in example \ref{ex:Irabu:scold:b}).

\subsection{Applicative-antipassive} \label{sec:simple-syncretism:appl-antp}
Applicative-antipassive syncretism has received some attention in the literature, though discussions of the syncretism remain largely sporadic. The syncretism has notably been discussed repeatedly in relation to the Eskimo language Central Alaskan Yupik\il{Yupik, Central Alaskan} (\lang{na}; e.g. \citealt[121f.]{malchukov:2015};; \citeyear[405ff.]{malchukov:2016};; \citeyear[13ff.]{malchukov:2017};; \citealt[243]{zuniga:kittila:2019}; \citealt[210ff.]{basilico:2019}). \citeauthor{malchukov:2015} and \citeauthor{zuniga:kittila:2019} also mention applicative-antipassive syncretism in the Northern Chukotko-Kamchatkan language \ili{Chukchi} (\lang{ea}), and \citeauthor{malchukov:2015} describes the syncretism for the Interior Salish language \ili{Sliammon} and the Central Salish language \ili{Halkomelem} (\lang{na}). Both Central Alaskan Yupik\il{Yupik, Central Alaskan} and \ili{Chukchi} are included in the language sample of this book and therefore discussed in this section. By contrast, the Interior and Central Salish genera\is{genus} are represented in the language sample by the languages \ili{Nxa’amxcin} and \ili{Musqueam}, respectively, but applicative-antipassive syncretism is not attested in these languages \citep{willett:2003, suttles:2004}. Additionally, it is worth observing that \cite[524ff.]{valenzuela:2016} has explicitly argued for applicative-antipassive syncretism in the Cahuapanan language \ili{Shiwilu} (\lang{sa}), and even suggests that “\ili{Shiwilu}’s sister language” \ili{Shawi} features the syncretism in question as well. Consider, for instance, \ili{Shiwilu} \example{lamapu’-} ‘to scream’ ↔ \example{lamapu’-tu-} ‘to scream at sb.’ and \example{panu-} ‘to give sth. as a present to sb.’ ↔ \example{panu-tu-} ‘to give sth. as a present [to sb.]’ \citep[524f.]{valenzuela:2016}.

In addition to Central Alaskan Yupik\il{Yupik, Central Alaskan} and \ili{Chukchi}, applicative-antipassive syncretism has only been attested in one other language in the language sample, the Gunwinyguan language \ili{Nunggubuyu} (\lang{au}). The syncretism in question is of type 1\is{voice syncretism, full resemblance -- type 1} in Central Alaskan Yupik\il{Yupik, Central Alaskan} and \ili{Chukchi}, but of type 3\is{voice syncretism, reverse resemblance -- type 3} in \ili{Nunggubuyu} (\sectref{resemblance-type3}). Interestingly, Central Alaskan Yupik\il{Yupik, Central Alaskan} possesses two suffixes that can serve as voice marking in both the applicative\is{applicative voice} and antipassive\is{antipassive voice} voices, \example{-ut} and \example{-i} (with the respective underlying forms \example{-uc} and \example{-ɣi}, \citealt[830ff.]{miyaoka:2012}). The former suffix even serves as voice marking in the reciprocal\is{reciprocal voice} voice and is treated in more detail in \sectref{sec:complex-syncretism:appl-antp-recp}. In turn, the functions of the latter suffix are here illustrated by an applicative\is{applicative voice} \isi{voice relation} (\ref{ex:CAY:die:a}↔\ref{ex:CAY:die:b}) and an antipassive\is{antipassive voice} \isi{voice relation} (\ref{ex:CAY:lose:a}↔\ref{ex:CAY:lose:b}).

\ea Central Alaskan Yupik\il{Yupik, Central Alaskan} \citep[396, 517, 836]{miyaoka:2012}
\ea\label{ex:CAY:die:a}
	\gll	nakmiilla-a tuqu-uq \\
			own-\textsc{abs.3sg.sg} die-\textsc{ind.3sg} \\
	\glt	‘His real offspring died’.
	\newpage
\ex\label{ex:CAY:die:b}
	\gll	tuqu-\textbf{i}-gaqa nulia-qa \\
			die-\textsc{appl-ind.1sg:3sg} wife-\textsc{abs.1sg.sg} \\
	\glt	‘My wife died on me’.
\ex\label{ex:CAY:lose:a}
	\gll	qimugta tamar-aqa \\
			dog.\textsc{abs.sg} lose-\textsc{ind.1sg:3sg} \\
	\glt	‘I lost the dog’.
\ex\label{ex:CAY:lose:b}
	\gll	angun {\ob}qimugte-mek{\cb} tamar-\textbf{i}-uq \\
			man.\textsc{abs.sg} {\db}dog-\textsc{abl.sg} lose-\textsc{antp-ind.3sg} \\
	\glt	‘The man lost [a dog]’.
	\z 
\z

In \ili{Chukchi} applicative-antipassive syncretism is characterised by the prefix \example{ine-/ena-} conditioned by \isi{vowel harmony} (\citealt[48]{dunn:1999}), as exemplified by the following applicative\is{applicative voice} (\ref{ex:Chukchi:hang:a}↔\ref{ex:Chukchi:hang:b}) and the antipassive\is{antipassive voice} voice relations\is{voice relation} (\ref{ex:Chukchi:inform:a}↔\ref{ex:Chukchi:inform:b}). Observe that the underlying stem in both (\ref{ex:Chukchi:hang:a}) and (\ref{ex:Chukchi:hang:b}) is the same (i.e. \example{jme}), the schwa in the former example is simply epenthetic \citep[39ff.]{dunn:1999}. 

\ea \ili{Chukchi} \citep[212, 215f.]{dunn:1999}
\ea\label{ex:Chukchi:hang:a}
	\gll	ətlʔa-ta jəme-nenat ewirʔ-ə-t \\
			mother-\textsc{erg} hang-\textsc{3sg:3pl} clothing-\textsc{ep-3pl.abs} \\
	\glt	‘Mother hung up the clothes’.
\ex\label{ex:Chukchi:hang:b}
	\gll	ətlʔa-ta \textbf{ena}-jme-nen tətəl meniɣ-e \\
			mother-\textsc{erg} \textsc{appl}-hang-\textsc{3sg:3sg} door.\textsc{3sg.abs} cloth-\textsc{ins} \\
	\glt	‘Mother hung the door with cloth’.
\ex\label{ex:Chukchi:inform:a}
	\gll	ɣəmnan t-ə-n-walom-at-ə-nat ənpənacɣ-ə-t \\
			\textsc{1sg.erg} \textsc{1sg.a-ep-caus}-hear/understand-\textsc{caus-ep-3pl.obj} old.man-\textsc{ep-3pl.abs} \\
	\glt	‘I informed the old men’.
\ex\label{ex:Chukchi:inform:b}
	\gll	ɣəmo t-\textbf{ena}-n-walom-at-ə-k \\
			\textsc{1sg.abs} \textsc{1sg-antp-caus}-hear/understand-\textsc{caus-ep-1sg} \\
	\glt	‘I made an announcement’.
	\z 
\z

While applicative-antipassive syncretism in Central Alaskan Yupik\il{Yupik, Central Alaskan} is likely of applicative(-reciprocal) origin\is{applicative origin} (\sectref{diachrony:appl2antp}), little is otherwise known about the \isi{diachronic development} of such syncretism. \cite[24]{malchukov:2017} suggests that “applicatives\is{applicative voice} of transitives\is{transitive} share the feature of P-\isi{demotion} with antipassives\is{antipassive voice}” which provides a plausible explanation for the syncretism, at least from a syntactic point of view. Semantically, all semantic participants\is{semantic participant} remain in place in the passive\is{passive voice} voice.

\subsection{Applicative-reflexive} \label{sec:simple-syncretism:appl-refl}
It has not been possible to find any discussion nor mentioning of applicative-reflexive syncretism in the literature, and the syncretism has only been marginally attested in the language sample: as type 2 syncretism\is{voice syncretism, partial resemblance -- type 2} in the language isolate \ili{Kutenai} (\lang{na}), and as type 3 syncretism\is{voice syncretism, reverse resemblance -- type 3} in the language isolate \ili{Mosetén} (\lang{sa}) and in the Gunwinyguan language \ili{Nunggubuyu} (\lang{au}). Applicative-reflexive type 1 syncretism\is{voice syncretism, full resemblance -- type 1} remains unattested. The applicative-reflexive syncretism in \ili{Mosetén} and \ili{Nunggubuyu} has already been discussed in \sectref{resemblance-type3}, whereas it is described for \ili{Kutenai} here. In this language applicative-reflexive type 2 syncretism\is{voice syncretism, partial resemblance -- type 2} is characterised by an “Associative Suffix” \example{-m} \citep[209]{morgan:1991}, which forms part of the voice marking in both the applicative\is{applicative voice} (i.e. \example{-m-aɬ}) and reflexive\is{reflexive voice} voices (\example{-m-ik}), as illustrated in the following voice relations\is{voice relation} (\ref{ex:Kutenai:fish:a}↔\ref{ex:Kutenai:fish:b}) and (\ref{ex:Kutenai:wash:a}↔\ref{ex:Kutenai:wash:b}). \cite[313, 321]{morgan:1991} calls the additional suffix \example{-ik} in the reflexive\is{reflexive voice} voice marking a “Reflexive\is{reflexive voice} Suffix”, and the additional suffix \example{-aɬ} in the applicative\is{applicative voice} voice marking a “Co-Participant\is{co-participation} Suffix” which “occurs nowhere else in the language”. Neither suffix seems to have a reflexive\is{reflexive voice} or applicative\is{applicative voice} function without the suffix \example{-m}. 

\ea \ili{Kutenai} \citep[292, 313, 363, 381]{morgan:1991}
\ea\label{ex:Kutenai:fish:a}
	\gll	kaʔ ku-¢ ʔaˑ-qaɬ haɬuqɬawut \\
			how \textsc{subord-fut} \textsc{im}-be.thus-\textsc{adv} fish \\
	\glt	‘[I wondered] how I was going to fish [in order to get the char to bite]’. 
\ex\label{ex:Kutenai:fish:b}
	\gll	taxa-s hu n-aɬuqɬawut-\textbf{m-aɬ}-ni-¢ \\
			then-\textsc{3} \textsc{1sg} \textsc{pred}-fish-\textsc{assoc-appl-ind}-and \\
	\glt	‘Then I went out fishing with her’. 
\ex\label{ex:Kutenai:wash:a}
	\gll	hu-n ʔiktuquʔ-ni \\
			\textsc{1sg-pred} wash-\textsc{ind} \\
	\glt	‘I washed him/her/it/them’. 
\ex\label{ex:Kutenai:wash:b}
	\gll	hu-n ʔiktuquʔ-\textbf{m-ik} \\
			\textsc{1sg-pred} wash-\textsc{assoc-refl} \\
	\glt	‘I washed myself’. 
	\z 
\z

\newpage

Given the limited data available on applicative-reflexive syncretism -- and the apparent absence of applicative-reflexive type 1 syncretism\is{voice syncretism, full resemblance -- type 1} -- it is not possible to draw any conclusions about the diachrony of the syncretism in question. It is, for instance, not clear how the \ili{Kutenai} suffix \example{-m} has become part of both applicative\is{applicative voice} and reflexive\is{reflexive voice} voice marking. For comparison, reciprocity\is{reciprocal voice} in the language is characterised by the suffix \example{-nam} which “appears to have originated as the inflectional Indefinite Human (Subject) Suffix /\example{-am}/, preceded by the N-Connector Suffix /\example{-n-}/” \citep[376]{morgan:1991}.

\subsection{Applicative-reciprocal} \label{sec:simple-syncretism:appl-recp}
In the literature applicative-reciprocal syncretism has been discussed most notably by \citeauthor{nedjalkov:2007a} who has noted the syncretism in the Eskimo language West Greenlandic\il{Greenlandic, West} (\lang{na}; \citeyear[174]{nedjalkov:2007c}), in the Bantu language \ili{Kinyarwanda} (\lang{af}; \citeyear[42]{nedjalkov:2007b}; \citeyear[275]{nedjalkov:2007d}), and in the Turkic language \ili{Yakut} (\lang{ea}; \citeyear[237]{nedjalkov:2007d}; \citealt{nedjalkov:nedjalkov:2007}). The applicative-reciprocal syncretism in \ili{Kinyarwanda} has also been addressed by \cite{maslova:2007}, while applicative-reciprocal syncretism among Bantu languages in general has been discussed by \cite{bostoen:al:2015}. These languages are all discussed in more detail in \sectref{diachrony:recp2appl} and \sectref{diachrony:appl2recp}. Discussions of applicative-reciprocal syncretism are otherwise rather uncommon, yet the syncretism in question is undoubtedly the most common pattern of applicative\is{applicative voice} syncretism in the language sample of this book. It is, however, primarily attested as type 2 syncretism\is{voice syncretism, partial resemblance -- type 2}, including in the Siouan language \ili{Assiniboine} (\lang{na}), the Tibeto-Burman language \ili{Galo} (\lang{ea}), the Malayo-Sumbawan language \ili{Madurese} (\lang{pn}), the Central Cushitic language \ili{Khimt’anga} (\lang{af}), and the Arauan language \ili{Kulina} (\lang{sa}). It is otherwise attested as type 1 syncretism\is{voice syncretism, full resemblance -- type 1} in the Eskimo language Central Alaskan Yupik\il{Yupik, Central Alaskan}, the language isolate \ili{Yuchi} (both \lang{na}), and the Ju-Kung language Western !Xun\il{!Xun, Western} (\lang{af}); and as type 3 syncretism\is{voice syncretism, reverse resemblance -- type 3} in the language isolate \ili{Mosetén}, the Nadahup language \ili{Hup} (both \lang{sa}), and the Gunwinyguan language \ili{Nunggubuyu} (\lang{au}). Applicative-reciprocal syncretism has been exemplified for Western !Xun\il{!Xun, Western} in \tabref{tab:ch3:type1a-examples-1} on page \pageref{tab:ch3:type1a-examples-1}, and for \ili{Mosetén}, \ili{Hup}, and \ili{Nunggubuyu} in \sectref{resemblance-type3}. The syncretism is described for the remaining languages here.

Applicative-reciprocal syncretism in Central Alaskan Yupik\il{Yupik, Central Alaskan} is illustrated by glossed applicative\is{applicative voice} (\ref{ex:CAY:eat:a}↔\ref{ex:CAY:eat:b}) and reciprocal\is{reciprocal voice} voice relations\is{voice relation} (\ref{ex:CAY:see:a}↔\ref{ex:CAY:see:b}) which show that the suffix \example{-ut} (with the underlying form \example{-uc}, \citealt[830ff.]{miyaoka:2012}) can serve as voice marking in both the applicative\is{applicative voice} (\ref{ex:CAY:eat:b}) and reciprocal\is{reciprocal voice} voices (\ref{ex:CAY:see:b}) in addition to the antipassive\is{antipassive voice} voice (\sectref{sec:complex-syncretism:appl-antp-recp}). The suffix can optionally be accompanied by a reciprocal pronoun in the reciprocal\is{reciprocal voice} voice \citep[928]{miyaoka:2012} as in (\ref{ex:CAY:see:b}). The diachrony of the suffix is discussed in \sectref{diachrony:recp2antp}, \sectref{diachrony:appl2recp}, and \sectref{diachrony:appl2antp}.

\ea Central Alaskan Yupik\il{Yupik, Central Alaskan} \citep[656, 844, 929, 953]{miyaoka:2012}
\ea\label{ex:CAY:eat:a}
	\gll	angute-m ner-aa neqa \\
			man-\textsc{rel.sg} eat-\textsc{ind.3sg:3sg} fish.\textsc{abs.sg} \\
	\glt	‘The man is eating the fish’.
\ex\label{ex:CAY:eat:b}
	\gll	ner-\textbf{ut}-aa neq-mek angun \\
			eat-\textsc{appl-ind.3sg:3sg} fish-\textsc{abl.sg} man.\textsc{abs.sg} \\
	\glt	‘She is eating fish with the man’.
\ex\label{ex:CAY:see:a}
	\gll	tangrr-aqa kenurraq qull-ra-mni \\
			see-\textsc{ind.1sg:3sg} lamp.\textsc{abs.sg} area.above-just-\textsc{loc.1sg.sg} \\
	\glt	‘I saw the lamp just right above me’.
\ex\label{ex:CAY:see:b}
	\gll	aana-ka kass’aq=llu tangrr-\textbf{ut}-uk ellmeg-nek \\
			mother-\textsc{abs.1sg.sg} white.man.\textsc{abs.sg}=and see-\textsc{recp-ind.3du} \textsc{3du-abl} \\
	\glt	‘My mother and the white man see each other’.
	\z 
\z

Applicative-reciprocal type 1 syncretism\is{voice syncretism, full resemblance -- type 1} in \ili{Yuchi} appears to have developed rather recently. \cite[251, 265]{linn:2000} argues that historically the “accompaniment” prefix \example{k’ã-} has served as voice marking in the applicative\is{applicative voice} voice, while the prefix \example{k’a-} has served as voice marking in the reciprocal\is{reciprocal voice} voice. However, \cite[251]{linn:2000} further remarks that “[s]ome speakers today make no difference in pronunciation between the reciprocal\is{reciprocal voice} prefix and the accompaniment prefix” and that “some speakers pronounce both \example{k’æ}” or \example{k’a-}. The present resemblance between the voice marking in the applicative\is{applicative voice} and reciprocal\is{reciprocal voice} voices in the language is illustrated in \tabref{tab:ch4:appl-recp:yuchi}. Nevertheless, note that the variation in pronunciation of the prefix in the applicative\is{applicative voice} voice remains visible to some extent in the language -- at least in \citeauthor{linn:2000}’s grammar. For instance, \cite[254]{linn:2000} lists the applicative\is{applicative voice} verbs in \tabref{tab:ch4:appl-recp:yuchi} elsewhere in her grammar as \example{k’ã-thla} and \example{k’ã-gõ}. 

\begin{table}
	\begin{tabularx}{0.99\textwidth}{llllll}
		\lsptoprule
		\multicolumn{6}{l}{\ili{Yuchi} \citep[148f., 213, 226, 253f.]{linn:2000}} \\
		\midrule
		\textsc{appl} & \example{gõ} & ‘to come’ & ↔ & \example{\textbf{k’a}-gõ} & ‘to bring/come with sb.’ \\
		\textsc{appl} & \example{thla} & ‘to go’ & ↔ & \example{\textbf{k’a}-thla} & ‘to carry/go with sb.’ \\
		\textsc{recp} & \example{’nẽ} & ‘to see/meet sb.’ & ↔ & \example{\textbf{k’a}-’nẽ} & ‘to see/meet e.o.’ \\
		\textsc{recp} & \example{’yuhõ} & ‘to embrace sb.’ & ↔ & \example{\textbf{k’a}-’yuhõ} & ‘to embrace e.o.’ \\
		\lspbottomrule
	\end{tabularx}
	\caption{Applicative-reciprocal type 1 syncretism}
	\label{tab:ch4:appl-recp:yuchi}
\end{table}

Applicative-reciprocal type 2 syncretism\is{voice syncretism, partial resemblance -- type 2} is illustrated for \ili{Assiniboine}, \ili{Galo}, \ili{Kulina}, \ili{Madurese}, and \ili{Khimt’anga} in \tabref{tab:ch4:appl-recp-type2}. In \ili{Assiniboine} the prefixes \example{ki-} and \example{kíci-} both serve as voice marking in the applicative\is{applicative voice} voice, while the prefix \example{kicʰi-} serves as voice marking in the reciprocal\is{reciprocal voice} voice. \cite[258]{cumberland:2005} calls these suffixes “\textsc{ki} morphemes” because “they have related meanings, share phonological characteristics, and have similar phonetic shapes that are likely due to a common historical source”. Note that the prefixes \example{ki-} and \example{kíci-} sometimes appear as infixes (e.g. \example{įcú} ‘to smoke’ ↔ \example{į<kí>cú} ‘to smoke for sb.’ and \example{iyúškį} ‘to admire sb.’ ↔ \example{i<kíci>yúškį} ‘to admire sb. for sb.’, \citealt[263ff.]{cumberland:2005}). Note also that the stress is fixed on the affix \example{kíci-} but not on the other two affixes (for instance, the stress pattern \example{kicʰí-} appears in the third person, while the stress pattern \example{kícʰi-} appears in the first person, \citealt[270]{cumberland:2005}). More examples of the prefixes \example{ki-} and \example{kicʰi-} are provided in \tabref{tab:ch3:type2-examples} on page \pageref{tab:ch3:type2-examples}. Next, in \ili{Galo} the suffix \example{-rɨ́k} serves as voice marking in both the applicative\is{applicative voice} and reciprocal\is{reciprocal voice} voices, in the latter in combination with the suffix \example{-hí}. \cite[530f.]{post:2007} speculates that the former suffix \example{-rɨ́k} can “presumably reconstruct\is{reconstruction} to \ili{Proto-Tani}” with the sense ‘to meet’, whereas the latter suffix has a reflexive\is{reflexive voice} function (e.g. \example{pá-} ‘to cut sth.’ ↔ \example{pá-hí-} ‘to cut self’, \citealt[137, 541]{post:2007}). 

\begin{table}
	\setlength{\tabcolsep}{3.6pt}
	\begin{tabularx}{\textwidth}{llllll}
		\lsptoprule
		\multicolumn{6}{l}{\ili{Assiniboine} \citep[263ff., 270f.]{cumberland:2005}} \\
		\midrule
		\textsc{appl} & \example{kté} & ‘to kill sb.’ & ↔ & \example{\textbf{ki}-kté} & ‘to kill sb. for sb.’ \\
		\textsc{appl} & \example{nową́} & ‘to sing’ & ↔ & \example{\textbf{kíci}-nową́} & ‘to sing for sb.’ \\
		\textsc{recp} & \example{kté} & ‘to kill sb.’ & ↔ & \example{\textbf{kicʰí}-kte} & ‘to kill e.o.’ \\
		\textsc{recp} & \example{yaʔį́škata} & ‘to tease sb.’ & ↔ & \example{\textbf{kicʰí}-yaʔįškata} & ‘to tease e.o.’ \\
		\midrule\midrule
		\multicolumn{6}{l}{\ili{Galo} \citep[134, 137, 152, 519, 530, 543, 725, 935]{post:2007}} \\
		\midrule
		\textsc{appl} & \example{dàk} & ‘to stand’ & ↔ & \example{dàk-\textbf{rɨ́k}} & ‘to stand up next to sb.’ \\
		\textsc{appl} & \example{ín} & ‘to go’ & ↔ & \example{ín-\textbf{rɨ́k}} & ‘to go to sb.’ \\
		\textsc{recp} & \example{pá} & ‘to chop sth.’ & ↔ & \example{pá-\textbf{rɨ́k-hí}} & ‘to cut e.o.’ \\
		\textsc{recp} & \example{záp} & ‘to talk to sb.’ & ↔ & \example{záp-\textbf{rɨ́k-hí}} & ‘to talk to e.o.’ \\
		\midrule\midrule
		\multicolumn{6}{l}{\ili{Kulina} \citep[78, 114, 128ff., 139, 175, 185, 249, 287ff.]{dienst:2014}} \\
		\midrule
		\textsc{appl} & \example{maiza} & ‘to lie’ & ↔ & \example{\textbf{ka}-maiza} & ‘to cheat/lie to sb.’ \\
		\textsc{appl} & \example{kha} & ‘to go’ & ↔ & \example{\textbf{ka}-kha} & ‘to bring/go with sb.’ \\
		\textsc{recp} & \example{bishi na} & ‘to pinch sb.’ & ↔ & \example{bishi \textbf{ka}-na-\textbf{ra}} & ‘to pinch e.o.’ \\
		\textsc{recp} & \example{ida} & ‘to beat sb.’ & ↔ & \example{\textbf{ka}-k-ida-\textbf{ra}} & ‘to beat e.o.’ \\
		\midrule\midrule
		\multicolumn{6}{l}{\ili{Madurese} \citep[104, 168, 252, 279, 425f.]{davies:2010}} \\
		\midrule
		\textsc{appl} & \example{gaggar} & ‘to fall’ & ↔ & \example{\textbf{ka}-gaggar-\textbf{an}} & ‘to fall to the \\
		& & & & & \multicolumn{1}{r}{detriment of sb.’} \\
		\textsc{appl} & \example{robbu} & ‘to collapse’ & ↔ & \example{\textbf{ka}-robbu-\textbf{wan}} & ‘to collapse to the \\
		& & & & & \multicolumn{1}{r}{detriment of sb.’} \\
		\textsc{recp} & \example{pokol} & ‘to hit sb.’ & ↔ & \example{\textbf{kol}\~{}pokol-\textbf{an}} & ‘to hit e.o.’ \\
		\textsc{recp} & \example{kerem} & ‘to send sb. sth.’ & ↔ & \example{\textbf{rem}\~{}kerem-\textbf{an}} & ‘to send e.o. sth.’ \\
		\midrule\midrule
		\multicolumn{6}{l}{\ili{Khimt’anga} \citep[162, 168, 231f., 235ff.]{belay:2015}} \\
		\midrule
		\textsc{appl} & \example{dʒɨβ-} & ‘to buy sth.’ & ↔ & \example{\textbf{dʒɨβ-ə}-dʒɨβ-\textbf{ɨs}-} & ‘to buy sth. with \\
		& & & & & \multicolumn{1}{r}{the support of sb.’} \\ 
		\textsc{appl} & \example{kil-} & ‘to break sth.’ & ↔ & \example{\textbf{kil-ə}-kil-\textbf{s}-} & ‘to break sth. with \\
		& & & & & \multicolumn{1}{r}{the support of sb.’} \\ 
		\textsc{recp} & \example{kil-} & ‘to break sth.’ & ↔ & \example{\textbf{kil-ə}-kil-\textbf{ʃit}-} & ‘to break e.o.’ \\
		\textsc{recp} & \example{qal-} & ‘to see sth.’ & ↔ & \example{\textbf{qal-ə}-qal-\textbf{ʃɨt}-} & ‘to see e.o.’ \\
		\lspbottomrule
	\end{tabularx}
	\caption{Applicative-reciprocal type 2 syncretism}
	\label{tab:ch4:appl-recp-type2}
\end{table}

In \ili{Kulina} the prefix \example{ka-} serves as voice marking in both the applicative\is{applicative voice} and reciprocal\is{reciprocal voice} voices, in the latter in combination with the suffix \example{-ra} forming a circumfix. The element \example{-k-} in the second reciprocal\is{reciprocal voice} example is simply epenthetic. In \ili{Kulina} “non-inflecting verbs are followed by an auxiliary, which takes the inflectional affixes” \citep[7]{dienst:2014}. Accordingly, the voice marking \example{ka-…-ra} is found on the auxiliary verb \example{na} (lit. ‘to say’) in the first reciprocal\is{reciprocal voice} example (cf. applicative\is{applicative voice} \example{haha ka-na} ‘to laugh at sb.’, \citealt[103]{dienst:2014}). Furthermore, note that Kulina makes a distinction between dual and plural reciprocals\is{reciprocal voice} \citep[129ff.]{dienst:2014}: the voice marking in \tabref{tab:ch4:appl-recp-type2} represents dual reciprocity\is{reciprocal voice}, while plural reciprocity\is{reciprocal voice} is expressed by the prefix \example{ka-} accompanied by full \isi{reduplication} (e.g. \example{bishi\~{}bishi ka-na} ‘to pinch e.o.’). Finally, observe that the verb \example{bishi na} appears as \example{bishi ta-[…]} in the original source \citep[78]{dienst:2014}. The form \example{ta-} results from the fusion of a third person marker \example{to-} and the auxiliary verb \example{na} (\example{*to-na-} > \example{ta-}, \citealt[141]{dienst:2014}). Next, in \ili{Madurese} the suffix \example{-an} (or \example{-wan/-yan} due to glide epenthesis, \citealt[41f.]{davies:2010}) serves as voice marking in both the applicative\is{applicative voice} and reciprocal\is{reciprocal voice} voices. The suffix is accompanied by the prefix \example{ka-} in the former voice and by partial \isi{reduplication} in the latter voice. Finally, in \ili{Khimt’anga} full \isi{reduplication} forms part of the voice marking in both the applicative\is{applicative voice} and reciprocal\is{reciprocal voice} voices, in the former accompanied by the suffix \example{-(ɨ)s} and in the latter by the suffix \example{-ʃit/-ʃɨt}. The schwa in the reduplicated forms is simply a “linking vowel” \citep[xxi]{belay:2015}. Observe that the former suffix in \ili{Khimt’anga} also serves as voice marking in the causative\is{causative voice} voice (see \tabref{tab:ch4:caus-appl-khimtanga} on page \pageref{tab:ch4:caus-appl-khimtanga}) and the latter suffix as voice marking in the passive\is{passive voice} voice (e.g. \example{χʷ-} ‘to eat sth.’ ↔ \example{χʷ-ɨʃit-} ‘to be eaten [by sb.]’, \citealt[235]{belay:2015}).

\subsection{Applicative-anticausative} \label{sec:simple-syncretism:appl-antc}
Applicative-anticausative syncretism appears to be the rarest of the 21 patterns of voice syncretism covered in this chapter and is not attested in a single language in the language sample. However, it can be mentioned here that there is potentially a vague hint of applicative-anticausative type 2 syncretism\is{voice syncretism, partial resemblance -- type 2} in the Northern Chukotko-Kamchatkan language \ili{Chukchi} (\lang{ea}), but it cannot be regarded as productive.\is{productivity} The language in question has a “verb deriver”\is{derivation} \example{-et/-at} conditioned by \isi{vowel harmony} which performs “a range of generally unpredictable morphological functions, including \isi{derivation} of verbs from other word classes, acting as thematic suffixes with other derivational\is{derivation} prefixes, and marking certain forms as having unpredictable semantic or syntactic features” \citep[48, 243]{dunn:1999}. The suffix in question can, for instance, serve as voice marking in the anticausative\is{anticausative voice} voice together with three verbs (\citealt[187]{kurebito:2012}; \citealt[256]{dunn:1999}), and apparently also in the applicative\is{applicative voice} voice with a single verb, \example{wetɣaw-} ‘to speak’. In the latter case, the suffix is accompanied by the prefix \example{r-/n-} \citep[199, 213]{dunn:1999}. Examples are provided in \tabref{tab:ch4:appl-antc}.

\begin{table}
	\begin{tabularx}{.98\textwidth}{llllll}
		\lsptoprule
		\multicolumn{6}{l}{\ili{Chukchi} (\citealt[256]{dunn:1999}; \citealt[187]{kurebito:2012})} \\
		\midrule
		\textsuperscript{?}\textsc{appl} & \example{wetɣaw} & ‘to speak’ & ↔ & \example{\textbf{r-/n}-wetɣa-\textbf{at}} & ‘to speak to sb.’ \\
		\textsc{antc} & \example{ejp-} & ‘to close sth.’ & ↔ & \example{ejp-\textbf{et}} & ‘to close’ \\
		\textsc{antc} & \example{tejwŋ-} & ‘to divide sth.’ & ↔ & \example{tejwŋ-\textbf{et}} & ‘to divide’ \\
		\textsc{antc} & \example{pela-} & ‘to leave sth.’ & ↔ & \example{pela-\textbf{(e)t}} & ‘to remain’ \\
		\lspbottomrule
	\end{tabularx}
	\caption{Hints of applicative-anticausative syncretism in Chukchi}
	\label{tab:ch4:appl-antc}
\end{table}

\cite[223]{mcgill:2009} claims that in the Kainji language \ili{Cicipu} (\lang{af}) “[t]he anticausative\is{anticausative voice} suffix \example{-wA} is formally identical to the applicative\is{applicative voice} suffix”. Consider, for instance, the examples \example{dúkwà} ‘to go’ ↔ \example{dúkwà-wà} ‘to go with sb.’ and \example{síɗù} ‘to heat sth.’ ↔ \example{síɗù-wà} \textsuperscript{?}‘to spoil,’ i.e. ‘to get hot’ \citep[134, 142, 223f.]{mcgill:2009}. However, \cite[224]{mcgill:2009} also argues that “[t]he function of the anticausative\is{anticausative voice} is to downplay the role of the \isi{agent}/\isi{causer} in the event denoted by the verb, so much so that it cannot be expressed at all”. This description suggests that there is an \isi{agent} (although it is “downplayed” and cannot be expressed syntactically) in the purported anticausative\is{anticausative voice}, in which case the voice is probably better treated as absolute passive\is{passive voice}. Note, for instance, that \citeauthor{mcgill:2009} translates the verb \example{síɗù-wà} elsewhere ‘[the water] got heated’. The limited data provided by \citeauthor{mcgill:2009} do not shed further light upon the matter, and for the time being it remains inconclusive whether or not \ili{Cicipu} features applicative-anticausative syncretism. 

\section{Overview} \label{sec:simple-syncretism:overview}
As demonstrated in this chapter, nineteen of the 21 patterns of voice syncretism listed at the beginning of this chapter (see \tabref{tab:ch4:simplex-patterns} on page \pageref{tab:ch4:simplex-patterns}) have been attested as type 1 syncretism\is{voice syncretism, full resemblance -- type 1} in the language sample of this book. The remaining two patterns are applicative-reflexive\is{reflexive voice} syncretism and applicative-anticausative syncretism, the former pattern of which is attested as type 2 syncretism\is{voice syncretism, partial resemblance -- type 2} in the language isolate \ili{Kutenai} (\lang{na}) while the latter pattern remains unattested altogether. This is not particularly surprising considering the seemingly disparate functions of the voices involved in the syncretism. For instance, in the latter case the anticausative\is{anticausative voice} voice is generally associated with a reduction in semantic participants\is{semantic participant}, while the applicative\is{applicative voice} voice is associated with an increase. Likewise, it is difficult to conceive a hypothetical context in which applicative-reflexive syncretism would arise. As noted in \sectref{sec:simple-syncretism:appl-refl}, it has not been possible to resolve the diachrony of the syncretism in \ili{Kutenai}. Furthermore, this chapter has shown that some patterns of voice syncretism are more prone to form part of complex voice syncretism\is{voice syncretism, complex} than others. For instance, while many languages discussed in this chapter feature voice marking restricted exclusively to two voices associated with \isi{middle syncretism} (e.g. \ili{Baraïn} passive-reciprocal \example{-ɟó}), most voice marking found in patterns of antipassive\is{antipassive voice} syncretism are syncretic with marking in other voices as well (cf. \ili{Ese Ejja} antipassive-reflexive-reciprocal-anticausative \example{xa-…-ki}). The wider syncretic scope of marking of the latter kind is discussed in more detail in terms of complex voice syncretism\is{voice syncretism, complex} in the next chapter. Likewise, the distribution and frequency of voice syncretism as well as its diachrony have only been addressed briefly in this chapter, and are treated more comprehensively in Chapters \ref{sect:distribution} and \ref{sec:diachrony}, respectively.

An overview of the 19 patterns of type 1 voice syncretism\is{voice syncretism, full resemblance -- type 1} covered in this chapter are provided in \tabref{tab:ch4:overview} for easy reference. The table does not include the applicative-reflexive syncretism only attested as type 2 syncretism\is{voice syncretism, partial resemblance -- type 2} nor the unattested applicative-anticausative syncretism mentioned in the beginning of this section. The examples in the table are listed in the same order as they have been discussed in the previous sections, and page numbers provide references to more information about them. However, note that the \ili{Chukchi} example \example{lʔu-tku-} comes from \tabref{tab:ch5:antp-refl-recp-antc} on page \pageref{tab:ch5:antp-refl-recp-antc} and the \ili{Wolaytta} example \example{meeC-ett-} from \tabref{tab:ch5:caus-pass-refl-recp} on page \pageref{tab:ch5:caus-pass-refl-recp} in the next chapter.

\begin{sidewaystable}
	\setlength{\tabcolsep}{3.0pt}
	\begin{tabularx}{\textwidth}{llllllll}
		\lsptoprule
		\ili{Arapaho} & \textsc{refl-recp} & \example{eeneti3-\textbf{eti}-} & ‘to speak to self’ & ↔ & \example{eeneti3-\textbf{eti}-} & ‘to speak to e.o.’ & (p. \pageref{ex:Arapaho:speak:a}) \\
		\ili{Yeri} & \textsc{refl-antc} & \example{\textbf{d}-altou} & ‘to cover self’ & ↔ & \example{\textbf{d}-awɨl} & ‘to hang’ & (p. \pageref{ex:Yeri:hang:a}) \\
		\ili{Chukchi} & \textsc{recp-antc} & \example{lʔu-\textbf{tku}-} & ‘to see e.o.’ & ↔ & \example{ejpə-\textbf{tku}-} & ‘to close’ & (p. \pageref{ex:Chukchi:hug:a}) \\
		\ili{Kayardild} & \textsc{pass-refl} & \example{bala-\textbf{a}-} & ‘to be hit [by sb.]’ & ↔ & \example{bala-\textbf{a}-} & ‘to hit self’ & (p. \pageref{ex:Kayardild:hit:a}) \\
		\ili{Baraïn} & \textsc{pass-recp} & \example{ɲárō-\textbf{ɟó}} & ‘to be looked for [by sb.]’ & ↔ & \example{ɲárō-\textbf{ɟó}} & ‘to look for e.o.’ & (p. \pageref{ex:barain:look:a}) \\
		\ili{Dhimal} & \textsc{pass-antc} & \example{cuma-\textbf{nha}-} & ‘to be taken [by sb.]’ & ↔ & \example{oŋ-\textbf{nha}-} & ‘to cook’ & (p. \pageref{ex:Dhimal:take:a}) \\
		\midrule
		\ili{Ese Ejja} & \textsc{antp-refl} & \example{\textbf{xa}-ishwa-\textbf{ki}-} & ‘to wait for [sth.]’ & ↔ & \example{\textbf{xa}-jabe-\textbf{ki}-} & ‘to comb self’ & (p. \pageref{ex:EseEjja:wait:a}) \\
		\ili{Tatar} & \textsc{antp-recp} & \example{jaz-\textbf{əš}-} & ‘to write [sth.]’ & ↔ & \example{sug-\textbf{əš}-} & ‘to hit e.o.’ & (p. \pageref{tab:ch4:antp-recp}) \\
		\ili{Majang} & \textsc{antp-antc} & \example{káw-\textbf{ɗìː}} & ‘to bite [sb.]’ & ↔ & \example{ŋùːl-\textbf{ɗìː}} & ‘to break’ & (p. \pageref{ex:Majang:bite:a}) \\
		\ili{Mosetén} & \textsc{pass-antp} & \example{raem’ya-\textbf{ki}-} & ‘to be bitten [by sb.]’ & ↔ & \example{raem’ya-\textbf{ki}-} & ‘to bite [sb.]’ & (p. \pageref{ex:Moseten:bite:a}) \\
		\midrule
		\ili{Ternate} & \textsc{caus-appl} & \example{\textbf{si}-hotu} & ‘to make sb. sleep’ & ↔ & \example{\textbf{si}-hoi} & ‘to open sth. for sb.’ & (p. \pageref{ex:Ternate:sleep:a}) \\
		\ili{Kilen} & \textsc{caus-pass} & \example{tanta-\textbf{wu}} & ‘to make sb. hit sb.’ & ↔ & \example{tanta-\textbf{wu}} & ‘to be hit [by sb.]’ & (p. \pageref{tab:ch4:caus-pass}) \\
		Balanta\il{Balanta, Ganja} & \textsc{caus-antp} & \example{sιιg-\textbf{t}-} & ‘to make sb. drink sth.’ & ↔ & \example{lɔt-\textbf{t}-} & ‘to cook [sth.]’ & (p. \pageref{tab:ch4:caus-antp-ganja}) \\
		\ili{Wolaytta} & \textsc{caus-refl} & \example{maLL-\textbf{ett}-} & ‘to make sth. tasty’ & ↔ & \example{meeC-\textbf{ett}-} & ‘to wash self’ & (p. \pageref{ex:Wolaytta:tasty:a}) \\
		\ili{Yine} & \textsc{caus-recp} & \example{hasɨka-\textbf{kaka}-} & ‘to make sb. run’ & ↔ & \example{hiylaka-\textbf{kaka}-} & ‘to hit e.o.’ & (p. \pageref{ex:Yine:run:a}) \\
		\ili{Korean} & \textsc{caus-antc} & \example{cwul-\textbf{li}-} & ‘to reduce sth.’ & ↔ & \example{yel-\textbf{li}-} & ‘to open’ & (p. \pageref{tab:ch4:caus-antc}) \\
		\midrule
		\ili{Irabu} & \textsc{appl-pass} & \example{tur-\textbf{ai}-} & ‘to take sth. affecting sb.’ & ↔ & \example{žž-\textbf{ai}-} & ‘to be scolded [by sb.]’ & (p. \pageref{ex:Irabu:take:a}) \\
		Yupik\il{Yupik, Central Alaskan} & \textsc{appl-antp} & \example{tuqu-\textbf{i}-} & ‘to die affecting sb.’ & ↔ & \example{tamar-\textbf{i}-} & ‘to lose [sb.]’ & (p. \pageref{ex:CAY:die:a}) \\
		\ili{Yuchi} & \textsc{appl-recp} & \example{\textbf{k’a}-gõ} & ‘to come with sb.’ & ↔ & \example{\textbf{k’a}-’yuhõ} & ‘to embrace e.o.’ & (p. \pageref{tab:ch4:appl-recp:yuchi}) \\
		\lspbottomrule
	\end{tabularx}
	\caption{Overview of minimal simplex voice syncretism}
	\label{tab:ch4:overview}
\end{sidewaystable}