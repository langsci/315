\chapter{Defining voice syncretism} \label{voice-syncretism}
As explained in Chapter \ref{introduction}, \textsc{voice syncretism} refers to formal verbal marking shared by two or more of the seven voices of focus in this book (i.e. passive\is{passive voice}, antipassive\is{antipassive voice}, reflexive\is{reflexive voice}, reciprocal\is{reciprocal voice}, anticausative\is{anticausative voice}, causative\is{causative voice}, applicative\is{applicative voice}). By contrast, \textsc{dedicated voice marking}\is{dedicated voice marking} can be defined as formal verbal marking restricted to a single voice (cf. \citealt[233]{zuniga:kittila:2019}). Voice syncretism is the primary focus of this and subsequent chapters, while a discussion of \isi{dedicated voice marking} is restricted mainly to \sectref{dist:dedicated}. This chapter more precisely provides an overview of previous typological research on voice syncretism (\sectref{previous-research}) with special attention to \citeauthor{geniusiene:1987}’s (\citeyear{geniusiene:1987}) study of reflexive\is{reflexive voice} syncretism and \citeauthor{haspelmath:1990}’s (\citeyear{haspelmath:1990}) study of passive syncretism. The chapter also establishes three main types of voice syncretism based on resemblance in voice marking (\sectref{resemblance}). All patterns of voice syncretism discussed in subsequent chapters belong to one of these types.

\section{Previous research} \label{previous-research}
Two main approaches to the study of voice syncretism can be discerned in the literature. One approach has traditionally been closely associated with the infamous \isi{middle voice} and entails a semantic core meaning (often characterised as a \isi{subject}’s \isi{affectedness}; cf. \citealt{klaiman:1991}) as point of reference in investigations of voices and their syncretism. The scope of this approach is accordingly restricted to voices complying with the semantic core meaning, typically considered to include passives\is{passive voice}, reflexives\is{reflexive voice}, reciprocals\is{reciprocal voice}, and anticausatives\is{anticausative voice} (in addition to various other semantic functions not of primary interest to this book). By contrast, the other approach has formal marking as its point of reference and its semantic and functional scope is therefore largely unrestricted. Consequently, the latter approach is considerably more explicit in relation to voice syncretism (as formal marking is investigated with regard to semantics) than the former approach (in which semantics is examined with regard to formal marking).

\subsection{Middle voice and semantics} \label{middle-voice}
The conceptualisation of a \isi{middle voice} in linguistics can be traced to the grammatical traditions pertaining to Classical \ili{Greek} (cf. \example{mesótēs} or \example{mésē diathesis} ‘middle diathesis’) and \ili{Sanskrit} (cf. \example{ātmanepada} ‘word for oneself’) though discussions of the phenomenon in a broader theoretical perspective are of more recent date \citep[168]{zuniga:kittila:2019}. \citeauthor{zuniga:kittila:2019} cite early characterisations of the \isi{middle voice} by \citet{kruger:1846} and \citet{kurylowicz:1964}, but note that \citet{lyons:1968} “is generally credited with reinterpreting the original idea of an ‘action performed with special reference to the \isi{subject}’ for \ili{English} phenomena” \citep[172]{zuniga:kittila:2019}. In his classic \textit{Introduction to Theoretical Linguistics}, \citeauthor{lyons:1968} describes the \isi{middle voice} in the following manner:

\begin{quote}
	As the term suggests, the \textit{middle}\is{middle voice} was thought of as intermediate between the primary opposition of active and passive (signifying either an ‘action’, like the active, or a ‘state’, like the passive, according to the circumstances or the inherent meaning of the verb in question). [...] The implications of the middle\is{middle voice} (when it is in opposition with the active) are that the ‘action’ or ‘state’ affects\is{affectedness} the \isi{subject} of the verb or his interests. (\citealt[373]{lyons:1968})
\end{quote} 

\citet[18f.]{barber:1975} further elaborates that “the \isi{middle voice} is expressing the fact that the \isi{subject} is not only performing the action, as \isi{agent}, but receiving some benefit from it as well”. For example, the Classical \ili{Greek} \isi{middle voice} can be used to express meanings such as \isi{autobenefactive} (e.g. ‘to take sth. for self’), reflexive\is{reflexive voice} (e.g. ‘to wash self’), and reciprocal\is{reciprocal voice} (e.g. ‘to crown e.o.’), inter alia. At the time of writing, \cite[17]{barber:1975} argued that the “linguistic literature on the \isi{middle voice} is almost nonexistent”, yet it can be noted that the label “\isi{middle voice}” has been applied in descriptive studies of non-Indo-European languages since at least the 1950s (e.g. \citealt{arnott:1956} on the Atlantic language \ili{Fula}, \citealt{chafe:1960} on the Northern Iroquoian language \ili{Seneca}, and \citealt{wallis:1964} on the Oto-Manguean language Mezquital Otomí\il{Otomí, Mezquital}). The first comprehensive typological investigations of the phenomenon are provided by \citet{klaiman:1982, klaiman:1991} and \citet{kemmer:1993, kemmer:1994} who both argue that \isi{affectedness} of the \isi{subject} or the self lies at the semantic core of the \isi{middle voice}. In the words of \citet[104f.]{klaiman:1991}, “the middle\is{middle voice} implicates the logical \isi{subject}’s \isi{affectedness}” as well as “\isi{detransitivisation} (valence reduction) and reflexivity\is{reflexive voice}”. \citeauthor{klaiman:1991} and \citeauthor{kemmer:1993} thus reiterate \citeauthor{lyons:1968}’ characterisation of the \isi{middle voice} quoted above:

\begin{quote}
	[...] there is a coherent, although complex, linguistic category subsuming many of the phenomena discussed under the name of middle\is{middle voice} [...] and this category receives grammatical instantiation in many languages. The category of the middle,\is{middle voice} although without fixed and precise boundaries, nevertheless has a clearly discernible semantic core that fits the traditional characterization of the \isi{middle voice} [by] \citeauthor{lyons:1968}. \citep[3]{kemmer:1993}
\end{quote} 

Evidently, the \isi{middle voice} has traditionally been regarded as a category loosely defined primarily according to a set of presumably related semantic criteria (e.g. \citealt[238]{kemmer:1993}) and secondarily on similarities in marking (e.g. \citealt[15ff.]{kemmer:1993}). In turn, this category can seemingly manifest itself in different ways in different languages, and neither \citeauthor{klaiman:1991} nor \citeauthor{kemmer:1993} claims that the functional scope of the \isi{middle voice} is necessarily the same in different languages. In fact, as argued by \cite[1149]{shibatani:2004}, “[t]he middle\is{middle voice} (or medial) voice is considered to be the most heterogeneous voice category”. For instance, although the \isi{middle voice} in Classical \ili{Greek} can be used to express reflexivity\is{reflexive voice}, reciprocity\is{reciprocal voice}, passivity\is{passive voice}, and anticausativity\is{anticausative voice}, the same functions “are expressed by distinct constructions such as the spontaneous, the reflexive\is{reflexive voice}, the reciprocal\is{reciprocal voice}, and the passive construction in \ili{English} and other languages” \citep[1157]{shibatani:2004}. While studies within the tradition described here rarely focus explicitly on voice syncretism, they provide valuable implicit insights into such syncretism due to their extensive focus on semantic similarities between the passive, reflexive\is{reflexive voice}, reciprocal\is{reciprocal voice}, and anticausative\is{anticausative voice} voices. Nevertheless, being semantically and syntactically heterogenous and based largely on vaguely defined fuzzy boundaries\is{fuzzy categorisation}, a “\isi{middle voice}” can hardly be defined as a \isi{comparative concept}, and the term is avoided entirely in subsequent chapters. However, due to the prevalent perception of passives\is{passive voice}, reflexives\is{reflexive voice}, reciprocals\is{reciprocal voice}, and anticausatives\is{anticausative voice} being associated with one other in the literature, voice syncretism involving at least two of the four voices will henceforth be referred to as \textsc{middle syncretism}\is{middle syncretism}. As shown in the following chapters, this kind of syncretism is cross-linguistically prevalent, so the grouping of these voices is not unfounded. As discussed in the next section, a similar solution has previously been suggested by \citet{shibatani:2004} in terms of the \isi{middle voice} being a “family of constructions”, and by \citeauthor{kulikov:2010} (\citeyear[394f.]{kulikov:2010};; \citeyear[265ff.]{kulikov:2013}) and \citet[175ff.]{zuniga:kittila:2019} in terms of a “middle cluster”.\is{cluster, middle} 

A few prominent investigations dealing implicitly or explicitly with \isi{middle syncretism} predate \citeauthor{klaiman:1991}’s (\citeyear{klaiman:1991}) and \citeauthor{kemmer:1993}’s (\citeyear{kemmer:1993}) observations on the phenomenon. For instance, in an early pioneering study, \citet[40ff.]{nedjalkov:silnickij:1969} investigate and exemplify various patterns of voice syncretism involving anticausatives\is{anticausative voice}, as further described in the next section. Voice syncretism involving passives\is{passive voice} has been discussed at length by \citet{siewierska:1984}, \citet{shibatani:1985} and \citet{haspelmath:1990}; while voice syncretism involving reflexives\is{reflexive voice} has been examined most notably by \citet{geniusiene:1987}. It is, however, worth noting that voice syncretism is not of primary interest to any of these studies. Nevertheless, the latter two studies are particularly noteworthy for their systematic sample-based approach which makes it possible to extract cross-linguistic data on voice syncretism. In fact, it seems that these two studies still stand as the most comprehensive surveys of voice syncretism despite being published more than three decades ago and not explicitly dedicated to the matter. In this respect, the studies in question differ from other inquiries into voice syncretism which have generally provided more sporadic observations on the phenomenon. For these reasons, \citeauthor{geniusiene:1987}’s (\citeyear{geniusiene:1987}) and \citeauthor{haspelmath:1990}’s (\citeyear{haspelmath:1990}) studies are discussed in more detail in \sectref{geniusiene-syncretism} and \sectref{haspelmath-syncretism}, respectively.

\subsection{Families, clusters, and voice ambivalence} \label{families-clusters}
As mentioned in the previous section, \citet[1147f.]{shibatani:2004} suggests that the \isi{middle voice} and other voices can be perceived as “families of constructions” and argues that “it is the morphological unity [...] that overtly indicates the nature of voice as something comprising of a family of constructions”. Thus, \citeauthor{shibatani:2004} defines unity in terms of similarities in formal marking, not similarities in semantics. \citeauthor{kulikov:2010} (\citeyear[394f.]{kulikov:2010};; \citeyear[265ff.]{kulikov:2013}) and \citet[175ff.]{zuniga:kittila:2019} adopt a similar view but use the term “cluster” instead of “family”. In addition to a “middle cluster”\is{cluster, middle} (or “detransitivizing\is{detransitivisation} cluster”\is{cluster, detransitivising}, \citealt[237]{zuniga:kittila:2019}), both \citet[395]{kulikov:2010} and \citet[234ff.]{zuniga:kittila:2019} also recognise a “\isi{transitive}” or “transitivising\is{transitivisation} cluster”\is{cluster, transitivising} encompassing the causative\is{causative voice} and applicative\is{applicative voice} voices, in other words causative-applicative syncretism. Evidently, the scope of this approach is not restricted by any specific semantic core meaning, and the approach may thus be applied to the study of the seven voices of interest in this book. This kind of systematic approach in which formal marking is considered with regard to its semantics (rather than semantics being considered with regard to its formal marking) can be traced to the Leningrad-St. Petersburg Typology Group established in the early 1960s at the Institute of Linguistics of the USSR Academy of Sciences. The fundamental ideology of the group has been described in the following manner:

\begin{quote}
	[...] meanings of comparable grammatical categories in different languages coincide to a greater or lesser degree. Partial coincidence is characteristic not only of meanings whose relatedness is obvious [...] but also of those meanings that at first glance may appear totally unrelated and occur within the semantic limits of the grammatical form by accident, as is the case with the causative\is{causative voice} and passive meanings in some languages. [...] We have reason to assume that at least for some comparable grammatical categories in different languages there exists a certain limit (or limits) of possible syncretism. [...] According to the range of various meanings expressed by comparable forms in them, individual languages differ from one another and can be subject to classification. (\citealt[301f.]{nedjalkov:1964};; cited via \citealt[xii]{nedjalkov:1988} and \citealt[vii]{comrie:polinsky:1993})
\end{quote} 

Causatives\is{causative voice} in particular were an early subject of interest to the Leningrad-St. Petersburg Typology Group which “first achieved international eminence” \citep[vii]{comrie:polinsky:1993} following the publication of a “typology of causative\is{causative voice} constructions” (\textit{Типология каузативных конструкций}) edited by \citet{xolodovic:1969}. In the publication’s chapter on morphological and lexical causatives\is{causative voice}, \citet[35ff., 40ff.]{nedjalkov:silnickij:1969} explicitly discuss syncretism of causatives\is{causative voice} and anticausatives\is{anticausative voice} (for an English translation of the chapter, see \citealt{nedyalkov:silnitsky:1973}). \citeauthor{nedjalkov:silnickij:1969} mention causative-applicative, causative-reciprocal, causative-passive, passive-anticausative, reflexive-anticausative, and reciprocal-anticausative syncretism. More recent prominent studies associated within the same tradition have been published by \citet{kulikov:nedjalkov:1992} who provide a “questionnaire for \isi{causativisation}” (\textit{Questionnaire zur Kausativierung}) in which the same patterns of voice syncretism observed by \citet{nedjalkov:silnickij:1969} are reiterated; by \citet{kazenin:1994, kazenin:2001a} and \citet{kulikov:2001} who both examine various patterns of voice syncretism, albeit rather briefly; and by \citet{nedjalkov:2007d} who has provided the most comprehensive account of syncretism focused on a specific voice to date, namely the reciprocal\is{reciprocal voice} voice. As already mentioned in Chapter \ref{introduction}, \citeauthor{nedjalkov:2007d} is also notable for explicitly acknowledging different degrees of resemblance in voice marking, a topic described in more detail in the following sections. Nevertheless, despite six decades of research on voice syncretism, \citet{malchukov:2015, malchukov:2016, malchukov:2017} has argued that voice syncretism may still be more widespread than generally acknowledged and that its typology has not yet been fully explored:

\begin{quote}
	One aspect of this topic that has not been sufficiently acknowledged so far is the pervasiveness of “ambivalence”\is{ambivalent voice} of voice categories, the fact that a certain voice marker (or, more broadly, a \isi{valency}-changing marker) performs different functions when applied to different \isi{valency} classes of verbs (in the first place to intransitives\is{intransitive} and transitives\is{transitive}). Admittedly, there have been occasional observations made about such polysemies\is{polysemy} in the literature on individual \isi{valency} categories […], but with a few exceptions [...] no extensive typological studies have been undertaken so a general picture is still lacking”. (\citealt[259]{malchukov:2016}; see also \citeyear[103]{malchukov:2015} and \citeyear[3]{malchukov:2017})
\end{quote} 


The terms “\isi{ambivalent voice}” and “voice ambivalence”\is{ambivalent voice} coined by \citeauthor{malchukov:2015} denote voice syncretism. \citeauthor{malchukov:2015} (\citeyear[123]{malchukov:2015}; \citeyear[414]{malchukov:2016}; \citeyear[24]{malchukov:2017}) notably goes on to design a semantic map of “voice categories capturing selective similarities between individual categories” which can be used to explain various patterns of voice syncretism involving the causative\is{causative voice}, applicative\is{applicative voice}, passive,\is{passive voice} and antipassive\is{antipassive voice} voices. This map is reproduced and discussed in \sectref{diachrony:overview}.

\subsection{Geniušienė (1987) on reflexive syncretism} \label{geniusiene-syncretism}
\citeauthor{geniusiene:1987}’s (\citeyear{geniusiene:1987}) widely cited typology of reflexives\is{reflexive voice} is notable for its systematic sample-based approach which makes it possible to extract cross-linguistic data on voice syncretism, although the syncretism in question is not of primary interest to her study. \citeauthor{geniusiene:1987}’s typology is based on a cross-linguistic survey of 50 languages: 25 Indo-European languages and 25 non-Indo-European languages which belong to seven and fifteen WALS genera\is{genus}, respectively. \cite[57, 220ff.]{geniusiene:1987} investigates these languages with regard to fifteen “derived R[eflexive] V[erb] diatheses”\is{derivation}\is{diathesis} (which she also calls “recessive diatheses”\is{diathesis}), all of which she gives a unique identifier in the form of a delta followed by a subscript numeral (Δ\textsubscript{i}). Of relevance to this book are the following seven “derived RV diatheses”\is{derivation}\is{diathesis} \citep[230]{geniusiene:1987}: “semantic reflexives\is{reflexive voice}” (Δ\textsubscript{1}), “absolute RVs” (Δ\textsubscript{2}), “reciprocal\is{reciprocal voice} RVs” (Δ\textsubscript{4}), “decausatives\is{decausative} RVs” (Δ\textsubscript{7}), and “reflexive passives\is{passive voice}” (Δ\textsubscript{9}) in addition to “\isi{autocausative} RVs” (Δ\textsubscript{3}) and “converse RVs” (Δ\textsubscript{11}). The first five diatheses\is{diathesis} roughly correspond to the reflexive\is{reflexive voice}, antipassive\is{antipassive voice}, reciprocal\is{reciprocal voice}, anticausative\is{anticausative voice}, and passive voices in this book, respectively. “Autocausative\is{autocausative} RVs” are also treated as anticausatives\is{anticausative voice} here, because this phenomenon appears to involve two voices which differ primarily in terms of a \isi{causer} (\sectref{def:causatives-anticausatives}), e.g. \ili{Estonian} (\lang{ea}) \example{lask-} ‘to put sth. down’ ↔ \example{lask-u-} ‘to go down’ \citep[316]{geniusiene:1987}. The same is true for “conversive RVs”, e.g. \ili{Swedish} (\lang{ea}) \example{vulkanen utspyr asken} ‘the volcano erupts the ashes’ and \example{asken utspy-s ut vulkanen} ‘the ashes erupt from the volcano’ (\citealt[273]{geniusiene:1987}; the English translations are slightly modified here). \cite[228]{geniusiene:1987} argues that the \isi{agent} in the former clause in paired examples of this kind “changes into some other semantic role” in the latter clause, and the \isi{voice relation} can hardly be considered passive. On the contrary, the former clause differs from the latter in having a \isi{causer}, and the \isi{voice relation} is thus treated as anticausative\is{anticausative voice} (‘to erupt sth.’ ↔ ‘to erupt’).

\begin{table}
	\begin{tabularx}{.90\textwidth}{lcccccccc}
		\lsptoprule
		& \multirow{2}{*}{Marking} & \textsc{refl} & \textsc{recp} & \multicolumn{3}{c}{\textsc{antc}} & \textsc{pass} & \textsc{antp} \\
		& & Δ\textsubscript{1} & Δ\textsubscript{4} & Δ\textsubscript{3} & Δ\textsubscript{7} & Δ\textsubscript{11} & Δ\textsubscript{9} & Δ\textsubscript{2} \\
		\midrule
		\ili{Swedish} 		& \example{-s} 		   & 	  & +   & + & +   & +   & + & +   \\
		\ili{Russian} 	& \example{-sja} 	   & +    & +   & + & +   & +   & + & +   \\
		\ili{Lithuanian} 	& \example{-s, -si-}   & +    & +   & + & +   & +   &   & +   \\
		Armenian\il{Armenian, Eastern} 	& \example{-v} 		   & +    & +   & + & +   & +   & + &     \\
		\ili{Greek} 		& * 				   & (+)  & +   & + & +   & (+) & + &     \\
		\ili{Latin}		& * 				   & (+)  & +   & + & +   &     & + &     \\
		\ili{Sanskrit} 	& * 				   & (--) & +   & + & +   &     & + &     \\
		\midrule
		\ili{Udmurt}		& \example{-śk} 	    & +    & +   & + & +   & +   & + & +   \\
		\ili{Hungarian} 	& \example{-d, -z}     & +    & +   & + & +   & +   & + & +   \\
		\ili{Veps}	 	& \example{-s} 		   & +    & +   & + & +   & (+) &   & +   \\
		Mordvin 	& \example{-v} 		   & (+)  &     & + & +   & +   & + &     \\
		\ili{Selkup}	 	& \example{-(c)y, -ī˱} & +    &     & + & +   & (+) &   &     \\
		\midrule
		\ili{Amharic} 	& \example{tə-} 	   & +    & +   & + & +   &     & + &     \\
		\ili{Shoshoni} 	& \example{na-, nɨɨ-}  & +    & +   & + & +   &     & + & (+) \\
		\ili{Georgian} 	& \example{i-} 		   & +    & (--) & + & +   & +   & + & +   \\
		\ili{Uzbek}	 	& \example{-n, -l} 	   & +    &     & + & +   & +   & + & (+) \\
		\ili{Fula}	 	& \example{-ii, -ike}  & +    &     & + & +   &     &   & +   \\
		\ili{Nivkh}	 	& \example{p‘-} 	   & +    &     & + & (+) & (--) &   &     \\
		\ili{Khmer}		& \example{rə-} 	   & (--)  & (+) & + & +   &     &   &     \\
		\ili{Aymara}		& \example{-si} 	   & +    & +   &   &     &     &   &     \\
		\lspbottomrule
	\end{tabularx}
	\caption{\citeauthor{geniusiene:1987}’s (\citeyear{geniusiene:1987}) survey of reflexive syncretism}
	\label{tab:ch3:geniusiene}
\end{table} 

The findings of \citeauthor{geniusiene:1987}’s (\citeyear[244, 258, 308, 320]{geniusiene:1987}) survey of the seven “derived RV diatheses”\is{derivation}\is{diathesis} are summarised in \tabref{tab:ch3:geniusiene}. The table only includes a subset of twenty languages, each representing a unique \isi{genus} and one or more voices featuring formal verbal marking. \citeauthor{geniusiene:1987} also discusses languages with various periphrastic constructions (e.g. English and the Oto-Manguean language Yatzachi Zapotec\il{Zapotec, Yatzachi}) which do not comply with any of the voice definitions in this book, and so these languages are excluded from the table. Change in verbal conjugation paradigm according to \isi{agreement} in the Chaplin dialect of Siberian Eskimo\il{Eskimo, Siberian} is not considered voice marking either (\sectref{def:principles}). A hyphen within parentheses (--) in \tabref{tab:ch3:geniusiene} denotes a “possible absence”, a plus sign within parentheses (+) denotes “a highly restricted class” \citep[353]{geniusiene:1987}, and an asterisk (*) denotes paradigmatic voice marking (i.e. fusion of voice marking and \isi{agreement}). “Inconclusive information” marked by a question mark in the original source is not included in the table. The first group of languages in the table represents Indo-European genera\is{genus}, the second group of languages Uralic genera\is{genus}, and the third group various unrelated genera\is{genus}. Note that \citeauthor{geniusiene:1987} treats the Finno-Ugric languages \ili{Erzya} and \ili{Moksha} collectively as “Mordvin”.

Observe that \citet{geniusiene:1987} includes more than one voice marker for some languages and does not make a clear distinction between them and their functions. For instance, \cite[305]{geniusiene:1987} remarks that “suffixes containing \example{-d-} or \example{-z-} are used in \ili{Hungarian}”, probably referring to suffixes like \example{-od}, \example{-oz}, \example{-kod}, and \example{-koz} (each with several allomorphs),\is{allomorphy} yet she does not differentiate them nor their specific uses. Thus, \tabref{tab:ch3:geniusiene} only gives an approximate idea of the extent of voice syncretism in the various languages, and no attempt has here been made to alter \citeauthor{geniusiene:1987}’s analysis of the languages. However, it can be mentioned that her analysis of languages also found in the language sample of this book (i.e. the Indo-European language Eastern Armenian\il{Armenian, Eastern}, the Permic language \ili{Udmurt}, and the language isolate \ili{Nivkh}; all \lang{ea}) does reflect the analysis of this book. By contrast, no passive-antipassive-reflexive-reciprocal-anticausative syncretism is recognised for the Ugric language \ili{Hungarian} (\lang{ea}) nor for the Uto-Aztecan language \ili{Shoshoni} (\lang{na}). In the former language the suffixes \example{-kod} and \example{-koz} are associated with antipassivity\is{antipassive voice}, reflexivity\is{reflexive voice}, and reciprocity\is{reciprocal voice}; whereas the suffixes \example{-od} and \example{-oz} are associated with anticausativity\is{anticausative voice} and \isi{resultative} state, but not passivity\is{passive voice} (for an overview of these and related markers as well as their various functions, see \citealt{karoly:1982}). \citet[306]{geniusiene:1987} only addresses \ili{Shoshoni} very briefly, simply mentioning the prefixes \example{na-} and \example{nɨɨ-}. Cognates of these prefixes are widely associated with passivity\is{passive voice}, reflexivity\is{reflexive voice}, reciprocity\is{reciprocal voice}, and/or anticausativity\is{anticausative voice} among the Numic languages (see, e.g., \citealt[118ff.]{crum:dayley:1993} on Western Shoshoni; \citealt[125ff.]{charney:1993} on Comanche; \citealt[104ff.]{dayley:1989} on \ili{Panamint}; \citealt[108ff.]{sapir:1930} on Southern Paiute\il{Paiute, Southern}; \citealt[373ff.]{thornes:2003} on Northern Paiute\il{Paiute, Northern}), but not antipassivity\is{antipassive voice}. In Numic languages antipassivity\is{antipassive voice} is more commonly associated with cognates of the prefix \example{tɨ-} (see, e.g., \citealt[122f.]{crum:dayley:1993} on Western Shoshoni; \citealt[128f.]{charney:1993} on Oklahoma Comanche\il{Comanche, Oklahoma}; \citealt[111f.]{dayley:1989} on \ili{Panamint}; \citealt[379ff.]{thornes:2003} on Northern Paiute\il{Paiute, Northern}). 

\tabref{tab:ch3:geniusiene-patterns} provides a statistical overview of the simplex\is{voice syncretism, simplex} and complex patterns of voice syncretism\is{voice syncretism, complex} that can be extracted from \citeauthor{geniusiene:1987}’s (\citeyear{geniusiene:1987}) findings summarised in \tabref{tab:ch3:geniusiene} according to frequency, if “possible absences” (--) of voices are ignored and the voices “of a highly restricted class” (+) are treated on par with other voices. The left-hand side of \tabref{tab:ch3:geniusiene-patterns} shows patterns of minimal\is{voice syncretism, minimal} simplex voice syncretism\is{voice syncretism, simplex}, whereas the right-hand side of the table shows patterns of maximal\is{voice syncretism, maximal} simplex\is{voice syncretism, simplex} and complex voice syncretism\is{voice syncretism, complex}. Thus, for example, the \ili{Aymara} reflexive-reciprocal marker \example{-si} is counted only under “\textsc{refl-recp}” on the left-hand side; while the Mordvin passive-reflexive-anticausative marker \example{-v} is counted under “\textsc{pass-refl-antc}” on the right-hand side and under “\textsc{pass-refl}”, “\textsc{pass-antc}” and “\textsc{refl-antc}” on the left-hand side. The distinction between minimal\is{voice syncretism, minimal} and maximal voice syncretism\is{voice syncretism, maximal} has been explained in Chapter \ref{introduction}.

\begin{table}
	\begin{tabularx}{.92\textwidth}{lrrrlrr}
		\lsptoprule
		\multicolumn{4}{l}{Minimal simplex syncretism} & \multicolumn{3}{l}{Maximal simplex/complex syncretism} \\
		\midrule
		\textsc{refl-antc} & 16 & (32 \%) & & \textsc{pass-antp-refl-recp-antc} & 4 & (8 \%) \\
		\textsc{pass-antc} & 13 & (26 \%) & & \textsc{pass-refl-recp-antc} & 4 & (8 \%) \\
		\textsc{recp-antc} & 13 & (26 \%) & & \textsc{pass-antp-refl-antc} & 2 & (4 \%) \\
		\textsc{refl-recp} & 11 & (22 \%) & & \textsc{antp-refl-recp-antc} &  2 & (4 \%) \\
		\textsc{pass-refl} & 11 & (22 \%) & & \textsc{refl-antc} & 2 & (4 \%) \\
		\textsc{pass-recp} & 10 & (20 \%) & & \textsc{refl-recp} & 1 & (2 \%) \\
		\textsc{antp-antc} & 10 & (20 \%) & & \textsc{recp-antc} & 1 & (2 \%) \\
		\textsc{antp-refl} &  9 & (18 \%) & & \textsc{pass-refl-antc} &  1 & (2 \%) \\
		\textsc{antp-recp} &  7 & (14 \%) & & \textsc{pass-recp-antc} &  1 & (2 \%) \\
		\textsc{pass-antp} &  7 & (14 \%) & & \textsc{antp-refl-antc} &  1 & (2 \%) \\
		& & & & \textsc{pass-antp-recp-antc} & 1 & (2 \%) \\
		\lspbottomrule
	\end{tabularx}
	\caption{Voice syncretism in \citeauthor{geniusiene:1987}’s (\citeyear{geniusiene:1987}) survey (\textit{n} = 50)}
	\label{tab:ch3:geniusiene-patterns}
\end{table}  

\tabref{tab:ch3:geniusiene-patterns} shows that ten patterns of minimal\is{voice syncretism, minimal} simplex voice syncretism\is{voice syncretism, simplex} are attested in \citeauthor{geniusiene:1987}’s (\citeyear{geniusiene:1987}) study, and \isi{middle syncretism} (\sectref{middle-voice}) is generally more prevalent cross-linguistically than syncretism involving the antipassive\is{antipassive voice} voice. This finding is confirmed by this book as well, although the specific frequencies only bear superficial resemblance (compare \tabref{tab:ch6:voice-syncretism-simplex} on page \pageref{tab:ch6:voice-syncretism-simplex}). Most notably, the frequencies attested in \citeauthor{geniusiene:1987}’s study are greatly inflated compared to those attested in this book. Such discrepancies can be explained by the smaller size of its language sample and its inclusion of several related languages (albeit of different genera\is{genus}) with rather similar patterns of voice syncretism, notably Indo-European and Uralic languages \citep[128f.]{geniusiene:1987}. In terms of maximal voice syncretism\is{voice syncretism, maximal}, all complex patterns\is{voice syncretism, complex} appear to be at least as common as simplex patterns\is{voice syncretism, simplex} in \citeauthor{geniusiene:1987}’s study. Indeed, only four of the twenty languages listed in \tabref{tab:ch3:geniusiene-patterns} feature maximal\is{voice syncretism, maximal} simplex voice syncretism\is{voice syncretism, simplex}, while the remaining languages feature maximal\is{voice syncretism, maximal} complex voice syncretism\is{voice syncretism, complex}. By contrast, in this book patterns of maximal\is{voice syncretism, maximal} simplex syncretism\is{voice syncretism, simplex} have been found to be considerably more prevalent cross-linguistically than suggested by the findings extracted from \citeauthor{geniusiene:1987}’s study.

\subsection{Haspelmath (1990) on passive syncretism} \label{haspelmath-syncretism}
In his study on “the grammaticization of passive morphology”, \cite[36]{haspelmath:1990} provides a survey of “[o]ther uses of passive morphemes” in a sample of 80 languages belonging to 72 different WALS genera\is{genus}. Seven of the Austronesian languages in his sample belong to the Oceanic \isi{genus}, and so does one of the “Indo-Pacific” languages, \ili{Magori} (\lang{pn}). According to \cite[28]{haspelmath:1990}, 31 of the 80 languages “were found to have a passive” and these languages constitute the focus of his discussion. In turn, fourteen of the 31 languages feature a passive voice characterised by some kind of formal verbal marking, and are thereby of interest to this book. The Bantu language \ili{Mwera} (\lang{af}) only features a \isi{potential passive} (“the \isi{subject} is capable of undergoing an action”, \citealt[33]{haspelmath:1990}) and is therefore ignored here. \citeauthor{haspelmath:1990}’s survey of passive syncretism covers reflexive\is{reflexive voice}, reciprocal\is{reciprocal voice}, anticausative\is{anticausative voice}, passive, and antipassive\is{antipassive voice} (“deobjective”) functions -- like \citeauthor{geniusiene:1987}’s (\citeyear{geniusiene:1987}) survey of reflexive\is{reflexive voice} syncretism described in the previous section -- in addition to various other functions not directly relevant to the discussion here (e.g. resultativity,\is{resultative} habituality,\is{habitual} collectivity). The findings of \citeauthor{haspelmath:1990}’s (\citeyear{haspelmath:1990}) survey are presented in \tabref{tab:ch3:haspelmath}, in which each language represents a unique \isi{genus}. An asterisk (*) indicates paradigmatic voice marking (i.e. fusion of voice marking and \isi{agreement}), while a plus sign within parentheses (+) indicates that “the passive morpheme does not express this use alone but in conjunction with some other morpheme” \citep[36]{haspelmath:1990}, in other words type 2 syncretism\is{voice syncretism, partial resemblance -- type 2} (\sectref{resemblance-type2}). As also remarked in relation to \citeauthor{geniusiene:1987}’s (\citeyear{geniusiene:1987}) survey in the previous section, no attempt has here been made to modify \citeauthor{haspelmath:1990}’s (\citeyear{haspelmath:1990}) analysis of the languages in \tabref{tab:ch3:haspelmath}, and the contents represent findings according to his own specific definitions of the various voices. Differences between their respective analyses are therefore also ignored. For example, \citeauthor{geniusiene:1987} recognises a reciprocal\is{reciprocal voice} function for \ili{Latin} “\example{-r} forms” whereas \citeauthor{haspelmath:1990} does not (compare \tabref{tab:ch3:geniusiene} on page \pageref{tab:ch3:geniusiene}). 

\begin{table}
	\begin{tabularx}{.78\textwidth}{lcccccc}
		\lsptoprule
		& Marking & \textsc{refl} & \textsc{recp} & \textsc{antc} & \textsc{pass} & \textsc{antp} \\
		\midrule
		\ili{Udmurt} 	 & \example{-śk} 	   & +   & +   & + & + & + \\
		\ili{Greek} 	 & * 				  & +   & +   & + & + &   \\
		\ili{ʼOʼodham}	 & * 				  & +   & +   & + & + &   \\
		\ili{Tigre} 	 & \example{tə-} 	  & +   & (+) & + & + &   \\
		\ili{Motu} 	 	 & \example{he-} 	  & (+) & (+) & + & + &   \\
		\ili{Kanuri} 	 & \example{tə-, -tə} & +   &     & + & + &   \\
		\ili{Latin} 	 & * 				  & +   &     & + & + &   \\
		\ili{Slave} 	 & \example{d-} 	  & +   &     &   & + &   \\
		\ili{Rukai} 	 & \example{ki-} 	  & +   &     &   & + &   \\
		\ili{Worrorra} 	 & \example{-ieŋu}    & +   & +   &   & + &   \\
		\ili{Tuareg} 	 & \example{mə-} 	  &     & +   &   & + &   \\
		\ili{Danish} 	 & \example{-s} 	  &     &     & + & + &   \\
		\ili{Uyghur} 	 & \example{-il} 	  &     &     & + & + &   \\
		\ili{Nimboran}	 & \example{-da} 	  &     &     & + & + &   \\
		\lspbottomrule
	\end{tabularx}
	\caption{\citeauthor{haspelmath:1990}’s (\citeyear{haspelmath:1990}) survey of passive syncretism}
	\label{tab:ch3:haspelmath}
\end{table} 

The approach of \citeauthor{haspelmath:1990}’s (\citeyear{haspelmath:1990}) survey differs from that of \citeauthor{geniusiene:1987}’s (\citeyear{geniusiene:1987}) survey, and analogous tables to those presented for the latter study in the previous section can therefore not be produced for the former. More specifically, \cite{haspelmath:1990} only includes information about reflexive\is{reflexive voice}, reciprocal\is{reciprocal voice}, anticausative\is{anticausative voice}, and antipassive\is{antipassive voice} voices \textit{if} they share voice marking with the passive voice in any given language. Consequently, although \citeauthor{haspelmath:1990}’s survey is based on a sample of 80 languages, he only investigates patterns of voice syncretism involving the passive voice which he attests in 31 languages. Other patterns of syncretism lie outside the scope of his investigation. Thus, the frequencies for patterns of syncretism extracted from \citeauthor{haspelmath:1990}’s findings must be calculated according to different sample sizes: 80 languages for frequencies of patterns involving the passive voice, and 31 languages for frequencies of all other patterns. The patterns alongside their frequencies are listed in \tabref{tab:ch3:haspelmath-patterns-1} and \tabref{tab:ch3:haspelmath-patterns-2}. By analogy with the summary of \citeauthor{geniusiene:1987}’s (\citeyear{geniusiene:1987}) findings related to voice syncretism (see \tabref{tab:ch3:geniusiene-patterns} on page \pageref{tab:ch3:geniusiene-patterns}), the left-hand side of \tabref{tab:ch3:haspelmath-patterns-1} shows patterns of minimal\is{voice syncretism, minimal} simplex voice syncretism\is{voice syncretism, simplex}, while the right-hand side of the table shows patterns of maximal\is{voice syncretism, maximal} simplex\is{voice syncretism, simplex} and complex voice syncretism\is{voice syncretism, complex}. By contrast, \ref{tab:ch3:haspelmath-patterns-2} covers only minimal\is{voice syncretism, minimal} simplex voice syncretism\is{voice syncretism, simplex}, as \citeauthor{haspelmath:1990}’s (\citeyear{haspelmath:1990}) findings do not include any patterns of complex voice syncretism\is{voice syncretism, complex} that do not involve the passive voice.

\begin{table}
	\begin{tabularx}{.94\textwidth}{lrrrlrr}
		\lsptoprule
		\multicolumn{4}{l}{Minimal simplex syncretism} & \multicolumn{3}{l}{Maximal simplex/complex syncretism} \\
		\midrule
		\textsc{pass-antc} & 10 & (12.5 \%) & & \textsc{pass-refl-recp-antc} & 4 & (5.0 \%) \\
		\textsc{pass-refl} &  9 & (11.3 \%) & & \textsc{pass-antc} & 3 & (3.8 \%) \\
		\textsc{pass-recp} &  5 &  (6.3 \%) & & \textsc{pass-refl} & 2 & (2.5 \%) \\
		\textsc{pass-antp} &  1 &  (1.3 \%) & & \textsc{pass-refl-antc} & 2 & (2.5 \%) \\
		& & & & \textsc{pass-refl-recp} & 1 & (1.3 \%) \\
		& & & & \textsc{pass-antp-refl-recp-antc} & 1 & (1.3 \%) \\
		& & & & \textsc{pass-recp} & 1 & (1.3 \%) \\
		\lspbottomrule
	\end{tabularx}
	\caption{Voice syncretism in \citeauthor{haspelmath:1990}’s (\citeyear{haspelmath:1990}) survey (\textit{n} = 80)}
	\label{tab:ch3:haspelmath-patterns-1}
\end{table} 

\begin{table}
	\begin{tabularx}{.40\textwidth}{lrrr}
		\lsptoprule
		\multicolumn{4}{l}{Minimal simplex syncretism} \\
		\midrule
		\textsc{refl-antc} & 6 & (19.4 \%) & \\
		\textsc{refl-recp} & 4 & (12.9 \%) & \\
		\textsc{recp-antc} & 3 &  (9.7 \%) & \\
		\textsc{antp-refl} & 1 &  (3.2 \%) & \\
		\textsc{antp-recp} & 1 &  (3.2 \%) & \\
		\textsc{antp-antc} & 1 &  (3.2 \%) & \\
		\lspbottomrule
	\end{tabularx}
	\caption{Voice syncretism in \citeauthor{haspelmath:1990}’s (\citeyear{haspelmath:1990}) survey (\textit{n} = 31)}
	\label{tab:ch3:haspelmath-patterns-2}
\end{table} 

Unlike the frequencies attested in \citeauthor{geniusiene:1987}’s study (\citeyear{geniusiene:1987}) (see \tabref{tab:ch3:geniusiene-patterns} on page \pageref{tab:ch3:geniusiene-patterns}), the frequencies attested in  \citeauthor{haspelmath:1990}’s (\citeyear{haspelmath:1990}) presented in \tabref{tab:ch3:haspelmath-patterns-1} and \tabref{tab:ch3:haspelmath-patterns-2} are only slightly higher than those attested in this book. The distribution of voice syncretism attested in both \citeauthor{geniusiene:1987}’s and \citeauthor{haspelmath:1990}’s studies can be compared to that attested in the survey of this book in Chapter \ref{sect:distribution} (see \tabref{tab:ch6:voice-syncretism-macroarea-minimal} on page \pageref{tab:ch6:voice-syncretism-macroarea-minimal}, \tabref{tab:ch6:voice-syncretism-maximal-simplex-macroarea} on page \pageref{tab:ch6:voice-syncretism-maximal-simplex-macroarea}, and \tabref{tab:ch6:voice-syncretism-maximal-complex-macroarea} on page \pageref{tab:ch6:voice-syncretism-maximal-complex-macroarea}). The various patterns attested by \citeauthor{geniusiene:1987} and \citeauthor{haspelmath:1990} are discussed and illustrated in the following two chapters, in which evidence for several additional patterns of voice syncretism is also presented.

\section{Resemblance in voice marking} \label{resemblance}
Descriptions and investigations of voice syncretism in the literature commonly focus on a complete resemblance in the voice marking of two or more voices, yet in many languages voices sharing some marking may differ slightly in one way or another. To account for such variation in voice marking, three overarching types of voice syncretism are established in this book: type 1 syncretism based on a full resemblance in voice marking\is{voice syncretism, full resemblance -- type 1}, type 2 syncretism based on a partial resemblance in voice marking\is{voice syncretism, partial resemblance -- type 2}, and type 3 syncretism based on a “reverse” resemblance in voice marking\is{voice syncretism, reverse resemblance -- type 3}. Type 1 syncretism has two subtypes: type 1a syncretism in which the voice marking in two voices bears full resemblance under \textit{all} conditions, and type 1b syncretism in which the voice marking in two voices bears full resemblance under only \textit{some} conditions. Type 1a syncretism will henceforth be labelled \textsc{unconditioned}, while type 1b syncretism will be labelled \textsc{conditioned}\is{voice syncretism, full resemblance -- type 1}. This difference is essentially dependent on \isi{allomorphy}: in type 1a syncretism the \isi{allomorphy} of voice marking in two voices is the same, unlike in type 1b syncretism in which the \isi{allomorphy} of the voice marking in two voices overlaps only under certain conditions. Consequently, one may argue that the voice marking in type 1b syncretism is not exactly identical, and they are therefore differentiated in this book for the sake of transparency. These two types of syncretism are discussed further and illustrated in the next two sections (\sectref{resemblance-type1a} and \sectref{resemblance-type1b}), followed by a more detailed description of the partial resemblance in voice marking in type 2 syncretism (\sectref{resemblance-type2}). The “reverse” resemblance in type 3 syncretism deserves a preparatory explanation before being properly described in \sectref{resemblance-type3}. This type of syncretism denotes a phenomenon whereby voice marking in a given language appears as a suffix in one voice but as a prefix in another voice. Thus, reverse resemblance does not refer to a reverse meaning, but to the reverse manner\is{voice syncretism, reverse resemblance -- type 3} in which the voice marking appears on a verb in the respective voices. Such voice syncretism is rare, and it is therefore not surprising that discussions of the phenomenon are almost non-existent in the literature. However, it is explicitly recognised and described in this book for the sake of linguistic diversity.  

\subsection{Type 1a: full resemblance (unconditioned)} \label{resemblance-type1a}
As noted in the previous section, type 1a syncretism entails full resemblance in the voice marking\is{voice syncretism, full resemblance -- type 1} of two or more voices under \textit{all} conditions and thus represents the kind of voice syncretism typically discussed in the literature. This type is also considerably more prevalent cross-linguistically than other types of syncretism, being attested in 91 of the 222 languages in the language sample (41 percent). By comparison, type 2 syncretism\is{voice syncretism, partial resemblance -- type 2} which follows type 1a syncretism\is{voice syncretism, full resemblance -- type 1} in terms of frequency is attested in 25 of the languages (approximately 11 percent). Type 1a syncretism\is{voice syncretism, full resemblance -- type 1} is here illustrated in the Burraran language \ili{Gurr-Goni} (\lang{au}) by a reflexive\is{reflexive voice} \isi{voice relation} (\ref{ex:GurrGoni:hit:a}↔\ref{ex:GurrGoni:hit:b}) and a reciprocal\is{reciprocal voice} \isi{voice relation} (\ref{ex:GurrGoni:hit:a}↔\ref{ex:GurrGoni:hit:c}). As seen in these voice relations\is{voice relation}, the suffix \example{-yi} in Gurr-Goni serves as voice marking in both the reflexive\is{reflexive voice} (\ref{ex:GurrGoni:hit:b}) and reciprocal\is{reciprocal voice} voices (\ref{ex:GurrGoni:hit:c}).

\newpage

\ea \ili{Gurr-Goni} \citep[214]{green:1995}
\ea\label{ex:GurrGoni:hit:a}
	\gll	nguna-bu-ni \\
			\textsc{2/3min.sbj:1obj}-hit-\textsc{real} \\
	\glt	‘S/he/you hit me’.
\ex\label{ex:GurrGoni:hit:b}
	\gll	ngu-bu-\textbf{yi}-ni \\
			\textsc{1min.sbj}-hit-\textsc{refl-real} \\
	\glt	‘I hit myself’.
\ex\label{ex:GurrGoni:hit:c}
	\gll	awuni-bu-\textbf{yi}-ni \\
			\textsc{3aug.nf.sbj}-hit-\textsc{recp-real} \\
	\glt	‘They are hitting each other’.
	\z
\z

Additional non-verbal marking accompanying voice marking does not affect the classification of the voice syncretism. For instance, in the West Bougainville language \ili{Rotokas} (\lang{pn}) the prefix \example{ora-} serves as voice marking in both the reflexive\is{reflexive voice} and reciprocal\is{reciprocal voice} voices (\ref{ex:Rotokas:kill:a}↔\ref{ex:Rotokas:kill:b}), but in the latter voice the prefix can optionally be accompanied by the reciprocal\is{reciprocal voice} adverb \example{oisiaropavira} (\ref{ex:Rotokas:kill:c}) unlike in the former. Nevertheless, the formal verbal voice marking clearly remains the same in both the reflexive\is{reflexive voice} and reciprocal\is{reciprocal voice} voices under all conditions, and the \ili{Rotokas} examples thus qualify as type 1a syncretism\is{voice syncretism, full resemblance -- type 1}.

\ea \ili{Rotokas} \citep[193, 221]{robinson:s:2011}
\ea\label{ex:Rotokas:kill:a}
	\gll	uuvau-va Rara kopii-pie-e-va \\
			tuberculosis-\textsc{sg.f} \textsc{name} die-\textsc{caus-3sg.f-pst} \\
	\glt	‘Tuberculosis killed Rara’.
\ex\label{ex:Rotokas:kill:b}
	\gll	\textbf{ora}-kopii-pie-pa-a-i \\
			\textsc{refl/recp}-die-\textsc{caus-cont-3pl-prs} \\
	\glt	‘They are killing themselves’.
	\glt	‘They are killing each other’.
\ex\label{ex:Rotokas:kill:c}
	\gll	oisiaropavira \textbf{ora}-kopii-pie-pa-ai \\
			reciprocally \textsc{recp}-die-\textsc{caus-cont-3pl-prs} \\
	\glt	‘They are killing each other’.
	\z
\z

In rare cases, non-verbal marking is obligatory in type 1a syncretism\is{voice syncretism, full resemblance -- type 1}, for example in the Ju-Kung language Western !Xun\il{!Xun, Western} (\lang{af}). In this language the suffix \example{-ā} serves as voice marking in both the applicative\is{applicative voice} and reciprocal\is{reciprocal voice} voices, in the latter obligatorily accompanied by the reciprocal pronoun \example{kòè}. Likewise, in the Timor-Alor-Pantar language \ili{Makalero} (\lang{pn}) the suffix \example{-ini} serves as voice marking in both the antipassive\is{antipassive voice} and causative\is{causative voice} voices, in the latter obligatorily accompanied by an auxiliary \isi{light verb}. These patterns of syncretism are exemplified in \tabref{tab:ch3:type1a-examples-1}. A subtype of type 1a syncretism\is{voice syncretism, full resemblance -- type 1} which takes obligatory non-verbal marking into account could potentially be established for languages like Western !Xun\il{!Xun, Western} and \ili{Makalero}, but these languages are the only two languages in which such marking has been attested in the language sample, so the establishment of such a subtype has been deemed superfluous for the time being.

\begin{table}
	\begin{tabularx}{\textwidth}{llllll}
		\lsptoprule
		\multicolumn{6}{l}{Western !Xun\il{!Xun, Western} \citep[88, 192, 210]{heine:konig:2015}} \\
		\midrule
		\textsc{appl} & \example{cŋ̏} & ‘to drink sth.’ & ↔ & \example{cŋ̏-\textbf{ā}} & ‘to drink sth. at sth.’ \\
		\textsc{recp} & \example{hŋ̄} & ‘to see sb.’ & ↔ & \example{hŋ̄-\textbf{ā} kòè} & ‘to see e.o.’ \\
		\midrule\midrule
		\multicolumn{6}{l}{\ili{Makalero} \citep[150, 340f., 248, 299, 456]{huber:2011}} \\
		\midrule
		\textsc{caus} & \example{da’al} & ‘to break’ & ↔ & \example{mei=ni da’al-\textbf{ini}} & ‘to break sth.’ \\
		\textsc{caus} & \example{dur} & ‘to wake up’ & ↔ & \example{mei=ni dur-\textbf{ini}} & ‘to wake sb. up’ \\
		\textsc{antp} & \example{heru} & ‘to weave sth.’ & ↔ & \example{heru-\textbf{ini}} & ‘to weave [sth.]’ \\
		\textsc{antp} & \example{isa} & ‘to bake sth.’ & ↔ & \example{isa-\textbf{ini}} & ‘to bake [sth.]’ \\
		\lspbottomrule
	\end{tabularx}
	\caption{Type 1a voice syncretism alongside non-verbal marking}
	\label{tab:ch3:type1a-examples-1}
\end{table} 

Next, consider the patterns of voice syncretism in \tabref{tab:ch3:type1a-examples-2}. The non-absolute passive and absolute antipassive\is{antipassive voice} voices in the Algonquian language \ili{Arapaho} (\lang{na}) share the same voice marking, and so do the causative\is{causative voice} and anticausative\is{anticausative voice} voices in the language isolate \ili{Ainu} and the Ugric language Northern Mansi\il{Mansi, Northern} (both \lang{ea}). Note that the schwa in the Northern Mansi verb \example{woŋən-l-} is simply epenthetic. Moreover, observe that in each of the absolute antipassive\is{antipassive voice} and anticausative\is{anticausative voice} voices the voice marking is in variation with some verbal marking in the contrasting \isi{diathesis} according to which it is defined (cf. \ili{Arapaho} \example{-oo} ↔ \example{-ee}, \ili{Ainu} \example{-e} ↔ \example{-ke}, and Northern Mansi\il{Mansi, Northern} \example{-t} ↔ \example{-l}). Nevertheless, as this book focuses strictly on voice marking, the verbal marking in the contrasting diatheses\is{diathesis} is irrelevant. The passive-antipassive syncretism in \ili{Arapaho} and the causative-anticausative syncretism in \ili{Ainu} and Northern Mansi\il{Mansi, Northern} thus both qualify as type 1a syncretism\is{voice syncretism, full resemblance -- type 1}. More examples of \ili{Arapaho} passive-antipassive syncretism are provided in \tabref{tab:ch4:pass-antp} on page \pageref{tab:ch4:pass-antp}, while additional examples of \ili{Ainu} and Northern Mansi\il{Mansi, Northern} causative-anticausative syncretism are given in \tabref{tab:ch4:caus-antc} on page \pageref{tab:ch4:caus-antc}.

\begin{table}
	\setlength{\tabcolsep}{3pt}
	\begin{tabularx}{\textwidth}{llllll}
		\lsptoprule
		\multicolumn{6}{l}{\ili{Arapaho} \citep[133ff., 155, 229, 323]{cowell:moss:2008}} \\
		\midrule
		\textsc{pass} & \example{neh’-} & ‘to kill sb.’ & ↔ & \example{neh’-\textbf{ee}-} & ‘to be killed [by sb.]’ \\
		\textsc{pass} & \example{to3ih-} & ‘to follow sb.’ & ↔ & \example{to3ih-\textbf{ee}-} & ‘to be followed [by sb.]’ \\
		\textsc{antp} & \example{niitow-oo-} & ‘to hear sth.’ & ↔ & \example{niitow-\textbf{ee}-} & ‘to hear [sth.]’ \\
		\textsc{antp} & \example{neeceew-oo-} & ‘to be in & ↔ & \example{neeceew-\textbf{ee}-} & ‘to be in charge [of sb.]’ \\
		& & charge of sb.’ & & & \\
		\midrule\midrule
		\multicolumn{6}{l}{\ili{Ainu} \citep[1760ff., 1780]{alpatov:al:2007}} \\
		\midrule
		\textsc{caus} & \example{ray} & ‘to die’ & ↔ & \example{ray-\textbf{ke}} & ‘to kill sb.’ \\
		\textsc{caus} & \example{ahuy} & ‘to burn’ & ↔ & \example{ahuy-\textbf{ke}} & ‘to burn sth.’ \\
		\textsc{antc} & \example{per-e} & ‘to break sth.’ & ↔ & \example{per-\textbf{ke}} & ‘to be broken’ \\
		\textsc{antc} & \example{moymoy-e} & ‘to move sth.’ & ↔ & \example{moymoy-\textbf{ke}} & ‘to move’ \\
		\midrule\midrule
		\multicolumn{6}{l}{Northern Mansi\il{Mansi, Northern} \citep[154, 160]{rombandeeva:1973}} \\
		\midrule
		\textsc{caus} & \example{lap-} & ‘to rise’ & ↔ & \example{lap-\textbf{l}-} & ‘to raise sth.’ \\
		\textsc{caus} & \example{woŋn-} & ‘to stretch’ & ↔ & \example{woŋən-\textbf{l}-} & ‘to stretch sth.’ \\
		\textsc{antc} & \example{āpram-t-} & ‘to hurry sb.’ & ↔ & \example{āpram-\textbf{l}-} & ‘to hurry’ \\
		\textsc{antc} & \example{toram-t-} & ‘to calm sb.’ & ↔ & \example{toram-\textbf{l}-} & ‘to calm down’ \\
		\lspbottomrule
	\end{tabularx}
	\caption{Type 1a syncretism alongside contrasting verbal marking}
	\label{tab:ch3:type1a-examples-2}
\end{table}

\newpage

The “antipassive-like” diathetic relations described for the Salishan languages \ili{Nxa’amxcin} and \ili{Musqueam} in \sectref{def:passives-antipassives} (see \tabref{tab:ch2:Nxaamxcin} on page \pageref{tab:ch2:Nxaamxcin}) are rather similar to the \ili{Arapaho} absolute antipassive\is{antipassive voice} voice relations\is{voice relation} and the \ili{Ainu} and Northern Mansi\il{Mansi, Northern} anticausative\is{anticausative voice} voice relations\is{voice relation} presented in \tabref{tab:ch3:type1a-examples-2}. As shown in \tabref{tab:ch3:type1a-examples-3}, the antipassive-like diatheses\is{diathesis} in both \ili{Nxa’amxcin} and \ili{Musqueam} are characterised by the suffix \example{-m}, which also serves as voice marking in the absolute passive\is{passive voice} voices in these languages. However, as already noted in \sectref{def:passives-antipassives}, the antipassive-like diatheses\is{diathesis} in the two languages do not qualify as proper antipassive\is{antipassive voice} voices, and the examples are consequently only presented here for the sake of comparison. Additional examples of type 1a syncretism\is{voice syncretism, full resemblance -- type 1} are provided throughout the subsequent chapters, so this type of syncretism is not discussed further here.

\begin{table}
	\setlength{\tabcolsep}{5pt}
	\begin{tabularx}{\textwidth}{llllll}
		\lsptoprule
		\multicolumn{6}{l}{\ili{Nxa’amxcin} \citep[104, 153, 158f., 164ff.]{willett:2003}} \\
		\midrule
		\textsc{pass} & \example{wík-ɫt-} & ‘to see sth.’ & ↔ & \example{wík-ɫt-\textbf{m}} & ‘to be seen [by sb.]’ \\
		\textsc{pass} & \example{x̣əlq’-nt-} & ‘to kill sb.’ & ↔ & \example{x̣əlq’-nt-\textbf{m}} & ‘to be killed [by sb.]’ \\
		\textsc{antp}-like & \example{wík-ɫt-} & ‘to see sth.’ & ↔ & \example{wík-\textbf{m}} & ‘to see [sth.]’ \\
		\textsc{antp}-like & \example{x̣əlq’-nt-} & ‘to kill sb.’ & ↔ & \example{x̣əlq’-\textbf{m}} & ‘to kill [sb.]’ \\
		\midrule\midrule
		\multicolumn{6}{l}{\ili{Musqueam} \citep[35, 43, 51, 231, 447f.]{suttles:2004}} \\
		\midrule
		\textsc{pass} & \example{k̓ʷłé-t} & ‘to spill sth.’ & ↔ & \example{k̓ʷłé-t-\textbf{əm}} & ‘to be spilled [by sb.]’ \\
		\textsc{pass} & \example{c̓éw-ət} & ‘to help sb.’ & ↔ & \example{c̓éw-ət-\textbf{əm}} & ‘to be helped [by sb.]’ \\
		\textsc{antp}-like & \example{kʷə́n-ət} & ‘to get sth.’ & ↔ & \example{kʷə́n-\textbf{əm}} & ‘to get [sth.]’ \\
		\textsc{antp}-like & \example{ʔáˑ-t} & ‘to call sb.’ & ↔ & \example{ʔáˑ-\textbf{m}} & ‘to call [sb.]’ \\
		\lspbottomrule
	\end{tabularx}
	\caption{Passive-antipassive-like syncretism in Salishan languages}
	\label{tab:ch3:type1a-examples-3}
\end{table}

\subsection{Type 1b: full resemblance (conditioned)} \label{resemblance-type1b}
On the one hand, type 1b syncretism entails full resemblance in the voice marking\is{voice syncretism, full resemblance -- type 1} of two voices, like type 1a syncretism. On the other hand, in type 1b syncretism the full resemblance in question is found only under certain conditions, unlike in type 1a syncretism\is{voice syncretism, full resemblance -- type 1}. Type 1b syncretism is notably rarer than type 1a syncretism\is{voice syncretism, full resemblance -- type 1}, and has only been attested in six languages in the language sample. A very illustrative example of type 1b syncretism\is{voice syncretism, full resemblance -- type 1} is provided in \tabref{tab:ch3:type1b-examples-1}. In the North Omotic language \ili{Wolaytta} (\lang{af}) the suffix \example{-ett} without a high pitch serves as voice marking in both the causative\is{causative voice} and passive\is{passive voice} voices. This suffix can alternatively have a high pitch (i.e. \example{-étt}) in the passive\is{passive voice} voice, but never in the causative\is{causative voice} voice \citep[1008]{wakasa:2008}. In other words, the suffix serving as voice marking in the passive\is{passive voice} voice has two allomorphs\is{allomorphy} (i.e. \example{-ett} and \example{-étt}), while the suffix serving as voice marking in the causative\is{causative voice} voice has only one (i.e. \example{-ett}). The allomorphic\is{allomorphy} variation of the passive\is{passive voice} suffix is dependent on the “tonal prominence” of the stem to which it is attached: the allomorph\is{allomorphy} \example{-ett} is found on stems with tonal prominence, while the allomorph\is{allomorphy} \example{-étt} is found on stems without tonal prominence \citep[84ff., 1013]{wakasa:2008}. This conditioned \isi{allomorphy} is particularly clear if one compares the verbs \example{dóór-} and \example{door-} in \tabref{tab:ch3:type1b-examples-1}. Note that the voice marking in the passive\is{passive voice} voice also serves as voice marking in the reflexive\is{reflexive voice} and reciprocal\is{reciprocal voice} voices (see \tabref{tab:ch5:caus-pass-antc} on page \pageref{tab:ch5:caus-pass-antc}).

\begin{table}
	\begin{tabularx}{\textwidth}{llllll}
		\lsptoprule
		\multicolumn{6}{l}{\ili{Wolaytta} \citep[217, 381, 1008, 1013f.]{wakasa:2008}} \\
		\midrule
		\textsc{caus} & \example{boLL-} & ‘to get hot’ & ↔ & \example{boLL-\textbf{ett}-} & ‘to make sth. hot’ \\
		\textsc{caus} & \example{7uNN-} & ‘to get narrow’ & ↔ & \example{7uNN-\textbf{ett}-} & ‘to make sth. narrow’ \\
		\textsc{pass} & \example{7ánC-} & ‘to mince sth.’ & ↔ & \example{7ánC-\textbf{ett}-} & ‘to be minced [by sb.]’ \\
		\textsc{pass} & \example{dóór-} & ‘to pile sth. up’ & ↔ & \example{dóór-\textbf{ett}-} & ‘to be piled up [by sb.]’ \\
		\textsc{pass} & \example{door-} & ‘to choose sb.’ & ↔ & \example{door-\textbf{étt}-} & ‘to be chosen [by sb.]’ \\
		\textsc{pass} & \example{bonc-} & ‘to respect sb.’ & ↔ & \example{bonc-\textbf{étt}-} & ‘to be respected [by sb.]’ \\
		\lspbottomrule
	\end{tabularx}
	\caption{Examples of type 1b syncretism (I)}
	\label{tab:ch3:type1b-examples-1}
\end{table}

\newpage

Type 1b syncretism\is{voice syncretism, full resemblance -- type 1} is also attested in the language isolate \ili{Kutenai} (\lang{na}) which has various suffixes that can serve as voice marking in the causative\is{causative voice} voice, one of which is a glottal stop. Interestingly, a suffixal glottal stop can also serve as voice marking in the anticausative\is{anticausative voice} voice. As argued by \cite[336]{morgan:1991}, the underlying suffix \example{-p} generally serving as voice marking in the anticausative\is{anticausative voice} voice is “realized as glottal stop [\example{-ʔ}] before the invariantly encliticized Indicative Marker [\example{-ni}], and the invariantly encliticized Locative Marker [\example{-ki}]”. As illustrated in the following causative\is{causative voice} (\ref{ex:Kutenai:spill:a}↔\ref{ex:Kutenai:spill:b}) and anticausative\is{anticausative voice} diathetic relations (\ref{ex:Kutenai:light:a}↔\ref{ex:Kutenai:light:b}), under such conditions (here preceding the “Indicative Marker” \example{-ni}) the anticausative\is{anticausative voice} voice marking (\ref{ex:Kutenai:light:b}) fully resembles causative\is{causative voice} voice marking (\ref{ex:Kutenai:spill:b})\is{voice syncretism, full resemblance -- type 1}. More examples of the causative-anticausative syncretism in \ili{Kutenai} are provided in \sectref{sec:simple-syncretism:caus-antc} (see \tabref{tab:ch4:caus-antc} on page \pageref{tab:ch4:caus-antc}).

\ea \ili{Kutenai} \citep[25, 337]{morgan:1991}
\ea\label{ex:Kutenai:spill:a}
	\gll	yik̓ta-ni \\
			spill-\textsc{ind} \\
	\glt	‘It spilled’.
\ex\label{ex:Kutenai:spill:b}
	\gll	yik̓ta-\textbf{ʔ}-ni \\
			spill-\textsc{caus-ind} \\
	\glt	‘S/he/they spilled it’.
\ex\label{ex:Kutenai:light:a}
	\gll	¢uk-ni (< ¢uku-ni) \\
			light-\textsc{ind} \\
	\glt	‘S/he/they lit it’.
\ex\label{ex:Kutenai:light:b}
	\gll	¢uku-\textbf{ʔ}-ni \\
			light-\textsc{antc-ind} \\
	\glt	‘It became lit / ignited’.
	\z
\z

Next, consider the patterns of type 1b voice syncretism\is{voice syncretism, full resemblance -- type 1} in \tabref{tab:ch3:type1b-examples-2}. In the language isolate \ili{Sandawe} (\lang{af}) the causative\is{causative voice} suffix \example{-kù̥} and the applicative\is{applicative voice} suffix \example{-x\`{}} both have the allomorph\is{allomorphy} \example{-kw} before a vowel due to \isi{assimilation} \citep[46, 189]{steeman:2012}. In San Francisco del Mar Huave\il{Huave, San Francisco del Mar} (\lang{na}) the passive\is{passive voice} suffix \example{-Vch} is “homophonous with the unaspirated allomorph\is{allomorphy} of the causative\is{causative voice} suffix” \example{-V(j)ch}, though it is worth noting that the passive\is{passive voice} suffix is rare and represents a “non-productive\is{productivity} way of forming passives\is{passive voice}” \citep[305]{kim:2008}. The phonological variation in the stems \example{-ji(o)ng} and \example{-pi(o)r} is due to a regular morphophonological\is{morphophonology} process of \isi{vowel breaking}, in this case /io/ > /i/ \citep[52ff.]{kim:2008}. Additionally, in the Atlantic language Ganja Balanta\il{Balanta, Ganja} (\lang{af}) the antipassive\is{antipassive voice} suffix \example{-t} is similar to one of the allomorphs\is{allomorphy} of the causative\is{causative voice} suffix \example{-(V)t}. The suffix \example{-t} only has an antipassive\is{antipassive voice} use with four verbs though, two of which are exemplified in \tabref{tab:ch3:type1b-examples-2}, while the other two verbs are illustrated in \tabref{tab:ch4:caus-antp-ganja} on page \pageref{tab:ch4:caus-antp-ganja}. From a \isi{language-specific} perspective, the Ganja Balanta\il{Balanta, Ganja} verbs in the causative\is{causative voice} and antipassive\is{antipassive voice} voices belong to different verb classes\is{verb class}, as indicated by the final infinitive vowels \citep[208ff.]{creissels:biaye:2016}.

\begin{table} 
	\setlength{\tabcolsep}{4pt}
	\begin{tabularx}{\textwidth}{llllll}
		\lsptoprule
		\multicolumn{6}{l}{\ili{Sandawe} \citep[148f., 189f., 237]{steeman:2012}} \\
		\midrule
		\textsc{caus} & \example{kê} & ‘to ascend’ & ↔ & \example{kê-\textbf{kw}-} & ‘to let sth. ascend’ \\
		\textsc{caus} & \example{mântshà} & ‘to eat sth.’ & ↔ & \example{mântshà-\textbf{kw}-} & ‘to make sb. eat sth.’ \\
		\textsc{appl} & \example{mântshà} & ‘to eat sth.’ & ↔ & \example{mântshà-\textbf{kw}-} & ‘to eat sth. for sb.’ \\
		\textsc{appl} & \example{ǁhèmé} & ‘to pay sth.’ & ↔ & \example{ǁhèmé-\textbf{kw}-} & ‘to pay sth. for sb.’ \\
		\midrule\midrule
		\multicolumn{6}{l}{San Francisco del Mar Huave\il{Huave, San Francisco del Mar} \citep[305, 311]{kim:2008}} \\
		\midrule
		\textsc{caus} & \example{pal-} & ‘to end’ & ↔ & \example{-pal-\textbf{ach}} & ‘to end sth.’ \\
		\textsc{caus} & \example{-jiong} & ‘to dance’ & ↔ & \example{-jing-\textbf{ach}} & ‘to make sb. dance’ \\
		\textsc{pass} & \example{-rriujt} & ‘to choose sb.’ & ↔ & \example{-rriujt-\textbf{ach}} & ‘to be chosen [by sb.]’ \\
		\textsc{pass} & \example{-pior} & ‘to sow sth.’ & ↔ & \example{-pir-\textbf{ach}} & ‘to be sown [by sb.]’ \\
		\midrule\midrule
		\multicolumn{6}{l}{Ganja Balanta\il{Balanta, Ganja} \citep[209ff.]{creissels:biaye:2016}} \\
		\midrule
		\textsc{caus} & \example{sιιg} & ‘to drink sth.’ & ↔ & \example{sιιg-\textbf{t}.ι} & ‘to make sb. drink sth.’ \\
		\textsc{caus} & \example{sum} & ‘to get pleasant’ & ↔ & \example{sum-\textbf{t}.ι} & ‘to make sth. pleasant’ \\
		\textsc{antp} & \example{wɔm} & ‘to eat sth.’ & ↔ & \example{wɔm-\textbf{t}.ɛ} & ‘to eat [sth.]’ \\
		\textsc{antp} & \example{rʊŋ} & ‘to crush sth.’ & ↔ & \example{rʊŋ-\textbf{t}.ɛ} & ‘to crush [sth.]’ \\
		\lspbottomrule
	\end{tabularx}
	\caption{Examples of type 1b syncretism (II)}
	\label{tab:ch3:type1b-examples-2}
\end{table}

\newpage

Finally, \cite[343, 347]{montgomery-anderson:2008} argues that the Southern Iroquoian language \ili{Cherokee} (\lang{na}) has a “reflexive\is{reflexive voice} prefix” as well as a “\isi{middle voice} prefix” with “some similarities in form and meaning to the Reflexive\is{reflexive voice} and probably developed out of it”. The “reflexive\is{reflexive voice} prefix” serves as voice marking in the reflexive\is{reflexive voice}, reciprocal\is{reciprocal voice}, and antipassive\is{antipassive voice} voices and has three allomorphs:\is{allomorphy} \example{ataa-} before consonants, \example{at-} before the vowel /a/, and \example{ataat-} before all other vowels \citep[343]{montgomery-anderson:2008}. By contrast, the “\isi{middle voice} prefix” serves as voice marking in the anticausative\is{anticausative voice} voice, and also has three allomorphs:\is{allomorphy} \example{ali-} before the consonant /h/ (and seemingly also before /s/ and /n/), \example{ataa-} before all other consonants, and \example{at-} before all vowels \citep[372]{montgomery-anderson:2008}. Evidently, the allomorphs\is{allomorphy} of the two prefixes are identical under certain phonological conditions, namely before consonants other than /h/, /s/, and /n/ and before the vowel /a/. These prefixes in Cherokee are illustrated and discussed further in \sectref{sec:complex-syncretism:antp-refl} (see \tabref{tab:ch5:antp-refl-recp-antc-3} on page \pageref{tab:ch5:antp-refl-recp-antc-3}). 

As suggested by \citet{montgomery-anderson:2008}, the diachronic origin of type 1b syncretism\is{voice syncretism, full resemblance -- type 1} in Cherokee can probably be explained in terms of semantic and functional \isi{convergence}. It is not unlikely that type 1b syncretism\is{voice syncretism, full resemblance -- type 1} in \ili{Wolaytta} and San Francisco del Mar Huave\il{Huave, San Francisco del Mar} can be explained in the same manner considering the distinct forms of their respective voice markers, though there are currently little historical data available for the languages to support such a claim. By contrast, type 1b syncretism\is{voice syncretism, full resemblance -- type 1} in \ili{Kutenai} and \ili{Sandawe} is almost certainly the result of coincidental phonological \isi{convergence}, while \citet{creissels:biaye:2016} do not shed any light upon the origin of type 1b syncretism\is{voice syncretism, full resemblance -- type 1} in Ganja Balanta. \il{Balanta, Ganja}

\subsection{Type 2: partial resemblance} \label{resemblance-type2}
As noted in \sectref{resemblance}, investigations of voice syncretism in the literature tend to focus on a full resemblance in the voice marking\is{voice syncretism, partial resemblance -- type 2} of two or more voices, while partial resemblance has received comparatively little attention -- with the notable exception of \citeauthor{nedjalkov:2007d} (see \tabref{tab:ch1:nedjalkov-examples} on page \pageref{tab:ch1:nedjalkov-examples}). Nevertheless, although explicit discussions of type 2 syncretism\is{voice syncretism, partial resemblance -- type 2} are rare in the literature, this type of syncretism is not uncommon cross-linguistically. Type 2 syncretism\is{voice syncretism, partial resemblance -- type 2} is only attested in one tenth of all the languages in the language sample, yet the syncretism is attested in a quarter of all languages in the sample featuring voice syncretism (see \tabref{tab:ch6:voice-syncretism-type-2} on page \pageref{tab:ch6:voice-syncretism-type-2}). Thus, type 2 syncretism\is{voice syncretism, partial resemblance -- type 2} can be found in a rather large portion of languages with voice syncretism. Furthermore, type 2 syncretism\is{voice syncretism, partial resemblance -- type 2} is not restricted to the reflexive-reciprocal syncretism discussed by \citet{nedjalkov:2007d} but is attested for a wide range of different patterns of voice syncretism in the language sample. Consider for instance the examples of type 2 syncretism\is{voice syncretism, partial resemblance -- type 2} provided in \tabref{tab:ch3:type2-examples}. In the language isolate \ili{Kwaza} (\lang{sa}) the causative\is{causative voice} voice is characterised by the suffix \example{-dy} which has become lexicalised\is{lexicalisation} in a number of verbs, including \example{wady} ‘to give’ in which the suffix appears after the root \example{*wa} of unknown origin and meaning \citep[372f.]{van-der-voort:2004}. In turn, this verb has grammaticalised\is{grammaticalisation} into the morpheme \example{=wady} which serves as voice marking in the applicative\is{applicative voice} voice. In the Siouan language \ili{Assiniboine} (\lang{na}) the applicative\is{applicative voice} prefix \example{ki-} forms part of the reciprocal\is{reciprocal voice} prefix \example{kicʰi-} (these prefixes are further discussed in \sectref{sec:simple-syncretism:appl-recp}). Coincidentally, the Kxa language \ili{ǂHȍã} (\lang{af}) also features a prefix \example{ki-} of interest to this discussion. In this language the prefix in question serves as voice marking in both the causative\is{causative voice} and passive\is{passive voice} voices, but always features a high tone in the former voice “clearly distinguished from the low tone” employed in the latter voice \citep[166]{collins:gruber:2014}. Additional examples of type 2 syncretism\is{voice syncretism, partial resemblance -- type 2} are provided throughout the following chapters, and the syncretism is therefore not discussed further here.

\begin{table} 
	\setlength{\tabcolsep}{3.8pt}
	\begin{tabularx}{\textwidth}{lllll @{\hspace{1.0\tabcolsep}} l}
		\lsptoprule
		\multicolumn{6}{l}{\ili{Kwaza} \citep[110, 366, 373, 898]{van-der-voort:2004}} \\
		\midrule
		\textsc{caus} & \example{kãu-} & ‘to break’ & ↔ & \example{kãu-\textbf{dy}-} & ‘to break sth.’ \\
		\textsc{caus} & \example{mãmãñẽ-} & ‘to sing’ & ↔ & \example{mãmãñẽ-\textbf{dy}-} & ‘to make sb. sing’ \\
		\textsc{appl} & \example{mãmãñẽ-} & ‘to sing’ & ↔ & \example{mãmãñẽ=\textbf{wady}-} & ‘to sing for sb.’ \\
		\textsc{appl} & \example{hãte-} & ‘to count sth.’ & ↔ & \example{hãte=\textbf{wady}-} & ‘to count sth. for sb.’ \\
		\midrule\midrule
		\multicolumn{6}{l}{\ili{Assiniboine} \citep[263, 271]{cumberland:2005}} \\
		\midrule
		\textsc{appl} & \example{ná} & ‘to ask for sth.’ & ↔ & \example{\textbf{ki}-ná} & ‘to ask sb. for sth.’ \\
		\textsc{appl} & \example{yukʰą́} & ‘to give room’ & ↔ & \example{\textbf{ki}-yúkʰą́} & ‘to give room for sb.’ \\
		\textsc{recp} & \example{pažípa} & ‘to poke sb.’ & ↔ & \example{\textbf{kicʰí}-pažipa} & ‘to poke e.o.’ \\
		\textsc{recp} & \example{yaʔį́škata} & ‘to tease sb.’ & ↔ & \example{\textbf{kicʰí}-yaʔįškata} & ‘to tease e.o.’ \\
		\midrule\midrule
		\multicolumn{6}{l}{\ili{ǂHȍã} \citep[21, 142, 164f., 186]{cumberland:2005}} \\
		\midrule
		\textsc{caus} & \example{ču} & ‘to drink sth.’ & ↔ & \example{\textbf{kí}-ču} & ‘to make sb. drink sth.’ \\
		\textsc{caus} & \example{ʼám} & ‘to eat sth.’ & ↔ & \example{\textbf{kí}-ʼám} & ‘to make sb. drink sth.’ \\
		\textsc{pass} & \example{ʼám} & ‘to eat sth.’ & ↔ & \example{\textbf{kì}-ʼám} & ‘to be eaten [by sb.]’ \\
		\textsc{pass} & \example{ǁgȍõ} & ‘to strike sb.’ & ↔ & \example{\textbf{kì}-ǁgȍõ} & ‘to be struck [by sb.]’ \\
		\lspbottomrule
	\end{tabularx}
	\caption{Examples of type 2 syncretism}
	\label{tab:ch3:type2-examples}
\end{table}

\subsection{Type 3: reverse resemblance} \label{resemblance-type3}
Type 3 syncretism is based on reverse resemblance in voice marking\is{voice syncretism, reverse resemblance -- type 3} which denotes a peculiar phenomenon whereby voice marking in a given language appears as a suffix in one voice but as a prefix in another. The reverse resemblance does not refer to a “reverse” meaning but rather to the “reverse” manner in which the voice marking\is{voice syncretism, reverse resemblance -- type 3} appears on the respective verbs. Discussions of type 3 syncretism\is{voice syncretism, reverse resemblance -- type 3} are very rare in the literature, and it has only been possible to find one prior typological discussion of the phenomenon. In a description of reciprocity\is{reciprocal voice} in the Gunwinyguan language \ili{Nunggubuyu} (\lang{au}), \citet[252]{nedjalkov:2007d} briefly mentions that the applicative\is{applicative voice} prefix \example{anʸji-} is “most likely etymologically related” to the phonologically rather similar suffix \example{-nʸji} which serves as voice marking in the reflexive\is{reflexive voice}, reciprocal\is{reciprocal voice}, and antipassive\is{antipassive voice} voices. For example, compare the verbs \example{anʸji-nᵍama} ‘to swim with sb.’, \example{ṟi-nʸji} ‘to spear self’ or ‘to spear e.o.’, and \example{warguri-nʸji} ‘to carry [sb.] on the shoulders’ \citep[382, 392]{heath:1984}. More examples are provided in \sectref{sec:complex-syncretism:antp-refl} (see \tabref{tab:ch5:antp-refl-recp} on page \pageref{tab:ch5:antp-refl-recp}). Five other languages with type 3 voice syncretism\is{voice syncretism, reverse resemblance -- type 3} have been attested in the language sample of this book, and the syncretism in these languages is illustrated in \tabref{tab:ch3:type3-examples}.

\begin{table} 
	\setlength{\tabcolsep}{3pt}
	\begin{tabularx}{\textwidth}{ll @{\hspace{1.1\tabcolsep}} lll @{\hspace{1.0\tabcolsep}} l}
		\lsptoprule
		\multicolumn{6}{l}{\ili{Hup} \citep[408, 486, 500, 574, 672, 852]{epps:2008}} \\
		\midrule
		\textsc{appl} & \example{dʼoʔ} & ‘to take/get sth.’ & ↔ & \example{dʼoʔ-\textbf{ʔũh}} & ‘to take/get sth. for sb.’ \\
		\textsc{appl} & \example{mæh} & ‘to hit/kill sb.’ & ↔ & \example{mæh-\textbf{ʔũh}} & ‘to kill sb. for sb.’ \\
		\textsc{recp} & \example{nɔʔ} & ‘to give sb. sth.’ & ↔ & \example{\textbf{ʔũh}-nɔʔ} & ‘to give e.o. sth.’ \\
		\textsc{recp} & \example{mæh} & ‘to hit/kill sb.’ & ↔ & \example{\textbf{ʔũh}-mæh} & ‘to hit e.o.’ \\
		\midrule\midrule
		\multicolumn{6}{l}{\ili{Mosetén} \citep[64, 193, 212, 322, 391, 455]{sakel:2004}} \\
		\midrule
		\textsc{appl} & \example{tyar-i-} & ‘to be sad’ & ↔ & \example{\textbf{ti}-tyar-i-} & ‘to be sad about sth.’ \\
		\textsc{appl} & \example{baeʼ-i-} & ‘to live’ & ↔ & \example{\textbf{ti}-baeʼ-i-} & ‘to live with sb.’ \\
		\textsc{recp} & \example{tyaj-ki-} & ‘to meet sb.’ & ↔ & \example{tyaj-ki-\textbf{ti}-} & ‘to meet e.o.’ \\
		\textsc{recp} & \example{chha’sh-i-} & ‘to reach sb.’ & ↔ & \example{chha’sh-i-\textbf{ti}-} & ‘to reach e.o.’ \\
		\midrule\midrule
		\multicolumn{6}{l}{\ili{Alamblak} \citep[177, 209, 250, 255, 356, 431]{bruce:1979}} \\
		\midrule
		\textsc{caus} & \example{tat} & ‘to hit sb.’ & ↔ & \example{\textbf{hay}-tat} & ‘to make sb. hit sb.’ \\
		\textsc{caus} & \example{yi} & ‘to go’ & ↔ & \example{\textbf{hay}-ni} & ‘to make sb. go’ \\
		\textsc{appl} & \example{wikna} & ‘to buy sth.’ & ↔ & \example{wikna-\textbf{hay}} & ‘to buy sth. for sb.’ \\
		\textsc{appl} & \example{suh} & ‘to fall’ & ↔ & \example{suh-\textbf{hay}} & ‘to fall for the benefit of sb.’ \\
		\midrule\midrule
		\multicolumn{6}{l}{\ili{Ainu} (\citealt[44]{bugaeva:2004}; \citeyear[445]{bugaeva:2015}; \citealt[1770]{alpatov:al:2007})} \\
		\midrule
		\textsc{caus} & \example{kay} & ‘to break’ & ↔ & \example{kay-\textbf{e}} & ‘to break sth.’ \\
		\textsc{caus} & \example{nukar} & ‘to see sth.’ & ↔ & \example{nukar-\textbf{e}} & ‘to make sb. see sth.’ \\
		\textsc{appl} & \example{mina} & ‘to laugh’ & ↔ & \example{\textbf{e}-mina} & ‘to laught about/at sth.’ \\
		\textsc{appl} & \example{rayap} & ‘to be delighted’ & ↔ & \example{\textbf{e}-rayap} & ‘to be delighted about sth.’ \\
		\midrule\midrule
		\multicolumn{6}{l}{\ili{Nivkh} (\citealt[1726ff.]{otaina:nedjalkov:2007};; \citealt[133]{nedjalkov:otaina:2013})} \\
		\midrule
		\textsc{caus} & \example{vaχtʼ-} & ‘to tear’ & ↔ & \example{vaχtʼ-\textbf{u}} & ‘to tear sth.’ \\
		\textsc{caus} & \example{veta-} & ‘to get dressed’ & ↔ & \example{veta--\textbf{u}} & ‘to dress sb.’ \\
		\textsc{recp} & \example{i-γ-} & ‘to kill sb.’ & ↔ & \example{\textbf{u}-γ-} & ‘to kill e.o.’ \\
		\textsc{recp} & \example{(i-)ŋali-} & ‘to resemble sb.’ & ↔ & \example{\textbf{u}-ŋali-} & ‘to resemble e.o.’ \\
		\lspbottomrule
	\end{tabularx}
	\caption{Examples of type 3 syncretism}
	\label{tab:ch3:type3-examples}
\end{table}

Two languages in the sample feature applicative-reciprocal type 3 syncretism\is{voice syncretism, reverse resemblance -- type 3}, the Nadahup language \ili{Hup} and the language isolate \ili{Mosetén} (both \lang{sa}). In \ili{Hup} a so-called “Interactional” prefix \example{ʔũh-} representing “the primary strategy for indicating reciprocal\is{reciprocal voice} relations” \citep[487]{epps:2008} bears resemblance to the applicative\is{applicative voice} suffix \example{-ʔũh}. \cite[500]{epps:2008} explicitly argues that “[i]n contrast to the Interactional preform \example{ʔũh-}, which often functions to decrease \isi{valency}, Applicative\is{applicative voice} \example{-ʔũh-} is a \isi{valency}-increaser”. \cite[119f.]{epps:2008} argues that \example{ʔũh} can be understood as a unit “of segmental phonological material” that is “best treated as distinct morphemes on the synchronic level, but as a diachronically unitary entity, from which the functional variants have arguably been derived\is{derivation} through grammaticalization”\is{grammaticalisation}. \citeauthor{epps:2008} adds that \example{ʔũh} in \ili{Hup} can function as the lexical root ‘sibling of opposite sex’, as a jussive or optative marker, and as an epistemic modality marker. In turn, in \ili{Mosetén} the prefix \example{ti-} serves as voice marking in the applicative\is{applicative voice} voice, while the suffix \example{-ti} can serve as voice marking in the reciprocal\is{reciprocal voice} voice -- as well as in the reflexive\is{reflexive voice} and passive\is{passive voice} voices (see \tabref{tab:ch4:pass-refl} on page \pageref{tab:ch4:pass-refl}). \cite[186, 190, 233ff.]{sakel:2004} remarks that the suffix \example{-ti} can additionally function as a verbal stem marker or play a role in cross-referential marking. \cite[311ff.]{sakel:2004} also mentions an antipassive\is{antipassive voice} function but based on the limited data she provides in her description of this phenomenon, it has not been possible to assert whether or not it complies with the antipassive\is{antipassive voice} definitions employed in this book (\sectref{def:passives-antipassives}). \cite[233]{sakel:2004} only briefly addresses the reverse resemblance\is{voice syncretism, reverse resemblance -- type 3} of the affixes \example{ti-} and \example{-ti}, noting that the latter suffix “frequently occurs in relation to applicatives\is{applicative voice}”. Observe that the \ili{Mosetén} verbs \example{ti-tyar-i-} and \example{ti-bae’-i-} included in \tabref{tab:ch3:type3-examples} appear as \example{ti-tyar-a-} and \example{ti-bae’-e-} in the original source but represent the same stem \citep[322]{sakel:2004}. For information about the regular vowel changes, see the discussion of examples (\ref{ex:Moseten:eat:a}--\ref{ex:Moseten:work:b}) on page \pageref{ex:Moseten:eat:a}. Moreover, note that the verbal stem with the meaning ‘to reach sth.’ is given by \cite[121, 391]{sakel:2004} variably as \example{chha’ch-} and \example{chhash-}. The variation between the final consonant \example{ch} and \example{sh} likely represents a regular consonant alternation \citep[48f.]{sakel:2004}, while the glottal stop in the former form has possibly been omitted by accident in the latter form.

Likewise, two languages in the sample feature causative-applicative type 3 syncretism\is{voice syncretism, reverse resemblance -- type 3}, the Sepik language \ili{Alamblak} (\lang{pn}) and the language isolate Ainu (\lang{ea}). \cite[254]{bruce:1979} explicitly argues that in \ili{Alamblak} the “[p]arallels between causative\is{causative voice} and \isi{benefactive} constructions are obvious”, as “[o]ne of the formatives is the same (\example{hay} ‘give’ prefixed as a causative\is{causative voice} and suffixed as a \isi{benefactive}) and similar semantic features characterize both”. The verbal form \example{ni} in the second \ili{Alamblak} causative\is{causative voice} example in \tabref{tab:ch3:type3-examples} is a regular word-medial allomorph\is{allomorphy} of \example{yi} ‘to go’ \citep[250]{bruce:1979}. \cite[358]{bruce:1979} even provides an example featuring both affixes, \example{hay-noh-hay} ‘to kill sb. affecting sb. else’\is{affectedness} (cf. \example{noh} ‘to die’). \ili{Ainu} has several more or less productive\is{productivity} causative\is{causative voice} suffixes, one of which is \example{-e}. The language also possesses a phonologically similar prefix \example{e-} which serves as voice marking in the applicative\is{applicative voice} voice. While the reverse resemblance described for \ili{Alamblak} and \ili{Hup} (and possible also for \ili{Mosetén}) can be explained by semantic similarities in function, the reverse resemblance\is{voice syncretism, reverse resemblance -- type 3} in Ainu is likely the result of coincidental phonological \isi{convergence}. The suffix \example{-e} has two allomorphs,\is{allomorphy} \example{-re} and \example{-te}, and \cite[475]{bugaeva:2015} notes that the three variant forms likely can be traced back to \ili{Proto-Ainu} \example{*de} of unknown origin \citep{vovin:1993}. Alternatively, \cite[15ff.]{nonno:2015} argues that the allomorphs\is{allomorphy} in question can be traced back to the verb \example{*ki} ‘to do, act’ which has grammaticalised\is{grammaticalisation} and subsequently undergone a series of assimilations\is{assimilation}: \example{*ki} > \example{*-ki} > \example{-ke} > \example{-te} > \example{-re} > \example{-e} (e.g. \example{*nukar-ki}  > \example{*nukar-ke} > \example{*nukar-te} > \example{*nukar-re} > \example{nukar-e}). The causative\is{causative voice} suffix \example{-ke} is retained in the language, but generally treated separately from \example{-e/-re/-te} \citep{bugaeva:2015}. The use of \example{-ke} as causative\is{causative voice} marking was already illustrated in the discussion of type 1a syncretism\is{voice syncretism, full resemblance -- type 1} (see \tabref{tab:ch3:type1a-examples-2} on page \pageref{tab:ch3:type1a-examples-2}). In any case, the diachronic origin of the suffix \example{-e} seems to differ from that of the prefix \example{e-} which “probably originated in the relational noun with the meaning ‘head’ that is retained as a lexical prefix \example{e-} ‘(its) head/top’” \citep[762]{bugaeva:2010}.

\newpage

Last but not least, causative-applicative type 3 syncretism\is{voice syncretism, reverse resemblance -- type 3} has been attested in the language isolate \ili{Nivkh} (\lang{ea}). In this language the suffix \example{-u} serves as causative\is{causative voice} voice marking on its own with approximately 15 verbs having a word-initial sonorant and in combination with a plosive-fricative alternation with 40 additional verbs with a word-initial plosive, e.g. \example{pil-} ‘to be big’ ↔ \example{vil-u-} ‘to make sth. (be) big’ (\citealt[1721f.]{otaina:nedjalkov:2007};; \citealt[132f.]{nedjalkov:otaina:2013}). The phonologically similar prefix \example{u-} can be used with (at most) a handful of verbs as reciprocal\is{reciprocal voice} marking. Note that the prefix \example{u-} has an allomorph\is{allomorphy} \example{v-} found with “about 30 relic verbs” (\citealt[1726ff.]{otaina:nedjalkov:2007};; \citealt[107f.]{nedjalkov:otaina:2013}), and also that the prefix often is in variation with a prefix in non-reciprocal counterpart verbs (typically \example{i-} or \example{j-}). As in the case of the \ili{Ainu} affixes \example{-e} and \example{e-}, the reverse resemblance\is{voice syncretism, reverse resemblance -- type 3} between \ili{Nivkh} \example{-u} and \example{u-} is most likely coincidental. 

A seventh language in the sample, the Tibeto-Burman language \ili{Anong} (\lang{ea}), seems to possess something akin to type 3\is{voice syncretism, reverse resemblance -- type 3} causative-reflexive syncretism, yet the extent of the phenomenon in the language is difficult to ascertain due to lack of data. \cite[24, 82]{sun:liu:2009} state that the reflexive\is{reflexive voice} suffix \example{-ɕɯ³¹} has two allomorphs,\is{allomorphy} \example{-ʂɿ³¹} and \example{-sɛ³¹} (the superscript numerals here denote tone, while the grapheme ⟨ɿ⟩ represents a lateral approximant /ɭ/ after retroflex consonant). The former allomorph\is{allomorphy} \example{-ʂɿ³¹} is phonologically identical to the causative\is{causative voice} prefix \example{ʂɿ³¹-}. However, \cite[82]{sun:liu:2009} describe reflexive\is{reflexive voice} marking in \ili{Anong} as “unproductive”,\is{productivity} and they note that it in some cases has been “fossilized\is{fossilisation} with the verb root”, and “seems to include some middle marking” or “fossilized\is{fossilisation} remains of middle marking”. As no clear (glossed and translated) reflexive\is{reflexive voice} examples of the suffix \example{-ʂɿ³¹} are given by \citeauthor{sun:liu:2009}, it is not clear whether it qualifies as reflexive\is{reflexive voice} voice marking according to the reflexive\is{reflexive voice} definition employed in this book (\sectref{def:reflexives-reciprocals}). 