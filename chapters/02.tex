\chapter{Defining voices} \label{defining-voices}
Passives,\is{passive voice} antipassives\is{antipassive voice}, reflexives\is{reflexive voice}, reciprocals\is{reciprocal voice}, anticausatives\is{anticausative voice}, causatives\is{causative voice}, and applicatives\is{applicative voice} have been the topic of much debate in the literature and much effort has been put into identifying and pinpointing their properties and features. As a result, definitions of the individual voices differ to varying degrees in the literature, yet many definitions are rather similar with regard to the manner in which they are defined. More specifically, voice definitions commonly rely on notions like an \isi{argument-adjunct distinction}, \isi{transitivity}, \isi{grammatical roles}, and/or an \isi{active voice}. However, although these notions are intuitively clear, there does not seem to be any consensus as to how they are best defined and they have consequently been endlessly debated for decades (\sectref{voices-revisited}). Rather than attempting to (re)define the various notions, they are avoided altogether in this book. Instead, the book employs alternative voice definitions that have been designed specifically for the investigation of voice syncretism, based on i) a comparison of two clausal constructions, ii) the number of semantic participants in the constructions, iii) the semantic roles of certain semantic participants in the constructions, and iv) the formal verbal marking of the constructions (\sectref{voices-redefined}). It is worth reiterating here that this book focuses exclusively on syncretism of formal voice marking found on verbs for which reason “uncoded alternations”\is{uncoded alternation} of various sorts (\citealt[178ff.]{zuniga:kittila:2019}) are excluded from the discussion.

\section{Voices revisited} \label{voices-revisited}
The use of an \isi{argument-adjunct distinction}, \isi{transitivity}, \isi{grammatical roles}, and an \isi{active voice} in voice definitions is here illustrated by an oft-cited causative\is{causative voice} definition formulated by \cite{dixon:aikhenvald:2000}. It is evident from the wider context in which the definition is found that \citeauthor{dixon:aikhenvald:2000} consider the \isi{grammatical roles} \textsc{s}, \textsc{a}, and \textsc{o} arguments (“core arguments”) in contrast to adjuncts (“peripheral arguments”). Furthermore, \citeauthor{dixon:aikhenvald:2000} argue that a prototypical\is{prototype} causative\is{causative voice} derives\is{derivation} a \isi{transitive} clause from an \isi{intransitive} clause, and the adjective “underlying” refers to a certain type of voice believed to be more basic than others -- in other words what is traditionally called an \isi{active voice}. For similar formulations and definitions, see for example \cite[1f.]{peterson:2007} on applicatives\is{applicative voice}, \cite[151f.]{siewierska:bakker:2012} on passives\is{passive voice}, and \citeauthor{heaton:2017} (\citeyear[63f.]{heaton:2017}; \citeyear[132ff.]{heaton:2020}) on antipassives\is{antipassive voice}. 

\begin{quote}
	The characteristics of a prototypical causative are:
	\begin{enumerate}[label=(\alph*)]
		\item Causative applies to an underlying \isi{intransitive} clause and forms a derived\is{derivation} \isi{transitive}.
		\item The argument in underlying \textsc{s} function (the \isi{causee}) goes into \textsc{o} function in the causative.
		\item A new argument (the \isi{causer}) is introduced, in \textsc{a} function.
		\item There is some explicit formal marking of the causative construction.
		\begin{flushright}
			\citep[13]{dixon:aikhenvald:2000}
		\end{flushright}
	\end{enumerate}
\end{quote}

The various notions mentioned here are intuitively clear and therefore widely presupposed and employed ad libitum in the literature on voice without explicit definitions. Nevertheless, the notions have been the topic of ongoing debate for decades, and there does not seem to be any agreement as to how they can best be defined. Furthermore, definitions of the notions tend to rely on \isi{language-specific} criteria which impede their use in cross-linguistic typological research. These issues are discussed in more detail in the following sections.

\subsection{Arguments and adjuncts} \label{arguments-adjuncts}
The \isi{argument-adjunct distinction} refers to a dichotomy first formulated by \cite[102]{tesniere:1959} who in clauses distinguished “actants” (i.e. \textit{les êtres ou les choses} “the beings or things”) from “circumstances” (\textit{circonstances}, i.e. the time, place, manner, etc., according to which a process unravels). The terminology of this dichotomy varies considerably in the literature, and so do definitions thereof. The term “argument” is favoured over “actant” in more recent publications (e.g. \citealt{comrie:1993, kazenin:1994, dik:1997, croft:2001, croft:2012, haspelmath:muller-bardey:2004, kulikov:2010, wichmann:2014, haspelmath:hartmann:2015}) and also often appears in the compound “core argument” (e.g. \citealt{dixon:2000, dixon:aikhenvald:2000, kazenin:2001a, van-valin:2001, van-valin:2005, peterson:2007, malchukov:2015, malchukov:2016}) to distinguish it from a “peripheral argument”, another term for “circumstance” (e.g. \citealt{dixon:2000, dixon:aikhenvald:2000, peterson:2007, malchukov:2016}). The term “adjunct” is frequently employed in both older and more recent publications alongside or instead of “peripheral argument” and “circumstance” (e.g. \citealt{vater:1978, comrie:1993, croft:2001, croft:2012, van-valin:2001, van-valin:2005, peterson:2007, wichmann:2014, haspelmath:hartmann:2015, malchukov:2015}), and so is the term “oblique” (e.g. \citealt{cooreman:1994, kazenin:1994, kazenin:2001a, haspelmath:muller-bardey:2004, peterson:2007, kulikov:2010, malchukov:2015}). For an overview of obsolete terminology, see \cite[508]{somers:1984}. 

\cite{tesniere:1959} provides a few criteria for distinguishing arguments from adjuncts.\is{argument-adjunct distinction} For instance, \cite[128]{tesniere:1959} states that arguments are indispensable for completing the semantics of a verb while adjuncts are not; and adjuncts tend to need additional prepositional marking while arguments\is{argument-adjunct distinction} do not – unless a preposition is closely associated with the verb. While the distinction itself has been highly influential, subsequent research has repeatedly shown that \citeauthor{tesniere:1959}’s criteria cannot be applied to all languages. In fact, it has proven remarkably difficult to find any adequate criteria for distinguishing arguments from adjuncts and vice versa cross-linguistically. Almost two decades after \citeauthor{tesniere:1959}’s formulation of the dichotomy, \cite[21]{vater:1978} notes that “the problem of how to differentiate between [arguments] and adjuncts has not yet been solved satisfactorily”, and similar comments are provided by \cite{somers:1984} in a paper “[o]n the validity of the [argument]-adjunct distinction\is{argument-adjunct distinction} in \isi{valency} grammar”. A decade later \cite[906]{comrie:1993} remarks that “[t]he basic intuition behind this distinction is relatively clear, though difficulties arise as soon as one tries to make it more explicit, and there is as yet no generally accepted solution to these difficulties”. Similar thoughts have been reiterated in the new millennium. \cite[30]{farrell:2005} states that “[a]lthough the conceptual distinction between argument and adjunct\is{argument-adjunct distinction} is relatively clear, the empirical basis for it is problematic”, and \cite[165]{rickheit:sichelschmidt:2007} observe that “[t]he problem with the dichotomy is that the criteria for classifying an [argument or adjunct]\is{argument-adjunct distinction} are anything but clear”. As discussed by \cite[46ff.]{haspelmath:hartmann:2015}, many approaches to the differentiation of arguments and adjuncts are based on criteria pertaining to semantic entailment or \isi{verb-specificity} of various kinds which are notoriously problematic in a cross-linguistic context.

The difficulties in distinguishing arguments from adjuncts cross-linguistically have prompted \cite[4]{haspelmath:2014} to speculate that “it may be that no good cross-linguistic definition of arguments and adjuncts as syntactic elements that largely coincides with our intuitions will be possible”. Faced with this problem, it has been suggested sporadically in the literature that the distinction between different clausal elements is not necessarily binary. For instance, \cite[1]{wichmann:2014} argues that “instead of requiring a sharp distinction we may satisfy ourselves with a gradient one”. An early advocate of a non-binary approach is \cite[140f.]{matthews:1981} who proposes a trichotomy distinguishing so-called “non-peripheral complements”, “non-complements”, and “peripherals”. \cite[524]{somers:1984} extends this trichotomy to a hexachotomy encompassing “integral complements”, “obligatory complements”, “optional complements”, “middles”, “adjuncts”, and “extraperipherals”. More recently, \cite{forker:2014} has proposed “a canonical\is{canonicity} approach to the argument/adjunct distinction”\is{argument-adjunct distinction} (in the spirit of \citealt{corbett:2005, corbett:2007, corbett:2013, brown:al:2013}) in which canonical\is{canonicity} arguments and canonical adjuncts represent opposite poles on a continuum. Canonicity\is{canonicity} in this approach is determined according to five criteria \citep[28ff.]{forker:2014} and if it is assumed that each criterion can be either argument-like or adjunct-like,\is{argument-adjunct distinction} \citeauthor{forker:2014}’s polychotomy has 32 distinctions. Nevertheless, although such polychotomous distinctions are undoubtedly more complex than a binary \isi{argument-adjunct distinction}, both kinds of distinctions are subject to the same problems. For instance, \citeauthor{forker:2014}’s five criteria are either based on the problematic concept of \isi{verb-specificity} mentioned above \citep[46ff.]{haspelmath:hartmann:2015} or \isi{language-specific} features. Indeed, \cite[36]{forker:2014} explicitly remarks that not all criteria in her canonical\is{canonicity} approach necessarily apply to all languages which impedes cross-linguistic comparison.

A notable alternative to the \isi{argument-adjunct distinction} is the \isi{microrole} approach developed for the Leipzig Valency\is{valency} Classes Project to facilitate the cross-linguistic comparison of 70 verbal meanings and their syntactic structures in 30 languages \citep{hartmann:al:2013, malchukov:comrie:2015a, malchukov:comrie:2015b}. In this project the microroles\is{microrole} for each of the 70 verbal meanings were defined as comparative concepts\is{comparative concept} (e.g. ‘thinker’ and ‘thought content’ for the meaning ‘to think’) which meant that problems pertaining to argumenthood and adjuncthood described above could be avoided. For example, in Modern Standard Arabic\il{Arabic, Modern Standard} (\lang{af}) ‘thought content’ is marked by the preposition \example{fī} and intuitively resembles an adjunct \citep{kasz:2013} but in the Oceanic language \ili{Xârâcùù} (\lang{pn}) ‘thought content’ is seemingly not marked differently from other presumed arguments \citep{moyse-faurie:2013}. This approach is a satisfactory solution for typological studies of specific sets of verbs but is not readily applicable to studies which are unrestricted in their scope regarding verbal semantics, including this book. However, the \isi{microrole} approach importantly shows that an \isi{argument-adjunct distinction} is not necessarily a prerequisite for cross-linguistic investigations of verbs. In the spirit of this approach, an attempt has been made to avoid the \isi{argument-adjunct distinction} in the voice definitions presented in this book.

\subsection{Transitivity and valency} \label{transitivity-valency}
Transitivity\is{transitivity} is omnipresent in linguistics and perhaps one of the most debated phenomena within the field. Indeed, \cite[142]{lazard:2002} notes that \isi{transitivity} “belongs to the oldest tradition of grammatical thinking in the Western world”, and \cite[346]{kittila:2010} remarks that \isi{transitivity} is “one of the core areas of linguistics”. Furthermore, \cite[128]{dixon:1972} argues that “[a]ll languages appear to have \isi{transitive} and \isi{intransitive} sentences” (see also \citealt[102]{dixon:1979}; \citeyear[6]{dixon:1994}; \citeyear[30]{dixon:2000}; \citealt[2]{dixon:aikhenvald:2000}), and \cite[1]{hopper:thompson:1982} state that “[i]n many languages (and perhaps covertly in all languages) the \isi{transitivity} relationship lies at the explanatory core of most grammatical processes”. In fact, as observed by \cite[2]{nass:2007}, the notion of \isi{transitivity} appears to be so deeply rooted in linguistic tradition that it is “often used in a way which takes its content for granted, without any attempt at a precise definition”, and “there is no universally accepted definition which captures precisely the range of functions”. Instead, it is commonly assumed that a general abstract idea of the notion suffices.

\cite[471]{lapolla:al:2011} describe one conceptualisation of such an abstract idea in the following manner: “The traditional syntactic definition of \isi{transitivity} says that a language has one or more constructions where two arguments are given special status in the clause as core (obligatory) arguments, as opposed to only one argument\is{argument-adjunct distinction} being given that status” (see also \citealt[143]{croft:2003}). This approach essentially represents an \isi{intransitive}-\isi{transitive} dichotomy: clauses with one argument are \isi{intransitive} while clauses with more than one argument are \isi{transitive}. The perhaps most prominent advocate of this approach is \cite[116]{dixon:2010b} who has stated that “[o]ne point to be stressed -- and always kept in mind -- is that \textit{\isi{transitivity} is a syntactic matter}” and that “[w]hen a clause is said to have a certain \isi{transitivity} value, and when a verb is said to show certain \isi{transitivity} possibilities, these are syntactic -- not semantic -- specifications” (original italics). Another notion similar to \isi{transitivity} is \isi{valency}, which dates back at least to the late 1940s (e.g. \citealt[114f.]{de-groot:1949}) though its consolidation as a linguistic term is generally attributed to \cite[238, 670]{tesniere:1959}, who defines it as the number of arguments a verb is “susceptible to govern” (\textit{susceptible de régir}). Valency\is{valency} and \isi{transitivity} differ in this respect, as clauses with one argument are valent but not \isi{transitive}: intransitives\is{intransitive} are \isi{monovalent}, (mono)transitives\is{transitive} \isi{divalent}. However, in light of the discussion concerning the \isi{argument-adjunct distinction} in the previous section, syntactic approaches to \isi{transitivity} and \isi{valency} are inherently problematic if argumenthood cannot be properly defined. Even if it is assumed that arguments can be readily distinguished from adjuncts cross-linguistically, \cite[544]{haspelmath:2011a} argue that other problems ensue: “[i]n individual languages, precise criteria for distinguishing two major clause types (‘transitive’, ‘intransitive’) can be found (e.g., particular argument-indexing patterns, passivizability, or even inflectional classes), but they are quite diverse and not generalizable across languages”.

A semantic approach to \isi{transitivity} has been pioneered notably by \cite[253]{hopper:thompson:1980}, who argue that the \isi{transitivity} of a clause can be established according to ten features, each of which can be given a “high” or “low” value: i) participants, ii) \isi{kinesis}, iii) \isi{aspect}, iv) \isi{punctuality}, v) volitionality,\is{volition(ality)} vi) affirmation, vii) mode, viii) \isi{agency}, ix) \isi{affectedness}, and x) \isi{individuation}. The more “high” features a clause has, “the more Transitive\is{transitive} it is – the closer it is to \textsc{cardinal} \isi{transitivity}” (original small caps). In a similar spirit, \citeauthor{givon:2001a} (\citeyear[209]{givon:2001a}; \citeyear[93]{givon:2001b}) highlights the importance of \isi{agency}, \isi{affectedness}, and \isi{perfectivity} in particular: a prototypical\is{prototype} \isi{transitive} event involves “a volitional,\is{volition(ality)} controlling, active, initiating \isi{agent} responsible for the event” (“the salient cause”), “a non-volitional,\is{volition(ality)} inactive, non-controlling \isi{patient} that registers the event’s changes-of-state” (“the salient effect”), and its verb “codes an event that is telic (compact), perfective\is{perfectivity} (bounded), sequential (non-perfect) and realis (non-hypothetical)”. In turn, \cite[30]{nass:2007} places emphasis on volitionality,\is{volition(ality)} instigation, and \isi{affectedness}. According to her Maximally Distinct Arguments Hypothesis, “a prototypical\is{prototype} \isi{transitive} clause is one where the two participants are maximally semantically distinct in terms of their roles in the event described by the clause”. More precisely, the two participants (\isi{agent} and \isi{patient}) are maximally distinct when the \isi{agent} is volitional,\is{volition(ality)} instigating, and unaffected,\is{affectedness} and the \isi{patient} non-volitional,\is{volition(ality)} non-instigating, and affected\is{affectedness} \citep[44]{nass:2007}. Although these semantic approaches to \isi{transitivity} are certainly more nuanced than syntactic approaches, they are not unproblematic either, primarily due to their reliance on “fuzzy” categorisation\is{fuzzy categorisation} \citep{geeraerts:1989}. Semantic approaches to \isi{transitivity} tend to rely “on semantic prototype\is{prototype} definitions that do not allow precise delimitation of \isi{transitive} clauses from non-transitive clauses” \citep[544]{haspelmath:2011a} and “it is generally difficult to justify such prototypes,\is{prototype} and prototypical\is{prototype} definitions cannot be used for formulating testable generalizations” \citep[313]{haspelmath:2016b}].
 
For the reasons above, no attempt will be made to (re)define \isi{transitivity} here. That is not to say that the various criteria according to which semantic \isi{transitivity} is often defined are not of relevance themselves, only that they will not be treated collectively as defining criteria of an abstract notion of \isi{transitivity} but treated individually wherever relevant.  

\subsection{Grammatical roles} \label{grammatical-roles}
Purported arguments (in contrast to adjuncts) are often classified according to their semantic and/or syntactic role in a clause. Traditionally, arguments have been classified as subjects, direct objects\is{object, direct}, and indirect objects\is{object, indirect} -- notions originally modelled on Indo-European languages and strongly associated with grammatical \isi{case}: subject with the nominative case\is{case, nominative}, direct object\is{object, direct} with the accusative case\is{case, accusative}, and indirect object\is{object, indirect} with the dative case\is{case, dative}. However, it is well-known that this traditional classification does not perform well cross-linguistically because \isi{case} marking does not correlate with presumed subject-\is{subject} and objecthood in many languages -- or lacks altogether. To account for such cross-linguistic variation, \cite[59, 128]{dixon:1972} introduced the notions \textsc{s}, \textsc{a}, and \textsc{o} (alternatively, \textsc{p}), which he defined “intransitive subject”, “transitive subject”, and “transitive object”, respectively. This set of notions has later been complemented by notions relevant for ditransitives\is{ditransitive}: \textsc{t} and \textsc{g} (alternatively, \textsc{r}) corresponding to a ditransitive direct object\is{object, direct} and a ditransitive indirect object\is{object, indirect}, respectively \citep{croft:1990}. These five notions have become widespread in linguistics, yet their meanings are commonly taken for granted, and \cite[536]{haspelmath:2011a} argues that “it does not seem to be widely recognized yet that there are quite different and incompatible definitions of the \textsc{saptr} terms in the literature”.

\cite{haspelmath:2011a} discerns three major approaches to the definitions of \textsc{s}, \textsc{a}, \textsc{p}, \textsc{t}, and \textsc{r} in the literature: a Dixonian approach, a Comrian approach, and a Bickelian approach. The first approach is epitomised by \citeauthor{dixon:1972}’s (\citeyear{dixon:1972, dixon:1979, dixon:1994, dixon:2010a, dixon:2010b}) definitions of the notions based on \isi{transitivity} already mentioned above. In the Comrian approach the definitions of \textsc{a} and \textsc{p} are based more specifically on a so-called “prototypical\is{prototype} \isi{transitive} situation” in which the semantic \isi{agent} is regarded as \textsc{a} and the semantic \isi{patient} as \textsc{p} (e.g. \citealt[105]{comrie:1981}; \citeyear[11]{comrie:1989}). In turn, a prototypical\is{prototype} \isi{transitive} situation -- or a “typical two-argument clause”\is{argument-adjunct distinction} -- involves a physical effect verb like ‘to kill sb.’ and ‘to break sth.’ (\citealt[545ff.]{haspelmath:2011a};; see also \citealt{andrews:1985, andrews:2007}; \citealt{lazard:2002}; \citealt{creissels:2006}). Likewise, \textsc{t} and \textsc{r} can be defined as “the theme and the recipient of typical physical transfers verbs of possession (‘give’, ‘lend’, ‘send’, etc.)” (\citealt[558]{haspelmath:2011a}; see also \citealt{malchukov:al:2010}). By contrast, \textsc{s} can be defined as the sole argument in a one-argument clause, or as any argument that is marked or behaves like the sole argument in a one-argument clause \citep[549f.]{haspelmath:2011a}. Finally, in the Bickelian approach the notions \textsc{s}, \textsc{a}, \textsc{p}, \textsc{t}, and \textsc{r} represent generalised semantic roles which are not restricted to a specific type of verb \citep{nichols:2008, bickel:nichols:2009, bickel:al:2010, bickel:2011, witzlack-makarevich:2011}. Agents\is{agent} are characterised as causers of events, volitional,\is{volition(ality)} sentient, and exist independently of events; while patients\is{patient} are typically affected\is{affectedness} by events, stationary relative to movement of other semantic participants, and/or undergo changes of state or in experience (\citealt[554]{haspelmath:2011a}; see also \citealt{dowty:1991}; \citealt[384]{bickel:al:2010}). In turn, \textsc{a} and \textsc{p} are the more \isi{agent}-like and less \isi{agent}-like arguments of a two-place predicate, respectively; and \textsc{r} and \textsc{t} the less \isi{patient}-like and more \isi{patient}-like arguments of the non-\isi{agent}-like arguments of a three place-predicate, respectively (\citealt[307]{bickel:nichols:2009}; \citealt[384]{bickel:al:2010}). 

The Dixonian approach is inherently problematic due to its reliance on notions of arguments (in contrast to adjuncts) and \isi{transitivity} which have been discussed in the previous two sections. In turn, the Bickelian approach is essentially subject to the same criticism as semantic \isi{transitivity} mentioned in the previous section because its definitions of agents\is{agent} and patients\is{patient} are based on \isi{fuzzy categorisation}. It is, for instance, not clear why the specific criteria for \isi{agent}- and patienthood\is{patient} are chosen over others, how they are assessed consistently cross-linguistically, nor how semantic participants with presumably equal status are treated \citep[554ff.]{haspelmath:2011a}. The Comrian approach is not problematic per se, but the approach has not been adopted in this book due to its restrictive nature in terms of verbal semantics. While this is not an issue for \isi{alignment} typology (in relation to which \citeauthor{haspelmath:2011a} discusses \textsc{s}, \textsc{a}, \textsc{p}, \textsc{t}, and \textsc{r} in the first place), it will become evident in subsequent chapters that many examples presented in this book involve verbs that hardly qualify as prototypical\is{prototype} \isi{transitive} situations. Moreover, as demonstrated later in this chapter, the notions are not necessarily a prerequisite for voice definitions. In this book only two grammatical (or rather semanti) roles will be needed: a \textsc{causer}\is{causer} defined as a \isi{semantic participant} causing another \isi{semantic participant} to do an action, and an \textsc{agent}\is{agent} defined minimally as the initiator of an action. While the latter definition might seem overly simplistic, the semantic role is only relevant to the definitions of passives\is{passive voice} and antipassives\is{antipassive voice} for which it suffices (\sectref{def:passives-antipassives}).

\subsection{Active voice} \label{active-voice}
The notion of voice is prevalent in the literature and the tradition of distinguishing between different kinds of voices can be traced to the grammatical traditions pertaining to Classical \ili{Greek} and \ili{Sanskrit} (see, e.g., \citealt[369]{kulikov:2010}; \citealt[1f.]{zuniga:kittila:2019}). Voice can essentially be perceived as a category (or “super-category”, \citealt[1140]{haspelmath:muller-bardey:2004}) of one or more clausal structures defined according to pragmatic, semantic, and/or syntactic criteria. It is widely assumed that one voice is somehow more neutral and/or more frequent in discourse than other voices. This voice is traditionally called the \isi{active voice}, but other denotations have become increasingly common in the literature as well, often characterised by the adjective “basic” (e.g. \citealt{comrie:1989, melchuk:1993, cooreman:1994, dixon:2000, dixon:aikhenvald:2000, malchukov:2015, malchukov:2016, haspelmath:hartmann:2015}), “unmarked” (e.g. \citealt{kazenin:2001a, haspelmath:muller-bardey:2004}), or “neutral” (e.g. \citealt{kulikov:2010}). 

Despite its omnipresence in linguistics, the \isi{active voice} is rarely defined nor explicitly discussed, but it is generally assumed to be a highly productive\is{productivity} and non-restricted clause type which is more frequent and somehow less marked than others (e.g. \citealt[19ff.]{comrie:1988}). While a definition like this is seemingly straightforward in theory, it can be difficult to apply in practice. Firstly, for some languages it is difficult to argue that one clause type is more frequent than others. This is true for many languages with so-called “\isi{symmetrical voice}” (\citealt[120ff.]{zuniga:kittila:2019}). For instance, in the Malayo-Sumbawan language \ili{Madurese} (\lang{pn}) “the distribution of actor voice and object voice fluctuates between roughly 50/50 to a 40/60 split” (\citealt[257, 311]{davies:2010}). Secondly, in some languages the clause type intuitively assumed to represent an \isi{active voice} is not necessarily less marked in terms of morphosyntactic marking compared to other clause types (\sectref{resemblance-type1a}). Thirdly, it can be difficult to properly measure and compare \isi{productivity} cross-linguistically because few descriptive grammars include detailed information on the matter. For these reasons, the voice definitions presented in this book will not rely on the notion of an \isi{active voice}. Instead, the definitions will be based on a comparison between any two clausal constructions that fulfill a number of criteria specified later, and no construction will be required to be more neutral than the other.

\section{Voices redefined} \label{voices-redefined}
As demonstrated in the previous sections, the notions of an \isi{argument-adjunct distinction}, \isi{transitivity}, \isi{grammatical roles}, and an \isi{active voice} are difficult to define and are consequently avoided in all the voice definitions presented in the following sections. The definitions of the passive\is{passive voice}, antipassive\is{antipassive voice}, reflexive\is{reflexive voice}, reciprocal\is{reciprocal voice}, anticausative\is{anticausative voice}, causative\is{causative voice}, and applicative\is{applicative voice} voices are instead based upon i) a comparison of two clausal constructions, ii) the number of semantic participants\is{semantic participant} in the constructions, iii) the semantic roles of certain semantic participants\is{semantic participant} in the constructions, and iv) the formal verbal marking of the constructions. The definitions represent comparative concepts\is{comparative concept} avoiding \isi{language-specific} criteria designed according to principles of cross-linguistic comparison outlined and advocated notably by \citeauthor{haspelmath:2010a} (\citeyear{haspelmath:2010a, haspelmath:2010b, haspelmath:2011a, haspelmath:2011b}; \citeyear{haspelmath:2014, haspelmath:2016a, haspelmath:2016b, haspelmath:2018}) as well as \citeauthor{croft:1990} (\citeyear[11f.]{croft:1990}; \citeyear[88]{croft:1995}; \citeyear[13f.]{croft:2003}), \cite{dryer:1997, dryer:2016}, \citet[22ff.]{givon:2001a}, \citet[10ff.]{song:2001}, and \cite{stassen:2010}. The definitions have transparent rigid boundaries and are based on as few criteria as possible to allow for maximum cross-linguistic diversity and because “comparative concepts\is{comparative concept} based on fewer factors seem to have a greater chance of leading to deeper insights” \citep[677]{haspelmath:2010a}.

\subsection{Principles} \label{def:principles}
The voice definitions presented in the following sections are all based on a comparison between two clausal constructions because it is difficult to argue that any given construction represents a passive\is{passive voice}, antipassive\is{antipassive voice}, reflexive\is{reflexive voice}, reciprocal\is{reciprocal voice}, anticausative\is{anticausative voice}, or applicative\is{applicative voice} voice if it is considered entirely in isolation. For the purpose of the following discussions, a clausal construction will henceforth be called a \textsc{diathesis}\is{diathesis}. This term has notably been employed by the Leningrad-St. Petersburg Typology group according to which a “[d]iathesis is determined as a pattern of mapping of semantic arguments onto syntactic functions” (\citealt[370]{kulikov:2010}; for a similar and more recent use of the term, see \citealt[4]{zuniga:kittila:2019}). However, it will become evident in the following sections that the link between semantic participants\is{semantic participant} and their syntactic functions is of little importance to the voice definitions presented in this book, for which reason \citeauthor{kulikov:2010}’s (\citeyear{kulikov:2010}) definition is not adopted here. Instead, the term \textsc{diathesis}\is{diathesis} is intended to be a neutral denotation for a clausal construction which can be conceptualised by a syntactic and a semantic level. As visualised in \figref{fig:ch2:diathesis}, the semantic level of a \isi{diathesis} features a semantic action alongside one or more semantic participants\is{semantic participant}. As the topic of this book is syncretic voice marking, it is naturally required that the semantic action is expressed syntactically (hence the solid line in the figure). By contrast, the \isi{semantic participant}(s) can be expressed either syntactically or remain implicit and deductible from wider context (hence the dotted line in the figure). Each \isi{semantic participant} may have one or more semantic referents.\is{semantic referent} For instance, the \isi{semantic participant} ‘dog’ has one \isi{semantic referent} (one dog) while the \isi{semantic participant} ‘dogs’ has multiple referents\is{semantic referent} (\textit{n} number of dogs). This distinction between semantic participants\is{semantic participant} and their referents\is{semantic referent} is relevant for the definitions of the reflexive\is{reflexive voice} and reciprocal\is{reciprocal voice} voices (\sectref{def:reflexives-reciprocals}).

\begin{figure}
	\caption{Syntactic and semantic model of a diathesis}
	\label{fig:ch2:diathesis}
	\begin{tabular}{c c c c c}
		& \multirow{3}{*}{\Big\langle} & Syntactic level & [  ] & [  ] \\
		Diathesis & & & | & ⋮ \\
		& & Semantic level & Action & Participant(s) \\
		& & & & Referent(s)\is{semantic referent} \\
	\end{tabular}
\end{figure}

The abstract interrelationship between two diatheses\is{diathesis} being compared to each other will henceforth be known as a \textsc{diathetic relation}\is{diathetic relation} while the two diatheses\is{diathesis} themselves will be known arbitrarily as \diath{1} and \diath{2}. To ensure meaningful comparison of two diatheses\is{diathesis} in a \isi{diathetic relation}, it is required that the actions in \diath{1} and \diath{2} have corresponding meanings on the semantic level and share the same verbal stem on the syntactic level (to avoid, e.g., \isi{suppletion}). A \isi{diathetic relation} qualifies as a passive\is{passive voice}, antipassive\is{antipassive voice}, reflexive\is{reflexive voice}, reciprocal\is{reciprocal voice}, anticausative\is{anticausative voice}, causative\is{causative voice}, or applicative\is{applicative voice} \textsc{voice relation}\is{voice relation} if it complies with one of the respective voice definitions presented later (for an overview of these definitions, see \sectref{def:overview}). In a \isi{voice relation} either \diath{1} or \diath{2} qualifies as a passive\is{passive voice}, antipassive\is{antipassive voice}, reflexive\is{reflexive voice}, reciprocal\is{reciprocal voice}, anticausative\is{anticausative voice}, causative\is{causative voice}, or applicative\is{applicative voice} \textsc{voice}, as further specified in the respective voice definitions. In other words, a \isi{voice relation} refers to a specific kind of \isi{diathetic relation}, and a voice refers to a specific kind of \isi{diathesis}. Thus, in this book the term \textsc{voice} is strictly used in reference to passives\is{passive voice}, antipassives\is{antipassive voice}, reflexives\is{reflexive voice}, reciprocals\is{reciprocal voice}, anticausatives\is{anticausative voice}, causatives\is{causative voice}, and applicatives\is{applicative voice} and not to any other kinds of diatheses\is{diathesis}. The definitions of the seven voices are based on one of two types of \isi{diathetic relation}: in one type \diath{1} and \diath{2} feature the exact same number of semantic participants\is{semantic participant}, and in another type \diath{2} features exactly one \isi{semantic participant} more than \diath{1}. These two types are visualised in \figref{fig:ch2:diathetic-relations}. The bidirectional arrow in the figure indicates that \diath{1} and \diath{2} are compared on par with each other, and neither is considered “derived”.\is{derivation} Both \diath{1} and \diath{2} are here represented by the semantic level alone: \textsc{v} denotes a semantic action and \textsc{p} a \isi{semantic participant}. Subscript \textit{n} denotes a finite number of semantic participants\is{semantic participant}. It will become evident in the following sections that the first type of \isi{diathetic relation} (fig. \ref{fig:ch2:diathetic-relations}a) underlies the definitions of the passive\is{passive voice} and antipassive\is{antipassive voice} voices, while the second type of \isi{diathetic relation} (fig. \ref{fig:ch2:diathetic-relations}b) underlies the definitions of the reflexive\is{reflexive voice}, reciprocal\is{reciprocal voice}, causative\is{causative voice}, anticausative\is{anticausative voice}, and applicative\is{applicative voice} voices. 

\begin{figure}
	\caption{Types of diathetic relations}
	\label{fig:ch2:diathetic-relations}
	\begin{tabular}{l l l l}
		a. & \diath{1} (\textsc{v}, \textsc{p}\textsubscript{\textit{n}}) & ↔ & \diath{2} (\textsc{v}, \textsc{p}\textsubscript{\textit{n}}) \\
		b. & \diath{1} (\textsc{v}, \textsc{p}\textsubscript{\textit{n}}) & ↔ & \diath{2} (\textsc{v}, \textsc{p}\textsubscript{\textit{n}+1})
	\end{tabular}
\end{figure}

A difference in the verbal marking between \diath{1} and \diath{2} in a \isi{voice relation} constitutes \textsc{voice marking} and minimally consists of an affix. The term \textsc{affix} is here used in a generic sense and can refer to any marking on the verb (including suprasegmental features and \isi{reduplication}) or phonologically dependent on the verb (including clitics). A detailed discussion of what exactly constitutes an affix (or a clitic for that matter) lies beyond the scope of this book, and in practice the book relies on the analyses and word boundaries presented and preferred by the authors of the grammars and other publications from which the data for the typological survey in this book have been sourced (see Appendix A). Furthermore, observe that verbal marking that forms part of a language’s formal \isi{agreement} system is not regarded as voice marking. This restriction is adopted to limit the scope of the book and because such marking by itself is not traditionally considered a defining characteristic of voices. Thus, verbal marking dedicated solely to, say, person and/or number \isi{agreement} is not considered voice marking per se. It is important to note, however, that this restriction does not necessarily exclude \isi{fusional voice marking} found in, for example, many Indo-European languages. Consider, for instance, the Classical \ili{Greek} first and third person “active” markers \example{-ō} and \example{-ei} and the contrasting first and third person “middle”\is{middle voice} markers \example{-omai} and \example{-etai} (the language has many more such pairs, \citealt[169]{zuniga:kittila:2019}). If the markers in each pair are compared to each other in accordance with the principles outlined above (i.e. \diath{1} \example{-ō} ↔ \diath{2} \example{-omai} and \diath{1} \example{-ei} ↔ \diath{2} \example{-etai}) there is a difference in the verbal marking between \diath{1} and \diath{2} in each case, even though no single middle\is{middle voice} (or active) marker can be discerned. If a comparison between \diath{1} and \diath{2} fulfills any other criteria of a given voice definition, it qualifies as that particular voice -- and the associated verbal marking qualifies as voice marking. Accordingly, in the case of Classical \ili{Greek}, each of the language’s many fusional middle\is{middle voice} markers qualifies as passive\is{passive voice}, reflexive\is{reflexive voice}, and anticausative\is{anticausative voice} voice marking. In other words, the middle\is{middle voice} markers serve other function than merely indicating \isi{agreement}. The exact number of passive\is{passive voice}, reflexive\is{reflexive voice}, and anticausative\is{anticausative voice} voice markers in the language is irrelevant because one marker (say \example{-omai} or \example{-etai}) suffices to determine whether or not the language features any voice syncretism (e.g. passive-reflexive syncretism: \example{loú-ō} ‘I wash sth.’ ↔ \example{loú-omai} ‘I am washed [by sb.]’ or ‘I wash myself’). 

By contrast, “\isi{symmetrical voice}” \citep[120ff.]{zuniga:kittila:2019} is generally an intrinsic part of a language’s formal \isi{agreement} and is therefore not regarded as voice marking in this book. Languages featuring \isi{symmetrical voice} possess two (e.g. direct-inverse\is{direct-inverse marking} or \isi{Indonesian-type marking}) or more (e.g. \isi{Philippine-type marking}) types of diatheses\is{diathesis} with roughly equal status but with different marking patterns \citep[7]{arka:ross:2005}, and the use of a given \isi{diathesis} is based on various \isi{language-specific} criteria related to semantic participants\is{semantic participant} and their \isi{agreement}. By illustration, in the language isolate \ili{Movima} (\lang{pn}) direct marking (\example{-na} or \example{<a>}) is employed “when two third-person participants are ranked equally in terms of discourse status”, whereas inverse marking (\example{-kay}) is “restricted to the situation in which the \isi{undergoer} outranks the \isi{actor} with regard to person and discourse prominence” \citep[265]{haude:2012}. Likewise, in Austronesian alignment\is{alignment, Austronesian} specific voices are associated with certain syntactic marking patterns for semantic participants\is{semantic participant}. For instance, in the Greater Central Philippine language \ili{Tagalog} (\lang{pn}) an \isi{actor} is syntactically marked nominative\is{case, nominative} and a \isi{patient} genitive\is{case, genitive} in the “Actor Voice”, while their case marking\is{case} is swapped in the “Patient Voice”. In other marking patterns both semantic participants\is{semantic participant} are marked genitive,\is{case, genitive} while a location and an instrument are marked nominative\is{case, nominative} in the “Locative Voice” and the “Instrumental Voice”, respectively \citep[125ff.]{zuniga:kittila:2019}. The closely related language \ili{Cebuano} (\lang{pn}) is largely similar to \ili{Tagalog}, and \cite[40]{tanangkingsing:2009} notes that “through these voice forms we can generally predict the semantic role of the nominative\is{case, nominative} argument”. Nevertheless, this does not mean that languages with symmetrical marking\is{symmetrical voice} do not feature any of the voices of focus in this book. On the contrary, the majority of the Austronesian languages included in the language sample feature applicative\is{applicative voice}, causative\is{causative voice}, and/or reflexive\is{reflexive voice} voices (see Appendix C). For instance, in the Northern Luzon language \ili{Dupaningan Agta} (\lang{pn}) the prefix \example{i-} characterising the \isi{language-specific} Theme Voice can serve as voice marking in the applicative\is{applicative voice} voice when the prefix is added onto verbs in the \isi{language-specific} “Locative Voice”, e.g. \example{alap-an} ‘to get sth.’ ↔ \example{i-alap-an} ‘to get sth. for sb.’ \citep[157ff., 161ff.]{robinson:l:2011}.

Finally, for a certain voice to be attested in a language, its voice marking must be productive,\is{productivity} yet \isi{productivity} can be difficult to measure in a uniform manner cross-linguistically (\sectref{active-voice}). For the sake of consistency, voice marking is considered productive\is{productivity} if it is attested with \textit{more than one} verb in a given language. As a result, some cases of voice marking labelled unproductive\is{productivity} in the literature are here considered productive.\is{productivity} This broad inclusion is considered an advantage, however, as low-frequent voices can prove interesting in their own right, for instance from a diachronic perspective. Nevertheless, for the sake of transparency, syncretic voice marking labelled or described as unproductive\is{productivity} for one or more voices in the literature is duly marked by an obelus (†) in Appendix C.

\subsection{Passives and antipassives} \label{def:passives-antipassives}
As noted in the previous section, passive\is{passive voice} and antipassive\is{antipassive voice} voice relations\is{voice relation} are characterised by two diatheses\is{diathesis} (\diath{1} and \diath{2}) that both have the same number of semantic participants\is{semantic participant}, as visualised in \figref{fig:ch2:diathetic-relations}a on page \pageref{fig:ch2:diathetic-relations} and reproduced here for convenience. This interdiathetic comparison serves as the foundation for the passive\is{passive voice} and antipassive\is{antipassive voice} definitions presented in this section and differentiates them from the reflexive\is{reflexive voice}, reciprocal\is{reciprocal voice}, causative\is{causative voice}, anticausative\is{anticausative voice}, and applicative\is{applicative voice} voice relations\is{voice relation} discussed in subsequent sections. The interdiathetic comparison also reflects the general understanding of the passive\is{passive voice} and antipassive\is{antipassive voice} voices in the literature where some \isi{semantic participant} in either one of the voices is often described as being somehow demoted\is{demotion} and/or omitted syntactically yet semantically implicit.

\smallskip

\noindent
\begin{center}
	\diath{1} (\textsc{v}, \textsc{p}\textsubscript{\textit{n}}) ↔ \diath{2} (\textsc{v}, \textsc{p}\textsubscript{\textit{n}})
\end{center}

\smallskip

This book maintains a fundamental distinction between \textsc{absolute} passive\is{passive voice} and antipassive\is{antipassive voice} voices on the one hand, and \textsc{non-absolute} passive\is{passive voice} and antipassive\is{antipassive voice} voices on the other hand. The former kind of passive\is{passive voice} voice is generally known as “agentless passive\is{passive voice}” in the literature (e.g. \citealt[7]{dixon:aikhenvald:2000}; \citealt[374]{kulikov:2010}) but is here called absolute passive\is{passive voice} by analogy with the absolute antipassive\is{antipassive voice} voice (e.g. \citealt[1131]{haspelmath:muller-bardey:2004}; \citealt[98]{malchukov:2015}). The absolute (or agentless) passive\is{passive voice} and the absolute (or absolutive) antipassive\is{antipassive voice} voice relations\is{voice relation} basically involve one \isi{semantic participant} which cannot be expressed syntactically, unlike the non-absolute passive\is{passive voice} and the non-absolute antipassive\is{antipassive voice} voice relations\is{voice relation} which involve semantic participants\is{semantic participant} that can all be expressed syntactically, though one \isi{semantic participant} is less likely to be so. The absolute passive\is{passive voice} and antipassive\is{antipassive voice} are here discussed first, whereas the non-absolute passive\is{passive voice} and antipassive\is{antipassive voice} are discussed further below. The following absolute passive\is{passive voice} definition establishes both an absolute passive\is{passive voice} \isi{voice relation} and an absolute passive\is{passive voice} voice, while the absolute antipassive\is{antipassive voice} definition establishes both an absolute antipassive\is{antipassive voice} \isi{voice relation} and an absolute antipassive\is{antipassive voice} voice. As noted in \sectref{grammatical-roles}, an \textsc{agent}\is{agent} is defined minimally as the initiator of an action. However, observe that all verbs do not involve a readily identifiable \isi{agent}, including for example \isi{experiencer} verbs (e.g. ‘to fear’) which feature a stimulus and \isi{experiencer}, neither of which can really be said to initiate an action. Although such verbs can appear in \isi{language-specific} passive\is{passive voice} or antipassive\is{antipassive voice} constructions in some languages (cf. \ili{Danish} \example{frygte-s} ‘to be feared [by sb.]’), they do not comply with the passive\is{passive voice} and antipassive\is{antipassive voice} definitions presented here and are thereby excluded from the discussion. Nevertheless, in terms of voice marking and voice syncretism, this exclusion is unlikely to have any noticeable effect on the findings of this book. Even though some verbs in some languages might be excluded by the passive\is{passive voice} and antipassive\is{antipassive voice} definitions presented here, languages feature many other verbs that can potentially comply with the definitions -- given that the other criteria of the definitions are fulfilled as well (cf. \ili{Danish} \example{dræbe-s} ‘to be killed [by sb.]’). Thus, the minimal definition of an \isi{agent} employed in the passive\is{passive voice} and antipassive\is{antipassive voice} definitions presented here suffices for capturing passive\is{passive voice} and antipassive\is{antipassive voice} voice marking cross-linguistically (like the suffix \example{-s} in Danish). 

\smallskip

\noindent
\begin{center}
	\begin{minipage}{0.80\textwidth}
		\textbf{Definition of absolute passive} \newline
		An \textsc{absolute passive voice relation} denotes a \isi{diathetic relation} involving two diatheses\is{diathesis}, \diath{1} and \diath{2}, if a comparison between these diatheses\is{diathesis} fulfills the criteria below; while an \textsc{absolute passive voice} denotes \diath{2} in the beforementioned \isi{voice relation}.
		\begin{enumerate}[label=\roman*)]
			\item \diath{1} and \diath{2} feature the same number of semantic participants\is{semantic participant}, one of which is an \isi{agent}.
			\item One \isi{semantic participant} in \diath{2} cannot be expressed syntactically.
			\item The abovementioned \isi{semantic participant} is the \isi{agent}.
			\item The verbs in \diath{1} and \diath{2} differ in terms of verbal marking.
		\end{enumerate}
	\end{minipage}
\end{center}

\noindent
\begin{center}
	\begin{minipage}{0.80\textwidth}
		\textbf{Definition of absolute antipassive} \newline
		An \textsc{absolute antipassive voice relation} denotes a \isi{diathetic relation} involving two diatheses\is{diathesis}, \diath{1} and \diath{2}, if a comparison between these diatheses\is{diathesis} fulfills the criteria below; while an \textsc{absolute antipassive voice} denotes \diath{2} in the beforementioned \isi{voice relation}.
		\begin{enumerate}[label=\roman*)]
			\item \diath{1} and \diath{2} feature the same number of semantic participants\is{semantic participant}, one of which is an \isi{agent}.
			\item One \isi{semantic participant} in \diath{2} cannot be expressed syntactically.
			\item The abovementioned \isi{semantic participant} is \textit{not} the \isi{agent}.
			\item The verbs in \diath{1} and \diath{2} differ in terms of verbal marking.
		\end{enumerate}
	\end{minipage}
\end{center}

\smallskip

The absolute passive\is{passive voice} and absolute antipassive\is{antipassive voice} definitions are here illustrated by diathetic relations\is{diathetic relation} in the Finnic language \ili{Finnish} (\lang{ea}; \ref{ex:Finnish:eat:a}↔\ref{ex:Finnish:eat:b}) and the Oceanic language \ili{Tolai} (\lang{pn}; \ref{ex:Tolai:hit:a}↔\ref{ex:Tolai:hit:b}). In each \isi{diathetic relation} \diath{1} (i.e. \ref{ex:Finnish:eat:a} and \ref{ex:Tolai:hit:a}) and \diath{2} (i.e. \ref{ex:Finnish:eat:b} and \ref{ex:Tolai:hit:b}) feature the same number of semantic participants\is{semantic participant} (first criterion). Furthermore, one \isi{semantic participant} in \diath{2} (i.e. the \isi{semantic participant} eating the ice cream in \ref{ex:Finnish:eat:b} and the \isi{semantic participant} being hit in \ref{ex:Tolai:hit:b}) cannot be expressed syntactically (second criterion). Additionally, \diath{1} and \diath{2} in both diathetic relations\is{diathetic relation} differ in terms of verbal marking (cf. \example{-tiin} in \ili{Finnish} and \example{ki\~{}} in \ili{Tolai}; fourth criterion). Finally, the \isi{diathetic relation} in \ili{Finnish} qualifies as an absolute passive\is{passive voice} \isi{voice relation} because the \isi{semantic participant} eating ice cream is an \isi{agent}, while the \isi{diathetic relation} in \ili{Tolai} qualifies as an absolute antipassive\is{antipassive voice} \isi{voice relation} because the \isi{semantic participant} being hit is not an \isi{agent} (third criterion). According to the absolute passive\is{passive voice} definition, \diath{2} in the \ili{Finnish} absolute passive\is{passive voice} \isi{voice relation} represents an absolute passive\is{passive voice} voice; and according to the absolute antipassive\is{antipassive voice} definition, \diath{2} in the \ili{Tolai} absolute antipassive\is{antipassive voice} \isi{voice relation} represents an absolute antipassive\is{antipassive voice} voice.

\ea \ili{Finnish} (personal knowledge)
	\ea\label{ex:Finnish:eat:a}
	\gll	poika				söi						jäätelö-n				\\
			boy.\textsc{nom}	eat.\textsc{pst.3sg}	ice.cream-\textsc{acc}	\\
	\glt	‘The boy ate the ice cream’.
	\ex\label{ex:Finnish:eat:b}
	\gll	jäätelö					syö-\textbf{tiin}		\\
			ice.cream.\textsc{nom}	eat-\textsc{pst.pass}	\\
	\glt	‘The ice cream was eaten’.
	\z
\z
\ea \ili{Tolai} \citep[248]{mosel:1991}
	\ea\label{ex:Tolai:hit:a}
	\gll	a				vavina	i	kita	ra				bul 	\\
			\textsc{art}	woman	she	hit		\textsc{art}	child	\\
	\glt	‘The woman hit the child’.
	\ex\label{ex:Tolai:hit:b}
	\gll	a				vavina	i	\textbf{ki}\~{}kita	\\
			\textsc{art}	woman	she	\textsc{antp}\~{}hit				\\
	\glt	‘The woman hit’.
	\z
\z

In both the \ili{Finnish} and \ili{Tolai} voice relations\is{voice relation} the verb in \diath{2} features additional verbal marking compared to the verb in \diath{1}. However, the absolute passive\is{passive voice} and antipassive\is{antipassive voice} definitions also encompass diathetic relations\is{diathetic relation} in which \diath{1} and \diath{2} differ in terms of verbal marking (as required by the fourth criterion) but neither \isi{diathesis} features additional verbal marking compared to the other. For instance, consider the following absolute passive\is{passive voice} \isi{voice relation} in the Semitic language Modern Standard Arabic\il{Arabic, Modern Standard} (\lang{af}; \ref{ex:Arabic:write:a}↔\ref{ex:Arabic:write:b}) in which the verb in \diath{2} does not feature additional verbal marking compared to the verb in \diath{1} nor vice versa (cf. \example{kataba}~↔~\example{kutiba}). Similar examples of absolute antipassive\is{antipassive voice} voices relations in the Algonquian language \ili{Arapaho} (\lang{na}) are provided in \tabref{tab:ch4:pass-antp} on page \pageref{tab:ch4:pass-antp}.

\ea Modern Standard Arabic\il{Arabic, Modern Standard} \citep[130]{abu-chacra:2007}
	\ea\label{ex:Arabic:write:a}
	\gll	k\textbf{a}t\textbf{a}ba l-muʿallim-u l-kitāb-a \\
			write.\textsc{active.pst.3sg.m} \textsc{def}-teacher.\textsc{m-nom} \textsc{def}-book.\textsc{m-acc} \\
	\glt	‘The teacher wrote the book’.
	\ex\label{ex:Arabic:write:b}
	\gll	k\textbf{u}t\textbf{i}ba l-kitāb-u \\
			write.\textsc{pass.pst.3sg.m} \textsc{def}-book.\textsc{m-nom}\\
	\glt	‘The book was written’.
	\z
\z

Next, the following non-absolute passive\is{passive voice} definition establishes a non-absolute passive\is{passive voice} \isi{voice relation} and a non-absolute passive\is{passive voice} voice, while the non-absolute antipassive\is{antipassive voice} definition establishes a non-absolute antipassive\is{antipassive voice} \isi{voice relation} and a non-absolute antipassive\is{antipassive voice} voice. These definitions share the first and third criteria with the absolute passive\is{passive voice} and antipassive\is{antipassive voice} definitions but differ with regard to the second and third criteria. The non-absolute passive\is{passive voice} and non-absolute antipassive\is{antipassive voice} definitions are based on the assumption that one \isi{semantic participant} in one \isi{diathesis} (\diath{2}) is less likely to be expressed syntactically than others, as specified in the second criterion. This criterion reflects the \isi{demotion} often associated with the passives\is{passive voice} and antipassives\is{antipassive voice} in the literature, and even applies to languages in which semantic participants\is{semantic participant} are commonly omitted for a variety of reasons. If need be, certain semantic participants\is{semantic participant} are more likely to be expressed syntactically than others. Cases in which no \isi{semantic participant} seems to be less likely expressed syntactically are simply excluded by the definitions.

\smallskip

\noindent
\begin{center}
	\begin{minipage}{0.80\textwidth}
		\textbf{Definition of non-absolute passive} \newline
		A \textsc{non-absolute passive voice relation} denotes a \isi{diathetic relation} involving two diatheses\is{diathesis}, \diath{1} and \diath{2}, if a comparison between these diatheses\is{diathesis} fulfills the criteria below; while a \textsc{non-absolute passive voice} denotes \diath{2} in the beforementioned \isi{voice relation}.
		\begin{enumerate}[label=\roman*)]
			\item \diath{1} and \diath{2} feature the same number of semantic participants\is{semantic participant}, one of which is an \isi{agent}.
			\item One \isi{semantic participant} in \diath{2} is less likely to be expressed syntactically than other semantic participants\is{semantic participant}.
			\item The abovementioned \isi{semantic participant} is the \isi{agent}.
			\item The verb in \diath{2} has additional marking compared to the verb in \diath{1}.
		\end{enumerate}
	\end{minipage}
\end{center}

\noindent
\begin{center}
	\begin{minipage}{0.80\textwidth}
		\textbf{Definition of absolute antipassive} \newline
		A \textsc{non-absolute antipassive voice relation} denotes a \isi{diathetic relation} involving two diatheses\is{diathesis}, \diath{1} and \diath{2}, if a comparison between these diatheses\is{diathesis} fulfills the criteria below; while a \textsc{non-absolute antipassive voice} denotes \diath{2} in the beforementioned \isi{voice relation}.
		\begin{enumerate}[label=\roman*)]
			\item \diath{1} and \diath{2} feature the same number of semantic participants\is{semantic participant}, one of which is an \isi{agent}.
			\item One \isi{semantic participant} in \diath{2} is less likely to be expressed syntactically than other semantic participants\is{semantic participant}.
			\item The abovementioned \isi{semantic participant} is \textit{not} the \isi{agent}.
			\item The verb in \diath{2} has additional marking compared to the verb in \diath{1}.
		\end{enumerate}
	\end{minipage}
\end{center}

\smallskip

The non-absolute passive\is{passive voice} and non-absolute antipassive\is{antipassive voice} definitions are here illustrated by diathetic relations\is{diathetic relation} in the Central Cushitic language \ili{Khimt’anga} (\lang{af}; \ref{ex:Khimtanga:eat:a}↔\ref{ex:Khimtanga:eat:b}) and the Northern Chukotko-Kamchatkan language \ili{Chukchi} (\lang{ea}; \ref{ex:Chukchi:catch:a}↔\ref{ex:Chukchi:catch:b}). In each \isi{diathetic relation} \diath{1} (i.e. \ref{ex:Khimtanga:eat:a} and \ref{ex:Chukchi:catch:a}) and \diath{2} (i.e. \ref{ex:Khimtanga:eat:b} and \ref{ex:Chukchi:catch:b}) feature the same number of semantic participants\is{semantic participant} (first criterion). Furthermore, one \isi{semantic participant} in \diath{2} (i.e. the \isi{semantic participant} eating the bread in \ref{ex:Khimtanga:eat:b} and the \isi{semantic participant} being caught in \ref{ex:Chukchi:catch:b}) is less likely to be expressed syntactically than other semantic participants\is{semantic participant} (second criterion). Additionally, \diath{2} in both diathetic relations\is{diathetic relation} features additional verbal marking compared to \diath{1} (cf. \example{-ɨʃit} in \ili{Khimt’anga} and \example{ine-} in \ili{Chukchi}; fourth criterion). Finally, the \isi{diathetic relation} in \ili{Khimt’anga} qualifies as a non-absolute passive\is{passive voice} \isi{voice relation} because the \isi{semantic participant} eating the bread is an \isi{agent}, while the \isi{diathetic relation} in \ili{Chukchi} qualifies as a non-absolute antipassive\is{antipassive voice} \isi{voice relation} because the \isi{semantic participant} being caught is not an \isi{agent} (third criterion). According to the non-absolute passive\is{passive voice} definition, \diath{2} in the \ili{Khimt’anga} non-absolute passive\is{passive voice} \isi{voice relation} represents a non-absolute passive\is{passive voice} voice; and according to the non-absolute antipassive\is{antipassive voice} voice definition, \diath{2} in the \ili{Chukchi} non-absolute antipassive\is{antipassive voice} \isi{voice relation} represents a non-absolute antipassive\is{antipassive voice} voice.

\ea \ili{Khimt’anga} \citep[235]{belay:2015}
	\ea\label{ex:Khimtanga:eat:a}
	\gll	ədʒɨr-d 			χabəʃə-d 			χʷ-Ø-u					\\
			man-\textsc{def}	bread-\textsc{def}	eat-\textsc{3sg.m-pfv}	\\
	\glt	‘The man ate the bread’.
	\ex\label{ex:Khimtanga:eat:b}
	\gll	χabəʃə-d {\ob}ədʒɨr-iz{\cb} χʷ-\textbf{ɨʃit}-Ø-u \\
			bread-\textsc{def} {\db}man-\textsc{ins} eat-\textsc{pass-3sg.m-pfv} \\
	\glt	‘The bread was eaten [by the man]’.
	\z
\z
\ea \ili{Chukchi} \citep[314]{polinsky:2017}
	\ea\label{ex:Chukchi:catch:a}
	\gll	ʔətt-e 				melota-lɣən 		piri-nin 					\\
			dog-\textsc{erg}	hare-\textsc{abs}	catch-\textsc{aor.3sg:3sg}	\\
	\glt	‘The dog caught a/the hare’.
	\ex\label{ex:Chukchi:catch:b}
	\gll	ʔətt-ən \textbf{ine}-piri-ɣʔi {\ob}melot-etə{\cb} \\
			dog-\textsc{abs} \textsc{antp}-catch-\textsc{aor.3sg} {\db}hare-\textsc{dat} \\
	\glt	‘The dog caught [a/the hare]’.
	\z
\z

Observe that the verb in \diath{2} is required to have additional marking compared to the verb in \diath{1} according to the fourth criterion in the definitions of non-absolute passive\is{passive voice} and antipassive\is{antipassive voice} voice relations\is{voice relation}, unlike in the definitions of absolute passive\is{passive voice} and antipassive\is{antipassive voice} voice relations\is{voice relation}. This requirement ensures a successful identification of \diath{2} in non-absolute passive\is{passive voice} and antipassive\is{antipassive voice} voice relations\is{voice relation} in cases where two diatheses\is{diathesis} feature a \isi{semantic participant} which is less likely to be expressed syntactically. For example, although that which is eaten is expressed syntactically in the \ili{Khimt’anga} \isi{diathesis} in (\ref{ex:Khimtanga:eat:a}), it can alternatively be omitted depending on context \citep[345]{belay:2015}. If the verbs in \diath{1} and \diath{2} were only required to differ in terms of verbal marking, the \isi{diathesis} in (\ref{ex:Khimtanga:eat:a}) would qualify equally well as \diath{1} and \diath{2} -- and so would the \isi{diathesis} in (\ref{ex:Khimtanga:eat:b}). One consequence of the fourth criterion is that non-absolute passive\is{passive voice} and antipassive\is{antipassive voice} counterparts to the absolute passive\is{passive voice} and antipassive\is{antipassive voice} diathetic relations\is{diathetic relation} described for Modern Standard Arabic\il{Arabic, Modern Standard} and \ili{Arapaho} are excluded by the definitions. Consider, for instance, the three diathetic relations\is{diathetic relation} in the Interior Salish language \ili{Nxa’amxcin} (\lang{na}) in \tabref{tab:ch2:Nxaamxcin} in which neither \isi{diathesis} features additional verbal marking. From a \isi{language-specific} perspective, the “antipassive\is{antipassive voice}” suffix on the right side of the bidirectional arrow (i.e. \example{-m}) is simply in variation with a “\isi{transitive}” suffix (e.g. \example{-stu}, \example{-nt}, and \example{-ɫt}) on the left side of the arrow. The diathetic relations\is{diathetic relation} otherwise fulfill all criteria (but the fourth) in the non-absolute antipassive\is{antipassive voice} voice definition \textit{if} it is assumed that the diatheses\is{diathesis} featuring the suffix \example{-m} are identified as \diath{2}. The closely related Central Salish language \ili{Musqueam} (\lang{na}) features a very similar phenomenon \citep{suttles:2004}.

\begin{table}
	\begin{tabularx}{0.85\textwidth}{lllll}
		\lsptoprule
		\example{ʔawʼtap-\textbf{stu}-} & ‘to follow sb.’ & ↔ & \example{ʔawʼtap-\textbf{m}-} & ‘to follow [sb.]’ \\
		\example{pʼiq-\textbf{nt}-} & ‘to cook sth.’ & ↔ & \example{pʼiq-\textbf{m}-} & ‘to cook [sth.]’ \\
		\example{wik-\textbf{ɫt}-} & ‘to see sth.’ & ↔ & \example{wik-\textbf{m}-} & ‘to see [sth.]’ \\
		\lspbottomrule
	\end{tabularx}
	\caption{Diathetic relations in Nxa’amxcin \citep[103, 164, 190]{willett:2003}}
	\label{tab:ch2:Nxaamxcin}
\end{table} 

\newpage

It is difficult to establish a cross-linguistically comparable criterion according to which \diath{2} can be successfully identified in diathetic relations\is{diathetic relation} like those illustrated for \ili{Nxa’amxcin} in \tabref{tab:ch2:Nxaamxcin}. One solution would be to alter the fourth criterion of the non-absolute antipassive\is{antipassive voice} definition so that it requires only that the verbs in \diath{1} and \diath{2} differ in terms of verbal marking (as in the absolute antipassive\is{antipassive voice} definition) and then specify that \diath{1} represents an \isi{active voice} unlike \diath{2}. However, as already discussed in \sectref{active-voice}, an \isi{active voice} poses its own definitional problems. Instead, for the sake of consistency, the phenomena described for \ili{Nxa’amxcin} and \ili{Musqueam} are simply not recognised as proper non-absolute antipassives\is{antipassive voice} as they do not comply fully with the non-absolute antipassive\is{antipassive voice} definition. However, the phenomena in the two languages is henceforth called “antipassive-like” (referring to the \isi{language-specific} constructions in the respective languages) and will be mentioned a few times in subsequent chapters, although they are kept strictly separated from proper antipassives\is{antipassive voice}. No other languages in the language sample feature similar phenomena, and no corresponding “passive-like” phenomenon has been attested in the sample either. Likewise, as the definitions of both absolute and non-absolute passive\is{passive voice} and antipassive\is{antipassive voice} voice relations\is{voice relation} require that the verbal marking in \diath{1} and \diath{1} must differ somehow, “uncoded alternations”\is{uncoded alternation} of various kinds described as passive\is{passive voice} or antipassive\is{antipassive voice} in the literature (e.g. \citealt[188ff.]{zuniga:kittila:2019}) are not covered by this book. Consequently, diathetic relations\is{diathetic relation} like the following in the Western Mande language \ili{Bambara} (\lang{af}; \ref{ex:Bambara:eat:a}↔\ref{ex:Bambara:eat:b}) and the Oceanic language East Uvean\il{Uvean, East} (\lang{pn}; \ref{ex:EastUvean:weed:a}↔\ref{ex:EastUvean:weed:b}) qualify as neither passive\is{passive voice} nor antipassive\is{antipassive voice} voice relations\is{voice relation}.

\ea \ili{Bambara} \citep[112]{creissels:2016}
	\ea\label{ex:Bambara:eat:a}
	\gll	wùlû 				má 				sògô 				dún 	\\
			dog.\textsc{det} 	\textsc{neg} 	meat.\textsc{det} 	eat 	\\
	\glt	‘The dog did not eat the meat’.
	\ex\label{ex:Bambara:eat:b}
	\gll	sògô				má				dún		{\ob}wùlú				fɛ̀{\cb} \\
			meat.\textsc{det}	\textsc{neg}	eat		{\db}dog.\textsc{det}	beside	\\
	\glt	‘The meat was not eaten [by the dog]’.
	\z
\z
\ea East Uvean\il{Uvean, East} \citep[110]{creissels:2016}
	\ea\label{ex:EastUvean:weed:a}
	\gll	ʻe				huo		e				Soane	tana	gāueʻaga	ʻufi	\\
			\textsc{npst} 	weed 	\textsc{erg} 	Soane 	his 	field 		yam		\\
	\glt	‘Soane is weeding his yam field’.
	\ex\label{ex:EastUvean:weed:b}
	\gll	ʻe				huo		ia				Soane	\\
			\textsc{npst}	weed	\textsc{abs}	Soane	\\
	\glt	‘Soane is weeding’.
	\z
\z

Furthermore, note that it is not specified how semantic participants\is{semantic participant} ought to be marked morphosyntactically in the definitions of the passive\is{passive voice} and antipassive\is{antipassive voice} voices presented in this section. Such specifications are otherwise common in definitions of the voices in the literature. For instance, it is commonly stated than an object\is{object, direct} or \textsc{o/p} becomes or behaves like a subject or \textsc{s} in passives\is{passive voice}, and that \textsc{a} becomes or behaves like \textsc{s} in antipassives\is{antipassive voice} (e.g. \citealt[32]{dixon:2000}; \citealt[7ff.]{dixon:aikhenvald:2000};; \citealt[200f.]{peterson:2007};; \citealt[371, 380]{kulikov:2010}; \citealt[412]{malchukov:2016}). Likewise, it is often specified that demoted\is{demotion} agents\is{agent} in passives\is{passive voice} and demoted\is{demotion} patients\is{patient} in antipassives\is{antipassive voice} are marked in some oblique fashion,\ if they are not omitted in the first place -- in other words, treated like adjuncts. As already argued in \sectref{voices-revisited}, the various notions mentioned here are difficult to define and are often \isi{language-specific}, for which reason they are avoided here. Moreover, languages appear to differ greatly in terms of how they mark semantic participants\is{semantic participant} in passives\is{passive voice} and antipassives\is{antipassive voice}, and it is therefore hardly feasible to include one kind of marking in definitions thereof but exclude other kinds.

For the sake of illustration, consider the following diathetic relations\is{diathetic relation} in the Uto-Aztecan language \ili{Ute} (\lang{ea}; \ref{ex:Ute:eat:a}↔\ref{ex:Ute:eat:b}) and the Southeastern Pama-Nyungan language \ili{Bandjalang} (\lang{au}; \ref{ex:Bandjalang:drink:a}↔\ref{ex:Bandjalang:drink:b}). In \isi{language-specific} terms, in Ute the object\is{object, direct} ‘the meat’ marked by the oblique case\is{case, oblique} in the \isi{active voice} (\ref{ex:Ute:eat:a}) does not become nor behave like a subject in the passive\is{passive voice} voice but retains its oblique marking (\ref{ex:Ute:eat:b}). The distinction between the two cases in this language is visible only in the voicing of the last vowel: in the nominative case\is{case, nominative} it is devoiced,\is{devoicing} but in the oblique case\is{case, oblique} it is voiced \citep[93f.]{givon:2011}. Likewise, in \ili{Bandjalang} the demoted\is{demotion} object\is{object, direct} ‘water’ (\ref{ex:Bandjalang:drink:a}) retains its absolutive case\is{case, absolutive} marking in the antipassive\is{antipassive voice} voice (\ref{ex:Bandjalang:drink:b}). Diathetic relations\is{diathetic relation} like these are often regarded as problematic in relation to existing passive\is{passive voice} and antipassive\is{antipassive voice} definitions due to their \isi{argument marking}. Nevertheless, the diathetic relations\is{diathetic relation} in question comply perfectly with the passive\is{passive voice} and antipassive\is{antipassive voice} definitions presented in this section and accordingly qualify as passive\is{passive voice} and antipassive\is{antipassive voice} voice relations\is{voice relation}, respectively. Examples similar to that in Ute can be found in the language isolate \ili{Chabu} (\lang{af}; \citealt[282ff.]{kibebe:2015}), and examples similar to that in Bandjalang can be found in the Katukinan language \ili{Katukina-Kanamari} (\lang{sa}; \citealt[350]{dos-anjos:2011}). Another interesting example comes from the Samoyedic language Tundra Nenets\il{Nenets, Tundra} (\lang{ea}) in which the passive\is{passive voice} \isi{agent} is marked by the nominative case\is{case, nominative} if it is a pronoun, exactly like the passive\is{passive voice} subject (\citealt[240f.]{nikolaeva:2014};; p.c. June 27th, 2019).

\ea \ili{Ute} \citep[249f.]{givon:2011}
	\ea\label{ex:Ute:eat:a}
	\gll	ta'wach\underline{i}	tʉkuavi				tʉka-qha 			\\
			man.\textsc{nom} 		meat.\textsc{obl} 	eat-\textsc{pst} 	\\
	\glt	‘The man ate the meat’.
	\ex\label{ex:Ute:eat:b}
	\gll	tʉkuavi				tʉka-\textbf{ta}-qa 				\\
			meat.\textsc{obl}	eat-\textsc{pass-pst}	\\
	\glt	‘The meat was eaten’.
	\z 
\z
\ea \ili{Bandjalang} (\citealt[38]{austin:1982} via \citealt[201]{kittila:2002}; \citeyear[347]{kittila:2015})
	\ea\label{ex:Bandjalang:drink:a}
	\gll	ngaju				juga-ala			nyabay				\\
			\textsc{1sg.erg} 	drink-\textsc{prs} 	water.\textsc{abs}	\\
	\glt	‘Soane is weeding his yam field’.
	\ex\label{ex:Bandjalang:drink:b}
	\gll	ngay				juga-\textbf{le}-la			{\ob}nyabay{\cb}		\\
			\textsc{1sg.nom}	drink-\textsc{antp-prs}		{\db}water.\textsc{abs}	\\
	\glt	‘I am drinking [water] repeatedly’.
	\z
\z

Finally, observe that -- if not otherwise specified -- in this book absolute and non-absolute passive\is{passive voice} are mostly treated indiscriminately as passive\is{passive voice}, and absolute and non-absolute antipassives\is{antipassive voice} indiscriminately as antipassive\is{antipassive voice}. Thus, henceforth a \textsc{passive voice relation} serves as a shorthand for both an absolute passive\is{passive voice} \isi{voice relation} and a non-absolute passive\is{passive voice} \isi{voice relation}, while a \textsc{passive voice} serves as a shorthand for both an absolute passive\is{passive voice} voice and a non-absolute passive\is{passive voice} voice. Likewise, henceforth an \textsc{antipassive voice relation} serves as a shorthand for both an absolute antipassive\is{antipassive voice} \isi{voice relation} and a non-absolute antipassive\is{antipassive voice} \isi{voice relation}, while an \textsc{antipassive voice} serves as a shorthand for both an absolute antipassive\is{antipassive voice} voice and a non-absolute antipassive\is{antipassive voice} voice.

\subsection{Reflexives and reciprocals} \label{def:reflexives-reciprocals}
Unlike the passive\is{passive voice} and antipassive\is{antipassive voice} voice relations\is{voice relation} discussed in the previous section, reflexive\is{reflexive voice} and reciprocal\is{reciprocal voice} voice relations\is{voice relation} are characterised by one \isi{diathesis} (\diath{2}) featuring one \isi{semantic participant} more than another \isi{diathesis} (\diath{1}). This interdiathetic comparison has been visualised on page \pageref{fig:ch2:diathetic-relations} in \figref{fig:ch2:diathetic-relations}b which is reproduced below, and reflects the contrast found in the literature on reflexivity\is{reflexive voice} and reciprocity\is{reciprocal voice} between action upon self/selves or each other on the one hand, and action upon another \isi{semantic participant} on the other hand.

\smallskip

\noindent
\begin{center}
	\diath{1} (\textsc{v}, \textsc{p}\textsubscript{\textit{n}}) ↔ \diath{2} (\textsc{v}, \textsc{p}\textsubscript{\textit{n}+1})
\end{center}

\smallskip

The following reflexive\is{reflexive voice} definition establishes both a reflexive\is{reflexive voice} \isi{voice relation} and a reflexive\is{reflexive voice} voice, while the reciprocal\is{reciprocal voice} definition establishes both a reciprocal\is{reciprocal voice} \isi{voice relation} and a reciprocal\is{reciprocal voice} voice. The definitions are each based on four criteria. The first criterion reflects the interdiathetic comparison shown above, and is also shared by the causative\is{causative voice}, anticausative\is{anticausative voice}, and applicative\is{applicative voice} definitions presented in the following sections. The second and third criteria serve to differentiate the reflexive\is{reflexive voice} and reciprocal\is{reciprocal voice} voice relations\is{voice relation} from those voice relations\is{voice relation}. In turn, the fourth criterion serves to differentiate the reflexive\is{reflexive voice} and reciprocal\is{reciprocal voice} voice relations\is{voice relation} from each other. Note that the fourth criterion in the reflexive\is{reflexive voice} definition covers both so-called distributive and collective reflexivity\is{reflexive voice}, in other words it is not relevant whether or not the referents\is{semantic referent} are perceived as individuals or groups \citep[159ff.]{zuniga:kittila:2019}. Likewise, the fourth criterion in the reciprocal\is{reciprocal voice} definition covers most “semantic configurations” of reciprocity\is{reciprocal voice}, including so-called strong, pair, melee, radial, ring, and chain reciprocity\is{reciprocal voice} \citep{majid:al:2011, evans:al:2011}. No distinction is made here between these configurations. 

\smallskip

\noindent
\begin{center}
	\begin{minipage}{0.80\textwidth}
		\textbf{Definition of reflexive} \newline
		A \textsc{reflexive voice relation} denotes a \isi{diathetic relation} involving two diatheses\is{diathesis}, \diath{1} and \diath{2}, if a comparison between these diatheses\is{diathesis} fulfills the criteria below; while a \textsc{reflexive voice} denotes \diath{2} in the \isi{voice relation}.
		\begin{enumerate}[label=\roman*)]
			\item \diath{2} features one \isi{semantic participant} more than \diath{1}.
			\item The additional \isi{semantic participant} in \diath{2} is not a \isi{causer}.
			\item The verb in \diath{1} has additional marking compared to the verb in \diath{2}.
			\item One or more referents\is{semantic referent} of one \isi{semantic participant} in \diath{1} act(s) upon \textit{self/selves}.
		\end{enumerate}
	\end{minipage}
\end{center}

\noindent
\begin{center}
	\begin{minipage}{0.80\textwidth}
		\textbf{Definition of reciprocal} \newline
		A \textsc{reciprocal voice relation} denotes a \isi{diathetic relation} involving two diatheses\is{diathesis}, \diath{1} and \diath{2}, if a comparison between these diatheses\is{diathesis} fulfills the criteria below; while a \textsc{reciprocal voice} denotes \diath{2} in the \isi{voice relation}.
		\begin{enumerate}[label=\roman*)]
			\item \diath{2} features one \isi{semantic participant} more than \diath{1}.
			\item The additional \isi{semantic participant} in \diath{2} is not a \isi{causer}.
			\item The verb in \diath{1} has additional marking compared to the verb in \diath{2}.
			\item Two or more referents\is{semantic referent} of one \isi{semantic participant} in \diath{1} act upon \textit{each other}.
		\end{enumerate}
	\end{minipage}
\end{center}

\smallskip

The reflexive\is{reflexive voice} and reciprocal\is{reciprocal voice} voice relations\is{voice relation} differ from the causative\is{causative voice} and anticausative\is{anticausative voice} voice relations\is{voice relation} with regard to the second criterion. In the causative\is{causative voice} and anticausative\is{anticausative voice} voice relations\is{voice relation} the additional \isi{semantic participant} in \diath{2} is a \isi{causer}, unlike in the reflexive\is{reflexive voice} and reciprocal\is{reciprocal voice} voice relations\is{voice relation}. For instance, consider the following reflexive\is{reflexive voice} \isi{voice relation} in the Hokan language \ili{Chimariko} (\lang{na}; \ref{ex:Chimariko:kill:a}↔\ref{ex:Chimariko:kill:b}) as well as the causative\is{causative voice} \isi{voice relation} in the Barbacoan language \ili{Awa Pit} (\lang{sa}; \ref{ex:AwaPit:buy:a}↔\ref{ex:AwaPit:buy:b}). In the \ili{Chimariko} reflexive\is{reflexive voice} \isi{voice relation} the additional \isi{semantic participant} in \diath{2} (i.e. ‘this person’ in \ref{ex:Chimariko:kill:b}) is not a \isi{causer} unlike the additional \isi{semantic participant} in \diath{2} in the \ili{Awa Pit} causative\is{causative voice} \isi{voice relation} (i.e. ‘Carmen’ in \ref{ex:AwaPit:buy:b}).  By contrast, the applicative\is{applicative voice} \isi{voice relation} is similar to the reflexive\is{reflexive voice} and reciprocal\is{reciprocal voice} voice relations\is{voice relation} in terms of the second criterion. Consider, for example, the applicative\is{applicative voice} \isi{voice relation} in the Muskogean language \ili{Creek} (\lang{na}; \ref{ex:Creek:write:a}↔\ref{ex:Creek:write:b}) in which the additional \isi{semantic participant} in \diath{2} (i.e. ‘pen’ in \ref{ex:Creek:write:b}) is used to realise the action of writing, supplying ink, but does not cause ‘Bill’ to do the action itself. The reflexive\is{reflexive voice} and reciprocal\is{reciprocal voice} voice relations\is{voice relation} are instead differentiated from the applicative\is{applicative voice} \isi{voice relation} by the third criterion in their definitions. In the applicative\is{applicative voice} \isi{voice relation}, the verb in \diath{2} has additional marking compared to the verb in \diath{1} (cf. \ili{Creek} \example{is-} in \ref{ex:Creek:write:b}), unlike in the reflexive\is{reflexive voice} and reciprocal\is{reciprocal voice} voice relations\is{voice relation} in which the opposite is true (cf. \ili{Chimariko} \example{-yeˀw} in \ref{ex:Chimariko:kill:a}). 

\ea \ili{Chimariko} \citep[121]{jany:2009}
\ea\label{ex:Chimariko:kill:a}
	\gll	y-ekʰo-\textbf{yeˀw}-xana-t					noˀot			\\
			\textsc{1sg.a}-kill-\textsc{refl-fut-asp} 	\textsc{1sg} 	\\
	\glt	‘I am going to kill myself’.
\ex\label{ex:Chimariko:kill:b}
	\gll	noˀot pʰaˀmot čʼimar-ot y-ekʰo-xana-t \\
			\textsc{1sg} \textsc{det} person-\textsc{def} \textsc{1sg.a}-kill-\textsc{fut-asp} \\
	\glt	‘I am going to kill this person’.
	\z 
\z
\ea \ili{Awa Pit} \citep[159f.]{curnow:1997}
\ea\label{ex:AwaPit:buy:a}
	\gll	Jaime	maza	atal		pay-ti-zi					\\
			Jaime	one		chicken		buy-\textsc{pst-nlocut} 	\\
	\glt	‘Jaime bought a chicken’.
\ex\label{ex:AwaPit:buy:b} 
	\gll	Carmen=na Jaime=ta maza atal pay-\textbf{nin}-ti-zi 		\\
			Carmen=\textsc{top} Jaime=\textsc{acc} one chicken buy-\textsc{caus-pst-nlocut} \\
	\glt	‘Carmen caused Jaime to buy a chicken’.
	\z
\z
\ea \ili{Creek} \citep[392]{martin:2011}
\ea\label{ex:Creek:write:a}
	\gll	Bill 	có·ka-n 			hó·cceyc-ís				\\
			Bill 	letter-\textsc{obl}	write.\textsc{asp-ind}	\\
	\glt	‘Bill is writing a letter’.
	\ex\label{ex:Creek:write:b}
	\gll	Bill isho·ccéycka có·ka-n \textbf{is}-hó·cceyc-ís \\
			Bill pen letter-\textsc{obl} \textsc{appl}-write.\textsc{asp-ind} \\
	\glt	‘Bill is writing a letter with a pen’.
	\z
\z

The fourth criterion in the reflexive\is{reflexive voice} and reciprocal\is{reciprocal voice} definitions is used to distinguish the reflexive\is{reflexive voice} and reciprocal\is{reciprocal voice} voice relations\is{voice relation} from each other. Compare the reflexive\is{reflexive voice} \isi{voice relation} already discussed for \ili{Chimariko} (\ref{ex:Chimariko:kill:a}↔\ref{ex:Chimariko:kill:b}) to the following reciprocal\is{reciprocal voice} \isi{voice relation} in the Oceanic language \ili{Nêlêmwa} (\lang{pn}; \ref{ex:Nelemwa:watch:a}↔\ref{ex:Nelemwa:watch:b}). The referent\is{semantic referent} of the \isi{semantic participant} ‘I’ acts upon itself in \diath{1} in the \ili{Chimariko} reflexive\is{reflexive voice} voice (\ref{ex:Chimariko:kill:a}), while the referents\is{semantic referent} of the \isi{semantic participant} ‘those women’ are watching each other in \diath{1} in the \ili{Nêlêmwa} reciprocal\is{reciprocal voice} voice (\ref{ex:Nelemwa:watch:a}).

\ea \ili{Nêlêmwa} \citep[1490]{bril:2007}
\ea\label{ex:Nelemwa:watch:a}
	\gll	hli \textbf{pe}-alu-\textbf{i} hliili thaamwa	\\
			\textsc{3du} \textsc{recp}-watch-\textsc{recp} those woman \\
	\glt	‘Those women are watching each other’.
\ex\label{ex:Nelemwa:watch:b}
	\gll	hli alu i na a hliili thaamwa \\
			\textsc{3du} watch \textsc{conn} \textsc{1sg} \textsc{ag} those woman \\
	\glt	‘Those women are watching me’.
	\z 
\z

As a consequence of the third criterion in the reflexive\is{reflexive voice} and reciprocal\is{reciprocal voice} definitions, both periphrastic and “uncoded”\is{uncoded alternation} reflexives\is{reflexive voice} and reciprocals\is{reciprocal voice} of various kinds (e.g. \citealt[151ff., 195ff.]{zuniga:kittila:2019}) are excluded from the discussions in this book. Thus, diathetic relations\is{diathetic relation} like the following ones in the Lowland East Cushitic language \ili{Konso} (\lang{af}; \ref{ex:Konso:wash:a}↔\ref{ex:Konso:wash:b}) and the Finnic language Tver Karelian\il{Karelian, Tver} (\lang{ea}; \ref{ex:TverKarelian:give:a}↔\ref{ex:TverKarelian:give:b}), in which reflexivity\is{reflexive voice} and reciprocity\is{reciprocal voice} is marked solely by pronouns, qualify as neither reflexive\is{reflexive voice} nor reciprocal\is{reciprocal voice} voices as they feature no verbal voice marking. However, observe that diathetic relations\is{diathetic relation} which feature periphrastic marking in addition to voice marking do comply with the definitions (\sectref{resemblance-type1a}).

\ea \ili{Konso} \citep[51, 134]{orkaydo:2013}
\ea\label{ex:Konso:wash:a}
	\gll	anti-ʔ				isi		in=ʄaʛ-ay				\\
			\textsc{1sg-nom} 	self	1=wash-\textsc{pfv.3.m}	\\
	\glt	‘I washed myself’.
\ex\label{ex:Konso:wash:b}
	\gll	anti-ʔ toma-siʔ kutt-a in=ʄaʛ-ay \\
			\textsc{1sg-nom} bowl-\textsc{def.m/f} be.big-\textsc{m/f} 1=wash-\textsc{pfv.3.m} \\
	\glt	‘I washed the big bowl’.
	\z 
\z
\ea Tver Karelian\il{Karelian, Tver} (fieldwork)
\ea\label{ex:TverKarelian:give:a}
	\gll	hüö anne-ttih {toine toize-lla} dengua \\
			\textsc{3pl} give-\textsc{pst.3pl} {each other-\textsc{ade}} money.\textsc{part}	\\
	\glt	‘They gave each other money’.
\ex\label{ex:TverKarelian:give:b}
	\gll	hüö anne-ttih lapš-i-lla dengua \\
			\textsc{3pl} give-\textsc{pst.3pl} child-\textsc{pl-ade} money.\textsc{part} \\
	\glt	‘They gave the children money’.
	\z 
\z

As noted in the previous section, passive\is{passive voice} and antipassive\is{antipassive voice} definitions in the literature commonly specify how certain semantic participants\is{semantic participant} ought to be marked morphosyntactically, and this is also true for reflexive\is{reflexive voice} and reciprocal\is{reciprocal voice} definitions (e.g. \citealt[11]{dixon:aikhenvald:2000}; \citealt[16]{melchuk:1993}; \citealt[95ff.]{givon:2001b};; \citealt[384f.]{kulikov:2010}). However, in comparison with the passive\is{passive voice} and antipassive\is{antipassive voice} voices, there seems to be less cross-linguistic diversity concerning such marking in the reflexive\is{reflexive voice} and reciprocal\is{reciprocal voice} voices. In any case, as demonstrated in this section such specifications are not needed to define reflexives\is{reflexive voice} and reciprocals\is{reciprocal voice}.

\subsection{Causatives and anticausatives} \label{def:causatives-anticausatives}
The causative\is{causative voice} and anticausative\is{anticausative voice} voice relations\is{voice relation} are characterised by one \isi{diathesis} (\diath{2}) featuring one \isi{semantic participant} more than another \isi{diathesis} (\diath{1}), and in this respect the relations in question bear resemblance to the reflexive\is{reflexive voice} and reciprocal\is{reciprocal voice} voice relations\is{voice relation} described in the previous section. This interdiathetic comparison has already been visualised on page \pageref{fig:ch2:diathetic-relations} in \figref{fig:ch2:diathetic-relations}b reproduced below, and serves as the foundation for the causative\is{causative voice} and anticausative\is{anticausative voice} definitions presented in this section. The interdiathetic comparison complies with the general understanding of both causativity\is{causative voice} and anticausativity\is{anticausative voice} in the literature: the former phenomenon is often believed to add a \isi{semantic participant}, a \isi{causer}, into a situation (e.g. \citealt[11]{melchuk:1993}; \citealt[30ff.]{dixon:2000}; \citealt[13]{dixon:aikhenvald:2000}; \citealt[1136f.]{haspelmath:muller-bardey:2004};; \citealt[386]{kulikov:2010}; \citealt[96, 122]{malchukov:2015}; \citeyear[412]{malchukov:2017}), while the latter is believed to remove a \isi{causer} from a situation (e.g. \citealt[11]{melchuk:1993}; \citealt[7]{dixon:aikhenvald:2000}; \citealt[1132]{haspelmath:muller-bardey:2004}; \citealt[392]{kulikov:2010}; \citealt[90, 96f.]{malchukov:2015}).

\smallskip

\noindent
\begin{center}
	\diath{1} (\textsc{v}, \textsc{p}\textsubscript{\textit{n}}) ↔ \diath{2} (\textsc{v}, \textsc{p}\textsubscript{\textit{n}+1})
\end{center}

\smallskip

The following causative\is{causative voice} definition establishes both a causative\is{causative voice} \isi{voice relation} and a causative\is{causative voice} voice, while the anticausative\is{anticausative voice} definition establishes both an anticausative\is{anticausative voice} \isi{voice relation} and an anticausative\is{anticausative voice} voice. The definitions are each based on three criteria, the first criterion of which is also shared by the reflexive\is{reflexive voice}, reciprocal\is{reciprocal voice}, and applicative\is{applicative voice} definitions. In turn, the second criterion serves to differentiate the causative\is{causative voice} and anticausative\is{anticausative voice} voice relations\is{voice relation} from those three voice relations\is{voice relation}, as already illustrated in the previous section. Thus, the first and second criteria are the same in both the causative\is{causative voice} and anticausative\is{anticausative voice} definitions, and the voice relations\is{voice relation} are ultimately differentiated by the third criterion.

\smallskip

\noindent
\begin{center}
	\begin{minipage}{0.80\textwidth}
		\textbf{Definition of causative} \newline
		A \textsc{causative voice relation} denotes a \isi{diathetic relation} involving two diatheses\is{diathesis}, \diath{1} and \diath{2}, if a comparison between these diatheses\is{diathesis} fulfills the criteria below; while a \textsc{causative voice} denotes \diath{2} in the \isi{voice relation}.
		\begin{enumerate}[label=\roman*)]
			\item \diath{2} features one \isi{semantic participant} more than \diath{1}.
			\item The additional \isi{semantic participant} in \diath{2} is a \isi{causer}.
			\item The verb in \diath{2} has additional marking compared to the verb in \diath{1}.
		\end{enumerate}
	\end{minipage}
\end{center}

\noindent
\begin{center}
	\begin{minipage}{0.80\textwidth}
		\textbf{Definition of anticausative} \newline
		An \textsc{anticausative voice relation} denotes a \isi{diathetic relation} involving two diatheses\is{diathesis}, \diath{1} and \diath{2}, if a comparison between these diatheses\is{diathesis} fulfills the criteria below; while an \textsc{anticausative voice} denotes \diath{2} in the \isi{voice relation}.
		\begin{enumerate}[label=\roman*)]
			\item \diath{2} features one \isi{semantic participant} more than \diath{1}.
			\item The additional \isi{semantic participant} in \diath{2} is a \isi{causer}.
			\item The verb in \diath{1} has additional marking compared to the verb in \diath{2}.
		\end{enumerate}
	\end{minipage}
\end{center}

\smallskip

The difference between the third criterions in the two definitions is illustrated by the following diathetic relations\is{diathetic relation} in the Huitotoan language \ili{Bora} (\lang{sa}; \ref{ex:Bora:run:a}↔\ref{ex:Bora:run:b}) and the Mon-Khmer language \ili{Kammu} (\lang{ea}; \ref{ex:Kammu:shake:a}↔\ref{ex:Kammu:shake:b}). In the \ili{Bora} \isi{diathetic relation} \diath{2} features additional marking compared to \diath{1} (i.e. \example{-tsʰó} in \ref{ex:Bora:run:b}) and thus qualifies as a causative\is{causative voice} \isi{voice relation}. By contrast, in the \ili{Kammu} \isi{diathetic relation} \diath{1} features additional marking compared to \diath{2} (i.e. \example{hm-} in \ref{ex:Kammu:shake:a}) and thus qualifies as an anticausative\is{anticausative voice} \isi{voice relation}. According to the causative\is{causative voice} definition, \diath{2} in the \ili{Bora} causative\is{causative voice} \isi{voice relation} represents a causative\is{causative voice} voice; and according to the anticausative\is{anticausative voice} definition, \diath{1} in the \ili{Kammu} anticausative\is{anticausative voice} \isi{voice relation} represents an anticausative\is{anticausative voice} voice. Similar criteria are found in many existing definitions of the causative\is{causative voice} and anticausative\is{anticausative voice} voices. For example, \cite[888]{kulikov:2001} argues that “causatives\is{causative voice} sensu stricto” are “formally more complex than their non-causative counterparts” while anticausatives\is{anticausative voice} are “morphologically more complex than the causative\is{causative voice}”. Furthermore, observe that so-called “autocausatives”\is{autocausative} (e.g. ‘to stretch [oneself]’ or ‘to sit [oneself] down’) comply with the anticausative\is{anticausative voice} definition and are therefore treated accordingly. Despite the apparent use of the pronoun ‘oneself’ in the English meanings given here, \isi{autocausative} actions are hardly reflexive\is{reflexive voice} in the sense that a semantic participants\is{semantic participant} actually acts upon itself. On the contrary, the actions themselves are largely spontaneous like in the case of anticausatives\is{anticausative voice}. Thus, the \isi{animacy}-related distinction sometimes maintained between anticausatives\is{anticausative voice} and autocausatives\is{autocausative} in the literature is not adopted here and both are considered anticausative\is{anticausative voice}.

\ea \ili{Bora} \citep[144]{thiesen:weber:2012}
\ea\label{ex:Bora:run:a}
	\gll	ó				 tsɨ̀ːnɛ́-ʔì		\\
			\textsc{1sg}	run-\textsc{clf}	\\
	\glt	‘I ran’.
\ex\label{ex:Bora:run:b}
	\gll	òːʔí-ːpʲɛ́			ò-kʰɛ̀					 tsɨ́ːnɛ̀-\textbf{tsʰó}-ʔì	 \\
			dog-\textsc{sg.m}	 \textsc{1sg-obj.anim}	run-\textsc{caus-clf}		\\
	\glt	‘The dog made me run’.
	\z 
\z
\ea \ili{Kammu} (\citealt[111]{svantesson:1983} via \citealt[49]{zuniga:kittila:2019})
\ea\label{ex:Kammu:shake:a}
	\gll	tóʔ		\textbf{hm}-pɨr		\\
			table	\textsc{antc}-shake	\\
	\glt	‘The table is shaking’.
\ex\label{ex:Kammu:shake:b}
	\gll	ʔòʔ 			pɨr 	tóʔ 	\\
			\textsc{1sg}	shake	table	\\
	\glt	‘I shake the table’.
	\z 
\z

While the \isi{diathesis} characterised by additional verbal marking can be readily identified in most languages (like in \ili{Bora} and \ili{Kammu}), this can prove difficult in some languages. Consider, for instance, the four diathetic relations\is{diathetic relation} in the Tibeto-Burman language Northern Pumi\il{Pumi, Northern} (\lang{ea}) in \tabref{tab:ch2:NorthernPumi}. It is clear that the verbal marking in \diath{1} (on the left side of the bidirectional arrows) in each of these diathetic relations\is{diathetic relation} is characterised by an initial non-aspirated voiced consonant (i.e. \example{b-}, \example{dz-}, \example{d-}, and \example{ɖ-}), while the verbal marking in \diath{2} (on the right side of the bidirectional arrows) is characterised by an initial aspirated voiceless counterpart (i.e. \example{pʰ-}, \example{tsʰ-}, \example{tʰ-}, and \example{ʈʰ-}). Nevertheless, it can hardly be argued that the verb in either \isi{diathesis} has additional marking. One solution would be to simply exclude such diathetic relations\is{diathetic relation}. This would be in line with \citeauthor{kulikov:2010}’s (\citeyear{kulikov:2010}) definitions discussed above, but would also lead to the inevitable loss of linguistic diversity. Another solution would be to treat \diath{1} and \diath{2} indiscriminately as anticausative\is{anticausative voice} and causative\is{causative voice}, respectively. However, this would result in the \ili{Bora} \isi{diathesis} in (\ref{ex:Bora:run:a}) being labelled anticausative\is{anticausative voice} and the \ili{Kammu} \isi{diathesis} in (\ref{ex:Kammu:shake:b}) being labelled causative\is{causative voice}. While this is a cross-linguistically applicable solution, it contrasts with the general understanding of anticausativity\is{anticausative voice} and causativity\is{causative voice} in the literature. 

\begin{table}
	\begin{tabularx}{0.80\textwidth}{lllll}
		\lsptoprule
		\example{\textbf{b}î} & ‘to fall over’ & ↔ & \example{\textbf{pʰ}î} & ‘to push sth. over’ \\
		\example{\textbf{dz}æ̌ŋ} & ‘to be clogged up’ & ↔ & \example{\textbf{tsʰ}æ̌ŋ} & ‘to clog sth. up’ \\
		\example{\textbf{d}ǒŋ} & ‘to be dammed up’ & ↔ & \example{\textbf{tʰ}ǒŋ} & ‘to dam sth. up’ \\
		\example{\textbf{ɖ}wɐ̌} & ‘to break’ & ↔ & \example{\textbf{ʈʰ}wɐ̌} & ‘to break sth.’ \\
		\lspbottomrule
	\end{tabularx}
	\caption{Diathetic relations in Northern Pumi \citep[295]{daudey:2014}}
	\label{tab:ch2:NorthernPumi}
\end{table}

A third solution -- adopted here -- is to treat diathetic relations\is{diathetic relation} like the ones illustrated for Northern Pumi\il{Pumi, Northern} in \tabref{tab:ch2:NorthernPumi} as \isi{equipollent} causative-anticausative (\citealt[91f.]{haspelmath:1993};; called “double derivation”\is{derivation, double} by \citealt[153]{nichols:al:2004}). The following definition establishes an \isi{equipollent} causative-anticausative \isi{voice relation}. This definition is identical to the causative\is{causative voice} and anticausative\is{anticausative voice} definitions in terms of the first and second criteria but differs in its third criterion. In specifically this kind of \isi{voice relation}, \diath{1} can invariably be said to be anticausative\is{anticausative voice} and \diath{2} can invariably be said to be causative\is{causative voice}. Accordingly, in the Northern Pumi\il{Pumi, Northern} diathetic relations\is{diathetic relation} in \tabref{tab:ch2:NorthernPumi} the diatheses\is{diathesis} on the left side of the bidirectional arrow (i.e. \diath{1}) are considered anticausative\is{anticausative voice}, while the diatheses\is{diathesis} on the right side of the arrow (i.e. \diath{2}) are considered causative\is{causative voice}. If not otherwise specified, in this book the causative\is{causative voice} and anticausative\is{anticausative voice} voices in an \isi{equipollent} causative-anticausative \isi{voice relation} are treated on pair with other causatives\is{causative voice} and anticausatives\is{anticausative voice}. Thus, henceforth a \textsc{causative voice} serves as a shorthand for a causative\is{causative voice} voice in either a causative\is{causative voice} \isi{voice relation} or in an \isi{equipollent} causative-anticausative \isi{voice relation}, while an \textsc{anticausative voice} serves as a shorthand for an anticausative\is{anticausative voice} voice in either an anticausative\is{anticausative voice} \isi{voice relation} or in an \isi{equipollent} causative-anticausative \isi{voice relation}.

\smallskip

\noindent
\begin{center}
	\begin{minipage}{0.80\textwidth}
		\textbf{Definition of equipollent causative-anticausative} \newline
		An \is{equipollent}\textsc{equipollent causative-anticausative voice relation} denotes a \isi{diathetic relation} involving two diatheses\is{diathesis}, \diath{1} and \diath{2}, if a comparison between these diatheses\is{diathesis} fulfills the criteria below; while an \textsc{anticausative voice} denotes \diath{1} and a \textsc{causative voice} denotes \diath{2} in the \isi{voice relation}.
		\begin{enumerate}[label=\roman*)]
			\item \diath{2} features one \isi{semantic participant} more than \diath{1}.
			\item The additional \isi{semantic participant} in \diath{2} is a \isi{causer}.
			\item The verbs in \diath{1} and \diath{2} differ in terms of verbal marking but neither verb in \diath{1} and \diath{2} has additional verbal marking compared to the other.
		\end{enumerate}
		\label{def:equipollent:causative-anticausative}
	\end{minipage}
\end{center}

\smallskip

The causative\is{causative voice} and anticausative\is{anticausative voice} definitions presented in this section all entail a difference in verbal marking between the verbs in \diath{1} and \diath{2}. This ensures that “uncoded alternations”\is{uncoded alternation} (e.g. \citealt[181ff.]{zuniga:kittila:2019}) like the following \isi{diathetic relation} in the Berber language \ili{Ghomara} (\lang{af}; \ref{ex:Ghomara:break:a}↔\ref{ex:Ghomara:break:b}) are excluded from the discussions in this book.

\ea Ghomara \citep[317]{mourigh:2015}
\ea\label{ex:Ghomara:break:a}
	\gll	lkas 	i-ṛeẓ \\
			glass	\textsc{3sg.m}-break.\textsc{pfv}	\\
	\glt	‘The glass is broken’.
\ex\label{ex:Ghomara:break:b}
	\gll	argaz=ahen 				i-ṛeẓ								lkas	\\
			man=\textsc{sg.dem}	\textsc{3sg.m}-break.\textsc{pfv}	glass	\\
	\glt	‘The man broke the glass’.
	\z 
\z

Furthermore, observe that the causative\is{causative voice} and anticausative\is{anticausative voice} definitions do not specify the morphosyntactic marking of semantic participants\is{semantic participant}, unlike many existing definitions of the voices in the literature. For instance, it is often specified that the \isi{causer} in the causative\is{causative voice} voice is or becomes or behaves like a \isi{subject} or \textsc{a} (\citealt[31]{dixon:2000}; \citealt[13]{dixon:aikhenvald:2000}; \citealt[1137]{haspelmath:muller-bardey:2004}; \citealt[386]{kulikov:2010}; \citealt[122]{malchukov:2015}; \citeyear[412]{malchukov:2016}). Likewise, it is commonly stated that the single \isi{semantic participant} in anticausative\is{anticausative voice} voice is or behaves like a \isi{subject} or \textsc{s} (\citealt[144]{kazenin:1994}; \citealt[7]{dixon:aikhenvald:2000}; \citealt[1132]{haspelmath:muller-bardey:2004}). However, as already noted in the previous two sections, notions and specifications of this sort are avoided in the definitions presented in this book. Moreover, there seems to be considerable cross-linguistic variation with regard to the marking of non-causing semantic participants\is{semantic participant} in causatives\is{causative voice} (see, e.g., \citealt[45ff.]{dixon:2000}), and it can therefore be difficult to justify that one kind of marking is included in its definition but other kinds excluded (this issue is less pronounced for anticausatives\is{anticausative voice}). Consider, for example, the \ili{Bora} (\lang{sa}) causative\is{causative voice} \isi{voice relation} already discussed (\ref{ex:Bora:run:a}↔\ref{ex:Bora:run:a}) in which causees\is{causee} are generally marked like a direct object\is{object, direct} from a \isi{language-specific} perspective. By contrast, in the language isolate \ili{Nivkh} (\lang{ea}) causees\is{causee} can optionally be marked by the suffix \example{-ax} specifically dedicated to this very function (\ref{ex:Nivkh:go:a}↔\ref{ex:Nivkh:go:b}).

\ea \ili{Nivkh} \citep[78]{nedjalkov:al:1995}
\ea\label{ex:Nivkh:go:a}
	\gll	ōla	 vi-d’				 \\
			child	go-\textsc{fin}		\\
	\glt	‘The child went’.
\ex\label{ex:Nivkh:go:b}
	\gll	ətək	ōla(-ax)				 vi-\textbf{gu}-d’		 \\
			father	child(-\textsc{causee})\is{causee}	go-\textsc{caus-fin}	\\
	\glt	‘Father made/let the child go’.
	\z 
\z

Finally, it is worth noting that causatives\is{causative voice} differ considerably both within and across languages regarding the more precise semantic nature of the causation they denote. Indeed, some languages feature several different types of causative\is{causative voice} marking, and \cite[62]{dixon:2000} lists nine semantic parameters according to which two or more causatives\is{causative voice} may be differentiated: i) state/action, ii) \isi{transitivity}, iii) control, iv) volition,\is{volition(ality)} v) \isi{affectedness} vi) directness, vii) intention, viii) naturalness, and ix) involvement. The sixth parameter is particularly prominent in the literature, and a fundamental distinction is sometimes simply made between “direct causatives\is{causative voice}” and “indirect causatives\is{causative voice}” (e.g. \citealt[171]{comrie:1989}; \citealt[892]{kulikov:2001}; \citealt{shibatani:pardeshi:2002}; \citealt[34ff.]{zuniga:kittila:2019}). According to \cite[1138]{haspelmath:muller-bardey:2004}, in direct causatives\is{causative voice} “the \isi{causer} actively participates in the action, acting on the \isi{causee} (in order to get the content of the base verb realized), which will imply some sort of coercion in case the \isi{causee} is animate”,\is{animacy} whereas in indirect causatives\is{causative voice} “the \isi{causer} is conceived of as a mere instigator or distant cause of the realization of the verb content”. Unfortunately, many of the descriptive grammars covering the languages included in the language sample do not explore differences in causation in detail. Consequently, it has not been possible to obtain enough relevant and cross-linguistically comparable data to draw any conclusions about cross-linguistic differences regarding causation in relation to voice syncretism, and the differences are therefore largely ignored in this book and all causatives are treated on par with each other.

\subsection{Applicatives} \label{def:applicatives}
Like the reflexive\is{reflexive voice}, reciprocal\is{reciprocal voice}, causative\is{causative voice}, and anticausative\is{anticausative voice} voice relations\is{voice relation} discussed in the previous two sections, the applicative\is{applicative voice} \isi{voice relation} is characterised by one \isi{diathesis} (\diath{2}) featuring one \isi{semantic participant} more than another \isi{diathesis} (\diath{1}), as already visualised on page \pageref{fig:ch2:diathetic-relations} in \figref{fig:ch2:diathetic-relations}b and reproduced here. This interdiathetic comparison serves as the foundation for the applicative\is{applicative voice} definition presented in this section and complies with the general understanding of applicativity\is{applicative voice} involving an additional \isi{semantic participant} being added to a situation (e.g. \citealt[144f.]{kazenin:1994}; \citealt[31]{dixon:2000}; \citealt[13f.]{dixon:aikhenvald:2000}; \citealt[389]{kulikov:2010}; \citealt[90, 96]{malchukov:2015}; \citeyear[413]{malchukov:2016}; \citealt[47]{zuniga:kittila:2019}).

\smallskip

\noindent
\begin{center}
	\diath{1} (\textsc{v}, \textsc{p}\textsubscript{\textit{n}}) ↔ \diath{2} (\textsc{v}, \textsc{p}\textsubscript{\textit{n}+1})
\end{center}

\smallskip

The following applicative\is{applicative voice} definition establishes both an applicative\is{applicative voice} \isi{voice relation} and an applicative\is{applicative voice} voice. The definition is based on three criteria. The first criterion reflects the interdiathetic comparison shown above, while the second and third criteria serve to differentiate the applicative\is{applicative voice} \isi{voice relation} from the reflexive\is{reflexive voice}, reciprocal\is{reciprocal voice}, causative\is{causative voice}, and anticausative\is{anticausative voice} voice relations\is{voice relation}, as already illustrated in \sectref{def:reflexives-reciprocals}. 

\smallskip

\noindent
\begin{center} \label{def:applicative}
	\begin{minipage}{0.80\textwidth}
		\textbf{Definition of applicative} \newline
		An \textsc{applicative voice relation} denotes a \isi{diathetic relation} involving two diatheses\is{diathesis}, \diath{1} and \diath{2}, if a comparison between these diatheses\is{diathesis} fulfills the criteria below; while an \textsc{applicative voice} denotes \diath{2} in the \isi{voice relation}.
		\begin{enumerate}[label=\roman*)]
			\item \diath{2} features one \isi{semantic participant} more than \diath{1}.
			\item The additional \isi{semantic participant} in \diath{2} is not a \isi{causer}.
			\item The verb in \diath{2} has additional marking compared to the verb in \diath{1}.
		\end{enumerate}
	\end{minipage}
\end{center}

\smallskip

An applicative\is{applicative voice} \isi{voice relation} has already been illustrated in the Muskogean language \ili{Creek} (\lang{na}) on page \pageref{ex:Creek:write:a} (\ref{ex:Creek:write:a}↔\ref{ex:Creek:write:b}) but for the sake of illustration in this section, another applicative\is{applicative voice} \isi{voice relation} here follows in the South Guaicuruan language \ili{Pilagá} (\lang{sa}; \ref{ex:Pilaga:dance:a}↔\ref{ex:Pilaga:dance:b}). In this \isi{voice relation} \diath{2} features an additional \isi{semantic participant} which is not a \isi{causer} (i.e. ‘the woman’ in \ref{ex:Pilaga:dance:b}; first and second criteria) in addition to additional marking (i.e. \example{-lege}) not found in \diath{1} (third criterion). In accordance with the applicative\is{applicative voice} definition, \diath{2} represents an applicative\is{applicative voice} voice. The third criterion ensures that various periphrastic constructions do not qualify as applicative\is{applicative voice} voice. Consider, for instance, the following examples from the Central Pama-Nyungan language \ili{Diyari} (\lang{au}; \ref{ex:Diyari:go:a}--\ref{ex:Diyari:go:c}). The diatheses\is{diathesis} in (\ref{ex:Diyari:go:b}) and (\ref{ex:Diyari:go:c}) both feature one \isi{semantic participant} more than the \isi{diathesis} in (\ref{ex:Diyari:go:a}), but only the diatheses\is{diathesis} in (\ref{ex:Diyari:go:a}) and (\ref{ex:Diyari:go:c}) differ in terms of diathetic marking (i.e. \example{-lka}) and thereby qualify as an applicative\is{applicative voice} \isi{voice relation}. If no difference in verbal marking were required, the \isi{diathetic relation} (\ref{ex:Diyari:go:a}↔\ref{ex:Diyari:go:b}) would also qualify as applicative\is{applicative voice} voice -- a result that does not reflect the general understanding of applicativity\is{applicative voice} in the literature. For very similar examples in the related Northern Pama-Nyungan language \ili{Yidiny}, see \cite[109]{dixon:1977}.

\ea \ili{Pilagá} \citep[318]{vidal:2001}
\ea\label{ex:Pilaga:dance:a}
	\gll	d-asot				\\
			\textsc{3sg}-dance	\\
	\glt	‘S/he dances’.
\ex\label{ex:Pilaga:dance:b}
	\gll	d-asot-e-lege 							hada’ 			yawo	\\
			\textsc{3sg}-dance-\textsc{ep-appl} 	\textsc{dem.f} 	woman	\\
	\glt	‘S/he dances for the woman’.
	\z 
\z \newpage
\ea \ili{Diyari} (\citealt[4f.]{austin:2005};; see also \citealt[264]{kittila:2002})
\ea\label{ex:Diyari:go:a}
	\gll	karna				wapa-yi			\\
			man.\textsc{abs}	go-\textsc{prs}	\\
	\glt	‘The man is going’.
\ex\label{ex:Diyari:go:b}
	\gll	karna-li			wapa-yi			wilha-nhi			\\
			man-\textsc{erg}	go-\textsc{prs}	woman-\textsc{loc}	\\
	\glt	‘The man is going with the woman’.
\ex\label{ex:Diyari:go:c}
	\gll	karna-li			wilha				wapa-\textbf{lka}-yi	\\
			man-\textsc{erg}	woman.\textsc{abs}	go-\textsc{appl-prs}	\\
	\glt	‘The man is going with the woman’.
	\z 
\z

Some applicative\is{applicative voice} definitions in the literature explicitly declare that the additional \isi{semantic participant} found in \diath{2} but not in \diath{1} in an applicative \isi{voice relation} -- henceforth called \textsc{applicative participant}\is{applicative participant} -- reflects some kind of peripheral \isi{semantic participant} in another \isi{diathesis}. For instance, \cite[13]{dixon:aikhenvald:2000} claim that in the process of \isi{applicativisation}, a “peripheral argument (which could be explicitly stated in the underlying \isi{intransitive}) is taken into the core” (see also \citealt[32]{dixon:2000}). In turn, \cite[389]{kulikov:2010} notes that in applicatives\is{applicative voice} “[t]he Direct Object\is{object, direct} may denote an entirely new participant in the situation, or it can be promoted\is{promotion} from the periphery of the syntactic structure”. In a similar manner, \cite[53]{zuniga:kittila:2019} argue that in the applicative\is{applicative voice} voice the “primary/direct object\is{object, direct} corresponds to an adjunct or non-core argument in the non-applicative voice, or to a participant that is introduced to the clause as primary/direct object”.\is{object, direct} In some languages this does indeed seem to be the case. For instance, in the \ili{Diyari} examples discussed above, the semantic participants\is{semantic participant} ‘the woman’ in (\ref{ex:Diyari:go:b}) and (\ref{ex:Diyari:go:c}) do appear to reflect each other in terms of meaning and function. However, the distinction between arguments and adjuncts (or core and periphery) is not applicable cross-linguistically (\sectref{arguments-adjuncts}). Moreover, in some languages there is no alternative to the use of an applicative\is{applicative voice} voice for certain semantic functions, in which case the \isi{applicative participant} cannot be considered a reflection of any other \isi{semantic participant}. For example, in the Bantu language (ci)Lubà\il{Lubà} (\lang{af}) the applicative\is{applicative voice} voice must be employed when one wants to express a beneficiary\is{benefactive} or a recipient \citep[104f., 107, 116]{de-kind:bostoen:2012}. Consider the following \ili{Lubà} applicative\is{applicative voice} \isi{voice relation} (\ref{ex:Luba:carry:a}↔\ref{ex:Luba:carry:b}) in which the beneficiary\is{benefactive} ‘the mother’ in (\ref{ex:Luba:carry:b}) cannot be replaced by, say, a prepositional phrase with \example{bwà} ‘for’ (*\example{bwà maamù}) nor be expressed in any other way. \cite[85]{creissels:2016} observes that such “obligatory applicatives\is{applicative voice} are particularly common among the languages of Subsaharan Africa”. By contrast, applicatives\is{applicative voice} like the one discussed for \ili{Diyari} (\ref{ex:Diyari:go:a}↔\ref{ex:Diyari:go:c}) can be characterised as “optional” (\citealt[45ff.]{peterson:2007}). The applicative\is{applicative voice} definition presented in this section encompasses both optional and obligatory applicatives\is{applicative voice} and does not make any distinction between them.

\ea Lubà \citep[103]{de-kind:bostoen:2012}
\ea\label{ex:Luba:carry:a}
	\gll	ba-àna bà-di ù-ambul-a mi-kàndà \\
			\textsc{clf2}-child \textsc{clf2}-be \textsc{clf1}-carry-\textsc{fin} \textsc{clf4}-book \\
	\glt	‘The children are carrying the books’.
\ex\label{ex:Luba:carry:b}
	\gll	ba-àna bà-di bà-ambul-\textbf{il}-a maamù mi-kàndà \\
			\textsc{clf2}-child \textsc{clf2}-be \textsc{clf2}-carry-\textsc{appl-fin} mother \textsc{clf4}-book \\
	\glt	‘The children are carrying the books for the mother’.
	\z 
\z

Furthermore, note that the applicative\is{applicative voice} definition presented in this section does not specify the morphosyntactic marking of semantic participants\is{semantic participant}, unlike many existing definitions. For instance, it is commonly stated that an \isi{applicative participant} is treated like \textsc{p}, or that \textsc{s} becomes \textsc{a}, or that a \isi{subject} or \textsc{a} remains unchanged in the process of \isi{applicativisation} (e.g. \citealt[31]{dixon:2000}; \citealt[13f.]{dixon:aikhenvald:2000}; \citealt[412f.]{malchukov:2016}; \citealt[53]{zuniga:kittila:2019}). Nevertheless, as already mentioned repeatedly in the previous sections, criteria like these are entirely avoided in the definitions employed in this book. Moreover, the morphosyntactic marking of the various roles differ greatly cross-linguistically, and it would be difficult to argue for one kind of marking being included in a definition but other kinds excluded. \cite[63]{zuniga:kittila:2019} highlight this diversity and remark that “[m]ost formal variation with applicatives\is{applicative voice} stems from the fact that not all [applicative participants]\is{applicative participant} acquire all properties associated with direct/primary objects”\is{object, direct} (see also \citealt{beck:2009}). For the sake of illustration, consider the following \isi{diathetic relation} in the Japonic language \ili{Irabu} (\lang{ea}; \ref{ex:Irabu:fall:a}↔\ref{ex:Irabu:fall:b}). In this \isi{diathetic relation} ‘rain’ falls in both \diath{1} (i.e. \ref{ex:Irabu:fall:a}) and \diath{2} (i.e. \ref{ex:Irabu:fall:b}) yet is not treated like a \isi{subject} from a \isi{language-specific} perspective in the latter \isi{diathesis}. Instead, it is treated like an adjunct, while the \isi{semantic participant} being detrimentally affected\is{affectedness} by the falling ‘I’ is treated like a \isi{subject} and not like a direct object\is{object, direct} -- as otherwise expected in many existing applicative\is{applicative voice} definitions. \cite[776]{kishimoto:al:2015} provide very similar examples from the related language \ili{Japanese} where the \isi{diathetic relation} is generally called “\isi{adversative}” or “adversative passive” (\citealt[76ff., 244]{zuniga:kittila:2019}). However, the \isi{diathetic relation} in \ili{Irabu} complies well with the applicative\is{applicative voice} definition presented in this section and therefore qualifies as an applicative\is{applicative voice} \isi{voice relation} and is treated accordingly. Note that the verbal stem in both (\ref{ex:Irabu:fall:a}) and (\ref{ex:Irabu:fall:b}) is the same (\sectref{sec:simple-syncretism:appl-pass}).

\ea \ili{Irabu} \citep[495]{shimoji:2008}
\ea\label{ex:Irabu:fall:a}
	\gll	ami=nu=du 				fïï 	\\
			rain=\textsc{nom=foc} 	fall 	\\
	\glt	‘Rain falls’.
\ex\label{ex:Irabu:fall:b}
	\gll	ba=a 				ami=n=du 				ff-\textbf{ai}-r 		\\
			\textsc{1sg=top} 	rain=\textsc{dat=foc} 	fall-\textsc{appl-npst}	\\
	\glt	‘I am bothered by rain (that) falls’. (i.e. ‘Rain falls to my detriment.')
	\z 
\z

Finally, it is worth observing that applicatives\is{applicative voice} are functionally heterogeneous, as suggested by the various applicative\is{applicative voice} examples presented in this section, and tend to be grouped according to the semantic nature of their \isi{applicative participant}. In a typological study of applicatives\is{applicative voice} in 100 languages, \cite[202f.]{peterson:2007} observes that the most common semantic functions of the \isi{applicative participant} are \isi{benefactive} or \isi{malefactive}, comitative, locative, and instrumental. The \isi{benefactive} function has been illustrated in \ili{Pilagá} (\ref{ex:Pilaga:dance:b}) and \ili{Lubà} (\ref{ex:Luba:carry:b}), the \isi{malefactive} function in \ili{Irabu} (\ref{ex:Irabu:fall:b}), and the comitative function in \ili{Diyari} (\ref{ex:Diyari:go:c}). Another common function is variously characterised as dative, goal, or directive (see, e.g., \citealt[1135]{haspelmath:muller-bardey:2004}; \citealt[187]{peterson:2007}) and basically indicates that an action is somehow directed towards the \isi{applicative participant}. Less common functions exist as well. For instance, the Skou language \ili{Barupu} (\lang{pn}) features a caritive/privative applicative\is{applicative voice} in which an action is done without the \isi{applicative participant} \citep[258f.]{corris:2005}, and the Lower Sepik language \ili{Yimas} (\lang{pn}) possesses visual applicatives\is{applicative voice} indicating that an action is done “while carefully watching another animate\is{animacy} [applicative] participant” \citep[315]{foley:1991}. While the cross-linguistic differences in the nature of the \isi{applicative participant} are interesting in their own right, many of the descriptive grammars covering languages in the sample of the this book do not explore the functional extents of applicatives\is{applicative voice} in detail. It has therefore proven difficult to obtain sufficient relevant and cross-linguistically comparable data on the languages to allow for any conclusive statements to be made about the semantic function(s) of the \isi{applicative participant} in relation to voice syncretism. Consequently, the differences are largely ignored in this book and applicatives\is{applicative voice} are treated on par with each other regardless of the semantic function(s) of their \isi{applicative participant}. This is also the reason why the applicative\is{applicative voice} definition presented in this section on page \pageref{def:applicative} does not explicitly mention the \isi{applicative participant}.

\subsection{Overview} \label{def:overview}
The fundamental distinction in interdiathetic comparison between the passive\is{passive voice} and antipassive\is{antipassive voice} voice relations\is{voice relation} on the one hand, and the reflexive\is{reflexive voice}, reciprocal\is{reciprocal voice}, causative\is{causative voice}, anticausative\is{anticausative voice}, and applicative\is{applicative voice} voice relations\is{voice relation} on the other hand (see \tabref{fig:ch2:diathetic-relations} on \pageref{fig:ch2:diathetic-relations}) is illustrated once again in \tabref{tab:ch2:overview}. The passive\is{passive voice} and antipassive\is{antipassive voice} voice relations\is{voice relation} are defined according to a comparison of two diatheses\is{diathesis}, both of which feature the same number of semantic participants\is{semantic participant} (\textsc{p}\textsubscript{\textit{n}} = \textsc{p}\textsubscript{\textit{n}}), whereas the reflexive\is{reflexive voice}, reciprocal\is{reciprocal voice}, causative\is{causative voice}, anticausative\is{anticausative voice}, and applicative\is{applicative voice} voice relations\is{voice relation} are defined according to a comparison of two diatheses\is{diathesis}, one of which features one \isi{semantic participant} more than the other (\textsc{p}\textsubscript{\textit{n}} ≠ \textsc{p}\textsubscript{\textit{n}+1}). \tabref{tab:ch2:overview} also provides an overview of the various similarities and dissimilarities between the seven voice relations\is{voice relation}. 

\begin{table}\setlength{\tabcolsep}{4.2pt}
	\begin{tabularx}{\textwidth}{lccccc}
		\lsptoprule
		& & \multicolumn{2}{c}{Absolute} & \multicolumn{2}{c}{Non-absolute} \\
		\diath{1} (\textsc{v}, \textsc{p}\textsubscript{\textit{n}}) ↔ \diath{2} (\textsc{v}, \textsc{p}\textsubscript{\textit{n}}) & & \textsc{pass} & \textsc{antp} & \textsc{pass} & \textsc{antp} \\
		\midrule
		The \isi{agent} is the least likely semantic & & \multirow{2}{*}{+} & \multirow{2}{*}{--} & \multirow{2}{*}{+} & \multirow{2}{*}{--} \\
		participant to be expressed syntactically. & & & & & \\
		One \isi{semantic participant} in \diath{2} cannot & & \multirow{2}{*}{+} & \multirow{2}{*}{+} & \multirow{2}{*}{--} & \multirow{2}{*}{--} \\
		be expressed syntactically. & & & & & \\
		\midrule\midrule
		\diath{1} (\textsc{v}, \textsc{p}\textsubscript{\textit{n}}) ↔ \diath{2} (\textsc{v}, \textsc{p}\textsubscript{\textit{n}+1}) & \textsc{refl} & \textsc{recp} & \textsc{caus} & \textsc{antc} & \textsc{appl} \\
		\midrule
		Additional \isi{semantic participant} in \diath{2} & \multirow{2}{*}{--} & \multirow{2}{*}{--} & \multirow{2}{*}{+} & \multirow{2}{*}{+} & \multirow{2}{*}{--} \\
		is a \isi{causer}. & & & & & \\
		Verb in \diath{2} has additional marking com- & \multirow{2}{*}{--} & \multirow{2}{*}{--} & \multirow{2}{*}{+} & \multirow{2}{*}{--} & \multirow{2}{*}{+} \\
		pared to verb in \diath{1}. & & & & & \\
		Referents\is{semantic referent} of one \isi{semantic participant} & \multirow{2}{*}{--} & \multirow{2}{*}{+} & & & \\
		in \diath{1} act upon each other. & & & & & \\
		\lspbottomrule
	\end{tabularx}
	\caption{Overview of voice definitions}
	\label{tab:ch2:overview}
\end{table} 

The \isi{agent} is the least likely \isi{semantic participant} to be expressed syntactically in the passive\is{passive voice} \isi{voice relation}, unlike in the antipassive\is{antipassive voice} \isi{voice relation}. In turn, the absolute passive\is{passive voice} and antipassive\is{antipassive voice} voice relations\is{voice relation} feature one \isi{semantic participant} in \diath{2} that cannot be expressed syntactically, unlike in the non-absolute passive\is{passive voice} and antipassive\is{antipassive voice} voice relations\is{voice relation}. Next, the reflexive\is{reflexive voice}, reciprocal\is{reciprocal voice}, causative\is{causative voice}, anticausative\is{anticausative voice}, and applicative\is{applicative voice} voice relations\is{voice relation} are distinguished from each other by the semantic role of the additional \isi{semantic participant} in \diath{2} as well as verbal marking. The additional \isi{semantic participant} is a \isi{causer} in the causative\is{causative voice} and anticausative\is{anticausative voice} voice relations\is{voice relation}, unlike in the reflexive\is{reflexive voice}, reciprocal\is{reciprocal voice}, and applicative\is{applicative voice} voice relations\is{voice relation}. In turn, the verb in \diath{2} has additional marking compared to the verb in \diath{1} in the causative\is{causative voice} and applicative\is{applicative voice} voice relations\is{voice relation}, unlike in the reflexive\is{reflexive voice}, reciprocal\is{reciprocal voice} and anticausative\is{anticausative voice} voice relations\is{voice relation}. Finally, the reflexive\is{reflexive voice} and reciprocal\is{reciprocal voice} voice relations\is{voice relation} are differentiated according to the behaviour of the referent(s)\is{semantic referent} of a \isi{semantic participant} in \diath{1}. The referents\is{semantic referent} in question act upon each other in the reciprocal\is{reciprocal voice} voice, unlike in the reflexive\is{reflexive voice} \isi{voice relation}.
